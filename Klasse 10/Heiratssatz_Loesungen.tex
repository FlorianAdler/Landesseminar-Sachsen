
\subsection*{Lösungen zu Kapitel~\ref{kapitel:Heiratssatz}: \emph{Der Heiratssatz}}

\begin{proof}[Lösung zu Aufgabe~\ref{aufgabe:KugelnHeiratssatz}]
	Wir betrachten einen bipartiten Graphen $G=(A\cup B,E)$, der wie folgt konstruiert ist: $A$ ist die Menge der Schalen und $B$ ist die Menge der Farben. Zwischen $a\in A$ und $b\in B$ verläuft genau dann eine Kante, wenn die Schale~$a$ eine Kugel in der Farbe~$b$  enthält. Wenn wir ein Matching $M$ konstruieren können, das alle Knoten aus $A$ überdeckt (und damit auch alle Knoten aus $B$, denn $\abs{A}=n=\abs{B}$), dann sind wir fertig, denn dann können wir aus jeder Schale eine Kugel in der ihr zugeordneten Farbe entnehmen und haben insgesamt jede der $n$ Farben genau einmal abgedeckt.
	
	Wir müssen somit Bedingung des Heiratssatzes prüfen. Sei $S\subseteq A$ eine Teilmenge von Knoten, und seien $s_1,\dotsc,s_k$ die zugehörigen Schalen (hier ist also $k=\abs{S}$). In diesen $k$ Schalen liegen insgesamt $kn$ Kugeln. Da jede Farbe nur $n$-mal auftritt, müssen unter diesen $kn$ Kugeln mindestens $k$ Farben vertreten sein. Das heißt aber nichts anderes als $\abs{N_G(S)}\geqslant k=\abs{S}$, also ist der Heiratssatz anwendbar und wir sind fertig.
\end{proof}

\begin{proof}[Lösung zu Aufgabe~\ref{aufgabe:Tracey}]
	Um seinen Plan zu verwirklichen, muss Norman $n$ Erdbeeren so auswählen, dass in jeder Zeile und jeder Spalte genau eine dieser Erdbeeren liegt. Betrachte dazu einen bipartiten Graphen $G=(A\cup B,E)$, der wie folgt konstruiert ist: $A$ ist die Menge der Zeilen des Kuchens und $B$ ist die Menge der Spalten des Kuchens. Zwischen $a\in A$ und $b\in B$ verläuft genau dann eine Kante, wenn sich die Zeile~$a$ und die Spalte~$b$ bei einem Kuchenstück schneiden, auf dem eine positive Anzahl Erdbeeren liegt. Wenn wir in $G$ ein Matching $M$ konstruieren können, das alle Knoten aus $A$ überdeckt (und damit auch alle aus $B$, denn $\abs{A}=n=\abs{B}$), haben wir die Aufgabe gelöst! Die Kanten aus $M$ bestimmen nämlich $n$ Kuchenstücke, eines in jeder Zeile und Spalte, die jedes mindestens eine Erdbeere enthalten. Dann kann Norman eine Erdbeere von jedem dieser Stücke naschen und damit die gewünschte Bedingung erfüllen.
	
	Nehmen wir also umgekehrt an, es gäbe kein solches Matching $M$. Nach dem Heiratssatz muss eine nichtleere Teilmenge $S\subseteq A$ existieren, sodass $\abs{N_G(S)}<\abs{S}$. In unser ursprüngliches Problem übersetzt heißt das: Es gibt Zeilen $z_1,z_2,\dotsc,z_k$ und Spalten $s_1,s_2,\dotsc,s_m$, wobei $m<k$, sodass alle Erdbeeren, die sich in den Zeilen $z_1,z_2,\dotsc,z_k$ befinden, schon in den Spalten $s_1,s_2,\dotsc,s_m$ enthalten sind. In den Zeilen $z_1,z_2,\dotsc,z_k$ sind nach Annahme genau $\the\year k$ Erdbeeren enthalten. Da sich alle diese Erdbeeren auch in den Spalten $s_1,s_2,\dotsc,s_m$ befinden, die insgesamt genau $\the\year m$ Erdbeeren enthalten, muss also $\the\year k\leqslant \the\year m$ gelten. Das widerspricht aber der Annahme $m<k$.
\end{proof}

Wir befolgen den Ratschlag und lösen die reine Zahlentheorie-Aufgabe~\ref{aufgabe:PolynomPrimfaktor}, bevor wir uns Aufgabe~\ref{aufgabe:VerschiedenePrimfaktoren} zuwenden.

\begin{proof}[Lösung zu Aufgabe~\ref{aufgabe:PolynomPrimfaktor} \textmd{(\href{https://artofproblemsolving.com/community/c3962_2011_imo_shortlist}{IMO-Shortlist 2011/N2})}]
	Betrachte $d\coloneqq\max\{d_1,d_2,\dotsc,d_9\}$ und wähle $N\coloneqq d^8+1$. Angenommen, für irgendein $n\geqslant N$ hat $P(n)$ nur Primfaktoren $\leqslant 20$. Sei $p_i^{e_i}$ die größte Primpotenz, die $n+d_i$ teilt. Weil $n+d_i$ nur durch die acht verschiedenen Primzahlen $2$, $3$, $5$, $7$, $11$, $13$, $17$ oder $19$ teilbar sein kann, muss $n+d_i$ ein Produkt von höchstens acht Primpotenzen sein. Folglich gilt
	\begin{equation*}
		p_i^{e_i}\geqslant \sqrt[8]{n+d_i}\geqslant \sqrt[8]{d^8+1}>d\,.
	\end{equation*}
	Nach dem Schubfachprinzip muss es ferner zwei Indizes $1\leqslant i<j\leqslant 9$ mit $p_i=p_j$ geben. Sei $p\coloneqq p_i=p_j$. Danns sind sowohl $n+d_i$ als auch $n+d_j$ durch $p^{\min\{e_i,e_j\}}$ teilbar. Folglich ist auch $\abs{d_i-d_j}$ durch $p^{\min\{e_i,e_j\}}$ teilbar. Nach Annahme gilt $\abs{d_i-d_j}<d<p^{\min\{e_i,e_j\}}$, also kommt nur $d_i=d_j$ in Frage. Das widerspricht jedoch der Annahme, dass $d_1,d_2,\dotsc,d_9$ paarweise verschieden sind.
\end{proof}

\begin{proof}[Lösung zu Aufgabe~\ref{aufgabe:VerschiedenePrimfaktoren}]
	Wir führen die Aufgabe zuerst auf ein Matching-Problem zurück. Wir konstruieren einen bipartiten Graphen $G=(A\cup B,E)$ wie folgt: $A$ ist die Menge aller Zahlen $N+1,N+2,\dotsc,N+n$ und $B$ ist die Menge aller Primzahlen, die eine der Zahlen $N+1,N+2,\dotsc,N+n$ teilen. Zwischen $a\in A$ und $b\in B$ verläuft genau dann eine Kante, wenn~$a$ durch die Primzahl~$b$ teilbar ist. Um die Aufgabe zu lösen, müssen wir in $G$ ein Matching $M$ konstruieren, das alle Knoten aus $A$ enthält, denn dann können wir jedem $N+i$ eine Primzahl $p_i\mid N+i$ zuordnen, sodass $p_i\neq p_j$ für $i\neq j$.
	
	Angenommen, das wäre unmöglich. Nach dem Heiratssatz muss also eine Menge $S\subseteq A$ mit $\abs{N_G(S)}<\abs{S}$ existieren. In unser ursprüngliches Problem übersetzt heißt das: Es gibt ganze Zahlen $1\leqslant i_1<i_2<\dotsb<i_k\leqslant n$, sodass höchstens $k-1$ verschiedene Primzahlen existieren, die irgendeine der Zahlen $N+i_1,N+i_2,\dotsc,N+i_k$ teilen. Für alle $j=1,2,\dotsc,k$ sei $q_j^{e_j}$ die größte Primpotenz, die $N+i_j$ teilt. Nach Annahme kann $N+i_j$ durch höchstens $k-1\leqslant n-1$ verschiedene Primzahlen teilbar sein, ist also ein Produkt von höchstens $n-1$ Primpotenzen. Wegen $N\geqslant n^{n-1}$ folgt daraus
	\begin{equation*}
		q_j^{e_j}>n\quad\text{für alle }j=1,2,\dotsc,k\,.
	\end{equation*}
	Andererseits gibt es höchstens $k-1$ verschiedene Primzahlen, die überhaupt irgendeine der Zahlen $N+i_1,N+i_2,\dotsc,N+i_k$ teilen. Nach dem Schubfachprinzip muss es also Indizes $j\neq \ell$ geben, für die $q_j=q_\ell$ gilt. Sei $q\coloneqq q_j=q_\ell$. Dann sind sowohl $N+i_j$ als auch $N+i_\ell$ durch $q^{\min\{e_j,e_\ell\}}$ teilbar. Folglich ist $\abs{i_j-i_\ell}$ ebenfalls durch $q^{\min\{e_j,e_\ell\}}$ teilbar. Nun gilt aber auch $\abs{i_j-i_\ell}<n<q^{\min\{e_j,e_\ell\}}$, also kommt nur $i_j=i_\ell$ in Frage. Das widerspricht jedoch unserer Annahme $j\neq \ell$.
\end{proof}

\begin{proof}[Lösung zu Aufgabe~\ref{aufgabe:AzurBordeauxCitrin} \textmd{(\href{https://artofproblemsolving.com/community/c3963_2012_imo_shortlist}{IMO-Shortlist 2012/C5})}]
	Wir konstruieren einen bipartiten Graphen $G=(A\cup A',E)$ wie folgt: $A$ ist die Menge aller Azur-farbenen Spielsteine und $A'$ ist die Menge aller Azur-farbenen Felder. Zwischen $a\in A$ und $a'\in A'$ verläuft genau dann eine Kante, wenn der Abstand zwischen dem Feld, auf dem~$a$ steht, und dem Feld~$a'$ maximal $d+2$ beträgt. Wenn wir in $G$ ein Matching $M$ konstruieren können, das alle Knoten aus $A$ überdeckt (und damit auch alle aus $A'$, denn $\abs{A}=3n^2=\abs{A'}$), dann haben gezeigt, dass alle Azur-farbenen Spielsteine wie gewünscht verschoben werden können. Mit einem analogen Argument für die Farben Bordeaux und Citrin haben wir dann die Aufgabe gelöst.
	
	Betrachte eine Menge $S\subseteq A$ von Azur-farbenen Spielsteinen. Für alle positiven reellen Zahlen $r$ sei $S_r$ die Menge aller Felder, die höchstens Abstand $r$ von einem Feld haben, auf dem ein Spielstein aus $S$ steht. Um die Bedingung des Heiratssatzes zu überprüfen, müssen wir zeigen, dass es in $S_{d+2}$ mindestens $\abs{S}$ Azur-farbene Felder gibt. Intuitiv ist das sehr plausibel: Einerseits ist durchschnittlich ein Drittel aller Felder Azur-farben, also können wir annehmen, dass $\approx\frac13\abs{S_{d+2}}$ Felder in $S_{d+2}$ Azur-farben sind. Andererseits muss es in $S_d$ mindestens $\abs{S}$ Bordeaux- und mindestens $\abs{S}$ Citrin-farbene Felder geben, denn nach Annahme existiert eine Permutation der Spielsteine, die jeden Spielstein maximal im Abstand $d$ bewegt und Azur auf Bordeaux, Bordeaux auf Citrin und Citrin auf Azur schickt. Somit ist $3\abs{S}\leqslant \abs{S_d}$. Wegen $S_d\subseteq S_{d+2}$ erhalten wir also $\abs{S}\lessapprox \frac13\abs{S_{d+2}}$.
	
	Der formale Beweis ist nicht viel schwieriger. Wir können das $3n\times 3n$-Feld mit horizontalen $3\times 1$-Rechtecken parkettieren, sodass in jedem dieser Rechtecke jede Farbe mindestens einmal vertreten ist. Wenn $R$ ein solches $3\times 1$-Rechteck ist und ein Feld von $R$ in $S_d$ enthalten ist, dann sind alle drei Felder in $S_{d+2}$ enthalten. Somit enthält $S_{d+2}$ mindestens $\frac13\abs{S_d}$ Azur-farbene Felder. Das ist genau die Abschätzung, die wir brauchen, um die obigen Plausibilitätsüberlegungen wasserdicht zu machen.
\end{proof}