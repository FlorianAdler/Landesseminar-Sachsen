\usepackage{yhmath}
\usepackage{ngerman} % a4wide,, latexsym
\usepackage[T1]{fontenc}
\usepackage{lmodern}
\usepackage[utf8]{inputenc}
\usepackage{times}
\usepackage[slantedGreek]{mathptmx}
\usepackage{amsmath}
\usepackage{amssymb}
%\usepackage{amscd}
\usepackage{exscale}
\usepackage{enumerate}
\usepackage{amsthm}
\usepackage{graphics}
\usepackage{graphicx}	
\usepackage{longtable}
\usepackage{color}
\usepackage{dsfont} 
\usepackage{bbm}
%\usepackage{wasysym}
\usepackage{ifpdf}
%\usepackage{pst-all}
%\usepackage{pstricks,pstricks-add,pst-math,pst-xkey}
\usepackage{lscape}
\usepackage{eurosym, url, hyperref}
%\usepackage{fourier}


\frenchspacing

%--------------------------------------------------------------------

\setlength\parskip{\medskipamount}
\setlength\parindent{0pt}
\setlength{\voffset}{-3cm} %-1	%-1.5 %bis -3
%%\setlength{\topmargin}{0.625cm}		% oberer Rand bis Oberkante Kopfzeile
\setlength{\oddsidemargin}{0.0cm} \setlength{\evensidemargin}{0.0cm}%	Linker Rand 
%%\setlength{\headheight}{1.25cm}		% Höhe der Kopfzeile
%%\setlength{\headsep}{0.625cm}			% Abstand zw. Kopfzeile und 
\setlength{\topskip}{0cm}
%%\setlength{\footskip}{1cm}
\setlength{\textheight}{26cm} %24 %23.5; voffset ausblenden % bis 27
\setlength{\textwidth}{16cm} %16


\setcounter{secnumdepth}{3}								% Nummerierungstiefe
\setcounter{tocdepth}{1}									% Inhaltsverzeichnistiefe
%\numberwithin{equation}{section}					% Formeln abschnittsweise nummerieren

\flushbottom
\renewcommand{\baselinestretch}{1.0}



%Abkürzungen-------------------------------------------------------------
%Zahlenbereiche----------------------------------
\newcommand{\N}{{\mathbb{N}}}
\newcommand{\Z}{{\mathbb{Z}}}
\newcommand{\Q}{{\mathbb{Q}}}
\newcommand{\R}{{\mathbb{R}}}
\newcommand{\C}{{\mathbb{C}}}	
%(Komplexe) Zahlen
\newcommand{\I}{{\mathrm{i}}}					% Imaginäre Einheit
\newcommand{\real}{{\mathrm{Re}}}			% Realteil
\newcommand{\imag}{{\mathrm{Im}}}			% Imaginärteil
\newcommand{\dual}{{\mathrm{Du}}}			% Dualteil
%Abkürzung griechischer Buchstaben & Abbildungen etc.
\newcommand{\id}{\mathrm{id}}
\newcommand{\ve}{\varepsilon}
\newcommand{\vp}{\varphi}
\newcommand{\eul}{{\mathrm{e}}} 			% Eulersche Zahl e
\newcommand{\ld}{{\mathrm{ld}}} 			% duadischer Logarithmus
%Matrizen und Vektorrechnung---------------------
\newcommand{\T}{^{\mathrm{T}}}				% Transponiertzeichen
\newcommand{\rg}{{\mathrm{rg}}}				% Rang
\newcommand{\bild}{{\mathrm{bild}}}		% Bild
\newcommand{\Kern}{{\mathrm{kern}}}		% Kern
\newcommand{\lin}{{\mathrm{lin}}}			% Kern
\newcommand{\ul}[1]{\underline{#1}} 	% Vektorunterstrich
%Algebra----------------------
\newcommand{\ggT}{{\mathrm{ggT}}}				% ggT
\newcommand{\kgV}{{\mathrm{kgV}}}				% kgV
%Geometrie
\newcommand{\ol}[1]{\overline{#1}} 	  % Strecke
\newcommand{\TV}{{\mathrm{TV}}}				% Teilverhältnis
\newcommand{\DV}{{\mathrm{DV}}}				% Doppelverhältnis
\newcommand{\mbb}[1]{\mathbb{#1}}			% \mathbb	
%Einheiten
\newcommand{\mm}{{\mbox{\,} \mathrm{mm}}}
\newcommand{\cm}{{\mbox{\,} \mathrm{cm}}}				% Einheit cm
\newcommand{\dm}{{\mbox{\,} \mathrm{dm}}}	
\newcommand{\m}{{\mbox{\,} \mathrm{m}}}	
\newcommand{\LE}{{\mbox{\,} \mathrm{LE}}}	
\newcommand{\fe}{{\mbox{\,} \mathrm{FE}}}	
\newcommand{\km}{{\mbox{\,} \mathrm{km}}}
\newcommand{\gpkcm}{{\mbox{\,} \mathrm{g}/\mathrm{cm}^3}}
\newcommand{\kmh}{{\mbox{\,} \frac{\mathrm{km}}{\mathrm{h}}}}
\newcommand{\komma}{{,}}
\newcommand{\entspricht}{\mathrel{\widehat{=}}}   		% Entspricht

\newcommand{\h}{{\mbox{\,} \mathrm{h}}}
\newcommand{\s}{{\mbox{\,} \mathrm{s}}}
\newcommand{\g}{{\mbox{\,} \mathrm{g}}}
\newcommand{\kg}{{\mbox{\,} \mathrm{kg}}}	
\newcommand{\gc}{{\grad\mathrm{C}}}	
\newcommand{\grad}{^{\circ}}				% Einheit °
\newcommand{\mbm}[1]{\mathbbm{#1}}	
\newcommand{\arc}{{\mathrm{arc}}}	
%Schüler
\newcommand{\ds}{\displaystyle}
% shortcuts
\def\ol#1{\overline{#1}}
\def\ul#1{\underline{#1}}
\def\br#1{\left(#1\right)}             % brackets
\def\sbr#1{\left[#1\right]}            % square brackets
\def\cbr#1{\left\{#1\right\}}          % curly brackets
\def\iff{\Leftrightarrow\ }
\def\yields{\Rightarrow\ }
\newcommand{\rf}[1][]{\textup{\eqref{#1}}}
\newcommand{\half}{\frac{1}{2}}
\newcommand{\third}{\frac{1}{3}}
\newcommand{\ov}{\overline}
\newcommand{\nn}{\nonumber}
\newcommand{\RRA}{{\,\,\Longrightarrow}\,\,}
\newcommand{\LRA}{{\,\,\Leftrightarrow}\,\,}
%\newcommand{\qed}[1][\rule{1ex}{1ex}]{\nopagebreak\hspace*{2em}\hspace*{\fill}{$#1$}}


%Analysis --------------------
\newcommand{\inte}[4]{\int\limits_{#1}^{#2} {#3}\mathrm{d}{#4} } 

%---------------------------------------------------------------------------------------------
\newenvironment{description*}[2]
   {\begin{list}{}{%
      \settowidth{\labelwidth}{#2{#1}}
      \setlength{\leftmargin}{\labelwidth}
         \addtolength{\leftmargin}{\labelsep}
      \setlength{\parsep}{0.5ex plus0.2ex minus0.2ex}
      \setlength{\itemsep}{0.3ex}
      \renewcommand{\makelabel}[1]{#2{##1}\hfill}}}
   {\end{list}}
%*********************************************************************************************

\newcommand{\bsp}[1]{\begin{Bsp}\hspace*{1cm}\newline#1\end{Bsp}}
\newcommand{\kartauf}[3]{
\newpage
\normalsize
\begin{tabular}{p{13cm}}
\textbf{#1 \hfill #2}\\\hline
\end{tabular}
\begin{enumerate}\small
#3
\end{enumerate}}

\newcommand{\bem}[1]{\begin{Bem}\normalfont \hspace*{1cm}\newline#1\end{Bem}}

\newcommand{\defi}[2]{\begin{Def}[#1]\hspace*{1cm}\newline#2\end{Def}}

\newcommand{\theo}[2]{\begin{Theo}[#1]\hspace*{1cm} #2\end{Theo}} %\newline

\newcommand{\satz}[2]{\begin{Satz}[#1]\hspace*{1cm}\newline#2\end{Satz}}

\newcommand{\folg}[2]{\begin{Folg}[#1]\hspace*{1cm}\newline#2\end{Folg}}

\newcommand{\auf}[1]{\begin{Auf} \normalfont#1\end{Auf}} % \hspace*{1cm}\newline

\newcommand{\lem}[1]{\begin{Lem}\hspace*{1cm}\newline#1\end{Lem}}

\newcommand{\cor}[1]{\begin{Cor}\hspace*{1cm}\newline#1\end{Cor}}

\newcommand{\tipp}[2]{\begin{Tipp}[#1]\hspace*{1cm}\newline#2\end{Tipp}}

\newcommand{\bew}[1]{\textsl{Beweis: }#1 \hfill $\Box$}  %ge{\"a}ndert (\!)

\newcommand{\loes}[1]{\textit{Lösung: }#1 \hfill $\Box$}  %ge{\"a}ndert (\!)

\newcommand{\lema}[2]{\begin{Lem}[#1]\hspace*{1cm}\newline#2\end{Lem}}

\newtheoremstyle{definition}% name
     {3pt}%      Space above
     {5pt}%      Space below
     {\itshape}%         Body font % evtl 
     {0ex}%         Indent amount (empty = no indent, \parindent = para indent)
     {\bfseries}% Thm head font
     {:}%        Punctuation after thm head
     {.5em}%     Space after thm head: " " = normal interword space;
           %       \newline = linebreak
     {}%         Thm head spec (can be left empty, meaning `normal')

\newtheoremstyle{Beweis}% name
     {30pt}%      Space above
     {3pt}%      Space below
     {}%         Body font
     {}%         Indent amount (empty = no indent, \parindent = para indent)
     {\itshape}% Thm head font
     {:}%        Punctuation after thm head
     {.5em}%     Space after thm head: " " = normal interword space;
           %       \newline = linebreak
     {}%         Thm head spec (can be left empty, meaning `normal')


\newtheoremstyle{break}% name
     {3pt}%      Space above
     {7pt}%      Space below
     {\itshape}%         Body font
     {}%         Indent amount (empty = no indent, \parindent = para indent)
     {\bfseries}% Thm head font
     {:}%        Punctuation after thm head
     {.5em}%     Space after thm head: " " = normal interword space;
           %       \newline = linebreak
     {}%         Thm head spec (can be left empty, meaning `normal')

\theoremstyle{definition}

\newtheorem{Def}{Definition}[section]


\theoremstyle{break}
\newtheorem{Satz}[Def]{Satz}
\newtheorem{Lem}[Def]{Lemma}
\newtheorem{Cor}[Def]{Korollar}
\newtheorem{Tipp}[Def]{Tipp}

\theoremstyle{definition}
\newtheorem{Theo}[Def]{Theorem}
\newtheorem{Auf}[Def]{Aufgabe}
\newtheorem{Bem}[Def]{Bemerkung}
\newtheorem{Bsp}[Def]{Beispiel}
\newtheorem{beispiel}[Def]{Beispiel}
\newtheorem{aufgabe}[Def]{Aufgabe}
\newtheorem{Folg}[Def]{Folgerung}
\newtheorem{obs}[Def]{Beobachtung}

\theoremstyle{Beweis}
\newtheorem{Bew}{Beweis}



\makeindex

\endinput
