\section{Quersummen}\label{kapitel:Quersummen}
Manchmal kommen in Olympiade-Aufgaben Quersummen vor. So zum Beispiel in den beiden folgenden Aufgaben:
\begin{aufgabe*}\label{aufgabe:441244}
	Wir bezeichnen die Quersumme einer positiven ganzen Zahl $n$ mit $Q(n)$. Was ist
	\begin{equation*}
		Q\parens*{Q\parens*{Q\parens*{2005^{2005}}}}\,?
	\end{equation*}
\end{aufgabe*}
\begin{aufgabe*}\label{aufgabe:531042}
	Für eine positive ganze Zahl sei $Q(n)$ die Quersumme von $n$ und $P(n)$ das Querprodukt von $n$, also das Produkt aller Ziffern von $n$. Ferner sei $R(n)\coloneqq n+P(n)Q(n)$. Für ein gegebenes $n$ definieren wir nun eine Folge $(a_i)_{i\geqslant 0}$ rekursiv durch $a_0\coloneqq n$, $a_{i+1}\coloneqq R(a_i)$. Zeige, dass die Folge $(a_i)_{i\geqslant 0}$ ab irgendeinem Punkt konstant sein muss.
\end{aufgabe*}

Bei solchen Aufgaben gibt es zwei Tricks, deren Kombination fast immer zum Erfolg führt.

\textbf{1. ~Abschätzungen.} Wenn $n$ eine $k$-stellige Zahl ist, dann gilt offensichtlich $n\geqslant 10^{k-1}$ und andererseits $Q(n)\leqslant 9k$. Weil lineare Funktionen viel langsamer wachsen als Exponentialfunktionen, ist $10^{k-1}$ für große $k$ viel größer als $9k$. Somit ist $n$ im Allgemeinen viel größer als~$Q(n)$.

\textbf{2.~Modulo~9.} Wenn $n=a_0+10a_1+10^2a_2+\dotsb+10^ka_k$ die Dezimaldarstellung einer positiven ganzen Zahl $n$ ist, dann gilt
\begin{equation*}
	n\equiv a_0+1\cdot a_1+1^2\cdot a_2+\dotsb+1^k\cdot a_k\equiv Q(n)\mod 9
\end{equation*}
(das ist eine Verschärfung der bekannten Teilbarkeitsregel für~$9$).

Ihr sollt nun versuchen, die beiden Aufgaben selbstständig zu lösen. Wenn ihr nicht weiterkommt, helfen euch die folgenden Tipps weiter. Am Ende dieses Heftes findet ihr außerdem die Lösungen.

\subsection*{Tipps zu den Beispielaufgaben}

\textbf{Tipp zu Aufgabe~\ref{aufgabe:441244}.} Schätze ab, wie groß $Q(Q(Q(2005^{2005})))$ höchstens sein kann. Wie kannst du herausfinden, welche der (sehr wenigen) verbleibenden Möglichkeiten die richtige ist?

\textbf{Tipp zu Aufgabe~\ref{aufgabe:531042}.} Wenn die Behauptung falsch wäre, dann müsste die Folge $(a_i)_{i\geqslant 0}$ streng monoton steigend sein. Um einen Widerspruch zu erzeugen, konstruiere ein $a_i$, welches an geeigneter Stelle eine Ziffer~$0$ enthält.