\documentclass[a4paper, 12pt]{article}

\usepackage{../landesseminarheader}
% ************************************************************************************
% Alternativ kann auch die bisherige Makro-Datei verwendet werden. Dann müssen die folgenden Zeilen entkommentiert werden.
%\usepackage{yhmath}
\usepackage{ngerman} % a4wide,, latexsym
\usepackage[T1]{fontenc}
\usepackage{lmodern}
\usepackage[utf8]{inputenc}
\usepackage{times}
\usepackage[slantedGreek]{mathptmx}
\usepackage{amsmath}
\usepackage{amssymb}
%\usepackage{amscd}
\usepackage{exscale}
\usepackage{enumerate}
\usepackage{amsthm}
\usepackage{graphics}
\usepackage{graphicx}	
\usepackage{longtable}
\usepackage{color}
\usepackage{dsfont} 
\usepackage{bbm}
%\usepackage{wasysym}
\usepackage{ifpdf}
%\usepackage{pst-all}
%\usepackage{pstricks,pstricks-add,pst-math,pst-xkey}
\usepackage{lscape}
\usepackage{eurosym, url, hyperref}
%\usepackage{fourier}


\frenchspacing

%--------------------------------------------------------------------

\setlength\parskip{\medskipamount}
\setlength\parindent{0pt}
\setlength{\voffset}{-3cm} %-1	%-1.5 %bis -3
%%\setlength{\topmargin}{0.625cm}		% oberer Rand bis Oberkante Kopfzeile
\setlength{\oddsidemargin}{0.0cm} \setlength{\evensidemargin}{0.0cm}%	Linker Rand 
%%\setlength{\headheight}{1.25cm}		% Höhe der Kopfzeile
%%\setlength{\headsep}{0.625cm}			% Abstand zw. Kopfzeile und 
\setlength{\topskip}{0cm}
%%\setlength{\footskip}{1cm}
\setlength{\textheight}{26cm} %24 %23.5; voffset ausblenden % bis 27
\setlength{\textwidth}{16cm} %16


\setcounter{secnumdepth}{3}								% Nummerierungstiefe
\setcounter{tocdepth}{1}									% Inhaltsverzeichnistiefe
%\numberwithin{equation}{section}					% Formeln abschnittsweise nummerieren

\flushbottom
\renewcommand{\baselinestretch}{1.0}



%Abkürzungen-------------------------------------------------------------
%Zahlenbereiche----------------------------------
\newcommand{\N}{{\mathbb{N}}}
\newcommand{\Z}{{\mathbb{Z}}}
\newcommand{\Q}{{\mathbb{Q}}}
\newcommand{\R}{{\mathbb{R}}}
\newcommand{\C}{{\mathbb{C}}}	
%(Komplexe) Zahlen
\newcommand{\I}{{\mathrm{i}}}					% Imaginäre Einheit
\newcommand{\real}{{\mathrm{Re}}}			% Realteil
\newcommand{\imag}{{\mathrm{Im}}}			% Imaginärteil
\newcommand{\dual}{{\mathrm{Du}}}			% Dualteil
%Abkürzung griechischer Buchstaben & Abbildungen etc.
\newcommand{\id}{\mathrm{id}}
\newcommand{\ve}{\varepsilon}
\newcommand{\vp}{\varphi}
\newcommand{\eul}{{\mathrm{e}}} 			% Eulersche Zahl e
\newcommand{\ld}{{\mathrm{ld}}} 			% duadischer Logarithmus
%Matrizen und Vektorrechnung---------------------
\newcommand{\T}{^{\mathrm{T}}}				% Transponiertzeichen
\newcommand{\rg}{{\mathrm{rg}}}				% Rang
\newcommand{\bild}{{\mathrm{bild}}}		% Bild
\newcommand{\Kern}{{\mathrm{kern}}}		% Kern
\newcommand{\lin}{{\mathrm{lin}}}			% Kern
\newcommand{\ul}[1]{\underline{#1}} 	% Vektorunterstrich
%Algebra----------------------
\newcommand{\ggT}{{\mathrm{ggT}}}				% ggT
\newcommand{\kgV}{{\mathrm{kgV}}}				% kgV
%Geometrie
\newcommand{\ol}[1]{\overline{#1}} 	  % Strecke
\newcommand{\TV}{{\mathrm{TV}}}				% Teilverhältnis
\newcommand{\DV}{{\mathrm{DV}}}				% Doppelverhältnis
\newcommand{\mbb}[1]{\mathbb{#1}}			% \mathbb	
%Einheiten
\newcommand{\mm}{{\mbox{\,} \mathrm{mm}}}
\newcommand{\cm}{{\mbox{\,} \mathrm{cm}}}				% Einheit cm
\newcommand{\dm}{{\mbox{\,} \mathrm{dm}}}	
\newcommand{\m}{{\mbox{\,} \mathrm{m}}}	
\newcommand{\LE}{{\mbox{\,} \mathrm{LE}}}	
\newcommand{\fe}{{\mbox{\,} \mathrm{FE}}}	
\newcommand{\km}{{\mbox{\,} \mathrm{km}}}
\newcommand{\gpkcm}{{\mbox{\,} \mathrm{g}/\mathrm{cm}^3}}
\newcommand{\kmh}{{\mbox{\,} \frac{\mathrm{km}}{\mathrm{h}}}}
\newcommand{\komma}{{,}}
\newcommand{\entspricht}{\mathrel{\widehat{=}}}   		% Entspricht

\newcommand{\h}{{\mbox{\,} \mathrm{h}}}
\newcommand{\s}{{\mbox{\,} \mathrm{s}}}
\newcommand{\g}{{\mbox{\,} \mathrm{g}}}
\newcommand{\kg}{{\mbox{\,} \mathrm{kg}}}	
\newcommand{\gc}{{\grad\mathrm{C}}}	
\newcommand{\grad}{^{\circ}}				% Einheit °
\newcommand{\mbm}[1]{\mathbbm{#1}}	
\newcommand{\arc}{{\mathrm{arc}}}	
%Schüler
\newcommand{\ds}{\displaystyle}
% shortcuts
\def\ol#1{\overline{#1}}
\def\ul#1{\underline{#1}}
\def\br#1{\left(#1\right)}             % brackets
\def\sbr#1{\left[#1\right]}            % square brackets
\def\cbr#1{\left\{#1\right\}}          % curly brackets
\def\iff{\Leftrightarrow\ }
\def\yields{\Rightarrow\ }
\newcommand{\rf}[1][]{\textup{\eqref{#1}}}
\newcommand{\half}{\frac{1}{2}}
\newcommand{\third}{\frac{1}{3}}
\newcommand{\ov}{\overline}
\newcommand{\nn}{\nonumber}
\newcommand{\RRA}{{\,\,\Longrightarrow}\,\,}
\newcommand{\LRA}{{\,\,\Leftrightarrow}\,\,}
%\newcommand{\qed}[1][\rule{1ex}{1ex}]{\nopagebreak\hspace*{2em}\hspace*{\fill}{$#1$}}


%Analysis --------------------
\newcommand{\inte}[4]{\int\limits_{#1}^{#2} {#3}\mathrm{d}{#4} } 

%---------------------------------------------------------------------------------------------
\newenvironment{description*}[2]
   {\begin{list}{}{%
      \settowidth{\labelwidth}{#2{#1}}
      \setlength{\leftmargin}{\labelwidth}
         \addtolength{\leftmargin}{\labelsep}
      \setlength{\parsep}{0.5ex plus0.2ex minus0.2ex}
      \setlength{\itemsep}{0.3ex}
      \renewcommand{\makelabel}[1]{#2{##1}\hfill}}}
   {\end{list}}
%*********************************************************************************************

\newcommand{\bsp}[1]{\begin{Bsp}\hspace*{1cm}\newline#1\end{Bsp}}
\newcommand{\kartauf}[3]{
\newpage
\normalsize
\begin{tabular}{p{13cm}}
\textbf{#1 \hfill #2}\\\hline
\end{tabular}
\begin{enumerate}\small
#3
\end{enumerate}}

\newcommand{\bem}[1]{\begin{Bem}\normalfont \hspace*{1cm}\newline#1\end{Bem}}

\newcommand{\defi}[2]{\begin{Def}[#1]\hspace*{1cm}\newline#2\end{Def}}

\newcommand{\theo}[2]{\begin{Theo}[#1]\hspace*{1cm} #2\end{Theo}} %\newline

\newcommand{\satz}[2]{\begin{Satz}[#1]\hspace*{1cm}\newline#2\end{Satz}}

\newcommand{\folg}[2]{\begin{Folg}[#1]\hspace*{1cm}\newline#2\end{Folg}}

\newcommand{\auf}[1]{\begin{Auf} \normalfont#1\end{Auf}} % \hspace*{1cm}\newline

\newcommand{\lem}[1]{\begin{Lem}\hspace*{1cm}\newline#1\end{Lem}}

\newcommand{\cor}[1]{\begin{Cor}\hspace*{1cm}\newline#1\end{Cor}}

\newcommand{\tipp}[2]{\begin{Tipp}[#1]\hspace*{1cm}\newline#2\end{Tipp}}

\newcommand{\bew}[1]{\textsl{Beweis: }#1 \hfill $\Box$}  %ge{\"a}ndert (\!)

\newcommand{\loes}[1]{\textit{Lösung: }#1 \hfill $\Box$}  %ge{\"a}ndert (\!)

\newcommand{\lema}[2]{\begin{Lem}[#1]\hspace*{1cm}\newline#2\end{Lem}}

\newtheoremstyle{definition}% name
     {3pt}%      Space above
     {5pt}%      Space below
     {\itshape}%         Body font % evtl 
     {0ex}%         Indent amount (empty = no indent, \parindent = para indent)
     {\bfseries}% Thm head font
     {:}%        Punctuation after thm head
     {.5em}%     Space after thm head: " " = normal interword space;
           %       \newline = linebreak
     {}%         Thm head spec (can be left empty, meaning `normal')

\newtheoremstyle{Beweis}% name
     {30pt}%      Space above
     {3pt}%      Space below
     {}%         Body font
     {}%         Indent amount (empty = no indent, \parindent = para indent)
     {\itshape}% Thm head font
     {:}%        Punctuation after thm head
     {.5em}%     Space after thm head: " " = normal interword space;
           %       \newline = linebreak
     {}%         Thm head spec (can be left empty, meaning `normal')


\newtheoremstyle{break}% name
     {3pt}%      Space above
     {7pt}%      Space below
     {\itshape}%         Body font
     {}%         Indent amount (empty = no indent, \parindent = para indent)
     {\bfseries}% Thm head font
     {:}%        Punctuation after thm head
     {.5em}%     Space after thm head: " " = normal interword space;
           %       \newline = linebreak
     {}%         Thm head spec (can be left empty, meaning `normal')

\theoremstyle{definition}

\newtheorem{Def}{Definition}[section]


\theoremstyle{break}
\newtheorem{Satz}[Def]{Satz}
\newtheorem{Lem}[Def]{Lemma}
\newtheorem{Cor}[Def]{Korollar}
\newtheorem{Tipp}[Def]{Tipp}

\theoremstyle{definition}
\newtheorem{Theo}[Def]{Theorem}
\newtheorem{Auf}[Def]{Aufgabe}
\newtheorem{Bem}[Def]{Bemerkung}
\newtheorem{Bsp}[Def]{Beispiel}
\newtheorem{beispiel}[Def]{Beispiel}
\newtheorem{aufgabe}[Def]{Aufgabe}
\newtheorem{Folg}[Def]{Folgerung}
\newtheorem{obs}[Def]{Beobachtung}

\theoremstyle{Beweis}
\newtheorem{Bew}{Beweis}



\makeindex

\endinput

%\usepackage{mathtools} % für DeclarePairedDelimiter
%\usepackage{wasysym} % für das Winkel-Symbol
%\usepackage{tikz} % für die Skizzen
%\usetikzlibrary{positioning,calc,arrows.meta,shapes,decorations.pathmorphing,decorations.markings,hobby}
%\tikzset{every picture/.style={line width=0.6, line cap=round,line join=round}}
%\usepackage{csquotes} % für bessere Anführungszeichen
%\usepackage[shortlabels]{enumitem} % um Aufzählungen automatisch in der Form (a), (b), ... zu setzen
%\setlist[enumerate]{label={$(\alph*)$}, ref={$(\alph*)$},topsep=0pt}
%\setlist[itemize]{topsep=0pt,itemsep=0pt,parsep=\parskip}
%\usepackage{booktabs} % schönere Tabellen
%\usepackage{tabularx} % Ich benutze tabularx, um mehrere Bilder so nebeneinander anzuordnen, dass die Abstände alle gleich groß sind
%\usepackage{wrapfig} % für Bilder, die von Text umflossen werden sollen
%\usepackage{microtype} % typographische Mikrooptimierungen (verhindert overfull hboxes)
%\usepackage{xurl} % damit URLs automatisch umgebrochen werden (verhindert over/underfull hboxes).
% ************************************************************************************

% ************************* Kosmetik fürs Inhaltsverzeichnis *************************
% Ich finde es optisch ansprechend und inhaltlich sinnvoll, wenn die Beiträge im Inhaltsverzeichnis nach Theme sortiert auftauchen. Dazu wird das Erscheinungsbild des Inhaltsverzeichnisses ein wenig umgebaut. Wird das nicht gewünscht, dann können die folgenden Zeilen sowie alle \cftaddtitleline-Befehle im Dokument entfernt werden.
\usepackage{tocloft}

% passe Abstände und Erscheinungsbild für Sections an
\cftsetindents{section}{1.5em}{2.3em} % Einrückung
\setlength{\cftbeforesecskip}{0.25em} % Zeilenabstand
\renewcommand{\cftsecleader}{\cftdotfill{\cftdotsep}} % Punkte zwischen Section-Titel und Seitenzahl
\renewcommand{\cftsecfont}{} % Section-Titel nicht fett
\renewcommand{\cftsecpagefont}{} % Seitenzahl nicht fett

% passe Abstände und Erscheinungsbild für Parts an
\setlength{\cftbeforepartskip}{1.0em} % Zeilenabstand
\renewcommand{\cftpartfont}{\bfseries} % Part-Titel nur fett, aber nicht größer
\renewcommand{\cftpartpagefont}{\bfseries} % Seitenzahl nur fett, aber nicht größer
% ************************************************************************************


% **************************** Eine praktische Umgebungen ****************************
% Die amsthm-Umgebungen erlauben es nicht, Bilder in Form von Floats einzufügen. Deswegen wird hier die proof-Umgebungen neu definiert. Außerdem definieren wir neue Umgebungen für Aufgaben, Definitionen und Sätze mit Namen.

% Die proof-Umgebung wird neu definiert
\RenewDocumentEnvironment{proof}{ O{\proofname} }{
	\par\pushQED{\qed}
	\noindent\textbf{#1.}\ \ignorespaces
}{%
	\popQED\par
}

% Aufgaben-Umgebung. Das optionale Argument wird meistens benutzt, um schwere Aufgaben durch Asteriske zu kennzeichnen.
\newcounter{caufgabe}[section]
\NewDocumentEnvironment{aufgabe*}{ O{} }{
	\par\refstepcounter{caufgabe}
	\noindent\textbf{Aufgabe~\thecaufgabe#1.}\ \ignorespaces
}{
	\par
}

% Eine Umgebung für benannte Sätze. Der Name kommt in das optionale Argument. Zum Beispiel liefert "\begin{satzmitnamen}[Satz von Euler-Fermat] ..." im Text "Satz von Euler-Fermat. ..."
\NewDocumentEnvironment{satzmitnamen}{ O{Satz} }{
	\par\begingroup
	\noindent\textbf{#1.}\ \ignorespaces\itshape
}{
	\endgroup\par
}

% Eine Umgebung für Definitionen
\NewDocumentEnvironment{definition}{ O{Definition}}{
	\par
	\noindent\textbf{#1.}\ \ignorespaces
}{
	\par
}
% ************************************************************************************

% ****************************** Eine praktische Makros ******************************
% bessere Klammern in kursiven Umgebungen
\newcommand{\embrace}[1]{\textup{(}#1\textup{)}}

% Das \varangle-Symbol wird in \itshape-Umgebungen kursiv dargestellt; dieses Kommando behebt das Problem.
\newcommand{\winkel}{\textup{\varangle}}

% Ein Verkehrszeichen
\DeclareRobustCommand{\Warnung}{\smash{\tikz[baseline, anchor=center]\node[draw, regular polygon, regular polygon sides=3, rounded corners=2, thick, inner sep=-0.25pt] at (0,0) {\textbf{!}};}}

% Gepaarte Klammern
\DeclarePairedDelimiter{\parens}{\lparen}{\rparen}
\DeclarePairedDelimiter{\braces}{\lbrace}{\rbrace}
\DeclarePairedDelimiter{\brackets}{[}{]}
\DeclarePairedDelimiter{\abs}{\lvert}{\rvert}
% ************************************************************************************

% Wenn $\boldsymbol{...}$ in einem Titel verwendet wird, kommt es manchmal zu einer Warnung (aber alles sieht ok aus). Dieser Befehl unterdrückt die Warnung.
\SetSymbolFont{wasy}{bold}{U}{wasy}{m}{n}

\begin{document}
	\begin{titlepage}
		\centering\sffamily\Huge\bfseries
		\vspace*{0.75em}
		
		Sächsisches Landesseminar \\
		Mathematik 2024 \\
		\textmd{in Sayda} \\
		
		\vspace{1.75em}
		
		\textmd{\large 11.\,03. -- 15.\,03.\,2024} \\
		
		\vspace{6em}
		
		Begleitmaterial zum Seminarprogramm\\[.6\baselineskip]
		der Klassenstufe 9\\
		
		\vfill
		
		\raggedright\normalfont\normalsize
		Herausgegeben im Auftrag des Sächsischen Landeskomitees zur Förderung \\
		mathematisch-naturwissenschaftlich begabter und interessierter Schüler
	\end{titlepage}
	\setcounter{page}{2}
	
	\section*{Vorwort}
	
	Liebe Schülerinnen,\\
	liebe Schüler,
	
	ich freue mich, Euch im Sächsischen Landesseminar Mathematik begrüßen zu können.
	
	In den nächsten drei Tagen werdet Ihr zehn mathematische Seminare haben. Ich hoffe, dass Ihr während dieser Seminare nicht nur neue Lösungsmethoden oder mathematische Sachverhalte kennen lernt, sondern dass Ihr dabei auch Freude an der Mathematik und am Lösen von Aufgaben haben werdet.
	
	Am Donnerstag wird dann die Auswahlklausur geschrieben, für die ich Euch jetzt schon alles Gute und viel Erfolg wünsche. Während Ihr am vorbereiteten Freizeitprogramm teilnehmt, werden die Klausuren korrigiert. Am Freitag wird dann die Mannschaft, die Sachsen auf der Bundesrunde der Mathematik-Olympiade vertreten darf, feierlich bekannt gegeben.
	
	Ich hoffe, dass Ihr es in den nächsten Tagen neben der Mathematik auch genießen könnt, Euch mit Gleichgesinnten zu unterhalten, Euch auszutauschen und um Lösungsansätze gemeinsam zu ringen. Dazu soll insbesondere auch der MatBoj beitragen. Aber natürlich denke ich dabei auch an die vielen Spiele, die eine lange Tradition im Landesseminar haben.
	
	Ich wünsche Euch viel Erfolg und eine gute Woche.
	
	Joachim Lippert
	
	\section*{Über dieses Heft}
	Dieses Heft behandelt einige der wichtigsten Themen für die Mathematik-Olympiade in der Klassenstufe~9. Einige dieser Themen werden auch in den Seminaren besprochen, einige werdet ihr bestimmt schon kennen und andere werden euch neu sein. Es wird empfohlen, dass ihr das Heft zur Vorbereitung auf die Bundesrunde oder die nächste Olympiade durcharbeitet.
	
	In diesem Heft gibt es zwei Typen von Aufgaben: \emph{Beispielaufgaben} und \emph{Übungsaufgaben}. An Beispielaufgaben lassen sich die vorgestellten Methoden besonders gut vorführen. Manchmal lösen wir Beispielaufgaben direkt auf, aber meistens findet ihr am Ende des jeweiligen Kapitels Tipps und erst ganz am Ende des Heftes die Lösungen für die Beispielaufgaben. Die Beispielaufgaben solltet ihr zuerst bearbeiten, wenn ihr euch mit einer neuen Methode vertraut machen wollt. Bei den Übungsaufgaben hingegen seid ihr auf euch allein gestellt. Sie dienen zur weiteren Vertiefung der Inhalte.
	
	Schwere Aufgaben sind mit einem (*) bis drei (***) Sternen gekennzeichnet. Ein Stern bedeutet dabei, dass die Aufgabe schwerer als die durchschnittliche Bundesrunden-Aufgabe ist. Besonders bei solchen Aufgaben gilt: Wenn ihr nicht weiterkommt, dann holt euch einen Tipp und wenn ihr dann immer noch feststeckt, dann lest euch auch gern die Lösung durch -- dafür sind die Tipps und die Lösungen schließlich da.
	
	
	\vfill
	
	\scriptsize
	
	\emph{Texte:} Ferdinand Wagner. Mit tatkräftiger Unterstützung von Sebastian Bürger, Leo Gitin, Cara Hobohm, Tien Nguyen und Arne Wolf. Aufbauend auf früheren Texten von Ingolf Busch, Frank Göhring, Maximilian Keitel, Eric Müller, Jens Reinhold und Lisa Sauermann. Herzlichen Dank auch an alle weiteren, die in früheren Ausgaben dieses Begleitheftes Beiträge erstellt haben.
	
	\emph{Textsatz:} Joachim Lippert, Tien Nguyen Thanh, Ferdinand Wagner.
	
	\emph{Redaktion}: Joachim Lippert (\href{mailto:lippert@landesseminar-sachsen.de}{\texttt{lippert@landesseminar-sachsen.de}}).
	\normalsize
	
	\newpage
	\tableofcontents
	
	\newpage
	\phantomsection\cftaddtitleline{toc}{part}{Gleichungen und Ungleichungen}{\thepage}
	\section{Nichtlineare Gleichungssysteme}\label{kapitel:Gleichungssysteme}

Ein beliebtes Aufgaben-Thema in allen Klassenstufen sind nichtlineare Gleichungssysteme. Im Gegensatz zu linearen Gleichungssystemen, die ihr aus der Schule kennt, gibt es hier keine Methode, die sicher zum Ziel führt. Stattdessen ist bei jeder Aufgabe aufs Neue eure Kreativität gefragt. Auf den richtigen Trick zu kommen wird euch um so leichter fallen, je mehr Tricks ihr schon kennt. Deshalb werden in diesem Kapitel einige Tricks präsentieren, mit denen sich nichtlineare Gleichungssysteme attackieren lassen. Danach werdet ihr dazu eingeladen, diese Tricks an einigen Beispielaufgaben auszuprobieren.

Wir werden im Folgenden davon ausgehen, dass die Variablen unseres Gleichungssystems \emph{reell} sind. Wenn stattdessen nach \emph{ganzzahligen} Lösungen gefragt ist, dann könnt ihr die Methoden aus Kapitel~\ref{kapitel:Diophantastisch}: \emph{Diophantische Gleichungen} verwenden (ganz besonders solltet ihr in diesem Fall nach Faktorisierungen Ausschau halten).

Zu nichtlinearen Gleichungssystemen gibt es auch ein exzellentes Skript von Eric Müller.\footnote{Online verfügbar unter \url{http://www.mathematikinformation.info/pdf/MI38_43-52_Mueller.pdf}.}

\subsection*{Strategien, Tipps und Tricks für nichtlineare Gleichungssysteme}

\textbf{1.~Forme geeignet um.} Ein guter erster Schritt ist zunächst mit den gegebenen Gleichungen herumzuspielen. Wenn ihr einige der Gleichungen so addieren oder subtrahieren könnt, dass sich möglichst viele Terme wegheben, dann seid ihr  auf gutem Wege. Haltet auch nach Faktorisierungen Ausschau. Fast immer sind die Gleichungen in Olympiade-Aufgaben von polynomieller Natur. Dann kommt es bei geschickter Addition oder Subtraktion zweier Gleichungen schnell einmal vor, dass sich zum Beispiel ein Faktor $(x-y)$ auf beiden Seiten ausklammern lässt. Wenn euch so eine Umformung gelingt, seid ihr ziemlich sicher auf dem richtigen Weg. Mehr zu solchen Faktorisierungen findet ihr in Kapitel~\ref{kapitel:Diophantastisch}: \emph{Diophantische Gleichungen}.

Es wird auch gerne versucht, die Gleichungen nacheinander nach einer Variablen umzuformen und ineinander einzusetzen. Gelegentlich führt das zum Ziel, häufig aber eher zu langwierigen Rechnungen, die irgendwann an Rechenfehlern. Allgemein gilt: \emph{Macht euch einen Plan!} Wenn es so aussieht, als könntet ihr durch sukzessives Einsetzen zum Ziel kommen, dann probiert das auf jeden Fall aus. Ansonsten schaut lieber, ob ihr einen besseren Trick seht.

\textbf{2.~Substituiere geeignet.} Substitutionen sind geeignet, wenn sie das Gleichungssystem vereinfachen, gleichzeitig aber auch nichts verschenken: Die Lösungen des ursprünglichen Gleichungssystems sollten sich immer noch aus den Lösungen des substituierten Gleichungssystems rekonstruieren lassen.

Hier sind zwei spezielle Substitutionen, die hin und wieder zum Ziel führen:
\begin{itemize}
	\item Wenn in eurem Gleichungssystem zwei Variablen~$x$ und~$y$ vorkommen, dann könnt ihr oftmals $s=x+y$ und $t=xy$ substituieren. Nach dem Satz von Vieta könnt ihr~$x$ und~$y$ als Lösungen der quadratischen Gleichung $X^2-sX+t=0$ rekonstruieren. Diese Substitution ist auch für Ungleichungen nützlich, wie ihr in Kapitel~\ref{kapitel:Schiebemethode}: \emph{Ungleichungen und die Schiebemethode} sehen werdet.
	
	Das Ganze funktioniert natürlich auch mit mehr als zwei Variablen. Wenn ihr zum Beispiel~$x$,~$y$ und~$z$ gegeben habt, könnt ihr $u=x+y+z$, $v=xy+yz+zx$ und $w=xyz$ substituieren. Auch hier lassen sich~$x$,~$y$ und~$z$ rekonstruieren, nämlich als Lösungen der kubischen Gleichung $(X-x)(X-y)(X-z)=X^3-uX^2+vX-w=0$.
	
	\item In manchen Aufgaben kommen \emph{trigonometrische Substitutionen} zum Einsatz. Hier schreibt ihr die Variablen als Sinus, Kosinus oder Tangens von geeigneten Winkeln (bevor ihr $x=\sin\alpha$ substituieren könnt, müsst ihr natürlich beweisen, dass $x$ im Intervall $[-1,1]$ liegt). Dadurch ergeben sich manchmal ganz phantastische Vereinfachungen und besonders elegante Lösungen.
	
	Trigonometrische Substitutionen werden meistens dann verwendet, wenn Terme auftauchen, die wie Additionstheoreme aussehen. Zum Beispiel erinnert uns der Term $\frac{x+y}{1-xy}$ an das Additionstheorem 
	\begin{equation*}
		\tan(\alpha+\beta)=\frac{\tan\alpha+\tan\beta}{1-\tan\alpha\tan\beta}
	\end{equation*}
	und der Term $2x^2-1$ könnte ein Hinweis auf die Formel $\cos(2\alpha)=2\cos^2\alpha-1$ sein. Weil trigonometrische Substitutionen in der Klassenstufe~9/10 sehr selten sind, werden wir hier nicht weiter darauf eingehen. Im Heft für die Klasse~12 gibt es ein komplettes Kapitel dazu.
\end{itemize}

\textbf{3.~Benutze das Extremalprinzip.} Das Extremalprinzip für Gleichungssysteme ist vielleicht der wichtigste Trick in diesem Kapitel! Um damit Gleichungssysteme zu lösen, schaut ihr euch zuerst das Maximum oder das Minimum aller vorkommenden Variablen an. Sagen wir, $x_1$ ist maximal. Erstaunlich häufig sind die Gleichungen so beschaffen, dass ihr zeigen könnt, dass dann noch eine weitere Variable maximal oder minimal sein muss. Sagen wir, $x_2$ ist ebenfalls maximal. Wenn das Gleichungssystem \emph{zyklisch}\footnote{\emph{Zyklisch} bedeutet, dass das Gleichungssystem bei einem geeigneten Ringtausch der Variablen in sich selbst überführt wird.} ist, dann können wir das gleiche Argument mit $x_2$ wiederholen und erhalten noch eine maximale Variable. Auf diese Weise könnt ihr euch von Variable zu Variable hangeln und findet heraus, dass alle Variablen maximal, also alle Variablen gleich sein müssen. Damit ist die Aufgabe meistens gegessen.

Wenn ihr mit dem Extremalprinzip argumentiert, dann müsst ihr besonders auf der Hut sein, dass ihr nicht \enquote{ohne Beschränkung der Allgemeinheit} Annahmen trefft, die in Wirklichkeit die Allgemeinheit beschränken. Wenn ihr zum Beispiel ein zyklisches Gleichungssystem in den Variablen $x$, $y$ und $z$ vorliegen habt, dann dürf ihr \emph{nicht} ohne Beschränkung der Allgemeinheit $x\geqslant y\geqslant z$ annehmen. Das ist nur möglich, wenn das Gleichungssystem \emph{symmetrisch}\footnote{\emph{Symmetrisch} bedeutet, dass das Gleichungssystem bei jedem beliebigen Tausch der Variablen in sich selbst überführt wird} ist. Was ihr im zyklischen Fall jedoch tun dürft, ist $x=\max\{x,y,z\}$ anzunehmen. Und ihr dürft natürlich ebenfalls die beiden Fälle $x\geqslant y\geqslant z$ und $x\geqslant z\geqslant y$ unterscheiden. Ihr müsst sie nur beide betrachten.


\textbf{4.~Interpretiere das Gleichungssystem als Gleichheitsfall einer Ungleichung.} Manchmal lässt sich ein Gleichungssystem als Ungleichung mit Nebenbedingung interpretieren. Wenn ihr die Ungleichung beweisen und die Gleichheitsfälle identifizieren könnt, habt ihr das ursprüngliche Gleichungssystem gelöst. Zum Beweis einer solchen Ungleichung könnt ihr etwa AM-GM benutzen (siehe Kapitel~\ref{kapitel:AM-GM}: \emph{Mittelungleichungen}) oder die entsprechenden Terme als Summe von Quadraten schreiben.

Nach diesem Trick solltet ihr vor allem dann Ausschau halten, wenn euer Gleichungssystem \emph{unterbestimmt} ist, das heißt, wenn es weniger Gleichungen als Variablen gibt. Denn in diesem Fall muss es einen Grund geben, warum das Gleichungssystem nicht einfach unendlich viele Lösungen hat. Und dieser Grund ist meistens, dass eine der Gleichungen, bis auf wenige Gleichheitsfälle, in Wirklichkeit als Ungleichung gilt.

Natürlich kann es auch vorkommen, dass euer Gleichungssystem wirklich unendlich viele Lösungen hat und ihr diese klassifizieren müsst. In diesem Fall könnt ihr meistens eine oder mehrere der Variablen beliebig wählen und die restlichen Variablen in Abhängigkeit der frei gewählten Variablen darstellen.

Noch ein letzer Hinweis: \emph{Macht eine Probe!!!} Eine fehlende Probe ist einer der häufigsten Gründe für ärgerliche und vermeidbare Punktabzüge.

\subsection*{Beispielaufgaben}

Ihr sollt nun einige der folgenden Aufgaben selbstständig lösen. Am Ende des Kapitels findet ihr Tipps zu den Aufgaben und am Ende des Heftes könnt ihr die Lösungen nachlesen.

\begin{aufgabe*}\label{aufgabe:520943}
	Bestimme alle reellen Lösungen $(x,y,z)$ des Gleichungssystems
	\begin{equation*}
		x-\frac 1y=y-\frac 1z=z-\frac 1x\,.
	\end{equation*}
\end{aufgabe*}
\begin{aufgabe*}\label{aufgabe:521043}
	Bestimme alle reellen Lösungen $(x,y,z)$ des Gleichungssystems
	\begin{equation*}
		x+\frac 1y=y+\frac 1z=z+\frac 1x\,.
	\end{equation*}
\end{aufgabe*}
\begin{aufgabe*}\label{aufgabe:380943}
	Finde alle Tripel $(x,y,z)$ von positiven reellen Zahlen, die die folgende Gleichung erfüllen:
	\begin{equation*}
		x+3y^3+5z^5+\frac1x+\frac{3}{y^3}+\frac{5}{z^5}=18\,.
	\end{equation*}
\end{aufgabe*}
\begin{aufgabe*}\label{aufgabe:451046}
	Finde alle reellen Lösungen $(x,y,z)$ des Gleichungssystems
	\begin{equation*}
		\left\{\begin{aligned}
			x+y+\frac 1z &= 3\,,\\
			y+z+\frac 1x &= 3\,,\\
			z+x+\frac 1y &= 3\,.
		\end{aligned}\right.
	\end{equation*}
\end{aufgabe*}
\begin{aufgabe*}\label{aufgabe:461041}
	Finde alle reellen Lösungen $(x,y,z)$ des Gleichungssystems
	\begin{equation*}
		\left\{\begin{aligned}
			x+y+z &= 1\,,\\
			\frac 1x+\frac 1y+\frac 1z &= 1\,.
		\end{aligned}\right.
	\end{equation*}
\end{aufgabe*}
\begin{aufgabe*}\label{aufgabe:541241}
	Finde alle reellen Lösungen $(x,y)$ des Gleichungssystems
	\begin{equation*}
		\left\{\begin{alignedat}{2}
			x^3&+9x^2y &&= 10\,,\\
			y^3&+\phantom{9}xy^2 &&= 2\,.
		\end{alignedat}\right.
	\end{equation*}
\end{aufgabe*}
\begin{aufgabe*}\label{aufgabe:Sayda2013}
	Finde alle reellen Lösungen $(u,v,w,x,y)$ des Gleichungssystems
	\begin{equation*}
		\left\{\begin{alignedat}{4}
			v^2&+w^2&&+x^2&&+y^2 &&= 6-2u\,,\\
			w^2&+x^2&&+y^2&&+u^2 &&= 6-2v\,,\\
			x^2&+y^2&&+u^2&&+v^2 &&= 6-2w\,,\\
			y^2&+u^2&&+v^2&&+w^2 &&= 6-2x\,,\\
			u^2&+v^2&&+w^2&&+x^2 &&= 6-2y\,.
		\end{alignedat}\right.
	\end{equation*}
\end{aufgabe*}
\begin{aufgabe*}[*]\label{aufgabe:IMOSL1993VNM}
	Sei $a>1$ eine gegebene reelle Zahl. Finde, in Abhängigkeit von $a$, alle reellen Lösungen des Gleichungssystems
	\begin{equation*}
		\left\{\begin{alignedat}{2}
			x_1^2 &= ax_2&&+1\,,\\
			x_2^2 &= ax_3&&+1\,,\\
			&\mathrel{\tikz[inner sep=0,outer sep=0]{\node at (0,-0.5ex) {$\phantom{=}$};\node at (0,0) {$\vdots$};}}\\
			x_{41}^2 &= ax_{42}&&+1\,,\\
			x_{42}^2 &= ax_1&&+1\,.
		\end{alignedat}\right.
	\end{equation*}
\end{aufgabe*}
\vfill\hrule\vspace{-1em}

\subsection*{Tipps zu den Beispielaufgaben}
\textbf{Tipps zu Aufgabe~\ref{aufgabe:520943}.} Zeige zuerst, dass alle Variablen das gleiche Vorzeichen haben müssen.

Nimm an, dass $x$ das Maximum der Variablen $x$, $y$ und $z$ ist. Was kannst du dann über $y$ aussagen?

\textbf{Tipps zu Aufgabe~\ref{aufgabe:521043}.} Bezeichne den gemeinsamen Wert von $x+\frac 1y$, $y+\frac 1z$ und $z+\frac 1x$ mit $a$. Forme geschickt um.

Für bestimmte Werte von $a$ hat das Gleichungssystem nicht nur die triviale Lösung $x=y=z$. Kannst du herausfinden, welche Werte von~$a$ das sind?

\textbf{Tipp zu Aufgabe~\ref{aufgabe:380943}.} Zeige, dass die gewünschte Gleichung in Wirklichkeit als Ungleichung gilt und bestimme alle Gleichheitsfälle.

\textbf{Tipp zu Aufgabe~\ref{aufgabe:451046}.} Subtrahiere jeweils zwei Gleichungen und faktorisiere.

\textbf{Tipp zu Aufgabe~\ref{aufgabe:461041}.} Betrachte das Polynom $P(X)\coloneqq (X-x)(X-y)(X-z)$. Was kannst du über die Koeffizienten von $P(X)$ aussagen?

\textbf{Tipp zu Aufgabe~\ref{aufgabe:541241}.} Durch eine geschickte Umformumg lässt sich bei dieser Aufgabe eine sehr einfache Faktorisierung finden.

\textbf{Tipp zu Aufgabe~\ref{aufgabe:Sayda2013}.} Subtrahiere jeweils zwei Gleichungen und faktorisiere.

\textbf{Tipp zu Aufgabe~\ref{aufgabe:IMOSL1993VNM}.} Nimm an, dass der Betrag $\abs{x_1}$ maximal unter allen Variablen ist. Was kannst du über $x_2$ und $x_{42}$ aussagen, je nachdem, ob $x_1$ positiv oder negativ ist?\newpage
	\section{Mittelungleichungen}\label{kapitel:AM-GM}
Ungleichungen sind ein beliebtes Thema in Olympiadeaufgaben und gleichzeitig eines der Themen mit der meisten Theorie (vermutlich nur übertroffen von Geometrie). Erstaunlich oft lässt sich diese Theorie auch tatsächlich anwenden. Je mehr Theorie ihr also beherrscht, desto höher sind eure Chancen bei schweren Aufgaben und desto größer ist euer Vorteil gegenüber der Konkurrenz.

In diesen Heften werden wir euch Stück für Stück an Ungleichungstheorie heranführen. In diesem Kapitel beginnen wir mit den Grundlagen, die ihr auf jeden Fall beherrschen solltet.

\subsection*{Die Ungleichung vom Arithmetischen und Geometrischen Mittel}
Die Ungleichung vom Arithmetischen und Geometrischen Mittel, kurz \emph{AM-GM}, ist die wichtigste Ungleichung, die ihr kennen solltet. Selbst auf IMO-Niveau lassen sich viele Ungleichungen mit AM-GM lösen, wenn ihr euch nur geschickt genug anstellt.
\begin{satzmitnamen}[Ungleichung vom Arithmetischen und Geometrischen Mittel (AM-GM)]
	Gegeben seien nichtnegative reelle Zahlen $a_1,a_2,\dotsc,a_n\geqslant 0$. Dann gilt stets
	\begin{equation*}
		\frac{a_1+a_2+\dotsb+a_n}{n}\geqslant \sqrt[n]{a_1a_2\dotsm a_n}\,.
	\end{equation*}
	Gleichheit gilt genau dann, wenn alle Variablen gleich sind, also genau für $a_1=a_2=\dotsb=a_n$.
\end{satzmitnamen}
\begin{proof}
	Betrachten wir zuerst den Fall $n=2$. In diesem Fall müssen wir $\frac12(a_1+a_2)\geqslant \sqrt{a_1a_2}$ beweisen. Nach der zweiten binomischen Formel gilt aber
	\begin{equation*}
		\frac{a_1+a_2}2-\sqrt{a_1a_2}=\frac{\sqrt{a_1}^2+\sqrt{a_2}^2-2\sqrt{a_1} \sqrt{a_2}}2=\frac{\parens*{\sqrt{a_1}-\sqrt{a_2}}^2}{2}
	\end{equation*}
	Da Quadrate stets nichtnegativ sind, ist die rechte Seite $\geqslant 0$, was die gewünschte Ungleichung ist. Gleichheit gilt genau für $\sqrt{a_1}=\sqrt{a_2}$, was zu $a_1=a_2$ äquivalent ist.
	
	Den allgemeinen Fall beweisen wir per Induktion nach $n$. Den Induktionsanfang $n=2$ haben wir soeben erledigt. Im Induktionsschritt ergibt sich eine Besonderheit: Statt wie üblich von $n$ auf $n+1$ zu induzieren, was in diesem Fall sehr kompliziert wäre, induzieren wir einmal von $n$ auf $2n$ und einmal von $n$ auf $n-1$. Diese Variante der Induktion wird \emph{Vorwärts-Rückwärts-Induktion} genannt und wurde zum ersten Mal von dem französischen Mathematiker Augustin-Louis Cauchy (1789--1857) verwendet.
	
	\emph{Induktionsschritt, $n\rightarrow 2n$.} Gegeben seien nichtnegative reelle $a_1,a_2,\dotsc,a_{2n}\geqslant 0$. Dann gilt
	\begin{align*}
		\frac{a_1+a_2+\dotsb+a_{2n}}{2n}&=\frac12\parens*{\frac{a_1+a_2+\dotsb+a_n}{n}+\frac{a_{n+1}+a_{n+2}+\dotsb+a_{2n}}{n}}\\
		&\geqslant \frac12\parens[\big]{\sqrt[n]{a_1a_2\dotsm a_n}+\sqrt[n]{a_{n+1}a_{n+2}\dotsm a_{2n}}}\\
		&\geqslant \sqrt{\sqrt[n]{a_1a_2\dotsm a_n}\cdot \sqrt[n]{a_{n+1}a_{n+2}\dotsm a_{2n}}}\\
		&=\sqrt[2n]{a_1a_2\dotsm a_{2n}}\,.
	\end{align*}
	In der ersten Abschätzung haben wir die Induktionsannahme benutzt, also die AM-GM-Un-gleichung für $a_1,a_2,\dotsc,a_n$ und für $a_{n+1},a_{n+2},\dotsc,a_{2n}$. In der zweiten Abschätzung haben wir den Induktionsanfang benutzt. Gleichheit gilt genau dann, wenn in beiden Abschätzungen Gleichheit eintritt. Nach der Induktionsannahme gilt in der ersten Abschätzung genau dann Gleichheit, wenn $a_1=a_2=\dotsb=a_n$ sowie $a_{n+1}=a_{n+2}=\dotsb=a_{2n}$ gilt. In der zweiten Abschätzung gilt Gleichheit genau dann, wenn $\sqrt[n]{a_1a_2\dotsm a_n}=\sqrt[n]{a_{n+1}a_{n+2}\dotsm a_{2n}}$. Für $a_1=a_2=\dotsb=a_n$ gilt aber $\sqrt[n]{a_1a_2\dotsm a_n}=a_1$ und für $a_{n+1}=a_{n+2}=\dotsb=a_{2n}$ gilt $\sqrt[n]{a_{n+1}a_{n+2}\dotsm a_{2n}}=a_{n+1}$. Wir sehen also, dass Gleichheit genau für $a_1=a_2=\dotsb=a_{2n}$ eintritt, wie gewünscht.
	
	\emph{Induktionsschritt, $n\rightarrow n-1$.} Gegeben seien nichtnegative reelle Zahlen $a_1,a_2,\dotsc,a_{n-1}\geqslant 0$. Betrachte außerdem $A=\frac1{n-1}(a_1+a_2+\dotsb+a_{n-1})$ und $G=\sqrt[n-1]{a_1a_2\dotsm a_{n-1}}$. Wir wollen also $A\geqslant G$ zeigen. Nun gilt
	\begin{equation*}
		\frac{(n-1)A}{n}+\frac{G}{n}=\frac{a_1+a_2+\dotsb+a_{n-1}+G}{n}\geqslant \sqrt[n]{a_1a_2\dotsm a_{n-1}G}=\sqrt[n]{G^{n-1}\cdot G}=G\,,
	\end{equation*}
	wobei wir die Induktionsannahme für $a_1,a_2,\dotsc,a_{n-1},G$ benutzt haben. Durch äquivalentes Umformen erhalten wir dann $(n-1)A/n\geqslant G-G/n=(n-1)G/n$ und nach Multiplikation mit $n/(n-1)$ folgt $A\geqslant G$, wie gewünscht. Gleichheit gilt genau dann, wenn in der obigen Abschätzung Gleichheit eintritt, also genau für $a_1=a_2=\dotsb=a_{n-1}=G$. Insbesondere kann Gleichheit nur für $a_1=a_2=\dotsb=a_{n-1}$ eintreten. In diesem Fall ist aber automatisch $A=a_1=G$, sodass tatsächlich Gleichheit eintritt. Das beendet die Induktion und den Beweis.
\end{proof}

\subsection*{Gewichtetes AM-GM}
Die folgende Variante der AM-GM-Ungleichung ist ebenfalls häufig nützlich:
\begin{satzmitnamen}[Gewichtete AM-GM-Ungleichung]
	Gegeben seien reelle Zahlen $a_1,a_2,\dotsc,a_n\geqslant 0$ sowie Gewichte $\lambda_1,\lambda_2,\dotsc,\lambda_n> 0$ mit $\lambda_1+\lambda_2+\dotsb+\lambda_n=1$. Dann gilt stets
	\begin{equation*}
		\lambda_1a_1+\lambda_2a_2+\dotsb+\lambda_na_n\geqslant a_1^{\lambda_1}a_2^{\lambda_2}\dotsm a_n^{\lambda_n}\,.
	\end{equation*}
	Gleichheit gilt genau dann, wenn alle Variablen gleich sind, also genau für $a_1=a_2=\dotsb=a_n$.
\end{satzmitnamen}
Die gewichtete AM-GM-Ungleichung ist eine Verallgemeinerung der AM-GM-Ungleichung, denn im Fall $\lambda_1=\lambda_2=\dotsb=\lambda_n=\frac1n$ erhalten wir genau die AM-GM-Ungleichung zurück. Umgekehrt lässt sich die gewichtete AM-GM-Ungleichung aber auch aus der normalen AM-GM-Ungleichung herleiten, wie wir sogleich sehen werden.

\begin{proof}
	Betrachte zuerst den Fall, dass alle Gewichte $\lambda_i$ rationale Zahlen sind. In diesem Fall können wir sie auf einen Hauptnenner bringen und somit $\lambda_i=m_i/N$ für gewisse positive ganze Zahlen $m_i$ und $N$ annehmen. Die Bedingung $\lambda_1+\lambda_2+\dotsb+\lambda_n=1$ impliziert dann $m_1+m_2+\dotsb+m_n=N$. Aus der normalen AM-GM-Ungleichung folgt nun
	\begin{align*}
		\lambda_1 a_1+\lambda_2 a_2+\dotsb+\lambda_na_n&=\frac{\overbrace{a_1+\dotsb+a_1}^{m_1\text{ mal}}+\overbrace{a_2+\dotsb+a_2}^{m_2\text{ mal}}+\dotsb+\overbrace{a_n+\dotsb+a_n}^{m_n\text{ mal}}}{N}\\
		&\geqslant \sqrt[N]{a_1^{m_1}a_2^{m_2}\dotsm a_n^{m_n}}\\
		&=a_1^{\lambda_1}a_2^{\lambda_2}\dotsm a_n^{\lambda_n}\,.
	\end{align*}
	%wie gewünscht.Gleichheit gilt genau dann, wenn in der AM-GM-Abschätzung Gleichheit eintritt, also genau für $a_1=a_2=\dotsb=a_n$.
	
	Der allgemeine Fall lässt sich auf den rationalen Fall mithilfe eines Stetigkeitsargumentes zurückführen. Dieses Argument wird euch vermutlich erstmal seltsam und ungewohnt vorkommen, aber in Wirklichkeit ist es ein Standardargument aus der Analysis. Mit etwas Übung werdet ihr leicht selbst auf solche Argumente kommen. Das Argument funktioniert folgendermaßen: Angenommen, die Ungleichung wäre falsch. Dann gilt also
	\begin{equation*}
		\lambda_1 a_1+\lambda_2 a_2+\dotsb+\lambda_na_n=a_1^{\lambda_1}a_2^{\lambda_2}\dotsm a_n^{\lambda_n}-\varepsilon
	\end{equation*}
	für ein $\varepsilon>0$. Wir können $\lambda_1,\lambda_2,\dotsc,\lambda_n$ beliebig genau durch rationale Zahlen $\lambda_1',\lambda_2',\dotsc,\lambda_n'$ approximieren. Wenn wir das tun, dann wird auch $\lambda_1 a_1+\lambda_2 a_2+\dotsb+\lambda_na_n$ beliebig genau durch $\lambda_1' a_1+\lambda_2' a_2+\dotsb+\lambda_n'a_n$ approximiert und $a_1^{\lambda_1}a_2^{\lambda_2}\dotsm a_n^{\lambda_n}$  wird beliebig genau durch $a_1^{\lambda_1'}a_2^{\lambda_2'}\dotsm a_n^{\lambda_n'}$ approximiert.\footnote{Wir sagen auch, die beiden Ausdrücke $\lambda_1 a_1+\lambda_2 a_2+\dotsb+\lambda_na_n$ und $a_1^{\lambda_1}a_2^{\lambda_2}\dotsm a_n^{\lambda_n}$ sind \emph{folgenstetig in den Variablen $(\lambda_1,\lambda_2,\dotsc,\lambda_n)$}} Indem wir genau genug approximieren, können wir rationale Gewichte $\lambda_1',\lambda_2',\dotsc,\lambda_n'$ mit $\lambda_1'+\lambda_2'+\dotsb+\lambda_n'=1$ finden, sodass Folgendes gilt:
	\begin{align*}
		\abs[\big]{\parens[\big]{\lambda_1 a_1+\lambda_2 a_2+\dotsb+\lambda_na_n}-\parens[\big]{\lambda_1' a_1+\lambda_2' a_2+\dotsb+\lambda_n'a_n}}&<\frac{\varepsilon}{2}\\
		\abs[\big]{a_1^{\lambda_1}a_2^{\lambda_2}\dotsm a_n^{\lambda_n}-a_1^{\lambda_1'}a_2^{\lambda_2'}\dotsm a_n^{\lambda_n'}}&<\frac{\varepsilon}{2}\,.
	\end{align*}
	Da wir die gewichtete AM-GM-Ungleichung für die Gewichte $\lambda_1',\lambda_2',\dotsc,\lambda_n'$ bereits bewiesen haben, folgt nun aber
	\begin{equation*}
		\lambda_1 a_1+\lambda_2 a_2+\dotsb+\lambda_na_n-a_1^{\lambda_1}a_2^{\lambda_2}\dotsm a_n^{\lambda_n}>\lambda_1' a_1+\lambda_2' a_2+\dotsb+\lambda_n'a_n-\frac{\varepsilon}{2}-a_1^{\lambda_1'}a_2^{\lambda_2'}\dotsm a_n^{\lambda_n'}-\frac{\varepsilon}{2}\,.
	\end{equation*}
	Daraus folgt $\lambda_1 a_1+\lambda_2 a_2+\dotsb+\lambda_na_n-a_1^{\lambda_1}a_2^{\lambda_2}\dotsm a_n^{\lambda_n}>-\varepsilon$, im Widerspruch zu unserer Wahl von $\varepsilon$. Unsere Annahme, dass die gewichtete AM-GM-Ungleichung falsch wäre, muss also selber falsch gewesen sein.
	
	Leider können wir auf diese Weise nicht den Gleichheitsfall untersuchen. Zwar folgt aus dem Argument für den rationalen Fall sofort, dass für rationale Gewichte $\lambda_1,\dotsc,\lambda_n$ Gleichheit nur eintreten kann, falls $a_1,a_2,\dotsc,a_n$ alle gleich sind. Aber es ist überhaupt nicht klar, dass durch das Approximationsargument nicht neue Gleichheitsfälle hinzukommen können. Die Untersuchung des Gleichheitsfalles verlangt also ein neues Argument, das ihr euch in der folgenden Übungsaufgabe selbst überlegen sollt.
\end{proof}
\begin{aufgabe*}\leavevmode
	\begin{enumerate}[label={$(\alph*)$},ref={$(\alph*)$}]
		\item Zeige per Induktion nach~$k$, $k\geqslant 1$, dass sich im Fall $n=2^k$ die normale AM-GM-Ungleichung wie folgt verschärfen lässt:
		\begin{equation*}
			\frac{a_1+a_2+\dotsb+a_{2^k}}{2^k}\geqslant \sqrt[2^k]{a_1a_2\dotsm a_{2^k}}+\frac{R}{2^{k+1}}\,,
		\end{equation*}
		wobei $R\coloneqq \parens{\sqrt{a_1}-\sqrt{a_2}}^2+\parens{\sqrt{a_3}-\sqrt{a_4}}^2+\dotsb+\parens{\sqrt{a_{2^k-1}}-\sqrt{a_{2^k}}}^2$.\label{aufgabe:AMGMSchaerfer}
		\item Gegeben seien $a_1,\dotsc,a_n\geqslant 0$ sowie Gewichte $\lambda_1,\lambda_2,\dotsc,\lambda_n>0$ mit $\lambda_1+\lambda_2+\dotsb+\lambda_n=1$. Angenommen, alle $\lambda_i$ sind rationale Zahlen mit Zweierpotenzen als Nenner. Folgere aus~\ref{aufgabe:AMGMSchaerfer}, dass sich die gewichtete AM-GM-Ungleichung wie folgt verschärfen lässt:\label{aufgabe:GewichtetAMGMSchaerferRational}
		\begin{equation*}
			\lambda_1 a_1+\lambda_2 a_2+\dotsb+\lambda_na_n\geqslant a_1^{\lambda_1}a_2^{\lambda_2}\dotsm a_n^{\lambda_n}+\frac{\min\{\lambda_1,\lambda_2\}\parens*{\sqrt{a_1}-\sqrt{a_2}}^2}{2}\,.
		\end{equation*}
		Hierbei bezeichnet $\min\{\lambda_1,\lambda_2\}$ das Minimum von $\lambda_1$ und $\lambda_2$.
		\item Zeige, dass die Aussage aus~\ref{aufgabe:GewichtetAMGMSchaerferRational} immer noch wahr ist, wenn $\lambda_1,\lambda_2,\dotsc,\lambda_n$ beliebige positive reelle Zahlen mit $\lambda_1+\lambda_2+\dotsb+\lambda_n=1$ sind. (\emph{Tipp: Benutze ein Stetigkeitsargument wie oben und approximiere $\lambda_i$ durch rationale Zahlen $\lambda_i'$, deren Nenner Zweierpotenzen sind.})\label{aufgabe:GewichtetAMGMSchaerfer}
		\item Folgere aus~\ref{aufgabe:GewichtetAMGMSchaerfer}, dass Gleichheit in der gewichteten AM-GM-Ungleichung genau für $a_1=a_2=\dotsb=a_n$ eintritt.
	\end{enumerate}
\end{aufgabe*}

\subsection*{Die allgemeine Potenzmittel-Ungleichung}
\begin{definition}
	Gegeben sei eine reelle Zahl $p\neq 0$ sowie $a_1,a_2,\dotsc,a_n\geqslant 0$ (im Fall $p<0$ müssen wir $a_i>0$ annehmen). Das \emph{$p$-te Potenzmittel von $a_1,a_2,\dotsc,a_n$} ist definiert als
	\begin{equation*}
		\parens*{\frac{a_1^p+a_2^p+\dotsb+a_n^p}{n}}^{1/p}\,.
	\end{equation*}
	Im Fall $p=1$ erhalten wir das arithmetische Mittel der Zahlen $a_1,a_2,\dotsc,a_n$. Die Potenzmittel für $p=2$ und $p=3$ werden üblicherweise \emph{quadratisches Mittel} und \emph{kubisches Mittel von $a_1,a_2,\dotsc,a_n$} genannt. Im Fall $p=-1$ erhalten wir das \emph{harmonische Mittel}
	\begin{equation*}
		\frac{n}{\frac1{a_1}+\frac1{a_2}+\dotsb+\frac1{a_n}}\,.
	\end{equation*}
\end{definition}
Potenzmittel finden auch außerhalb der Mathematik vielfach Anwendung (das könnt ihr zum Beispiel im Spam-Ordner manch eines e-Mail-Postfaches sehen). Sie erfüllen die folgende allgemeine Ungleichung:
\begin{satzmitnamen}[Allgemeine Potenzmittel-Ungleichung]\leavevmode
	\begin{enumerate}
		\item Gegeben seien positive reelle Zahlen $p>q>0$ sowie $a_1,a_2,\dotsc,a_n\geqslant 0$. Dann gilt stets
		\begin{equation*}
			\parens*{\frac{a_1^p+a_2^p+\dotsb+a_n^p}{n}}^{1/p}\geqslant \parens*{\frac{a_1^q+a_2^q+\dotsb+a_n^q}{n}}^{1/q}\geqslant \sqrt[n]{a_1a_2\dotsm a_n}\,.
		\end{equation*}
		An jeder Stelle gilt Gleichheit genau dann, wenn $a_1=a_2=\dotsb=a_n$.
		\item Gegeben seien negative reelle Zahlen $p<q<0$ sowie $a_1,a_2,\dotsc,a_n> 0$. Dann gilt stets
		\begin{equation*}
			\sqrt[n]{a_1a_2\dotsm a_n}\geqslant \parens*{\frac{a_1^q+a_2^q+\dotsb+a_n^q}{n}}^{1/q}\geqslant  \parens*{\frac{a_1^p+a_2^p+\dotsb+a_n^p}{n}}^{1/p}\,.
		\end{equation*}
		An jeder Stelle gilt Gleichheit genau dann, wenn $a_1=a_2=\dotsb=a_n$.
	\end{enumerate}
\end{satzmitnamen}
In den nun folgenden Übungsaufgaben sollt ihr euch selbstständig einen Beweis der allgemeinen Potenzmittel-Ungleichung erarbeiten.
\begin{aufgabe*}\label{aufgabe:PotenzmittelGroesserGM}
	Zeige, dass für $a_1,a_2,\dotsc,a_n\geqslant 0$ und $p>0$ stets die Ungleichung
	\begin{equation*}
		\parens*{\frac{a_1^p+a_2^p+\dotsb+a_n^p}{n}}^{1/p}\geqslant \sqrt[n]{a_1a_2\dotsm a_n}
	\end{equation*}
	gilt, mit Gleichheit genau für $a_1=a_2=\dotsb=a_n$ (\emph{Tipp: Substituiere $b_i\coloneqq a_i^p$}).
\end{aufgabe*}
\begin{aufgabe*}\label{aufgabe:PotenzmittelPositiverFall}
	Ziel dieser Aufgabe ist, die allgemeine Potenzmittelungleichung für $p>q>0$ zu zeigen. Im folgenden sind $a_1,a_2,\dotsc,a_n\geqslant 0$ stets nichtnegative reelle Zahlen und $m\geqslant 1$ ist eine positive ganze Zahl.
	\begin{enumerate}[label={$(\alph*)$},ref={$(\alph*)$}]
		\item Zeige, dass Binomialkoeffizienten die folgende Gleichung für alle $0\leqslant i\leqslant m-1$ erfüllen:\label{aufgabe:PotenzmittelPositiverFallA}
		\begin{equation*}
			(m+1)i\binom{m}{i}+(m+1)(i+1)\binom{m}{i+1}=m(i+1)\binom{m+1}{i+1}\,.
		\end{equation*}
		\item Zeige, dass für $i$, $j$ mit $i+j=m$ und $0\leqslant i\leqslant m-1$ die folgende Ungleichung gilt:\label{aufgabe:PotenzmittelPositiverFallB}
		\begin{equation*}
			\binom{m}{i}a_1^{(m+1)i}a_2^{(m+1)j}+\binom{m}{i+1}a_1^{(m+1)(i+1)}a_2^{(m+1)(j-1)}\geqslant \binom{m+1}{i+1}a_1^{m(i+1)}a_2^{mj}
		\end{equation*}
		(\emph{Tipp: Verwende gewichtetes AM-GM}).
		\item Zeige, dass
		\begin{equation*}
			\parens*{\frac{a_1^{m+1}+a_2^{m+1}}{2}}^{1/(m+1)}\geqslant \parens*{\frac{a_1^{m}+a_2^{m}}{2}}^{1/m}\,,
		\end{equation*}
		mit Gleichheit genau für $a_1=a_2$ (\emph{Tipp: Verwende~\ref{aufgabe:PotenzmittelPositiverFallB} und den binomischen Lehrsatz}).
		\item Zeige, dass
		\begin{equation*}
			\parens*{\frac{a_1^{m+1}+a_2^{m+1}+\dotsb+a_n^{m+1}}{n}}^{1/(m+1)}\geqslant \parens*{\frac{a_1^{m}+a_2^{m}+\dotsb+a_n^m}{n}}^{1/m}\,,
		\end{equation*}
		wobei Gleichheit genau für $a_1=a_2=\dotsb=a_n$ eintritt (\emph{Tipp: Verwende die Cauchysche Vorwärts-Rückwärts-Induktion}).\label{aufgabe:PotenzmittelPositiverFallC}
		\item Seien $p>q>0$ reelle Zahlen. Zeige, dass\label{aufgabe:PotenzmittelPositiverFallD}
		\begin{equation*}
			\parens*{\frac{a_1^{p}+a_2^{p}+\dotsb+a_n^{p}}{n}}^{1/p}\geqslant \parens*{\frac{a_1^{q}+a_2^{q}+\dotsb+a_n^q}{n}}^{1/q}
		\end{equation*}
		(\emph{Tipp: Zeige zuerst den Fall, dass $p$ und $q$ rational sind, indem du die Aussage auf~\ref{aufgabe:PotenzmittelPositiverFallD} zurückführst. Zeige dann den allgemeinen Fall mit einem Stetigkeitsargument}).\label{aufgabe:PotenzmittelPositiverFallE}
	\end{enumerate}
\end{aufgabe*}
Um zu zeigen, dass in Aufgabe~\ref{aufgabe:PotenzmittelPositiverFall}\ref{aufgabe:PotenzmittelPositiverFallE} Gleichheit genau für $a_1=a_2=\dotsb=a_n$ gilt, müssen wir uns zum Glück nicht so verrenken wie bei der gewichteten AM-GM-Ungleichung. Stattdessen bemerken wir einfach, dass Gleichheit für zwei reelle Zahlen $p>q>0$ auch Gleichheit für alle rationalen Zahlen $p'>q'$ mit $p>p'>q'>q$ impliziert. Aus dem rationalen Fall folgt dann $a_1=a_2=\dotsb=a_n$, wie gewünscht.
\begin{aufgabe*}
	Zeige, dass für alle negativen reellen Zahlen $p<q<0$ und alle $a_1,a_2,\dotsc,a_n>0$ stets
	\begin{equation*}
		\sqrt[n]{a_1a_2\dotsm a_n}\geqslant \parens*{\frac{a_1^q+a_2^q+\dotsb+a_n^q}{n}}^{1/q}\geqslant  \parens*{\frac{a_1^p+a_2^p+\dotsb+a_n^p}{n}}^{1/p}
	\end{equation*}
	gilt, mit Gleichheit genau für $a_1=a_2=\dotsb=a_n$ (\emph{Tipp: Substituiere $b_i\coloneqq 1/a_i$}).
\end{aufgabe*}\newpage
	\section{Die Umordnungs-Ungleichung}\label{kapitel:Umordnung}
In diesem Kapitel behandeln wir eine sehr simple, aber sehr mächtige Ungleichung.
\begin{satzmitnamen}[Umordnungs-Ungleichung]
	Sei $a_1\geqslant a_2\geqslant \dotsb\geqslant a_n\geqslant 0$ eine absteigend geordnete Folge sowie $b_1,b_2,\dotsc,b_n\geqslant 0$ eine beliebig geordnete Folge von nichtnegativen reellen Zahlen. Sei $b_1',b_2',\dotsc,b_n'$ eine Permutation von $b_1,b_2,\dotsc,b_n$, für die $b_1'\geqslant b_2'\geqslant \dotsb\geqslant b_n'\geqslant 0$ erfüllt ist. Dann gelten stets die Ungleichungen
	\begin{equation*}
		a_1b_1'+a_2b_2'+\dotsb+a_nb_n'\geqslant a_1b_1+a_2b_2+\dotsb+a_nb_n\geqslant a_1b_n'+a_2b_{n-1}'+\dotsb+a_nb_1\,.
	\end{equation*}
\end{satzmitnamen}
Wenn ihr ein wenig darüber nachdenkt, ist die Umordnungs-Ungleichung völlig klar: Um die Summe $a_1b_1+a_2b_2+\dotsb+a_nb_n$ so groß wie möglich zu machen, muss $a_1$ in der Summe so oft wie möglich vorkommen, also müssen wir $a_1$ mit dem größten $b_i$ multiplizieren. Analog muss $a_2$ mit dem zweitgrößten $b_i$ multipliziert werden und so weiter. Der formale Beweis ist auch nicht schwieriger:
\begin{proof}
	Betrachten wir zuerst den Fall $n=2$. In diesem Fall müssen wir nur die Ungleichung $a_1b_1'+a_2b_2'\geqslant a_1b_2'+a_2b_1'$ zeigen. Aber $(a_1b_1'+a_2b_2')- (a_1b_2'+a_2b_1')=(a_1-a_2)(b_1'-b_2')\geqslant 0$. Im allgemeinen Fall gehen wir wie folgt vor: Immer dann, wenn Indizes $i>j$ mit $b_i<b_j$ existieren, vertauschen wir $b_i$ und $b_j$. Nach endlich vielen Tauschen sind $b_1,b_2,\dotsc,b_n$ absteigend geordnet und aus dem Fall $n=2$ folgt, dass die Summe $a_1b_1+a_2b_2+\dotsb+a_nb_n$ in jedem Tausch nicht kleiner wird. Also gilt $a_1b_1'+a_2b_2'+\dotsb+a_nb_n'\geqslant a_1b_1+a_2b_2+\dotsb+a_nb_n$. Die andere Ungleichung folgt völlig analog.
\end{proof}

\subsection*{Beispielaufgaben}
Ihr sollt nun die Umordnungsungleichung auf zwei Beispielaufgaben anwenden. Unter den Aufgaben findet ihr Tipps und am Ende des Heftes findet ihr Musterlösungen. Wenn ihr bei Aufgabe~\ref{aufgabe:AM-GM-MitUmordnung} nicht weiterkommt, dann benutzt ruhig den Tipp oder lest euch die Lösung durch, denn diese Aufgabe ist wirklich nicht einfach.
\begin{aufgabe*}\label{aufgabe:EasyUmordnung}
	Zeige, dass für nichtnegative reelle Zahlen $a,b,c\geqslant 0$ die folgende Ungleichung gilt:
	\begin{equation*}
		a^3+b^3+c^3\geqslant a^2b+b^2c+c^2a\,.
	\end{equation*}
\end{aufgabe*}

\begin{aufgabe*}[*]\label{aufgabe:AM-GM-MitUmordnung}
	Benutze die Umordnungs-Ungleichung, um die AM-GM-Ungleichung zu zeigen!
\end{aufgabe*}

\vfill\hrule\vspace{-1em}

\subsection*{Tipps zu den Beispielaufgaben}
\textbf{Tipp zu Aufgabe~\ref{aufgabe:EasyUmordnung}.} Die Folgen $(a,b,c)$ und $(a^2,b^2,c^2)$ sind stets auf die gleiche Weise geordnet.

\textbf{Tipp zu Aufgabe~\ref{aufgabe:AM-GM-MitUmordnung}.} Zeige zuerst die Ungleichung
\begin{equation*}
	\frac{a_2}{a_1}+\frac{a_3}{a_2}+\dotsb+\frac{a_n}{a_{n-1}}+\frac{a_1}{a_n}\geqslant n\,.
\end{equation*}
Wähle dann geeignete Werte für $a_1,a_2,\dotsc,a_n$.\newpage
	\section{Ungleichungen und die Schiebemethode}\label{kapitel:Schiebemethode}

Die Schiebemethode ist eine sehr naive, aber erstaunlich effektive Methode, an Ungleichungen heranzugehen: Wir wählen uns zuerst zwei der Variablen aus, sagen wir,~$a$ und~$b$. Alle anderen Variablen werden fixiert. Wir verschieben nun~$a$ und~$b$ gegeneinander (und zwar auf solch eine Weise, dass die Nebenbedingung, wenn sie existiert, erfüllt bleibt). Wenn wir Glück haben, können wir sehr einfach ablesen, dass das Minimum oder Maximum nur in speziellen Fällen angenommen werden kann, zum Beispiel nur für $a=0$ oder $b=0$ oder $a=b$. Dann müssen wir die Ungleichung nur in diesen Spezialfällen beweisen, was meistens deutlich einfacher ist.

Zum Beispiel kommt es häufig vor, dass wir eine Ungleichung in vier Variablen $a,b,c,d\geqslant 0$ mit der Nebenbedingung $a+b+c+d=1$ haben. Wir fixieren $c$ und $d$. Damit die Nebenbedingung erhalten bleibt, fixieren wir außerdem $s\coloneqq a+b$. Nun verschieben wir $a$ und $b$ gegeneinander. Weil wir $s$ fixieren, läuft das darauf hinaus, $b=s-a$ einzusetzen und $a$ zu variieren. Wenn wir Glück haben, ist der erhaltene Ausdruck linear in $a$. Lineare Funktionen nehmen in Minimum und Maximum stets am Rand ihres Definitionsbereiches an. Wegen $a,b\geqslant 0$ und $a+b=s$ ist der Definitionsbereich für $a$ genau das Intervall $[0,s]$. Wir müssen also nur die beiden Randfälle $a=0$ und $a=s$ (was auf $b=0$ führt) betrachten.

Natürlich kommt es eher selten (aber auch nicht nie) vor, dass der erhaltene Ausdruck linear in $a$ ist. Dafür ist die Methode aber auch sehr flexibel. Zum Beispiel würde sie genauso gut funktionieren, wenn der erhaltene Ausdruck bloß monoton in $a$ ist (egal ob steigend oder fallend). Aber auch für nicht-monotone Ausdrücke können wir uns häufig auf wenige Spezialfälle beschränken. Das werden wir nun an zwei Aufgaben demonstrieren. Wir empfehlen euch, die Aufgaben zuerst mit AM-GM oder der Umordnungsungleichung zu attackieren, um euch davon zu überzeugen, dass sie alles andere als trivial sind.
\begin{aufgabe*}\label{aufgabe:JuMaUngleichung}
	Seien $a,b,c,d\geqslant 0$ nichtnegative reelle Zahlen mit $a+b+c+d=4$. Beweise die Ungleichung
	\begin{equation*}
		abc+bcd+cda+dab\leqslant ac+bd+\frac12(ab+bc+cd+da)\,.
	\end{equation*}
\end{aufgabe*}
\begin{proof}
	Wir fixieren $b$, $d$ und die Summe $s=a+c$ und verschieben die Variablen $a$ und $c$ gegeneinander. Dabei bleibt die Nebenbedingung offenbar erhalten. Wir können die Ungleichung umschreiben zu
	\begin{equation*}
		0\leqslant ac\parens[\big]{1-(b+d)}+bd(1-s)+\frac12s(b+d)\,.
	\end{equation*}
	Der einzige Term, der sich beim gegenseitigen Verschieben von $a$ und $c$ ändert, ist $ac(1-(b+d))$. Wir wollen diesen Term so klein wie möglich machen. Je nach dem, ob der Faktor $1-(b+d)$ positiv oder negativ ist, wollen wir dafür $ac$ minimieren oder maximieren (im Fall $1-(b+d)=0$ ist es egal, ob wir maximieren oder minimieren). Das Produkt $ac$ nimmt offensichtlich für $a=0$ oder $c=0$ sein Minimum an. Für das Maximum erinnern wir uns an die AM-GM-Ungleichung: Es gilt $\sqrt{ac}\leqslant s/2$, also $ac\leqslant s^2/4$. Ferner gilt Gleichheit in AM-GM genau für $a=c$. Also nimmt das Produkt $ac$ genau für $a=c$ sein Maximum an.
	
	Also müssen wir die Ungleichung nur in den drei Spezialfällen $a=0$, $c=0$ und $a=c$ beweisen! Eine analoge Überlegung lässt sich für $b$ und $d$ durchführen: Wir müssen nur die Spezialfälle $b=0$, $d=0$ und $b=d$ betrachten. Bis auf Vertauschen von $a$ und $c$ oder $b$ und $d$ haben wir die Ungleichung damit auf die folgenden drei Spezialfälle reduziert:
	
	\emph{Fall~1: $a=0$ und $b=0$.} Dieser Fall führt auf die triviale Ungleichung $0\leqslant \frac12 cd$.
	
	\emph{Fall~2: $a=0$, $b=d$.} In diesem Fall müssen wir die Ungleichung $b^2c\leqslant b^2+bc$ unter der Nebenbedingung $2b+c=4$ beweisen. Indem wir $c=4-2b$ einsetzen und alles auf eine Seite bringen, erhalten wir die Ungleichung
	\begin{equation*}
		0\leqslant b\parens*{2b^2-5b+4}=b\,\parens*{\parens*{b-\frac{5}{4}}^2+\frac78}\,,
	\end{equation*}
	welche offensichtlich wahr ist.
	
	\emph{Fall~3: $a=c$ und $b=d$.} Hier müssen wir $2a^2b+2ab^2\leqslant a^2+b^2+2ab$ unter der Nebenbedingung $a+b=2$ zeigen. Die linke Seite lässt sich als $2ab(a+b)=4ab$ schreiben, und die rechte Seite als $(a+b)^2$. Wir müssen also $4ab\leqslant (a+b)^2$ zeigen, was direkt aus AM-GM folgt.
\end{proof}

Nehmt euch einen Moment, um diese Lösung zu verdauen: Wir haben, ohne uns ernsthaft anzustrengen, die Ungleichung auf wenige triviale Spezialfälle reduziert! Dabei betonen wir erneut, dass es sich hier keineswegs um eine triviale Aufgabe handelt.\footnote{In der JuMa-Klausur, in welcher der Autor dieses Textes mit Aufgabe~\ref{aufgabe:JuMaUngleichung} konfrontiert wurde, hat sie niemand rausbekommen.}

\begin{aufgabe*}
	Gegeben seien nichtnegative reelle Zahlen $x,y,z\geqslant 0$ mit $x+y+z=1$. Beweise die Ungleichung
	\begin{equation*}
		0\leqslant xy+yz+zx-2xyz\leqslant\frac{7}{27}\,.
	\end{equation*}
\end{aufgabe*}
\begin{proof}
	Wir fixieren $z$ sowie die Summe $s\coloneqq x+y$ und verschieben die Variablen $x$ und $y$ gegeneinander. Dabei bleibt die Nebenbedingung offenbar erhalten. Wir können die Ungleichung umschreiben zu
	\begin{equation*}
		0\leqslant xy(1-2z)+sz\leqslant \frac{7}{27}\,.
	\end{equation*}
	Der einzige Term, der sich beim Verschieben ändert, ist $xy(1-2z)$. Wir müssen also herausfinden, wann das Produkt $xy$ sein Minimum und Maximum annimmt. Genau wie in der Lösung von Aufgabe~\ref{aufgabe:JuMaUngleichung} finden wir heraus, dass das Minimum von $xy$ genau bei $x=0$ oder $y=0$ angenommen wird und das Maximum genau bei $x=y$. Bis auf Vertauschung von $x$ und $y$ müssen wir also nur die folgenden beiden Fälle betrachten:
	
	\emph{Fall~1: $x=0$.} In diesem Fall gilt $xy+yz+zx-2xyz=yz$. Die Ungleichung $0\leqslant yz$ ist nun trivial. Die Nebenbedingung wird zu $y+z=1$, sodass wir mithilfe der AM-GM-Ungleichung
	\begin{equation*}
		yz\leqslant\parens*{\frac{y+z}{2}}^2=\frac{1}{4}<\frac{7}{27}\,.
	\end{equation*}
	erhalten. Also gelten die gewünschten Ungleichungen in diesem Fall.
	
	\emph{Fall~2: $x=y$.} In diesem Fall folgt $xy+yz+zx-2xyz=x^2+2xz-2x^2z$. Wegen $x\leqslant 1$ gilt $2xz\geqslant 2x^2z$, also ist $x^2+2xz-2x^2z\geqslant 0$. Damit ist die erste Hälfte der Ungleichung gezeigt. Für die zweite Hälfte benutzen wir $z=1-(x+y)=1-2x$ laut Nebenbedingung. Indem wir das einsetzen, erhalten wir $x^2+2xz-2x^2z=4x^3-5x^2+2x$. Nun gilt
	\begin{equation*}
		4x^3-5x^2+2x-\frac{7}{27}=\parens*{x-\frac13}^2\parens*{4x-\frac{7}{3}}\leqslant 0\,,
	\end{equation*}
	denn der erste Faktor ist stets nichtnegativ, während der zweite Faktor für $x\leqslant \frac12$ (was aus $2x+z=1$ und $x,z\geqslant 0$ folgt) negativ ist. Damit haben wir die Ungleichung auch im zweiten Fall bewiesen.
\end{proof}

\textbf{Bemerkung~1.} Die Faktorisierung im zweiten Fall scheint auf den ersten Blick vom Himmel zu fallen, aber in Wirklichkeit ist sie sehr naheliegend. Es lässt sich sofort nachprüfen, dass in der Ungleichung $xy+yz+zy-2xyz\leqslant\frac7{27}$ für $x=y=z=\frac13$ Gleichheit eintritt (den Fall, dass alle Variablen gleich sind, solltet ihr immer als erstes ausprobieren). Also muss das Polynom $f(x)\coloneqq4x^3-5x^2+2x-\frac{7}{27}$ bei $x=\frac13$ eine Nullstelle haben. Dann muss es aber bei $x=\frac13$ automatisch eine Doppelnullstelle haben, denn sonst würde es dort sein Vorzeichen ändern und die Ungleichung wäre falsch. Also wissen wir, dass $f(x)$ durch $\parens[\big]{x-\frac13}^2$ teilbar sein muss. Durch Polynomdivision erhalten wir dann die gewünschte Faktorisierung. 

Merkt euch diese Technik! Wenn ihr es schafft, eure Ungleichung auf eine Variable zurückzuführen, seid ihr durch (quadratisches) Ausklammern der Gleichheitsfälle fast immer fertig.

Um Gleichheitsfälle zu erraten, solltet ihr die \enquote{üblichen Verdächtigen} durchprobieren: Alle Variablen sind gleich, alle bis auf eine Variable sind gleich oder einige Variablen sind $0$ und die anderen Variablen sind gleich.

Um allgemein Nullstellen von Polynomen höheren Grades zu raten, könnt ihr wie folgt vorgehen. Bringt das Polynom zuerst in eine Form $P(X)=a_0+a_1X+\dotsb+a_nX^n$, in der $a_0,a_1,\dotsc,a_n$ ganze Zahlen sind (wenn euer ursprüngliches Polynom keine rationalen Koeffizienten hat, ist meistens etwas schief gelaufen). Wir dürfen außerdem $a_0,a_n\neq 0$ annehmen (für $a_0=0$ ist $X=0$ eine Nullstelle). Wenn $X=r/s$ eine rationale Nullstelle ist, wobei $r/s$ vollständig gekürzt ist (sodass~$r$ und~$s$ teilerfremde ganze Zahlen sind), dann folgt
\begin{equation*}
	0=s^nP\parens*{\frac rs}=a_0s^n+a_1rs^{n-1}+\dotsb+a_nr^n\,.
\end{equation*}
Also ist $0\equiv a_0s^n\mod r$ und $0\equiv a_nr^n\mod s$. Weil $r$ und $s$ teilerfremd sind, folgt, dass $a_0$ durch~$r$ und $a_n$ durch~$s$ teilbar sein muss. Auf diese Weise müsst ihr nur endlich viele Fälle durchprobieren und findet alle rationalen Nullstellen von~$P$.

Für irrationale Nullstellen gibt es kein so einfaches Kriterium. In Olympiade-Aufgaben haben eure Polynome aber meistens mindestens eine rationale Nullstelle.

\textbf{Bemerkung~2.} Eine weitere Idee im Fall~2 wäre, die Schiebemethode als nächstes auf $y$ und $z$ anzuwenden und so die Ungleichung auf den Fall $x=y=z=\frac13$ zu reduzieren. Hier müssen wir allerdings aufpassen: Wenn wir $x=y$ erreicht haben und jetzt $y$ und $z$ gegeneinander verschieben, bis wir $y=z$ erreichen, geht die Gleichheit $x=y$ wieder verloren. Wir erhalten also \emph{nicht} direkt $x=y=z$. Trotzdem lässt sich auch dieser Ansatz zum Ziel führen: Wenn wir die Schiebemethode erst auf $x$ und $y$, dann auf $y$ und $z$, dann auf $z$ und $x$ anwenden und das Ganze sehr oft wiederholen, lässt sich tatsächlich zeigen, dass $x$, $y$ und $z$ gegen $\frac13$ konvergieren (es sei denn, irgendeiner dieser Schritte führt uns in den Fall~1, aber dieser Fall ist ja ebenfalls trivial).

Allerdings ist es etwas umständlich, diese Lösung so sauber aufzuschreiben, dass ihr keine Punkte verliert. In der Olympiade solltet ihr deshalb lieber den etwas weniger eleganten Weg gehen.



\subsection*{Weitere Übungsaufgaben}

\begin{aufgabe*}
	Gegeben seien nichtnegative reelle Zahlen $a,b,c, d\geqslant 0$ mit $a+b+c+d=1$. Beweise die Ungleichung
	\begin{equation*}
		abc+bcd+cda+dab\leqslant \frac{1}{27}+\frac{176}{27}abcd\,.
	\end{equation*}
\end{aufgabe*}

\begin{aufgabe*}\label{exc:Ungleichung1}
	Gegeben sei der Ausdruck
	\begin{equation*}
		T\coloneqq x_1x_2x_3+x_2x_3x_4+x_3x_4x_5+x_4x_5x_6+x_5x_6x_7+x_6x_7x_1+x_7x_1x_2
	\end{equation*}
	für nichtnegative reelle Zahlen $x_1,x_2,\dotsc,x_7\geqslant 0$ mit $x_1+x_2+\dotsb+x_7=1$. Beweise, dass $T$ einen maximalen Wert annimmt und bestimme diesen Wert.
\end{aufgabe*}\newpage
	
	\phantomsection\cftaddtitleline{toc}{part}{Geometrie}{\thepage}	
	\section{Die Sätze von Ceva und Menelaos}\label{kapitel:CevaMenelaos}
In vielen Geometrieaufgaben sollt ihr beweisen, dass sich drei Geraden in einem Punkt treffen oder dass drei Punkte auf einer Geraden liegen. In diesen Situationen sind die Sätze von Ceva und Menelaos häufig sehr nützlich.

\subsection*{Der Satz von Ceva und seine Varianten}
\begin{satzmitnamen}[Satz von Ceva]
	Sei $ABC$ ein Dreieck und seien $X$,~$Y$ und~$Z$ Punkte im Inneren der Seiten $\overline{BC}$, $\overline{CA}$ und~$\overline{AB}$. Die Geraden $AX$, $BY$ und~$CZ$ schneiden sich genau dann in einem Punkt, wenn die folgende Gleichung gilt:
	\begin{equation*}%\label{eq:Ceva}
		\frac{\abs{BX}}{\abs{XC}}\cdot \frac{\abs{CY}}{\abs{YA}}\cdot \frac{\abs{AZ}}{\abs{ZB}}=1\,.
	\end{equation*}
\end{satzmitnamen}

\begin{proof}
	Wir nehmen zuerst an, dass sich $AX$, $BY$ und~$CZ$ im Punkt~$T$ schneiden. Seien $h_B$~und~$h_C$ die Längen der Höhen von $B$ und~$C$ auf die Gerade~$AX$. Dann sind die Flächeninhalte der Dreiecke $ABT$ und $CAT$ gegeben durch $\mathrm{A}_{ABT}=\frac 12\abs{AT}\cdot h_B$ und $\mathrm{A}_{CAT}=\frac12 \abs{AT}\cdot h_C$. Es gilt also $\mathrm{A}_{ABT}/\mathrm{A}_{CAT}=h_B/h_C$. Die Höhen von $B$ und~$C$ auf~$AX$ sind parallel, also können wir den Strahlensatz anwenden und erhalten außerdem $h_B/h_C=\abs{BX}/\abs{XC}$. Es folgt $\mathrm{A}_{ABT}/\mathrm{A}_{CAT}=\abs{BX}/\abs{XC}$. Analoge Gleichungen gelten für $\abs{CY}/\abs{YA}$ und $\abs{AZ}/\abs{ZB}$. Dann ist also
	\begin{equation*}
		\frac{\abs{BX}}{\abs{XC}}\cdot \frac{\abs{CY}}{\abs{YA}}\cdot \frac{\abs{AZ}}{\abs{ZB}}=\frac{\mathrm{A}_{ABT}}{\mathrm{A}_{CAT}}\cdot \frac{\mathrm{A}_{BCT}}{\mathrm{A}_{ABT}}\cdot \frac{\mathrm{A}_{CAT}}{\mathrm{A}_{BCT}}=1\,,
	\end{equation*}
	wie gewünscht.
	
	Nun nehmen wir umgekehrt an, dass die Gleichung aus dem Satz von Ceva gilt. Sei $X'$ derjenige Punkt auf~$\overline{BC}$, für den sich die Geraden $AX'$, $BY$ und~$CZ$ in einem Punkt schneiden. Wir wollen $X=X'$ zeigen. Wie wir gerade gezeigt haben, gilt die Gleichung aus dem Satz von Ceva dann auch für~$X'$. Es muss also $\abs{BX}/\abs{XC}=\abs{BX'}/\abs{X'C}$ gelten. Für jede positive reelle Zahl~$\lambda$ gibt es aber genau einen Punkt~$X_\lambda$ auf~$\overline{BC}$, für den $\abs{BX_\lambda}/\abs{X_\lambda C}=\lambda$ gilt. Es muss also $X=X'$ gelten und die Geraden $AX$, $BY$ und~$CZ$ schneiden sich in der Tat in einem Punkt.
\end{proof}

\begin{figure}[ht]
	\centering
	\begin{tikzpicture}[x=0.8cm,y=0.8cm]
		\clip (-4.5,-4.4) rectangle (3.4,2.9);
		\coordinate (A) at (2.28,2.3);
		\coordinate (B) at (-3.9,-2.52);
		\coordinate (C) at (2.68,-3.4);
		\coordinate (T) at (0.54,-1);
		\coordinate (X) at (-0.5,-2.97);
		\coordinate (Y) at (2.47,-0.34);
		\coordinate (Z) at (-0.48,0.15);
		\fill[black!15!white] (A) to (B) to (T) to cycle;
		\fill[black!15!white] (C) to (A) to (T) to cycle;
		\draw (A) to (B);
		\draw (B) to (C);
		\draw (C) to (A);
		\draw[line width=0.3,shorten >=-3.25em] (A) to (X);
		\draw[line width=0.3,shorten >=-2em] (B) to (Y);
		\draw[line width=0.3,shorten >=-3em] (C) to (Z);
		\draw[line width=0.3,dashed] (B) to node[below,pos=0.5] {$h_B$} (-1.05,-4.02) ;
		\draw[line width=0.3,dashed] (C) to node[below,pos=0.7] {$h_C$} (0.02,-2);
		\draw[line width=0.3, shift={(-1.05,-4.02)}] (62.2:0.32cm) arc (62.2:152.2:0.32cm);
		\fill[shift={(-1.05,-4.02)}] (107.2:0.18cm) circle (1pt);
		\draw [line width=0.3,shift={(0.02,-2)}] (-117.8:0.32cm) arc (-117.8:-27.8:0.32cm);
		\fill[shift={(0.02,-2)}] (-72.8:0.18cm) circle (1pt);
		\draw[fill=black] (A) circle (2pt) node[shift={(60:2ex)}] {$A$};
		\draw[fill=black] (B) circle (2pt) node[shift={(200:2ex)}] {$B$};
		\draw[fill=black] (C) circle (2pt) node[shift={(-30:2ex)}] {$C$};
		\draw[fill=black] (T) circle (2pt) node[shift={(90:2.5ex)}] {$T$};
		\draw[fill=black] (X) circle (2pt) node[shift={(-70:2ex)}] {$X$};
		\draw[fill=black] (Y) circle (2pt) node[shift={(-40:2ex)}] {$Y$};
		\draw[fill=black] (Z) circle (2pt) node[shift={(80:2ex)}] {$Z$};
	\end{tikzpicture}
\end{figure}

Es gibt eine Version des Satzes von Ceva, die auch dann gilt, wenn die Punkte $X$,~$Y$ und~$Z$ nicht notwendigerweise im Inneren der Strecken $\overline{BC}$, $\overline{CA}$ und~$\overline{AB}$ liegen. Der größte Teil des Beweises funktioniert auch im allgemeineren Fall, nur das letzte Argument geht schief: Für jede positive reelle Zahl~$\lambda$ (außer $\lambda=1$) gibt es \emph{zwei} Punkte~$X_\lambda$ auf der Geraden~$BC$, für die $\abs{BX_\lambda}/\abs{X_\lambda C}=\lambda$ gilt: einen im Inneren von~$\overline{BC}$ und einen außerhalb. Dieses Problem lässt sich beheben, indem wir \emph{gerichtete Strecken} verwenden. Dafür wählen wir eine Richtung auf der Geraden~$BC$. Für zwei verschiedene Punkte $D$~und~$E$ auf der Geraden~$BC$ ordnen wir der Strecke~$\overline{DE}$ die positive Länge $+\abs{DE}$ oder die negative Länge $-\abs{DE}$ zu, je nachdem, ob $D$~und~$E$ in der gewählten Richtung oder entgegensetzt zur gewählten Richtung auf der Geraden~$BC$ liegen. Die gerichtete Streckenlänge von~$\overline{DE}$ notieren wir einfach als~$DE$ (das kollidiert zwar mit der Bezeichnung für die Gerade~$DE$, aber es wird stets aus dem Kontext klar sein, was gemeint ist). Das Vorzeichen der gerichteten Streckenlängen $BX$ und~$CX$ hängt von der Wahl der Richtung auf der Geraden~$BC$ ab, nicht aber das Verhältnis $BX/XC$! Wir erhalten, unabhängig von der Wahl der Richtung, dass das Verhältnis $BX/XC$ positiv ist, wenn $X$ im Inneren von~$\overline{BC}$ liegt, und negativ, wenn $X$ außerhalb liegt. Mit gerichteten Streckenlängen gibt es folglich für jede reelle Zahl $\lambda\neq 0,-1$ genau einen Punkt~$X_\lambda$ auf~$BC$ verschieden von $B$ und~$C$, für den $BX_\lambda/X_\lambda C=\lambda$ gilt.

Analog lassen sich \emph{gerichtete Flächeninhalte} definieren: Einem Dreieck $DEF$ ordnen wir den positiven Flächeninhalt $+\mathrm{A}_{DEF}$ oder den negativen Flächeninhalt $-\mathrm{A}_{DEF}$ zu, je nachdem, ob das Dreieck $DEF$ mathematisch positiv oder negativ orientiert ist.

In der allgemeineren Version des Satzes von Ceva tritt leider eine weitere Feinheit auf:
% Tien: Wirklich? "Subtilität"? Wie wäre es mit "Feinheit"?
Es kann passieren, dass die Geraden $AX$, $BY$ und~$CZ$ paarweise parallel sind. Dann schneiden sie sich gewissermaßen \enquote{im Unendlichen} und wir sollten erwarten, dass Gleichung aus dem Satz von Ceva auch in diesem Fall erfüllt ist. Somit erhalten wir den folgenden Satz:

\begin{satzmitnamen}[Gerichteter Satz von Ceva]
	Sei $ABC$ ein Dreieck und seien $X$,~$Y$ und~$Z$ Punkte auf den Geraden $BC$, $CA$ und~$AB$ \embrace{verschieden von $A$,~$B$ und~$C$}. Die Geraden $AX$, $BY$ und~$CZ$ schneiden sich genau dann in einem Punkt oder sind paarweise parallel, wenn die folgende Gleichung \embrace{mit gerichteten Streckenlängen} gilt:
	\begin{equation*}%\label{eq:GerichteterCeva}
		\frac{BX}{XC}\cdot \frac{CY}{YA}\cdot \frac{AZ}{ZB}=1\,.
	\end{equation*}
\end{satzmitnamen}

\begin{proof}
	Wenn die Geraden $AX$, $BY$ und~$CZ$ paarweise parallel sind, folgt die behauptete Gleichung leicht aus dem Strahlensatz (der auch für orientierte Streckenlängen gültig ist). Indem wir den Strahlensatz zuerst auf die Parallelen $AX\parallel BY$ und danach auf die Parallelen $BY\parallel CZ$ anwenden, erhalten wir
	\begin{equation*}
		\frac{BX}{XC}=\frac{BA}{AZ}\quad\text{and}\quad \frac{CY}{YA}=\frac{ZB}{BA}\,.
	\end{equation*}
	Aus diesen beiden Gleichungen folgt sofort die behauptete Gleichung
	\begin{equation*}
		\frac{BX}{XC}\cdot \frac{CY}{YA}\cdot \frac{AZ}{ZB}=\frac{BA}{BA}=1\,.
	\end{equation*}
	
	Wenn sich $AX$, $BY$ und~$CZ$ in einem Punkt~$T$ schneiden, können wir die behauptete Gleichung auf die gleiche Weise wie im Satz von Ceva beweisen, nur dass wir gerichtete Streckenlängen und die gerichteten Flächeninhalte der Dreiecke $ABT$, $BCT$ und $CAT$ verwenden. Die Umkehrung lässt sich ebenfalls völlig analog beweisen.
\end{proof}

Der Satz von Ceva hat auch eine trigonometrische Version, die ebenfalls sehr nützlich sein kann:
\begin{satzmitnamen}[Trigonometrischer Satz von Ceva]
	Sei $ABC$ ein Dreieck und seien $X$,~$Y$ und~$Z$ Punkte auf den Geraden $BC$, $CA$ und~$AB$ \embrace{verschieden von $A$,~$B$ und~$C$}. Die Geraden $AX$, $BY$ und~$CZ$ schneiden sich genau dann in einem Punkt, wenn die folgende Gleichung \embrace{mit gerichteten Winkeln} gilt:
	\begin{equation*}%\label{eq:TrigonometrischerCeva}
		\frac{\sin\winkel BAX}{\sin\winkel XAC}\cdot \frac{\sin \winkel CBY}{\sin\winkel YBA}\cdot \frac{\sin \winkel ACZ}{\sin\winkel ZCB}=1\,.
	\end{equation*}
\end{satzmitnamen}

\begin{proof}
	Übungsaufgabe. (\emph{Tipp: Verwende die Formel $\mathrm{A}_{ABT}=\frac12\abs{AB}\cdot \abs{AT}\cdot \sin\winkel BAT$ und argumentiere dann analog zum Beweis des normalen Satzes von Ceva.})
	% Tien: Den kommentierten Text habe ich Korrektur gelesen.
	%		Wir nehmen zuerst an, dass sich $AX$, $BY$ und~$CZ$ im Punkt~$T$ schneiden. Wie vorher betrachten wir die (gerichteten) Flächeninhalte von $ABT$ und $CAT$. Nach der trigonometrischen Flächeninhaltsformel ist $\mathrm{A}_{ABT}=\frac 12\abs{AB}\cdot \abs{AT}\cdot \sin\winkel BAT$ und $\mathrm{A}_{CAT}=\frac 12\abs{AC}\cdot \abs{AT}\cdot \sin\winkel TAC$. Es gilt $\sin\winkel BAT=\pm\sin\winkel BAX$ und $\sin\winkel TAC=\pm\sin\winkel XAC$, je nachdem, ob $X$ auf der gleichen Seite von~$A$ wie~$T$ liegt oder nicht. In jedem Fall heben sich die Vorzeichen auf und wir erhalten $\sin\winkel BAX/\sin\winkel XAC=\sin\winkel BAT/\sin\winkel TAC$. Folglich gilt
	%		\begin{equation*}
		%			\frac{\sin\winkel BAX}{\sin\winkel XAC}=\frac{\mathrm{A}_{ABT}}{\mathrm{A}_{CAT}}\cdot \frac{\frac12\abs{AC}\cdot\abs{AT}}{\frac12\abs{AB}\cdot \abs{AT}}=\frac{\mathrm{A}_{ABT}}{\mathrm{A}_{CAT}}\cdot \frac{\abs{AC}}{\abs{AB}}
		%		\end{equation*}
	%		Analoge Gleichungen gelten auch für die anderen Winkel. Folglich ist
	%		\begin{equation*}
		%			\frac{\sin\winkel BAX}{\sin\winkel XAC}\cdot \frac{\sin \winkel CBY}{\sin\winkel YBA}\cdot \frac{\sin \winkel ACZ}{\sin\winkel ZCB}=\frac{\mathrm{A}_{ABT}}{\mathrm{A}_{CAT}}\cdot \frac{\abs{AC}}{\abs{AB}}\cdot \frac{\mathrm{A}_{BCT}}{\mathrm{A}_{ABT}}\cdot \frac{\abs{AB}}{\abs{BC}}\cdot \frac{\mathrm{A}_{CAT}}{\mathrm{A}_{BCT}}\cdot \frac{\abs{BC}}{\abs{AC}}=1\,,
		%		\end{equation*}
	%		wie gewünscht. Wenn umgekehrt die gewünschte Gleichung gilt, können wir analog zum Beweis des Satzes von Ceva argumentieren, dass sich $AX$, $BY$ und~$CZ$ in der Tat in einem Punkt schneiden müssen.
\end{proof}

Aus dem Satz von Ceva und seiner trigonometrischen Variante folgen sofort einige klassische Resultate aus der Dreiecksgeometrie.

\begin{aufgabe*}
	Benutze den Satz von Ceva oder seine trigonometrische Version, um zu zeigen, dass sich in jedem Dreieck die Seitenhalbierenden, die Winkelhalbierenden und die Höhen jeweils in einem Punkt schneiden!
\end{aufgabe*}

\begin{aufgabe*}
	Sei $ABC$ ein Dreieck und seien $D$,~$E$ und~$F$ die Berührpunkt des Inkreises mit $\overline{BC}$, $\overline{CA}$ und~$\overline{AB}$. Beweise, dass sich die Geraden $AD$, $BE$ und~$CF$ in einem Punkt schneiden \embrace{dieser Punkt ist der sogenannte \emph{Gergonnepunkt} von $ABC$}.
\end{aufgabe*}

\subsection*{Der Satz von Menelaos}
\begin{satzmitnamen}[Satz von Menelaos]
	Sei $ABC$ ein Dreieck und seien $X$,~$Y$ und~$Z$ Punkte auf den Geraden $BC$, $CA$ und~$AB$ \embrace{verschieden von $A$,~$B$ und~$C$}. Dann liegen die Punkte $X$,~$Y$ und~$Z$ genau dann auf einer Geraden, wenn die folgende Gleichung \embrace{mit gerichteten Streckenlängen} gilt:
	\begin{equation*}%\label{eq:GerichteterMenelaos}
		\frac{BX}{XC}\cdot \frac{CY}{YA}\cdot \frac{AZ}{ZB}=-1\,.
	\end{equation*}
\end{satzmitnamen}

\begin{proof}
	Nehmen wir zunächst an, dass $X$,~$Y$ und~$Z$ auf einer Geraden~$\ell$ liegen. Sei $H_A$ der Lotfußpunkt von~$A$ auf~$\ell$ und definiere $H_B$,~$H_C$ analog. Die Höhen von $B$~und~$C$ auf~$\ell$ sind parallel. Aus dem Strahlensatz (der in dieser Form auch für gerichtete Streckenlängen gültig ist) erhalten wir also $BX/XC=BH_B/H_CC=-BH_B/CH_C$. Analoge Gleichungen gelten für $CY/YA$ und $AZ/ZB$. Es folgt
	\begin{equation*}
		\frac{BX}{XC}\cdot \frac{CY}{YA}\cdot \frac{AZ}{ZB}=\parens*{-\frac{BH_B}{CH_C}}\parens*{-\frac{CH_C}{AH_A}}\parens*{-\frac{AH_A}{BH_B}}=-1\,,
	\end{equation*}
	wie gewünscht. Wenn umgekehrt die Gleichung aus dem Satz von Menelaos gilt, können wir wie im Beweis des Satzes von Ceva argumentieren: Sei $X'$ derjenige Punkt auf~$BC$, für den $X'$,~$Y$ und~$Z$ kollinear sind. Wir müssen $X=X'$ zeigen. Wie wir gerade gesehen haben, ist für den Punkt~$X'$ ebenfalls die Gleichung aus dem Satz von Menelaos erfüllt. Es muss also $BX/XC=BX'/X'C$ gelten. Für jede reelle Zahl $\lambda\neq 0$ gibt es aber genau einen Punkt~$X_\lambda$ auf der Geraden~$BC$, für den $BX_\lambda/X_\lambda C=\lambda$ gilt. Folglich muss in der Tat $X=X'$ sein.
\end{proof}

\begin{center}
	\begin{tikzpicture}[x=0.75cm,y=0.75cm]
		\clip (-4.5,-4.7) rectangle (7.2,3.15);
		\coordinate (A) at (2.28,2.3);
		\coordinate (B) at (-3.9,-2.52);
		\coordinate (C) at (2.68,-3.4);
		\coordinate (Y) at (2.59,-2.05);
		\coordinate (X) at (6.56,-3.92);
		\coordinate (Z) at (-1.09,-0.33);
		\coordinate (HA) at (0.66,-1.15);
		\coordinate (HB) at (-2.55,0.36);
		\coordinate (HC) at (3.18,-2.33);
		\draw (A) to (B);
		\draw[shorten >=-8.5em] (B) to (C); % Tien: Ich habe die Gerade mal etwas länger gemacht (davor war das -6.5em)
		\draw[line width=0.3, dashed] (A) to (HA);
		\draw[line width=0.3, dashed] (B) to (HB);
		\draw[line width=0.3, dashed] (C) to (HC);
		\draw (C) to (A);
		\draw[line width=0.3,shorten <=-2ex,shorten >=-2ex] (HB) to (X);
		\draw[line width=0.3,shift={(HB)}] (-115.15:0.32cm) arc (-115.15:-25.15:0.32cm);
		\fill[shift={(HB)}] (-70.15:0.18cm) circle (1pt);
		\draw[line width=0.3,shift={(HA)}] (-25.15:0.32cm) arc (-25.15:64.85:0.32cm);
		\fill[shift={(HA)}] (19.85:0.18cm) circle (1pt);
		\draw[line width=0.3,shift={(HC)}] (-115.15:0.32cm) arc (-115.15:-25.15:0.32cm);
		\fill[shift={(HC)}] (-70.15:0.18cm) circle (1pt);
		\draw[fill=black] (A) circle (2pt) node[shift={(60:2ex)}] {$A$};
		\draw[fill=black] (B) circle (2pt) node[shift={(240:2ex)}] {$B$};
		\draw[fill=black] (C) circle (2pt) node[shift={(240:2ex)}] {$C$};
		\draw[fill=black] (X) circle (2pt) node[shift={(240:2ex)}] {$X$};
		\draw[fill=black] (Y) circle (2pt) node[shift={(220:2ex)}] {$Y$};
		\draw[fill=black] (Z) circle (2pt) node[shift={(90:2ex)}] {$Z$};
		\draw[fill=white] (HA) circle (2pt) node[shift={(240:2ex)}] {$H_A$};
		\draw[fill=white] (HB) circle (2pt) node[shift={(60:2ex)}] {$H_B$};
		\draw[fill=white] (HC) circle (2pt) node[shift={(60:2ex)}] {$H_C$};
	\end{tikzpicture}
	% Tien: Was macht das hier? Wenn du unbedingt willst, dass das Bild auf derselben Seite passen soll, dann solltest du am besten auf die figure Umgebung verzichten (der Zweck ist automatisches Setzen von Floats).
	% Vielleicht willst du \begin{center}...\end{center} verwenden?
	%\vspace{-2em}
\end{center}

\subsection*{Weitere Übungsaufgaben}
\begin{aufgabe*}
	Sei $ABC$ ein Dreieck und sei $P$ ein Punkt verschieden von $A$,~$B$ und~$C$. 
	\begin{enumerate}
		\item Sei $\ell_A$ das Spiegelbild der Geraden~$AP$ an der Winkelhalbierenden von $\winkel BAC$ und definiere $\ell_B$,~$\ell_C$ analog. Zeige, dass sich die Geraden $\ell_A$,~$\ell_B$ und~$\ell_C$ in einem Punkt~$Q$ schneiden \embrace{$Q$ wird der \emph{zu~$P$ isogonal konjugierte Punkt} genannt}.
	\end{enumerate}
	Die folgenden beiden Teilaufgaben haben nichts mit den Sätzen von Ceva und Menelaos zu tun haben, aber die Theorie von isogonal konjugierten Punkte kann gelegentlich in der Olympiade ganz nützlich sein, deshalb sollt ihr sie euch hier selbstständig erarbeiten.
	\begin{enumerate}[resume]
		\item Seien $P_a$, $P_b$ und $P_c$ die Spiegelbilder von $P$ an den Geraden $BC$, $CA$ und $AB$. Zeige, dass der isogonal konjugierte Punkt $Q$ der Umkreismittelpunkt des Dreiecks $P_aP_bP_c$ ist.\label{teilaufgabe:SatzVomIsogonalKonjugiertenPunkt}
		\item Finde heraus, was die isogonal konjugierten Punkte des Umkreismittelpunkts, des Inkreismittelpunkts und des Höhenschnittpunkts sind. Was besagt Teilaufgabe~\ref{teilaufgabe:SatzVomIsogonalKonjugiertenPunkt} in diesen drei Fällen?
	\end{enumerate}
	Der isogonal konjugierte Punkt des Schwerpunkts (der sogenannte \emph{Symmedianpunkt}) kommt auch gelegentlich in Aufgaben vor, allerdings eher in der Oberstufe. Im Heft der Klasse~12 gibt es ein Kapitel dazu.
\end{aufgabe*}

\begin{aufgabe*}
	Sei $ABC$ ein Dreieck und sei $P$ ein Punkt auf dem Umkreis $\odot ABC$. Ferner seien $P_a$,~$P_b$ und~$P_c$ die Lotfußpunkte von~$P$ auf die Geraden $BC$, $CA$ und~$AB$. Zeige, dass die Punkte $P_a$, $P_b$ und~$P_c$ auf einer Geraden liegen \embrace{diese Gerade wird üblicherweise \emph{Simson-Gerade} genannt}.
\end{aufgabe*}

\begin{aufgabe*}
	Sei $ABCD$ ein Tangentenviereck. Der Inkreis~$\omega$ berühre die Seiten $\overline{AB}$, $\overline{BC}$, $\overline{CD}$ und~$\overline{DA}$ in den Punkten $W$,~$X$, $Y$ und~$Z$. Zeige, dass sich die Geraden $WX$, $YZ$ und~$AC$ in einem Punkt schneiden oder paarweise parallel sind.
\end{aufgabe*}

\begin{aufgabe*}
	% Tien: Lol, ist das eine Shortlist?
	Für ein konvexes Fünfeck $ABCDE$ gelte $\winkel BAC=\winkel CAD=\winkel DAE$ sowie $\winkel CBA=\winkel DCA=\winkel EDA$. Der Schnittpunkt der Diagonalen $BD$ und~$CE$ wird mit~$P$ bezeichnet. Beweise, dass die Gerade~$AP$ durch den Mittelpunkt der Seite~$\overline{CD}$ verläuft.
\end{aufgabe*}

\begin{aufgabe*}
	Sei $ABC$ ein spitzwinkliges Dreieck mit $\abs{BC}>\abs{CA}$. Die Mittelsenkrechte der Strecke~$\overline{AB}$ schneide die Gerade~$BC$ in~$P$ und die Gerade~$CA$ in~$Q$. Sei $R$ der Lotfußpunkt von~$R$ auf~$CA$ und $S$ der Lotfußpunkt von~$Q$ auf~$BC$. Zeige, dass $R$,~$S$ und der Mittelpunkt der Strecke~$\overline{AB}$ auf einer Geraden liegen. 
\end{aufgabe*}

\begin{aufgabe*}[*]
	Sei $ABC$ ein Dreieck mit Inkreismittelpunkt~$I$ und sei $\omega$ ein Kreis mit Mittelpunkt~$I$ im Inneren des Dreiecks. Die Lote von~$I$ auf die Dreiecksseiten $BC$, $CA$ und~$AB$ schneiden~$\omega$ in $A'$,~$B'$ und~$C'$. Zeige, dass sich die Geraden $AA'$, $BB'$ und~$CC'$ in einem Punkt schneiden.
\end{aufgabe*}

\begin{aufgabe*}[***]
	Beweise den Satz von Pascal:
	\begin{satzmitnamen}[Satz von Pascal]
		Sei $ABCDEF$ ein Sehnensechseck, dessen gegenüberliegende Seiten paarweise nicht-parallel sind. Die Seiten $AB$~und~$DE$ schneiden sich in~$X$, die Seiten $BC$~und~$EF$ schneiden sich in~$Y$ und die Seiten $CD$~und~$FA$ schneiden sich in~$Z$. Dann sind $X$,~$Y$ und~$Z$ kollinear.
	\end{satzmitnamen}
\end{aufgabe*}\newpage
	\section{Inversion am Kreis}\label{kapitel:Inversion}
Inversion am Kreis ist eine geometrische Transformation. Die \enquote{klassischen} geometrischen Transformationen, wie Drehungen, Spiegelungen oder zentrische Streckungen, sind häufig nützliche Hilfsmittel. Allerdings bringt es wenig, sie direkt auf eine Aufgabenstellung anzuwenden: Wenn ihr eine Aufgabe um einen Punkt dreht, erhaltet ihr die gleiche Aufgabe zurück. Mit der Inversion verhält sich das anders: Wenn ihr eine Aufgabe an einem Kreis invertiert, erhaltet ihr im Allgemeinen eine völlig andere (aber äquivalente!) Aufgabe. Und wenn ihr euch geschickt anstellt, ist die neue Aufgabe einfacher als die alte!

Wir werden zuerst die Theorie besprechen und dann an mehreren Olympiadeaufgaben demonstrieren, wie Inversion am Kreis effektiv eingesetzt werden kann.

\subsection*{Formale Eigenschaften der Inversion}
\begin{definition}
	Sei $\Omega$ ein Kreis mit Mittelpunkt~$O$ und Radius~$r$. Die \emph{Inversion an~$\Omega$} ist diejenige Abbildung der Ebene (ohne~$O$) auf sich selbst, die einen beliebigen Punkt $A\neq O$ auf denjenigen Punkt~$A'$ auf dem Strahl~$\overrightarrow{OA}$ schickt, für den $\abs{OA}\cdot \abs{OA'}=r^2$ gilt.
\end{definition}

\begin{satzmitnamen}[Eigenschaften der Inversion]
	Sei $\iota$ die Inversion an~$\Omega$.
	\begin{enumerate}
		\item \label{itm:Involution}
		Für jeden Punkt $A\neq O$ ist $\iota(\iota(A))=A$. Mit anderen Worten: Die zweimalige Hintereinanderausführung $\iota\circ\iota$ ist die Identität.
		\item \label{itm:Vertauschen}
		Punkte im Inneren von~$\Omega$ werden durch~$\iota$ auf Punkte außerhalb von~$\Omega$ geschickt und umgekehrt. Punkte auf~$\Omega$ bleiben unter~$\iota$ fixiert.
		\item \label{itm:Winkel}
		Seien $A,B\neq O$ zwei beliebige Punkte und $A'$,~$B'$ ihre Bilder unter~$\iota$. Dann gilt $\winkel OAB=\winkel A'B'O$ und $\winkel ABO=\winkel OA'B'$. %Insbesondere sind die Dreiecke $OAB$ und $OA'B'$ gegensinning ähnlich.
		\item \label{itm:GeradeGerade}
		Geraden, die durch~$O$ verlaufen, werden von~$\iota$ auf sich selbst abgebildet.
		\item \label{itm:GeradeKreis}
		Geraden, die nicht durch~$O$ verlaufen, werden durch~$\iota$ auf Kreise abgebildet, die durch~$O$ verlaufen, und umgekehrt.
		\item \label{itm:KreisKreis}
		Kreise, die nicht durch~$O$ verlaufen, werden durch~$\iota$ wieder auf Kreise abgebildet, die nicht durch~$O$ verlaufen.
	\end{enumerate}
\end{satzmitnamen}

Um Eigenschaft~\ref{itm:GeradeKreis} besser zu verstehen, ist es hilfreich, einen \enquote{unendlich weit entfernten Punkt}~$\infty$ zur Euklidischen Ebene hinzuzufügen und $\iota(O)\coloneqq \infty$, $\iota(\infty)\coloneqq O$ zu definieren. Wir betrachten dann Geraden und Kreise als \emph{verallgemeinerte Kreise}: Ein verallgemeinerter Kreis ist eine Gerade, wenn er durch den unendlich weit entfernten Punkt~$\infty$ geht, sonst ist er ein normaler Kreis. Damit werden die Eigenschaften \ref{itm:GeradeKreis} und~\ref{itm:KreisKreis} zu einem Spezialfall der folgenden Eigenschaft:
\begin{enumerate}[label={$(\alph*')$}, ref={$(\alph*')$}, start=6]\itshape
	\item \label{itm:veralgKreis}
	Verallgemeinerte Kreise werden durch $\iota$ auf verallgemeinerte Kreise abgebildet.
\end{enumerate}
Aus Eigenschaft~\ref{itm:veralgKreis} folgt nämlich, dass $\iota$ verallgemeinerte Kreise durch~$O$ auf verallgemeinerte Kreise durch $\iota(O)=\infty$ abbildet und umgekehrt; das liefert uns genau Eigenschaft~\ref{itm:GeradeKreis}. Und verallgemeinerte Kreise, die weder durch~$O$ noch durch~$\infty$ gehen, werden auch nach der Inversion nicht durch~$O$ oder~$\infty$ gehen, sodass wir Eigenschaft~\ref{itm:KreisKreis} zurückbekommen.

Für verallgemeinerte Kreise lässt sich ein \emph{Schnittwinkel} definieren. Für zwei Geraden ist der Schnittwinkel wie üblich definiert: Die beiden Geraden spannen zwei Paare von Scheitelwinkeln auf, sodass wir zwei Werte erhalten, die sich zu~$180^\circ$ ergänzen; der Schnittwinkel ist dann der kleinere der beiden Werte (sodass der Schnittwinkel immer zwischen $0^\circ$~und~$90^\circ$ liegt). Der Schnittwinkel zweier Kreise ist definiert als der Schnittwinkel der Tangenten in einem Schnittpunkt der Kreise (sofern mindestens ein Schnittpunkt existiert; ansonsten ist der Schnittwinkel undefiniert). Analog definieren wir den Schnittwinkel zwischen einer Gerade und einem Kreis. Mit dieser Definition können wir zwei weitere Eigenschaften der Inversion formulieren:
\begin{enumerate}[resume]\itshape
	\item \label{itm:Schnitt}
	Wenn sich zwei verallgemeinerte Kreise im Winkel~$\varphi$ schneiden \embrace{wobei $0^\circ\leqslant \varphi\leqslant 90^\circ$}, dann schneiden sich auch ihre Bilder unter~$\iota$ im Winkel~$\varphi$.
	\item \label{itm:Schnitt90}
	Verallgemeinerte Kreise, die $\Omega$ im Winkel~$90^\circ$ schneiden, werden durch $\iota$ auf sich selbst abgebildet.
\end{enumerate}

Zwei Warnungen, bevor wir die Eigenschaften beweisen:
\begin{itemize}[\Warnung]
	\item In vielen Texten wird davon gesprochen, dass Inversion \emph{winkeltreu} ist. Damit ist Eigenschaft~\ref{itm:Schnitt} gemeint. Es folgt jedoch \emph{nicht}, dass $\winkel BAC=\winkel B'A'C'$ für beliebige drei Punkte $A$,~$B$ und~$C$ sowie ihre Bildpunkte $A'$,~$B'$ und~$C'$ gilt.
	\item Eigenschaft~\ref{itm:KreisKreis} besagt nur, dass Inversion Kreise, die nicht durch $O$ verlaufen, auf ebensolche Kreise abbildet. Es folgt \emph{nicht}, dass auch die Mittelpunkte aufeinander abgebildet werden.
\end{itemize}

\begin{proof}[Beweis der Eigenschaften]
	Die Eigenschaften~\ref{itm:Involution},~\ref{itm:Vertauschen} und~\ref{itm:GeradeGerade} folgen direkt aus der Definition.
	
	\emph{Eigenschaft~\ref{itm:Winkel}.} Wir bemerken zuerst, dass $\winkel AOB=\winkel A'OB'$ gilt, denn $A'$~und~$B'$ liegen auf den Strahlen $\overrightarrow{OA}$ und~$\overrightarrow{OB}$. Ferner ist
	\begin{equation*}
		\frac{\abs{OA'}}{\abs{OB'}}=\frac{r^2\!/\abs{OA}}{r^2\!/\abs{OB}}=\frac{\abs{OB}}{\abs{OA}}\,.
	\end{equation*}
	Nach dem Ähnlichkeitssatz~\emph{sws} sind die Dreiecke $OAB$ und $OA'B'$ also gegensinnig ähnlich. Die behauptete Winkelgleichheit in Eigenschaft~\ref{itm:Winkel} folgt unmittelbar.
	\begin{figure}[ht]
		\centering
		\begin{tabularx}{\textwidth}{X c X c X}
			& \begin{tikzpicture}[x=2.35cm,y=2.35cm]
				\draw[line width=0.3] (-22:1) arc (-22:110:1);
				\coordinate (O) at (0,0);
				\coordinate (A) at (1.21,0.23);
				\coordinate (A1) at (0.8,0.15);
				\coordinate (B) at (1.14,0.97);
				\coordinate (B1) at (0.51,0.43);
				\draw (O) to (A);
				\draw (O) to (B);
				\draw (A) to (B);
				\draw (A1) to (B1);
				\draw[line width=0.3, shift={(A)}] (95.1:0.32cm) arc (95.1:190.9:0.32cm);
				\draw[line width=0.3, shift={(B1)}] (220.29:0.32cm) arc (220.29:316.09:0.32cm);
				\draw[line width=0.3, shift={(B)}] (220.29:0.32cm) arc (220.29:275.1:0.32cm);
				\draw[line width=0.3, shift={(B)}] (220.29:0.37cm) arc (220.29:275.1:0.37cm);
				\draw[line width=0.3, shift={(A1)}] (136.09:0.32cm) arc (136.09:190.9:0.32cm);
				\draw[line width=0.3, shift={(A1)}] (136.09:0.37cm) arc (136.09:190.9:0.37cm);
				\draw[fill=white] (O) circle (2pt) node[shift={(220:2ex)}] {$O$};
				\draw[fill=black] (A) circle (2pt) node[shift={(-60:2ex)}] {$A$};
				\draw[fill=black] (A1) circle (2pt) node[shift={(-90:2ex)}] {$A'$};
				\draw[fill=black] (B) circle (2pt) node[shift={(80:2ex)}] {$B$};
				\draw[fill=black] (B1) circle (2pt) node[shift={(140:2ex)}] {$B'$};
				\node at (0,0.85) {$\Omega$};
			\end{tikzpicture} & & \begin{tikzpicture}[x=2.35cm,y=2.35cm]
				\draw[line width=0.3] (-22:1) arc (-22:110:1);
				\coordinate (O) at (0,0);
				\draw[line width=0.3,dashed] (0.406,0.038) circle (0.408);
				\coordinate (P) at (1.22,0.11);
				\coordinate (P1) at (0.81,0.07);
				\coordinate (A) at (1.14,0.97);
				\coordinate (A1) at (0.509,0.433);
				\draw (O) to (A);
				\draw (O) to (P);
				\draw[shorten <=-3ex,shorten >=-6.9ex] (A) to (P);
				\draw (A1) to (P1);
				\draw[line width=0.3, shift={(P)}] (95.1:0.32cm) arc (95.1:185.1:0.32cm);
				\fill[shift={(P)}] (140.1:0.18cm) circle (1pt);
				\draw[line width=0.3, shift={(A1)}] (220.29:0.32cm) arc (220.29:310.29:0.32cm);
				\fill[shift={(A1)}] (265.29:0.18cm) circle (1pt);
				\draw[fill=white] (O) circle (2pt) node[shift={(220:2ex)}] {$O$};
				\draw[fill=black] (P) circle (2pt) node[shift={(-25:2ex)}] {$P$};
				\draw[fill=black] (P1) circle (2pt) node[shift={(-135:2.5ex)}] {$P'$};
				\draw[fill=black] (A) circle (2pt) node[shift={(20:2ex)}] {$A$};
				\draw[fill=black] (A1) circle (2pt) node[shift={(110:2ex)}] {$A'$};
				\node at (0,0.85) {$\Omega$};
				\node at (0.4,-0.25) {$\omega$};
				\node at (1.35,-0.25) {$\ell$};
			\end{tikzpicture} & \\\addlinespace
			& Eigenschaft~\ref{itm:Winkel} & & Eigenschaft~\ref{itm:GeradeKreis} & 
		\end{tabularx}
	\end{figure}
	
	\emph{Eigenschaft~\ref{itm:GeradeKreis}.} Betrachte eine Gerade~$\ell$, die nicht durch~$O$ verläuft. Sei $P$ der Lotfußpunkt von~$O$ auf~$\ell$. Seien $P'$~und~$\ell'$ die Bilder von $P$~und~$\ell$ unter~$\iota$. Wir werden zeigen, dass $\ell'$ der Kreis~$\omega$ mit Durchmesser~$\overline{OP'}$ ist. Für jeden Punkt~$A$ auf~$\ell$ ist $\winkel APO=90^\circ$. Aus Eigenschaft~\ref{itm:Winkel} folgt $\winkel OA'P'=90^\circ$. Nach dem Satz von Thales liegt $A'$ somit auf~$\omega$. Damit ist gezeigt, dass $\ell'$ in~$\omega$ enthalten ist. Umgekehrt zeigt das gleiche Argument, dass für jeden Punkt~$B$ auf~$\omega$ der Bildpunkt~$\iota(B)$ auf~$\ell$ liegen muss (das stimmt auch für $B=O$, denn dann ist $\iota(B)=\infty$). Wegen $\iota(\iota(B))=B$ folgt, dass $\ell'$ jeden Punkt auf~$\omega$ enthält. Also wird $\ell$ unter der Inversion tatsächlich bijektiv auf~$\omega$ abgebildet. Damit ist gezeigt, dass $\iota$ Geraden auf Kreise durch~$O$ abbildet. Die umgekehrte Aussage folgt aus dem gleichen Argument, oder alternativ aus Eigenschaft~\ref{itm:Involution}.
	
	\emph{Eigenschaft~\ref{itm:KreisKreis}.} Sei $\omega$ ein Kreis, der nicht durch~$O$ verläuft. Die Gerade, die durch~$O$ und den Mittelpunkt von~$\omega$ verläuft, schneide $\omega$ in den Punkten $A$~und~$B$, sodass $\overline{AB}$ ein Durchmesser von~$\omega$ ist. Wir nehmen im Folgenden an, dass $A$~und~$B$ auf der gleichen Seite von~$O$ liegen (sodass $O$ außerhalb von~$\omega$ liegt); der andere Fall geht analog. Seien $\omega'$, $A'$~und~$B'$ die jeweiligen Bilder unter~$\iota$. Wir werden zeigen, dass $\omega'$ ein Kreis mit Durchmesser~$\overline{A'B'}$ ist. Dazu betrachte einen beliebigen Punkt~$C$ auf~$\omega$ und seinen Bildpunkt $C'=\iota(C)$. Nach Thales gilt $\winkel ACB=90^\circ$ und nach Außenwinkelsatz im Dreieck $ABC$ ist $\winkel CAO=90^\circ+\winkel CBO$. Aus Eigenschaft~\ref{itm:Winkel} folgt $\winkel OC'B'=\winkel CBO$ und $\winkel OC'A'=\winkel CAO=90^\circ+\winkel CBO$. Also ist
	\begin{equation*}
		\winkel B'C'A'=\winkel OC'A'-\winkel OC'B'= \parens{90^\circ+\winkel CBO}-\winkel CBO=90^\circ\,.
	\end{equation*}
	Aus der Umkehrung des Satzes von Thales folgt, dass $C'$ auf dem Kreis mit Durchmesser~$\overline{A'B'}$ liegt. Also ist $\omega'$ in diesem Kreis enthalten. Analog zu Eigenschaft~\ref{itm:GeradeKreis} argumentierten wir, dass $\omega'$ tatsächlich mit diesem Kreis identisch sein muss. Also ist $\omega'$ in der Tat ein Kreis.
	
	\begin{figure}[ht]
		\centering
		\begin{tabularx}{\textwidth}{X c X c X}
			& \begin{tikzpicture}[x=2.35cm,y=2.35cm]
				\draw[line width=0.3] (-22:1) arc (-22:110:1);
				\draw[line width=0.3] (0.652,0.058) circle (0.265);
				\draw[line width=0.3,dashed,shift={(1.819,0.162)}, dash phase=1] (-50:0.738) arc (-50:230:0.738);
				\coordinate (O) at (0,0);
				\coordinate (A) at (0.388,0.035);
				\coordinate (A1) at (2.554,0.228);
				\coordinate (B) at (0.916,0.082);
				\coordinate (B1) at (1.083,0.097);
				\coordinate (C) at (0.617,0.321);
				\coordinate (C1) at (1.276,0.663);
				\draw (O) to (A1);
				\draw (O) to (C1);
				\draw (A) to (C);
				\draw (B) to (C);
				\draw (B1) to (C1);
				\draw (A1) to (C1);
				\draw [line width=0.3,shift={(A)}] (51.358:0.27cm) arc (51.358:185.104:0.27cm);
				\draw [line width=0.3,shift={(A)}] (51.358:0.22cm) arc (51.358:185.104:0.22cm);
				\draw [line width=0.3,shift={(C)}] (231.358:0.32cm) arc (231.358:321.358:0.32cm);
				\fill [line width=0.3,shift={(C)}] (275.358:0.18cm) circle (1pt);
				\draw [line width=0.3,shift={(B)}] (141.358:0.37cm) arc (141.358:185.104:0.37cm);
				\draw [line width=0.3,shift={(C1)}] (207.457:0.27cm) arc (207.457:341.203:0.27cm);
				\draw [line width=0.3,shift={(C1)}] (207.457:0.47cm) arc (207.457:251.203:0.47cm);
				\draw [line width=0.3,shift={(C1)}] (207.457:0.22cm) arc (207.457:341.203:0.22cm);
				\draw[fill=white] (O) circle (2pt) node[shift={(220:2ex)}] {$O$};
				\draw[fill=black] (A) circle (2pt) node[shift={(230:2ex)}] {$A$};
				\draw[fill=black] (A1) circle (2pt) node[shift={(225:2.25ex)}] {$A'$};
				\draw[fill=black] (B) circle (2pt) node[shift={(225:2.25ex)}] {$B$};
				\draw[fill=black] (B1) circle (2pt) node[shift={(-30:2.5ex)}] {$B'$};
				\draw[fill=black] (C) circle (2pt) node[shift={(120:2ex)}] {$C$};
				\draw[fill=black] (C1) circle (2pt) node[shift={(120:2ex)}] {$C'$};
				\node at (0,0.85) {$\Omega$};
				\node at (0.65,-0.3) {$\omega$};
				\node at (1.8,0.78) {$\omega'$};
			\end{tikzpicture} & & \begin{tikzpicture}[x=2.35cm,y=2.35cm]
				\draw[line width=0.3] (-22:1) arc (-22:110:1);
				\draw[line width=0.3] (0.613,0.698) circle (0.476);
				\draw[line width=0.3] (1.212,0.521) circle (0.597);
				\coordinate (O) at (0,0);
				\coordinate (A) at (0.505,1.162);
				\coordinate (P) at (0.691,0.228);
				\coordinate (Q) at (0.934,1.05);
				\coordinate (B) at (1.622,0.086);
				\draw (P) to (Q);
				\draw (A) to (P) to (B) to (Q) to cycle;
				\draw[shorten <=-2.5em,shorten >=-4.2em] (0.362,0.812) to (P);
				\draw[shorten <=-3em,shorten >=-7.5em] (0.162,0.14) to (P);
				\draw [line width=0.3,shift={(P)}] (9.412:0.32cm) arc (9.412:73.511:0.32cm);
				\draw [line width=0.3,shift={(P)}] (73.511:0.37cm) arc (73.511:119.337:0.37cm);
				\draw [line width=0.3,shift={(P)}] (73.511:0.42cm) arc (73.511:119.337:0.42cm);
				\draw [line width=0.3,shift={(A)}] (281.23:0.32cm) arc (281.23:345.328:0.32cm);
				\draw [line width=0.3,shift={(B)}] (125.524:0.32cm) arc (125.524:171.367:0.32cm);
				\draw [line width=0.3,shift={(B)}] (125.524:0.37cm) arc (125.524:171.367:0.37cm);
				\draw[fill=white] (O) circle (2pt) node[shift={(220:2ex)}] {$O$};
				\draw[fill=black] (A) circle (2pt) node[shift={(140:2ex)}] {$A$};
				\draw[fill=black] (B) circle (2pt) node[shift={(-40:2ex)}] {$B$};
				\draw[fill=black] (P) circle (2pt) node[shift={(238:2.25ex)}] {$P$};
				\draw[fill=black] (Q) circle (2pt) node[shift={(85:2ex)}] {$Q$};
				\node at (0,0.85) {$\Omega$};
				\node at (0.3,0.55) {$\omega_1$};
				\node at (1.6,0.78) {$\omega_2$};
				\node at (0.05,0.245) {$t_1$};
				\node at (0.8,-0.2) {$t_2$};
			\end{tikzpicture} & \\\addlinespace
			& Eigenschaft~6 & & Eigenschaft~7 & 
		\end{tabularx}
	\end{figure}
	
	\emph{Eigenschaft~\ref{itm:veralgKreis}.} Bei dieser Eigenschaft sind mehrere Fälle zu unterscheiden, je nachdem, welche der verallgemeinerten Kreise durch $O$ oder~$\infty$ verlaufen. Wir werden nur den (schwierigsten) Fall betrachten, in dem wir es mit zwei Kreisen zu tun haben, die nicht durch~$O$ (und nicht durch~$\infty$) verlaufen, und überlassen es euch als Übungsaufgabe, die Argumente auf die anderen Fälle zu übertragen. Ferner ergeben sich in dem betrachteten Fall je nach Lage der Kreise mehrere Unterfälle. Wir werden der Einfachheit halber nur einen Lagefall betrachten. Für diejenigen unter euch, die bereits mit orientierten Winkeln modulo~$180^\circ$ vertraut sind, ist der Beweis so formuliert, dass er unter Verwendung von orientierten Winkeln modulo~$180^\circ$ für alle Lagefälle gültig ist.
	
	Betrachte also zwei Kreise $\omega_1$~und~$\omega_2$, die sich in den Punkten $P$~und~$Q$ im Winkel~$\varphi$ schneiden. Wähle ferner einen Punkt~$A$ auf~$\omega_1$ und einen Punkt~$B$ auf~$\omega_2$, sodass $A$~außerhalb von~$\omega_2$ und $B$~außerhalb von~$\omega_1$ liegt. Die Bilder unter~$\iota$ werden mit $A'$,~$B'$, $P'$, $Q'$, $\omega_1'$ und~$\omega_2'$ bezeichnet. Seien $t_1$~und~$t_2$ die Tangenten an $\omega_1$~und~$\omega_2$ in~$P$. Nach dem Sehnen-Tangentenwinkelsatz ist $\winkel (t_1,PQ)=\winkel PAQ$ und $\winkel (PQ,t_2)=\winkel QBP$. Folglich schneiden sich $t_1$~und~$t_2$ im Winkel $\winkel(t_1,PQ)+\winkel (PQ,t_2)=\winkel PAQ+\winkel QBP$. Das heißt allerdings nicht unbedingt, dass $\winkel PAQ+\winkel QBP=\varphi$ gelten muss: Dadurch, dass wir für Schnittwinkel stets $0^\circ\leqslant \varphi\leqslant 90^\circ$ fordern, kann $\winkel PAQ+\winkel QBP$ auch die Werte $180^\circ-\varphi$, $180^\circ+\varphi$ oder sogar $360^\circ-\varphi$ annehmen. In jedem Fall ist es aber ausreichend, $\winkel PAQ+\winkel QBP=\winkel Q'A'P'+\winkel P'B'Q'$ zu zeigen, denn wir können das gleiche Argument für $\omega_1'$~und~$\omega_2'$ anwenden. Dazu erinnern wir uns an Eigenschaft~\ref{itm:Winkel} und erhalten $\winkel Q'A'P'=\winkel OA'P'-\winkel OA'Q'=\winkel APO-\winkel AQO$ und analog $\winkel P'B'Q'=\winkel OPB-\winkel OQB$. Sodann rechnen wir
	\begin{align*}
		\winkel Q'A'P'+\winkel P'B'Q'&=\parens{\winkel APO+\winkel OPB}-\parens{\winkel AQO+\winkel OQB}\\
		&=360^\circ-\winkel BPA-\winkel AQB\\
		&=\winkel PAQ+\winkel QBP\,.
	\end{align*}
	Im zweiten Schritt haben wir benutzt, dass sich $\winkel APO$, $\winkel OPB$ und $\winkel BPA$ zu~$360^\circ$ ergänzen. Im dritten Schritt haben wir die Innenwinkelsumme im Viereck $APBQ$ eingesetzt. Das beendet den Beweis von Eigenschaft~\ref{itm:veralgKreis}.
	
	\emph{Eigenschaft~\ref{itm:Schnitt90}.} Sei $\omega$ ein verallgemeinerter Kreis, der $\Omega$ im Winkel~$90^\circ$ schneidet, und seien $A$~und~$B$ die beiden Schnittpunkte. Die Punkte $A$~und~$B$ werden durch~$\iota$ auf sich selbst abgebildet. Nach Eigenschaft~\ref{itm:Schnitt} ist das Bild von~$\omega$ also wiederum ein verallgemeinerter Kreis durch $A$~und~$B$, der $\Omega$ im Winkel~$90^\circ$ schneidet. Für je zwei Punkte $A$~und~$B$ gibt es aber genau einen verallgemeinerten Kreis durch $A$~und~$B$, der $\Omega$ im Winkel~$90^\circ$ schneidet (wenn $A$~und~$B$ diametral gegenüberliegen, ist dieser verallgemeinerte Kreis eine Gerade, ansonsten ein echter Kreis). Also muss $\omega$ auf sich selbst abgebildet werden.
\end{proof}

\subsection*{Strategien, Tipps und Tricks für Inversionslösungen}
Fast immer ist es egal, welchen Radius der Kreis hat, an dem wir invertieren. Wenn im Folgenden die Rede davon ist, dass \emph{an einem Punkt~$O$ invertiert wird}, so ist gemeint, dass wir an einem Kreis um~$O$ mit beliebigem Radius invertieren.
\begin{itemize}
	\item Die wichtigste Eigenschaft der Inversion ist Nummer~\ref{itm:GeradeKreis}, denn sie erlaubt uns, Aufgaben mit Kreisen in Aufgaben mit Geraden zu überführen, die häufig einfacher sind (und dank Eigenschaft~\ref{itm:Involution} ist die neue Aufgabe zur alten äquivalent).
	\item Invertieren an einem Punkt~$O$ lohnt sich besonders dann, wenn viele Kreise und viele Geraden durch~$O$ gehen. Denn so werden möglichst viele Kreise zu Geraden und möglichst wenige Geraden zu Kreisen.
	\item Zwei Kreise (oder ein Kreis und eine Gerade) berühren sich in einem Punkt~$O$ genau dann, wenn ihre Bilder unter Inversion an~$O$ zwei parallele Geraden sind, denn zwei Kreise (oder ein Kreis und eine Gerade) haben genau dann nur den Punkt~$O$ gemeinsam, wenn ihre Bilder unter Inversion an~$O$ nur den Punkt~$\infty$ gemeinsam haben. Wenn ihr also in einer Aufgabe zeigen sollt, dass sich zwei Kreise (oder ein Kreis und eine Gerade) berühren, ist es oft eine gute Idee, am gewünschten Berührpunkt zu invertieren. Dann müsst ihr nur noch zeigen, dass zwei Geraden parallel sind, was wesentlich weniger gruselig aussieht. Siehe dazu die Beispielaufgaben.
	\item Wir haben euch bereits gewarnt, dass Inversion die Mittelpunkte von Kreisen nicht erhält. Trotzdem lässt sich häufig herausfinden, worauf der Mittelpunkt eines Kreises geschickt wird. Wenn zum Beispiel $M$~der Mittelpunkt eines Kreises~$\omega$ ist, der durch das Inversionszentrum~$O$ verläuft, dann betrachte denjenigen Punkt~$P$ auf~$\omega$, für den $\overline{OP}$ ein Durchmesser ist. Dann ist $P'$ der Lotfußpunkt von~$O$ auf die Gerade~$\omega'$. Weil $M$ der Mittelpunkt von~$\overline{OP}$ ist, liegt $M'$ auf der Geraden~$OP'$. Andererseits folgt aus $\abs{OM}=\frac 12\abs{OP}$, dass $\abs{OM'}=2\abs{OP'}$ gelten muss. Also ist $M'$ das Spiegelbild von~$O$ an der Geraden~$\omega'$.
	
	Für allgemeinere Mittelpunkte könnt ihr mit ähnlichen Überlegungen, oder manchmal auch mit Eigenschaft~\ref{itm:Winkel}, herausfinden, worauf diese durch die Inversion geschickt werden.
	\item Streckenlängen bleiben natürlich auch nicht unter Inversion erhalten. Trotzdem gibt es eine relativ einfache Formel. Wenn $A$,~$B$ zwei Punkte sind und $A'$,~$B'$ ihre Bildpunkte unter Inversion an einem Kreis um~$O$ mit Radius~$r$, dann sind nach Eigenschaft~\ref{itm:Winkel} die Dreiecke $OAB$ und $OA'B'$ gegensinning ähnlich. Der Ähnlichkeitsfaktor muss dann $\abs{OA'}/\abs{OB}$ sein, also ist
	\begin{equation*}%\label{eq:Laengenformel}
		\abs{A'B'}=\abs{AB}\cdot \frac{\abs{OA'}}{\abs{OB}}=\abs{AB}\cdot \frac{r^2}{\abs{OA}\cdot\abs{OB}}\,.
	\end{equation*}
	\item Gelegentlich treten in Aufgabenstellungen sehr komische Winkelbedingungen auf. Durch geschickte Inversion könnt ihr dank Eigenschaft~\ref{itm:Winkel} dafür sorgen, dass diese Winkel an anderer Stelle in eurer Skizze auftauchen, und dann könnt ihr vielleicht mehr mit der Winkelbedingung anfangen. Siehe dazu die Beispielaufgaben.
	\item Manchmal sieht eure Skizze nach einer Inversion genauso aus wie vorher. Dann ist eure Aufgabe zwar nicht einfacher geworden, aber aus der Erkenntnis, dass die Aufgabe \enquote{symmetrisch unter Inversion} ist, könnt ihr trotzdem nichttriviale Schlüsse ziehen oder sogar die Aufgabe lösen. Siehe dazu die Beispielaufgaben.
	\item Versteift euch nicht zu sehr auf die invertierte Skizze. Wenn ihr nicht weiter kommt, dann versucht, eure Erkenntnisse aus der invertierten Skizze in die ursprüngliche Skizze zu übersetzen und die ursprüngliche Aufgabe damit zu lösen.
\end{itemize}

\subsection*{Beispielaufgaben}
Nun sollt ihr die Theorie aus den vorherigen Unterabschnitten auf Olympiadeaufgaben anwenden. Am Ende des Kapitels findet ihr Tipps zu den Beispielaufgaben und am Ende des Heftes könnt ihr die Lösungen nachlesen.

Diese Aufgaben sind ziemlich schwierig für die 9.\ Klasse. Bevor ihr in die Tipps oder die Lösungen schaut, solltet ihr euch trotzdem selbstständig überlegen, welche der Punkte als Inversionszentren in Frage kommen, um dafür Intuition zu entwickeln. Versucht außerdem, mithilfe der Eigenschaften~\ref{itm:Involution}--\ref{itm:Schnitt90} herauszufinden, wie eure Skizze nach der Inversion aussehen würde.

\begin{aufgabe*}\label{aufgabe:450943}
	In einem spitzwinkligen Dreieck $ABC$ sei $H$ der Lotfußpunkt von~$C$ auf~$AB$. Ferner seien $P$~und~$Q$ die Lotfußpunkte von~$H$ auf $AC$ und~$BC$. Die Umkreise $\odot AHP$ und $\odot HBQ$ schneiden die Strecke~$\overline{PQ}$ in von $P$~und~$Q$ verschiedenen Punkten $S$~und~$T$. Zeige, dass der Umkreis $\odot HTS$ die Gerade~$AB$ berührt.
\end{aufgabe*}

\begin{aufgabe*}[*]\label{aufgabe:IMO1996}
	Sei $P$ ein Punkt im Inneren eines Dreiecks $ABC$ mit der Eigenschaft $\winkel APB-\winkel ACB=\winkel CPA-\winkel CBA$. Beweise, dass sich die Winkelhalbierende von $\winkel PBA$ und die Winkelhalbierende von $\winkel ACP$ auf der Geraden~$AP$ schneiden.
\end{aufgabe*}

\begin{aufgabe*}[*]\label{aufgabe:521243}
	Gegeben sind zwei Kreise $\omega_1$~und~$\omega_2$, die sich in zwei Punkten schneiden. Einer der Schnittpunkte sei~$Q$. Der Punkt~$P$ liege außerhalb des Kreises~$\omega_1$ auf dem Kreis~$\omega_2$. Dabei sei die Lage von~$P$ so gewählt, dass folgende zwei Bedingungen erfüllt sind:
	\begin{enumerate}[label={$(\Alph*)$},ref={$(\Alph*)$}]
		\item Die Gerade~$PQ$ schneidet den Kreis~$\omega_1$ in einem von~$Q$ verschiedenen Punkt~$X$.
		\item Die Tangente an~$\omega_1$ in~$X$ schneidet $\omega_2$ in zwei Punkten $A$~und~$B$.
	\end{enumerate}
	Der Kreis~$\omega$ verlaufe durch die Punkte $A$~und~$B$ und berühre die Parallele~$\ell$ zu~$AB$ durch~$P$. Beweise, dass sich die Kreise $\omega$~und~$\omega_1$ berühren.
\end{aufgabe*}

\begin{aufgabe*}[*]\label{aufgabe:Ptolemaeus}
	Beweise die Ungleichung von Ptolemäus:
	\begin{satzmitnamen}[Ungleichung von Ptolemäus]
		Für beliebige vier paarweise verschiedene Punkte $A$,~$B$, $C$ und~$D$ in der Ebene gilt stets
		\begin{equation*}
			\abs{AB}\cdot \abs{CD}+\abs{BC}\cdot\abs{DA}\geqslant \abs{AC}\cdot \abs{BD}\,.
		\end{equation*}
		Gleichheit gilt genau dann, wenn $ABCD$ ein konvexes Sehnenviereck ist.
	\end{satzmitnamen}
	(Der Gleichheitsfall dieser Ungleichung ist als \emph{Satz von Ptolemäus} bekannt und häufig nützlich, wenn ihr Streckenlängen in einem Sehnenviereck berechnen wollt.)
\end{aufgabe*}

\subsection*{Weitere Übungsaufgaben}
\begin{aufgabe*}
	Sei $ABC$ ein spitzwinkliges Dreieck mit Umkreismittelpunkt~$O$. Die Umkreismittelpunkte der Dreiecke $AOC$ und $BCO$ werden mit $Q$~und~$R$ bezeichnet. Zeige, dass sich die Umkreise $\odot ABO$ und $\odot QOR$ berühren.
\end{aufgabe*}

\begin{aufgabe*}
	Gegeben sei ein Kreis~$\Omega$. Wir definieren eine unendliche Folge weiterer Kreise
	\begin{wrapfigure}[6]{r}{0.25\textwidth} % Tien: Statt \vspace{-...} soll das nur 6 Zeilen einnehmen
		\centering
		\begin{tikzpicture}
			%\draw (0,0) to (3.896,0);
			%\draw [fill=black] (1.222,0) circle (2pt) node[above] {$\omega_0$};
			%\draw [fill=black] (3.17,0) circle (2pt) node[above] {$\omega_1$};
			\draw [shift={(1.948,0)}] (-22:1.948) arc (-22:202:1.948);
			\draw [shift={(1.222,0)}] node {$\omega_0$} (-36:1.222) arc (-36:216:1.222);
			\pgfresetboundingbox % Tien: Engere bounding box
			\draw (0,0); % Tien: Manuell linker Rand
			\draw (3.17,0) node {$\omega_1$} circle (0.726);
			\draw (2.594,1.189) node {$\omega_2$} circle (0.595);
			\draw (1.68,1.54) node {$\omega_3$} circle (0.385);
			\draw (1.058,1.455)  circle (0.242); %node {$\omega_4$}
			\draw (0.697,1.278) circle (0.16);
			\draw (0.484,1.11) circle (0.111);
			\draw (0.353,0.971) circle (0.081);
			\draw (0.267,0.857) circle (0.061);
			\draw (0.209,0.756) circle (0.048);
			\draw (0.167,0.69) circle (0.038);
			\draw (0.137,0.627) circle (0.031);
			\draw (0.114,0.574) circle (0.026);
			\draw (0.096,0.529) circle (0.022);
			\draw (0.082,0.491) circle (0.019);
			\draw (0.082,0.491) circle (0.019);
			\draw (0.071,0.458) circle (0.016);
		\end{tikzpicture}
	\end{wrapfigure}
	$\omega_0,\omega_1,\omega_2,\dotsc$ wie folgt:
	\begin{itemize}%[label=\textup{\arabic*.},ref=\textup{\arabic*.}]
		\item Die Kreise $\omega_0$~und~$\omega_1$ berühren einander von außen und~$\Omega$ jeweils von innen. Ferner liegen die Mittelpunkte von $\Omega$,~$\omega_0$ und~$\omega_1$ auf einer Geraden.
		\item Für alle $n\geqslant 2$ berührt der Kreis~$\omega_n$ die Kreise $\omega_0$~und~$\omega_{n-1}$ von außen sowie~$\Omega$ von innen.
	\end{itemize}
	Sei $r_n$ der Radius von~$\omega_n$. Berechne $r_n$ in Abhängigkeit von $n$,~$r_0$ und~$r_1$.
\end{aufgabe*}

%\begin{aufgabe*}
%	Sei $ABCD$ ein Trapez mit $AB\parallel CD$. Sei $P$ ein innerer Punkt des Trapezes, für den $\winkel BPC=\winkel DPA$ gilt. Zeige, dass sich die Umkreise $\odot ABP$ und $\odot CDP$ berühren.
%\end{aufgabe*}

\begin{aufgabe*}
	Sei $\Omega$ ein Kreis und $A$,~$B$ zwei Punkte auf~$\Omega$. Der Kreis~$\omega$ berührt~$\Omega$ von innen in~$P$ und die Strecke~$\overline{AB}$ in~$Q$. 
	% Tien: Es gibt zwei Kreise, vielleicht ist das uneindeutig.
	Schließlich sei $S$ der Bogenmittelpunkt des Bogens~$\wideparen{AB}$ von~$\Omega$, der $P$ nicht enthält.
	\begin{enumerate}% Tien: Sollen die Buchstaben in der Aufzählung wirklich kursiv?
		\item Zeige, dass die Gerade~$PQ$ durch~$S$ verläuft.
		\item Zeige, dass $\abs{SP}\cdot \abs{SQ}=\abs{SA}^2=\abs{SB}^2$.
	\end{enumerate}
\end{aufgabe*}

\begin{aufgabe*}
	Gegeben ist ein Halbkreis~$\Omega$ mit Durchmesser~$\overline{AB}$. Auf~$\Omega$ liege ein Punkt~$C$, der von $A$~und~$B$ verschieden ist. Der Lotfußpunkt von~$C$ auf~$AB$ heiße~$D$. Ein Kreis~$\omega$ liege außerhalb des Dreiecks $ADC$ und berühre gleichzeitig den Halbkreis~$\Omega$ sowie die Strecken $\overline{AB}$ und~$\overline{CD}$. Der Berührpunkt von~$\omega$ mit~$\overline{AB}$ sei~$E$. Zeige, dass die Strecken $\overline{AC}$ und~$\overline{AE}$ gleich lang sind. 
\end{aufgabe*}

\begin{aufgabe*}
	Sei $ABC$ ein Dreieck mit Umkreis~$\Omega$. Ein weiterer Kreis~$\omega$ liege so, dass er sowohl die Strecke~$\overline{BC}$ berührt, 
	% Tien: Hier hört etwas abrupt auf. Wahrscheinlich muss das weg.
	als auch den Bogen~$\wideparen{BC}$ von~$\Omega$, der $A$ nicht enthält. Seien $P$~und~$Q$ die Berührpunkte und sei $M$ der Mittelpunkt von~$\omega$. Zeige: Wenn $\winkel BAM=\winkel MAC$, dann gilt auch $\winkel BAP=\winkel QAC$.
\end{aufgabe*}

\begin{aufgabe*}
	Sei $ABC$ ein spitzwinkliges Dreieck mit $\abs{AB}>\abs{AC}$. Beweise, dass es einen Punkt~$D$ mit folgender Eigenschaft gibt: Immer wenn $X$~und~$Y$ Punkte in $ABC$ sind, für die $BCXY$ ein Sehnenviereck ist und die Bedingung $\winkel AXB-\winkel ACB=\winkel CYA-\winkel CBA$ gilt, verläuft die Gerade~$XY$ durch~$D$.
\end{aufgabe*}

\begin{aufgabe*}
	Sei $ABCD$ ein konvexes Viereck, in dem keine zwei Seiten parallel sind. Die Punkte $P$~und~$Q$ liegen so innerhalb von $ABCD$, dass $PQDA$ und $QPBC$ Sehnenvierecke sind. Wir nehmen außerdem an, dass es einen Punkt~$E$ auf der Strecke~$\overline{PQ}$ gibt, für den $\winkel PAE=\winkel EDQ$ und $\winkel EBP=\winkel QCE$ ist. Beweise, dass $ABCD$ dann ein Sehnenviereck sein muss.
\end{aufgabe*}

\begin{aufgabe*}[*]
	Sei $ABC$ ein Dreieck mit $\winkel CBA>\winkel ACB$. Auf der Gerade~$AC$ liegen zwei verschiedene Punkte $P$~und~$Q$, sodass $\winkel PBA=\winkel ABQ=\winkel ACB$ gilt und $A$~zwischen $P$~und~$C$ liegt. Ferner gebe es einen Punkt~$D$ im Inneren der Strecke $\overline{BQ}$, für den $\abs{PD}=\abs{PB}$ gilt. Schließlich sei $R$ der von~$A$ verschiedene Schnittpunkt der Geraden~$AD$ mit dem Umkreis $\odot ABC$. Beweise, dass $\abs{QB}=\abs{QR}$. 
\end{aufgabe*}

\subsection*{Tipps zu den Beispielaufgaben}
\textbf{Tipp zu Aufgabe~\ref{aufgabe:450943}.} Invertiere an $H$. Wie sieht die Skizze nach der Inversion aus?

\textbf{Tipp zu Aufgabe~\ref{aufgabe:IMO1996}.} Invertiere an $A$. Wie lässt sich die Winkelbedingung nach der Inversion ausdrücken? Was fällt dir dabei auf?

\textbf{Tipp zu Aufgabe~\ref{aufgabe:521243}.} Bei dieser Aufgabe bringt es wenig, am gewünschten Berührpunkt zu invertieren, denn wir wissen ja anfangs gar nicht, wo der liegt. Invertiere statdessen an $X$. Wie sieht die Skizze nach der Inversion aus? Was fällt dir dabei auf?

\textbf{Tipp zu Aufgabe~\ref{aufgabe:Ptolemaeus}.} Invertiere an $A$ und führe die Ungleichung von Ptolemäus mithilfe der Längenformel auf eine bekannte Ungleichung zurück. Was kannst du über den Gleichheitsfall dieser bekannten Ungleichung aussagen?\newpage
	
	\phantomsection\cftaddtitleline{toc}{part}{Kombinatorik}{\thepage}	
	\section{Graphentheorie}\label{kapitel:Graphentheorie}
Die Graphentheorie ist ein relativ junges Teilgebiet der Mathematik, aber dafür eines mit vielen praktischen Anwendungen. Auch in der Mathematik-Olympiade kommt Graphentheorie regelmäßig vor. In diesem Kapitel werden wir einige grundlegende Begriffe einführen. Die Beispielaufgaben sollen euch überdies dabei helfen, einige Beweistechniken aus der Kombinatorik zu wiederholen.

\subsection*{Grundlegende Begriffe}
Ein \emph{Graph} $G=(V,E)$ besteht aus einer Menge $V$ von Knoten und einer Menge~$E$ von Kanten, wobei jede Kante zwischen zwei Knoten verläuft. Dabei darf es zwischen zwei Knoten auch mehrere Kanten geben und es darf auch Kanten geben, die einen Knoten mit sich selbst verbinden. In diesem Fall sprechen wir von \emph{parallelen Kanten} und \emph{Schleifen}. Ein Graph ohne parallele Kanten und ohne Schleifen heißt \emph{schlicht} (der unten abgebildete Graph ist also nicht schlicht). Eine Kante $e$ zwischen zwei Knoten $u$ und $v$ notieren wir oft als $e=uv$. Wenn $G$ nicht schlicht ist, ist diese Notation etwas ungenau, denn $e$ wird dann nicht eindeutig durch $u$ und $v$ bestimmt. In der Praxis führt das aber so gut wie nie zu Verwirrung.

\begin{figure}[ht]
	\centering
	\begin{tabularx}{\textwidth}{X c X c X}
		& 
		\begin{tikzpicture}
			\coordinate (a) at (18:1);
			\coordinate (b) at (90:1);
			\coordinate (c) at (162:1);
			\coordinate (d) at (234:1);
			\coordinate (e) at (306:1);
			\draw (c) to[in=130,out=230,loop,min distance=10mm,looseness=10] (c);
			\draw (a) to (b) to (e);
			\draw (a) to (d) to[bend left=20] (b) to[bend left=20] (d); 
			\draw[fill=black] (a) circle (2pt);
			\draw[fill=black] (b) circle (2pt);
			\draw[fill=black] (c) circle (2pt);
			\draw[fill=black] (d) circle (2pt);
			\draw[fill=black] (e) circle (2pt);
		\end{tikzpicture} & & \begin{tikzpicture}
		\coordinate (a) at (18:1);
		\coordinate (b) at (90:1);
		\coordinate (c) at (162:1);
		\coordinate (d) at (234:1);
		\coordinate (e) at (306:1);
		\path (a) edge[in=-50,out=50,loop,min distance=10mm,looseness=10,decoration={markings, mark=at position 0.55 with {\arrow{>}}},postaction={decorate}] (a);
		\path[decoration={markings, mark=at position 0.5 with {\arrow{>}}}] 
		(a) edge[postaction={decorate}] (b)
		(b) edge[postaction={decorate}] (c)
		(b) edge[postaction={decorate}] (d)
		(d) edge[postaction={decorate}] (a)
		(d) edge[postaction={decorate}] (c)
		(b) edge[postaction={decorate}] (e);
		\draw[fill=black] (a) circle (2pt);
		\draw[fill=black] (b) circle (2pt);
		\draw[fill=black] (c) circle (2pt);
		\draw[fill=black] (d) circle (2pt);
		\draw[fill=black] (e) circle (2pt);
		\end{tikzpicture} & \\\addlinespace
		& ein Graph $G_1$ & & ein gerichteter Graph $G_2$ & 
	\end{tabularx}
\end{figure}


In einem \emph{gerichteten Graphen} ist jede Kante außerdem mit einer Richtung ausgestattet. In einem gerichteten Graphen schreiben wir immer noch $e=uv$, um zu verdeutlichen, dass $e$ von $u$ nach $v$ verläuft. Im Gegensatz zu ungerichteten Graphen ist in dieser Schreibweise die Reihenfolge von $u$ und $v$ relevant. 

Wenn wir im Folgenden einfach nur \emph{Graph} schreiben, dann ist damit stets ein ungerichteter Graph gemeint. Außerdem beschränken wir uns auf \emph{endliche} Graphen, also solche bei denen die Mengen der Knoten $V$ und Kanten $E$ beide endlich sind.

Weiterhin gelten folgende Definitionen:

\textbf{Teilgraphen.} Ein \emph{Teilgraph} $G'=(V',E')$ eines Graphen $G=(V,E)$ ist ein Graph, dessen Knoten und Kanten in $G$ enthalten sind. Es muss also $V' \subseteq V$ und $E'\subseteq E$ gelten.

\textbf{Knotengrad.} In einem Graphen ist der \emph{Grad} $d(v)$ eines Knoten $v$  die Anzahl der angrenzenden Kanten. Dabei zählen Schleifen doppelt. In einem gerichteten Graphen ist der \emph{Ausgangsgrad} $d^+(v)$ die Anzahl der von $v$ ausgehenden Kanten und der \emph{Eingangsgrad} $d^-(v)$ die Anzahl der in $v$ eingehenden Kanten. Die Knoten, die mit $v$ durch eine Kante verbunden sind werden auch \emph{Nachbarn} von $v$ genannt.
	
\textbf{Wege, Pfade und Kreise.} Ein \emph{Weg} $W_n=v_1v_2\ldots v_n$ mit $n$ Knoten ist ein Graph mit den Knoten $v_1,v_2,\dotsc,v_n$ und den Kanten $v_iv_{i+1}$ für $i=1,2,\dotsc,n-1$. Ein \emph{Pfad} $P_n$ mit $n$ Knoten ist ein Weg, in dem kein Knoten doppelt vorkommt. Ein \emph{Kreis} $C_n$ der Länge $n$ ist ein Pfad $P_{n}$ zusammen mit der Kante $v_nv_1$. Ebenso lassen sich \emph{gerichtete Wege}, \emph{gerichtete Pfade} und \emph{gerichtete Kreise} definieren.
Wenn für einen Graphen $G=(V,E)$ von einem \emph{Weg/Pfad/Kreis in $G$} die Rede ist, so ist damit stets ein Teilgraph von~$G$ gemeint, der ein Weg/Pfad/Kreis ist.
	
\textbf{Zusammenhang.} Ein Graph heißt \emph{zusammenhängend}, wenn es zu je zwei Knoten immer einen Weg gibt, der diese beiden miteinander verbindet. Ein Graph heißt \emph{kreisfrei}, wenn er keinen Kreis als Teilgraph besitzt (der oben abgebildete ungerichtete Graph $G_1$ ist weder zusammenhängend noch kreisfrei).

Jeder Graph~$G$ lässt sich in zusammenhängende Teilgraphen zerlegen. Diese werden \emph{Zusammenhangskomponenten von~$G$} genannt. In vielen Aufgaben könnt ihr jede Zusammenhangskomponente einzeln behandeln und die Aufgabe so auf den Fall reduzieren, dass der betrachtete Graph zusammenhängend ist.
	
\textbf{Bäume.} Ein \emph{Baum} ist ein Graph, der zusammenhängend und kreisfrei ist. Ein Knoten $v$ in einem Baum ist ein \emph{Blatt} falls $d(v)=1$. Bei einem gerichteten Graphen reden wir von einem Baum, wenn der unterliegende ungerichtete Graph ein Baum ist und außerdem $d^-(v)\leqslant 1$ für alle Knoten $v\in V$ gilt. Ein Knoten mit $d^-(v)=0$ heißt \emph{Wurzel}.
\begin{figure}[ht]
	\centering
	\begin{tabularx}{\textwidth}{X c X c X}
		& \begin{tikzpicture}
			\coordinate (a) at (-1.732,2);
			\coordinate (b) at (-1.732,1);
			\coordinate (c) at (-1.732,0);
			\coordinate (d) at (-0.866,1.5);
			\coordinate (e) at (-0.866,0.5);
			\coordinate (f) at (0,2);
			\coordinate (g) at (0,1);
			\coordinate (h) at (0,0);
			\coordinate (i) at (0.866,1.5);
			\coordinate (j) at (0.866,0.5);
			\coordinate (k) at (1.732,2);
			\coordinate (l) at (1.732,1);
			\coordinate (m) at (1.732,0);
			\draw (c) to (e) to (h) to (j) to (m);
			\draw (b) to (e) to (g) to (i) to (l);
			\draw (f) to (i) to (k);
			\draw (a) to (d) to (g); 
			\draw[fill=white] (a) circle (2pt);
			\draw[fill=white] (b) circle (2pt);
			\draw[fill=white] (c) circle (2pt);
			\draw[fill=black] (d) circle (2pt);
			\draw[fill=black] (e) circle (2pt);
			\draw[fill=white] (f) circle (2pt);
			\draw[fill=black] (g) circle (2pt);
			\draw[fill=black] (h) circle (2pt);
			\draw[fill=black] (i) circle (2pt);
			\draw[fill=black] (j) circle (2pt);
			\draw[fill=white] (k) circle (2pt);
			\draw[fill=white] (l) circle (2pt);
			\draw[fill=white] (m) circle (2pt);
		\end{tikzpicture} & & \begin{tikzpicture}
			\coordinate (a) at (-1.732,2);
			\coordinate (b) at (-1.732,1);
			\coordinate (c) at (-1.732,0);
			\coordinate (d) at (-0.866,1.5);
			\coordinate (e) at (-0.866,0.5);
			\coordinate (f) at (0,2);
			\coordinate (g) at (0,1);
			\coordinate (h) at (0,0);
			\coordinate (i) at (0.866,1.5);
			\coordinate (j) at (0.866,0.5);
			\coordinate (k) at (1.732,2);
			\coordinate (l) at (1.732,1);
			\coordinate (m) at (1.732,0);
			\path[decoration={markings, mark=at position 0.5 with {\arrow{>}}}] 
			(i) edge[postaction={decorate}] (g)
			(g) edge[postaction={decorate}] (d)
			(d) edge[postaction={decorate}] (a);
			\path[decoration={markings, mark=at position 0.5 with {\arrow{>}}}] 
			(i) edge[postaction={decorate}] (k);
			\path[decoration={markings, mark=at position 0.5 with {\arrow{>}}}]
			(i) edge[postaction={decorate}] (f);
			\path[decoration={markings, mark=at position 0.5 with {\arrow{>}}}]
			(i) edge[postaction={decorate}] (l);
			\path[decoration={markings, mark=at position 0.5 with {\arrow{>}}}] 
			(g) edge[postaction={decorate}] (e)
			(e) edge[postaction={decorate}] (h)
			(h) edge[postaction={decorate}] (j)
			(j) edge[postaction={decorate}] (m);
			\path[decoration={markings, mark=at position 0.5 with {\arrow{>}}}]
			(e) edge[postaction={decorate}] (b);
			\path[decoration={markings, mark=at position 0.5 with {\arrow{>}}}]
			(e) edge[postaction={decorate}] (c);
			\draw[fill=black] (a) circle (2pt);
			\draw[fill=black] (b) circle (2pt);
			\draw[fill=black] (c) circle (2pt);
			\draw[fill=black] (d) circle (2pt);
			\draw[fill=black] (e) circle (2pt);
			\draw[fill=black] (f) circle (2pt);
			\draw[fill=black] (g) circle (2pt);
			\draw[fill=black] (h) circle (2pt);
			\draw[fill=white] (i) circle (2pt);
			\draw[fill=black] (j) circle (2pt);
			\draw[fill=black] (k) circle (2pt);
			\draw[fill=black] (l) circle (2pt);
			\draw[fill=black] (m) circle (2pt);
		\end{tikzpicture} & \\
		& Ein Baum mit Blättern & & Ein gerichteter Baum mit Wurzel &
	\end{tabularx}
\end{figure}	
	
\textbf{Bipartite Graphen.} Ein Graph $G=(V,E)$ heißt \emph{bipartit}, wenn sich die Knotenmenge von $G$ so in zwei disjunkte Teilmengen $A$ und $B$ zerlegen lässt, dass nur Kanten zwischen $A$ und $B$ verlaufen, aber keine Kanten innerhalb von $A$ oder innerhalb von $B$. Die disjunkte Zerlegung $V=A\cup B$ wird auch \emph{Bipartition} genannt.
\begin{figure}[ht]
	\centering
	\begin{tabularx}{\textwidth}{X c X c X c X}
		& \begin{tikzpicture}
			\draw[fill=black] (0,0) circle (2pt) node[left=0.1em] {$a_1$};
			\draw[fill=black] (0,-1) circle (2pt) node[left=0.1em] {$a_2$};
			\draw[fill=black] (2,0.5) circle (2pt) node[right=0.1em] {$b_1$};
			\draw[fill=black] (2,-0.167) circle (2pt) node[right=0.1em] {$b_2$};
			\draw[fill=black] (2,-0.833) circle (2pt) node[right=0.1em] {$b_3$};
			\draw[fill=black] (2,-1.5) circle (2pt) node[right=0.1em] {$b_4$};
			\draw (0,0) -- (2,-0.833);
			\draw (0,-1) -- (2,-0.167);
			\draw (0,-1) -- (2,0.5);
			\draw (0,-1) -- (2,-0.833);
			\path (0,0) -- (0,-1) node[pos=0.5, left=2.2em] {$A$} node[pos=0.5, below=1.2em, sloped] {$\underbrace{\hspace{1.35cm}}$};
			\path (2,-1.5) -- (2,0.5) node[pos=0.5,right=2.2em] {$B$} node[pos=0.5, below=1.2em, sloped] {$\underbrace{\hspace{2.35cm}}$};
		\end{tikzpicture} & & \begin{tikzpicture}
			\draw[fill=black] (0,0) circle (2pt) node[left=0.1em] {$a_1$};
			\draw[fill=black] (0,-1) circle (2pt) node[left=0.1em] {$a_2$};
			\draw[fill=black] (2,0.5) circle (2pt) node[right=0.1em] {$b_1$};
			\draw[fill=black] (2,-0.167) circle (2pt) node[right=0.1em] {$b_2$};
			\draw[fill=black] (2,-0.833) circle (2pt) node[right=0.1em] {$b_3$};
			\draw[fill=black] (2,-1.5) circle (2pt) node[right=0.1em] {$b_4$};
			\draw (0,0) -- (2,-0.833);
			\draw (0,0) -- (2,0.5);
			\draw (0,0) -- (2,-0.167);
			\draw (0,0) -- (2,-1.5);
			\draw (0,-1) -- (2,-0.833);
			\draw (0,-1) -- (2,0.5);
			\draw (0,-1) -- (2,-0.167);
			\draw (0,-1) -- (2,-1.5);
		\end{tikzpicture} & & \begin{tikzpicture}
		\coordinate (a) at (18:1);
		\coordinate (b) at (90:1);
		\coordinate (c) at (162:1);
		\coordinate (d) at (234:1);
		\coordinate (e) at (306:1);
		\draw (a) to (b) to (c) to (d) to (e) to (a);
		\draw (a) to (c) to (e) to (b) to (d) to (a);
		\draw[fill=black] (a) circle (2pt);
		\draw[fill=black] (b) circle (2pt);
		\draw[fill=black] (c) circle (2pt);
		\draw[fill=black] (d) circle (2pt);
		\draw[fill=black] (e) circle (2pt);
		\end{tikzpicture} &\\
		& bipartiter Graph & & vollständiger bipatiter & & vollständiger Graph $K_5$ & \\
		& & &  Graph $K_{2,4}$ & & &
		\end{tabularx} 
	\end{figure}	
	
\textbf{Vollständige Graphen.} Der \emph{vollständige Graph $K_n$} ist ein Graph mit $n$ Knoten, in dem jeder Knoten mit jedem anderen verbunden ist. Der \emph{vollständige bipartite Graph $K_{m,n}$} ist ein bipartiter Graph mit $\abs{A}=m$, $\abs{B}=n$, in dem jeder Knoten aus $A$ mit jedem Knoten aus $B$ verbunden ist.
	
\textbf{Planare Graphen.} Ein \emph{planarer Graph} ist ein Graph, der so auf ein Blatt Papier gezeichnet kann, dass sich keine zwei Kanten schneiden. Jede Kante darf dabei beliebig krumm sein, aber sie darf keine Lücken enthalten. Zum Beispiel sind alle Bäume planare Graphen. 
\begin{figure}[ht]
	\centering
	\begin{tabularx}{\textwidth}{X c X c X}
		& \begin{tikzpicture}
			\coordinate (a) at (18:1);
			\coordinate (b) at (90:1);
			\coordinate (c) at (162:1);
			\coordinate (d) at (234:1);
			\coordinate (e) at (306:1);
			\draw (c) to[in=130,out=230,loop,min distance=10mm,looseness=10] (c);
			\draw (a) to (b) to (e);
			\draw (e) to (d) to[bend left=20] (b) to[bend left=20] (d); 
			\draw[fill=black] (a) circle (2pt);
			\draw[fill=black] (b) circle (2pt);
			\draw[fill=black] (c) circle (2pt);
			\draw[fill=black] (d) circle (2pt);
			\draw[fill=black] (e) circle (2pt);
		\end{tikzpicture} & & \begin{tikzpicture}
			\coordinate (a) at (0,0);
			\coordinate (b) at (0,2);
			\coordinate (c) at (2,2);
			\coordinate (d) at (2,0);
			\coordinate (e) at (0.5,0.5);
			\coordinate (f) at (0.5,1.5);
			\coordinate (g) at (1.5,1.5);
			\coordinate (h) at (1.5,0.5);
			\draw (a) to (e) to (f) to (b) to (a);
			\draw (c) to (d) to (h) to (g) to (c);
			\draw (a) to (d);
			\draw (b) to (c);
			\draw (f) to (g);
			\draw (e) to (h);
			\draw[fill=black] (a) circle (2pt);
			\draw[fill=black] (b) circle (2pt);
			\draw[fill=black] (c) circle (2pt);
			\draw[fill=black] (d) circle (2pt);
			\draw[fill=black] (e) circle (2pt);
			\draw[fill=black] (f) circle (2pt);
			\draw[fill=black] (g) circle (2pt);
			\draw[fill=black] (h) circle (2pt);
		\end{tikzpicture} &  \\
		& $G_1$ als planarer Graph & & Ein Würfel als planarer Graph &
	\end{tabularx} 
\end{figure}	
	
Wie man in diesem Beispielbild seht, ist der Graph $G_1$ planar, obwohl er oben mit einer Überschneidung gezeichnet wurde.

\subsection*{Beispielaufgaben zur Graphentheorie}
Hier sind einige sehr interessante Resultate aus der Graphentheorie, die auch in Olympiade-Aufgaben weiterhelfen können. In Graphentheorie-Aufgaben kommt es sehr häufig vor, dass mit Induktion über die Menge der Knoten oder Kanten argumentiert wird. Auch Algorithmen kommen sehr prominent zum Einsatz; in Kapitel~\ref{kapitel:Algorithmen}: \emph{Algorithmen in der Kombinatorik} wird auf dieses Thema für allgemeine Aufgaben im Detail eingegangen. Ansonsten braucht ihr natürlich auch die Standardmethoden\footnote{Wenn ihr eine schnelle Wiederholung zu diesen Themen wollt, dann sei euch das Buch \emph{Geometria -- Scientiae Atlantis~1} von Eckard Specht und Robert Strich ans Herz gelegt. Dieses Buch enthält nicht nur, wie der Name nahelegt, jede Menge interessante Geometrie, sondern auch sehr praktische Zusammenfassungen zu anderen olympiaderelevanten Themen. Ihr könnt es relativ kostengünstig hier bestellen: \url{http://hydra.nat.uni-magdeburg.de/math4u/fix/bestellung.html}.} in der Kombinatorik, wie das Schubfachprinzip, das Extremalprinzip und das Invarianzprinzip.
\begin{aufgabe*}\label{aufgabe:Handschlagslemma}
	Beweise das Handschlagslemma. Folgere, dass in jedem Graphen die Anzahl der Knoten von ungeradem Grad gerade sein muss.
	\begin{satzmitnamen}[Handschlagslemma]
		In jedem Graphen $G=(V,E)$ gilt
		\begin{equation*}
			\sum_{v\in V} d(v) = 2\abs{E}\,.
		\end{equation*}
		In jedem gerichteten Graphen $G=(V,E)$ gilt
		\begin{equation*}
			\sum_{v\in V}d^+(v)=\sum_{v\in V}d^-(v)=\abs{E}\,
		\end{equation*}
	\end{satzmitnamen}
\end{aufgabe*}
\begin{aufgabe*}\label{aufgabe:Blatt}
	\begin{enumerate}
		\item \label{teilaufgabe:BaumHatBlaetter}Zeige, dass jeder Baum mit mindestens zwei Knoten mindestens ein Blatt hat.
		\item \label{teilaufgabe:BaumKnotenKanten}Beweise: Für jeden Baum $G=(V,E)$ gilt $\abs{E} = \abs{V} -1$. Folgere, dass jeder Baum mit $\abs*{V}\geqslant 2$ sogar mindestens zwei Blätter hat.
	\end{enumerate}
\end{aufgabe*}
\begin{aufgabe*}\label{aufgabe:Bipartit}
	Beweise, dass ein Graph $G=(V,E)$ genau dann bipartit ist, wenn es in $G$ keine Kreise von ungerader Länge gibt.
\end{aufgabe*}
\begin{aufgabe*}\label{aufgabe:Schlicht}
	Beweise, dass in jedem schlichten Graphen (mit mindestens zwei Knoten) zwei Knoten existieren, die den gleichen Grad haben.
\end{aufgabe*}
\begin{aufgabe*}[*]\label{aufgabe:Euler-Hierholzer}
	Beweise den Satz von Euler-Hierholzer.
	\begin{satzmitnamen}[Satz von Euler-Hierholzer]
		In einem zusammenhängenden Graphen $G=(V,E)$ existiert genau dann ein Weg, der jede Kante genau einmal durchläuft, wenn $G$ höchstens zwei Knoten mit ungeradem Grad besitzt. Ferner existiert genau dann ein geschlossener Weg \embrace{also ein Weg, dessen Anfangs- und Endknoten übereinstimmen}, der alle Kanten durchläuft, wenn alle Knoten in $G$ geraden Grad haben.
	\end{satzmitnamen}
\end{aufgabe*}

Ein solcher Weg wird auch als \emph{Eulerweg} bezeichnet. Die Frage, ob für einen gegebenen Graphen ein Eulerweg existiert, ist auch als \enquote{Königsberger Brückenproblem} bekannt. 

\begin{aufgabe*}[*]\label{aufgabe:Polyeder}
	Beweise den Eulerschen Polyedersatz:
	\begin{satzmitnamen}[Eulerscher Polyedersatz]
		Sei $G$ ein planarer Graph. Sei $Z$ die Menge der Zusammenhangskomponenten von~$G$ und sei $F$ die Menge der Flächen in einer überschneidungsfreien Zeichnung von~$G$ \embrace{die \enquote{unendlich große Fläche außenrum} wird mitgezählt}. Dann gilt stets:
		\begin{equation*}
			\abs{V}-\abs{E}+\abs*{F} = \abs*{Z}+1\,.
		\end{equation*}
	\end{satzmitnamen}
\end{aufgabe*}
\begin{aufgabe*}[*]\label{aufgabe:Unplanar}
	Zeige, dass die vollständigen Graphen $K_5$ und $K_{3,3}$ nicht planar sind.
\end{aufgabe*}
\begin{aufgabe*}[**]\label{aufgabe:Dirac}
	Beweise den Satz von Dirac:
	\begin{satzmitnamen}[Satz von Dirac]
		Sei $G=(V,E)$ ein schlichter Graph mit $n=\abs*{V}$ Knoten. Wenn jeder Knoten $v\in V$ einen Grad $d(v)\geqslant n/2$ hat, dann gibt es in $G$ einen Kreis, der durch jeden Knoten genau einmal geht.
	\end{satzmitnamen}
\end{aufgabe*}
Ein solcher Kreis wird auch~\emph{Hamiltonkreis genannt}. Im Gegensatz zu Eulerwegen, für deren Existenz es ein sehr einfaches hinreichendes und notwendiges Kriterium gibt (wie wir in Aufgabe~\ref{aufgabe:Euler-Hierholzer} gesehen haben), ist es im Allgemeinen sehr kompliziert zu entscheiden, ob ein gegebener Graph~$G$ einen Hamiltonkreis enthält.

\subsection*{Tipps zu den Aufgaben}
\textbf{Tipp zu Aufgabe~\ref{aufgabe:Handschlagslemma}.} Der Grund für den Namen \enquote{Handschlagslemma} ist folgender: Stell dir vor, eine Menge $V$ von sehr wichtigen Leuten trifft sich zu einer sehr wichtigen Konferenz. Einige davon begrüßen einander mit Händeschütteln. Aus Infektionsschutzgründen soll jeder Teilnehmende~$v$ am Ende angeben, wie viele Hände er geschüttelt hat. Außerdem gibt es eine Infektionsschutzbeauftragte, die sehr aufmerksam zählt, wie viele Handschläge stattfinden.

\textbf{Tipps zu Aufgabe~\ref{aufgabe:Blatt}.} Beweise~\ref{teilaufgabe:BaumHatBlaetter} durch Widerspruch.

Für~\ref{teilaufgabe:BaumKnotenKanten} benutze~\ref{teilaufgabe:BaumHatBlaetter} und schneide sukzessive Blätter ab.
	
\textbf{Tipp zu Aufgabe~\ref{aufgabe:Bipartit}.} Überlegen dir zuerst, warum ein Graph mit einem ungeraden Kreis nicht bipartit sein kann. Kannst du dein Argument umdrehen und ein Verfahren angeben, mit dem zu jedem Graphen ohne ungeraden Kreis eine Bipartition konstruiert werden kann?

\textbf{Tipp zu Aufgabe~\ref{aufgabe:Schlicht}.}
Wie groß kann $d(v)$ maximal sein, sodass der Graph schlicht bleibt? Angenommen, es gäbe einen Graphen der schlicht ist und bei dem jeder Knoten einen unterschiedlichen Grad hat. Was kannst du in diesem Fall über die $d(v)$ aussagen?

\textbf{Tipp zu Aufgabe~\ref{aufgabe:Euler-Hierholzer}.}
Der schwierige Teil der Aufgabe besteht darin, zu zeigen, dass in einem Graphen mit lauter geraden Knotengraden auch tatsächlich ein geschlossener Eulerweg existiert.  Benutze dafür Induktion über die Anzahl der Kanten.

\textbf{Tipp zu Aufgabe~\ref{aufgabe:Polyeder}.} Diese Aufgabe lässt sich auf verschiedene Art und Weise mit Induktion lösen. Pass dabei auf, dass dein Argument keine Fälle übersieht.

\textbf{Tipp zu Aufgabe~\ref{aufgabe:Unplanar}.} Benutze den Eulerschen Polyedersatz. Wie viele Kanten haben die Flächen in $K_5$ mindestens? Wie viele in $K_{3,3}$?

\textbf{Tipps zu Aufgabe~\ref{aufgabe:Dirac}.} Wenn es ein Gegenbeispiel gäbe es auch auch ein Gegenbeispiel~$G$ mit maximal vielen Kanten.

Verbinde zwei Knoten in~$G$, die noch nicht verbunden waren. Danach gibt es einen Hamiltonkreis (überlege dir, warum). Benutze das Schubfachprinzip, um diesen Hamiltonkreis zu einem Hamiltonkreis in~$G$ umzubauen.


\newpage
	\section{Algorithmen in der Kombinatorik}\label{kapitel:Algorithmen}
Es gibt eine bestimmte Sorte von Kombinatorik-Aufgaben, die sich nur lösen lassen, indem ihr einen geeigneten Algorithmus hinschreibt. In diesem Kapitel seht ihr vier solche Aufgaben.\footnote{Diese vier Aufgaben wurden (in leicht anderer Formulierung) dem Autoren dieses Textes bei der Bundesrunde in den Klassenstufen 8, 9, 10 und 11 gestellt. Nachdem er keine der Aufgaben in den Klassen 8, 9 und 10 lösen konnte, hat er zehn Minuten vor Abgabe in der Klasse~11 endlich verstanden, wie solche Aufgaben funktionieren. Der vorliegende Text soll dafür sorgen, dass euch diese Erkenntnis möglichst schon früher ereilt (und zwar am besten schon in der Klasse~9).} Wie üblich findet ihr am Ende des Kapitels Tipps und am Ende des Heftes die Lösungen. Ihr solltet euch aber zuerst selbst an diesen Aufgaben versuchen.\footnote{Lasst euch nicht davon beeindrucken, dass Aufgabe~\ref{aufgabe:541143} in der 11.\ Klasse gestellt wurde. Sie ist definitiv nicht die schwerste der vier Aufgaben.} 
\begin{aufgabe*}\label{aufgabe:510846}
	Gegeben seien positive ganze Zahlen $z_1,z_2,\dots,z_m$ und $s_1,s_2,\dotsc,s_n$, für die
	\begin{equation*}
		z_1+z_2+\dotsb+z_m=s_1+s_2+\dotsb+s_n
	\end{equation*} 
	gilt. Gegeben sei weiter eine leere Tabelle mit $m$ Zeilen und $n$ Spalten. In die Tabelle sollen nichtnegative ganze Zahlen derart eingetragen werden, dass für alle $i=1,2,\dotsc,m$ und alle $j=1,2,\dotsc,n$ die Summe der Einträge in der $i$-ten Zeile genau $z_i$ und die Summe der Einträge in der $j$-ten Spalte genau $s_j$ beträgt. Außerdem darf in höchstens $m+n-1$ Feldern eine positive Zahl stehen. Beweise, dass das stets möglich ist.
\end{aufgabe*}
\begin{aufgabe*}[*]\label{aufgabe:520945}
	Arne und Basti spielen ein Spiel. Arne hat \the\year\ Zettel vorbereitet. Auf jedem dieser Zettel steht eine endliche nichtleere Menge von natürlichen Zahlen. Bastis Aufgabe besteht darin, ebenfalls eine endliche nichtleere Menge von natürlichen Zahlen auf einen Zettel zu schreiben, sodass folgende Bedingungen erfüllt sind:
	\begin{enumerate}[label={$(\Alph*)$},ref={$(\Alph*)$}]
		\item Für jeden von Arnes Zetteln steht mindestens eine Zahl von diesem Zettel auch auf Bastis Zettel.\label{bedingung:EineZahlVonJedemZettel}
		\item Es gibt mindestens einen von Arnes Zetteln, dessen kleinste Zahl auch die kleinste Zahl auf Bastis Zettel ist. Zudem soll keine weitere Zahl von diesem Zettel auch auf Bastis Zettel stehen.\label{bedingung:KleinsteZahlVonEinemZettel}
	\end{enumerate}
	Zeige, dass Basti diese Aufgabe stets lösen kann.
\end{aufgabe*}
\begin{aufgabe*}[*]\label{aufgabe:531046}
	Gegeben seien positive ganze Zahlen $0<a<b$. Eine $3$-elementige Menge $\{\ell,m,n\}$ von positiven ganzen Zahlen heißt \emph{$ab$-normal}, wenn die \embrace{positiven} Abstände dieser drei Zahlen genau $a$, $b$ und $a+b$ betragen. Die Reihenfolge ist dabei egal. Ist es stets möglich, die Menge $\mathbb Z_{>0}$ aller positiven ganzen Zahlen in disjunkte $ab$-normale Teilmengen zu zerlegen?
\end{aufgabe*}
\begin{aufgabe*}[*]\label{aufgabe:541143}
	Bei einem Mathematik-Wettbewerb kennen sich einige der Teilnehmenden bereits (Bekanntschaft ist immer gegenseitig). Insgesamt gibt es dabei $k$ Bekanntschaften. Für einen Ausflug sollen die Teilnehmenden auf zwei Busse aufgeteilt werden. Damit nicht immer die gleichen Gruppen miteinander rumhängen, soll in beiden Bussen die Anzahl der Bekanntschaften möglichst gering sein. Zeige, dass es stets eine Aufteilung gibt, für die die Anzahl der Bekanntschaften in jedem der beiden Busse maximal $\frac{k}{3}$ beträgt.
\end{aufgabe*}
Bevor wir zu den konkreten Tipps kommen, sammeln wir einige allgemeine Strategien, wie ihr an solche Aufgaben herangehen könnt.
\begin{itemize}
	\item Traut euch und probiert Dinge aus! Schreibt einen Algorithmus hin und schaut, ob er funktioniert. Wenn nicht, überlegt euch, an welcher Stelle er scheitert, und versucht, euren Algorithmus Schritt für Schritt zu verbessern, bis er die Aufgabe löst.
	\item Tut das \enquote{Offensichtliche}. Algorithmen in Olympiade-Aufgaben sind meistens nicht besonders kompliziert. Ihr könnt davon ausgehen, dass der gesuchte Algorithmus die gegebene Situation in jedem Schritt verbessert und sich damit Schritt für Schritt der gewünschten Situation nähert.
	\item Benutze einen \enquote{Greedy-Algorithmus}. Ein Greedy-Algorithmus nimmt sich einfach in jedem Schritt so viel wie möglich. In der Praxis finden solche Algorithmen oftmals nicht eine optimale Lösung, aber in den Problemen, die euch in Olympiade-Aufgaben begegnen, sind sie häufig ausreichend.
\end{itemize}
Genauso wichtig wie das Finden einer Lösung ist aber auch, dass ihr eure Lösung sauber aufschreiben könnt. Gerade Lösungen mit Algorithmen tendieren dazu, konfus und unverständlich zu sein, wenn sie schlecht aufgeschrieben sind. Damit macht ihr dem Korrekturteam das Leben schwer (und das Korrekturteam revanchiert sich natürlich mit Punktabzügen). Deshalb:
\begin{itemize}
	\item Erwähnt am Anfang eurer Lösung, dass ihr die Aufgabe mit einem Algorithmus lösen werdet.
	\item Formuliert euren Algorithmus so klar wie möglich. Wenn es nicht anders geht, könnt ihr sogar Pseudocode verwenden.
	\item Ihr müsst zeigen, dass jeder Schritt eures Algorithmus durchführbar ist, dass euer Algorithmus nach endlich vielen Schritten endet und dass euer Algorithmus ein korrektes Ergebnis liefert. Am besten strukturiert ihr euren Aufschrieb so, dass diese drei Schritte, also Durchführbarkeit, Endlichkeit und Korrektheit, klar voneinander getrennt sind.
\end{itemize}
Allgemein lohnt es sich, darüber nachzudenken, ob ihr eure algorithmische Lösung nicht zu einer Lösung mit dem Extremalprinzip umformulieren könnt. Das ist nicht immer möglich, aber recht häufig, und es verlangt etwas Übung. Dafür sind solche Lösungen leichter aufzuschreiben (und leichter zu korrigieren) als algorithmische Lösungen. In den Musterlösungen seht ihr ein Beispiel, wie sich eine algorithmische Lösung in eine Extremalprinzips-Lösung umformulieren lässt. Um auf die Lösung zu kommen, könnt (und solltet) ihr natürlich nach wie vor über Algorithmen nachdenken.

\vfill\hrule\vspace{-1em}

\subsection*{Tipps zu den Beispielaufgaben}
\textbf{Tipp zu Aufgabe~\ref{aufgabe:510846}.} Fülle die Tabelle Feld für Feld aus.

\textbf{Tipps zu Aufgabe~\ref{aufgabe:520945}.} Beginne damit, das Minimum aller Zahlen auf Arnes Zetteln auf Bastis Zettel zu schreiben. In welchem Fall geht das schief?

Zeige, dass Basti zuerst alle Zettel beiseite legen kann, die einen anderen Zettel komplett enthalten. Kannst du danach die Aufgabe lösen?

\textbf{Tipp zu Aufgabe~\ref{aufgabe:531046}.} Füge schrittweise Tripel der Form $\{n,n+a,n+a+b\}$ hinzu. Wenn das nicht geht, füge ein Tripel der Form $\{n,n+b,n+a+b\}$ hinzu.

\textbf{Tipp zu Aufgabe~\ref{aufgabe:541143}.} Starte mit einer beliebigen Verteilung und verbessere sie Schritt für Schritt auf die \enquote{offensichtliche} Weise.
\newpage
	
	\phantomsection\cftaddtitleline{toc}{part}{Zahlentheorie}{\thepage}
	\section{Teiler und Teilerfremdheit}\label{kapitel:Teilerfremdheit}
In diesem Theoriekapitel werden wir die Grundlagen der für Olympiadezwecke relevanten Zahlentheorie besprechen, damit wir sie in späteren Kapiteln frei verwenden können. Euch wird sicherlich vieles (wenn nicht gar alles) in diesem Kapitel bereits bekannt sein.

\subsection*{Der größte gemeinsame Teiler}
\begin{definition}
	Seien $a$ und $b$ ganze Zahlen, die nicht beide Null sind. Der \emph{größte gemeinsame Teiler von $a$ und $b$}, kurz $\operatorname{ggT}(a,b)$, ist genau das, was der Name sagt: nämlich die größte positive ganze Zahl, die sowohl~$a$ als auch~$b$ teilt.
\end{definition}
Wenn die Primfaktorzerlegungen\footnote{Indem wir auch $\alpha_i=0$ oder $\beta_j=0$ erlauben, können wir erreichen, dass in den Primfaktorzerlegungen von~$a$ und~$b$ die gleichen Primfaktoren vorkommen, wie es die Notation suggeriert.} $a=\pm p_1^{\alpha_1}p_2^{\alpha_2}\dotsm p_r^{\alpha_r}$ und $b=\pm p_1^{\beta_1}p_2^{\beta_2}\dotsm p_r^{\beta_r}$ bekannt sind, dann lässt sich der $\operatorname{ggT}$ leicht ablesen: Es gilt
\begin{equation*}
	\operatorname{ggT}(a,b)=p_1^{\min\{\alpha_1,\beta_1\}}p_2^{\min\{\alpha_2,\beta_2\}}\dotsm p_r^{\min\{\alpha_r,\beta_r\}}\,.
\end{equation*}
Aber der $\operatorname{ggT}$ lässt sich auch ohne Kenntnis der Primfaktorzerlegungen bestimmen. Das geht folgendermaßen:
\begin{satzmitnamen}[Euklidischer Algorithmus]
	Gegeben seien nichtnegative ganze Zahlen $a$ und $b$, die nicht beide gleich Null sind. Dann wird $\operatorname{ggT}(a,b)$ durch folgenden rekursiven Algorithmus berechnet:
	\begin{enumerate}
		\item[$(*)$] Sei $r$ der Rest von $a$ modulo $b$. Wenn $r=0$, dann ist $\operatorname{ggT}(a,b)=b$. Ansonsten ersetze $(a,b)$ durch $(b,r)$ und wiederhole den gleichen Schritt.
	\end{enumerate}
\end{satzmitnamen}
\begin{proof}
	Sei $(a_i,b_i)$ das Paar von ganzen Zahlen, das im $i$-ten Schritt betrachtet wird. Dann ist also $(a_1,b_1)=(a,b)$; ferner gilt $a_{i+1}=b_i$ und $b_{i+1}$ ist der Rest von $a_i$ modulo $b_i$. Insbesondere ist $b_1>b_2>\dotsb>b_i>\dotsb$. Nach endlich vielen Schritten muss also $b_{n+1}=0$ gelten und der Algorithmus endet.
	
	Weil $b_{i+1}$ der Rest von $a_i$ modulo $b_i$ ist, gibt es eine nichtnegative ganze Zahl $q$ mit $b_{i+1}=a_i-qb_i$. Jeder gemeinsame Teiler von $a_i$ und $b_i$ ist somit auch ein Teiler von $b_{i+1}$ und natürlich auch von $a_{i+1}=b_i$. Umgekehrt ist $a_i=b_{i+1}+qb_i=b_{i+1}+qa_{i+1}$, also ist jeder gemeinsame Teiler von $a_{i+1}$ und $b_{i+1}$ auch ein gemeinsamer Teiler von $a_i$ und $b_i$. Es folgt $\operatorname{ggT}(a_i,b_i)=\operatorname{ggT}(a_{i+1},b_{i+1})$ und damit
	\begin{equation*}
		\operatorname{ggT}(a,b)=\operatorname{ggT}(a_1,b_1)=\operatorname{ggT}(a_2,b_2)=\dotsb=\operatorname{ggT}(a_n,b_n)\,.
	\end{equation*}
	Wegen $b_{n+1}=0$ ist~$a_n$ durch~$b_n$ teilbar somit $\operatorname{ggT}(a_n,b_n)=b_n$. Nach Konstruktion endet der Algorithmus im $n$-ten Schritt und gibt~$b_n$ aus. Also ist der Algorithmus korrekt.
\end{proof}

Eine weitere, konzeptuell wichtige Beschreibung des $\operatorname{ggT}$ ist folgendermaßen gegeben:
\begin{satzmitnamen}[Lemma von Bézout]
	Gegeben seien ganze Zahlen $a$ und $b$, die nicht beide Null sind. Dann ist der $\operatorname{ggT}$ von $a$ und $b$ die kleinste positive ganze Zahl $d$, die sich in der Form $d=ma+n b$ für ganze Zahlen $m$ und $n$ schreiben lässt.
\end{satzmitnamen}
\begin{proof}
	Jede Zahl der Form $ma+nb$ ist offensichtlich durch $\operatorname{ggT}(a,b)$ teilbar. Wenn $ma+nb$ also eine positive ganze Zahl ist, dann folgt sofort $ma+nb\geqslant \operatorname{ggT}(a,b)$. Wir müssen somit nur zeigen, dass sich auch $\operatorname{ggT}(a,b)$ in dieser Form schreiben lässt. Indem wir gegebenenfalls $a$ oder $b$ durch $-a$ oder $-b$ ersetzen, dürfen wir annehmen, dass $a$ und $b$ nichtnegativ sind, sodass der Euklidische Algorithmus anwendbar ist. Dann zeigen wir, dass sich alle $a_i$ und $b_i$ im Euklidischen Algorithmus in dieser Form schreiben lassen. Dafür benutzen wir Induktion nach $i$. Der Induktionsanfang $i=1$ ist trivial: Es gilt $(a_1,b_1)=(a,b)$ und $a=1\cdot a+0\cdot b$ sowie $b=0\cdot a+1\cdot b$.
	
	Nun nehmen wir an, dass wir bereits ganze Zahlen $m_i$, $n_i$, $k_i$ und $\ell_i$ gefunden haben, für die $a_i=m_i a+n_i b$ und $b_i=k_ia+\ell_ib$ gilt. Nach Konstruktion gilt $a_{i+1}=b_i=k_ia+\ell_ib$. Ferner ist $b_{i+1}$ der Rest von $a_i$ modulo $b_i$, folglich gibt es eine nichtnegative ganze Zahl $q$ mit $b_{i+1}=a_i-qb_i$. Es folgt
	\begin{equation*}
		b_{i+1}=\parens*{m_ia+n_ib}-q\parens*{k_i a+\ell_i b}=\parens*{m_i-qk_i}a+\parens*{n_i-q\ell_i}b\,.
	\end{equation*}
	Somit lässt sich auch $b_{i+1}$ in der gewünschten Form darstellen. Das beendet den Induktionsschritt, die Induktion und den Beweis.
\end{proof}

\begin{definition}
	Wir nennen $a$ und $b$ \emph{teilerfremd}, wenn $\operatorname{ggT}(a,b)=1$.
\end{definition}
Das Lemma von Bézout hat eine sehr interessante Konsequenz in dem Fall, dass $a$ und $b$ teilerfremd sind: Wir können modulo $b$ nicht nur addieren, subtrahieren und multiplizieren, sondern auch \emph{durch $a$ dividieren!} Um das einzusehen, wählen wir zunächst ganze Zahlen $m$ und $n$ mit $ma+nb=1$. Insbesondere gilt $ma\equiv 1\mod b$. Wir sagen dann, \emph{die Restklasse von $m$ modulo $b$ ist das multiplikative Inverse der Restklasse von $a$ modulo $b$}. Division durch $a$ modulo $b$ können wir schließlich als Multiplikation mit $m$, dem multiplikativen Inversen von $a$, definieren.

Ein besonderer Spezialfall ergibt sich, wenn $b=p$ eine Primzahl ist. In diesem Fall ist $a$ genau dann teilerfremd zu $p$, wenn $a$ nicht durch $p$ teilbar ist. Also können wir modulo $p$ durch $a$ dividieren, solange $a\not\equiv 0\mod p$ gilt. Mit anderen Worten: \emph{Wir können modulo $p$ durch jede Restklasse außer durch Null dividieren!} Auf diese Weise verhalten sich die Restklassen modulo $p$ genau wie die rationalen Zahlen oder die reellen Zahlen, in welchen die Division, außer durch Null, ebenfalls immer erlaubt ist.

Division modulo einer Primzahl ist bestimmt erstmal ungewohnt für euch. Um damit ein bisschen warm zu werden, wollen wir untersuchen, was \enquote{$\frac 32$ modulo 7} ist. Wir stellen uns also die Frage: \emph{Was passiert, wenn wir~3 modulo~7 durch~2 dividieren?} Wenn~$x$ das Ergebnis ist, dann sollte $x$ die Kongruenz $2x\equiv 3\mod 7$ erfüllen. Wegen $2\cdot 4\equiv 8\equiv 1\mod 7$ ist~$4$ das multiplikative Inverse von~$2$ modulo~$7$. Also ist $x\equiv 8x\equiv 4\cdot 2x\equiv 4\cdot 3\equiv 5\mod 7$. Und damit haben wir unsere Antwort: \emph{Wenn wir~3 modulo~7 durch~2 dividieren, erhalten wir~5.}


\subsection*{Der Chinesische Restsatz}
Der Chinesische Restsatz ist auf den ersten Blick eine sehr technische Aussage, die aber trotzdem in vielen Olympiade-Aufgaben eine wichtige Rolle spielt.
\begin{satzmitnamen}[Chinesischer Restsatz/Satz von Sun Zi]
	Gegeben seien paarweise teilerfremde positive ganze Zahlen $m_1,m_2,\dotsc,m_n$ sowie Restklassen $a_1$ modulo $m_1$, $a_2$ modulo $m_2$, \ldots, $a_n$ modulo $m_n$. Dann hat das System von Kongruenzen
	\begin{equation*}
		\left\{\begin{aligned}
			x&\equiv a_1\mod m_1\,,\\
			x&\equiv a_2\mod m_2\,,\\
			&\mathrel{\tikz[inner sep=0,outer sep=0]{\node at (0,-0.5ex) {$\phantom{\equiv}$};\node at (0,0) {$\vdots$};}}\\
			x&\equiv a_n\mod m_n
		\end{aligned}\right.
	\end{equation*}
	genau eine Lösung $x$ modulo $m_1m_2\dotsm m_n$.
\end{satzmitnamen}
\begin{proof}
	Wir zeigen zuerst, dass für jedes $j=1,2,\dotsc,n$ eine ganze Zahl $x_j$ existiert, für die $x_j\equiv 1\mod m_j$ und $x_j\equiv 0\mod m_i$ für alle $i\neq j$ gilt. Dazu sei $m$ das Produkt aller $m_i$ für $i\neq j$. Weil $m_1,m_2,\dotsc,m_n$ paarweise teilerfremd sind, sind auch $m_j$ und $m$ teilerfremd. Nach dem Lemma von Bézout existiert also eine ganze Zahl~$y$ mit $my\equiv 1\mod m_j$ (sodass die Restklasse von~$y$ das multiplikative Inverse zur Restklasse von $m$ modulo $m_j$ ist). Somit ist $x_j\coloneqq my$ eine Lösung von $x_j\equiv 1\mod m_j$ und $x_j\equiv 0\mod m_i$ für alle $i\neq j$.
	
	Nun sehen wir, dass $x\coloneqq a_1x_1+a_2x_2+\dotsb+a_nx_n$ das obige System von Kongruenzen löst. Somit existiert immer eine Lösung und wir müssen nur noch zeigen, dass diese Lösung eindeutig modulo $m_1m_2\dotsm m_n$ ist. Das folgt aus einem einfachen Abzählargument: Wenn wir $a_1$ modulo~$m_1$, $a_2$ modulo~$m_2$, \ldots, $a_n$ modulo~$m_n$ variieren, erhalten wir $m_1m_2\dotsm m_n$ verschiedene Systeme von Kongruenzen. Wie wir gerade gesehen haben, hat jedes dieser Systeme mindestens eine Lösung modulo $m_1m_2\dotsm m_n$. Es gibt aber nur $m_1m_2\dotsm m_n$ Restklassen modulo $m_1m_2\dotsm m_n$. Also muss jedes System genau eine Lösung haben.
\end{proof}

Eine typische Anwendung des Chinesischen Restsatzes ist die folgende Olympiadeaufgabe.
\begin{aufgabe*}\leavevmode\label{aufgabe:ChinesischerRestsatz}
	\begin{enumerate}[label={$(\alph*)$},ref={$(\alph*)$}]
		\item Sei $n$ eine vorgegebene positive ganze Zahl. Zeige, dass eine positive ganze Zahl $N$ existiert, sodass keine der Zahlen $N+1,N+2,\dotsc,N+n$ eine Primpotenz ist.\label{teilaufgabe:Primpotenzen}
		\item Sei $n$ eine vorgegebene positive ganze Zahl. Zeige, dass eine positive ganze Zahl $N$ existiert, sodass keine der Zahlen $N+1,N+2,\dotsc,N+n$ eine echte Potenz ist. (\emph{Eine echte Potenz ist eine ganze Zahl der Form $m^k$, wobei $m$ und $k$ ganze Zahlen sind und $k\geqslant 2$ gilt.})\label{teilaufgabe:EchtePotenzen}
	\end{enumerate}
\end{aufgabe*}
\begin{proof}[Lösung]
	Wir beginnen mit~\ref{teilaufgabe:Primpotenzen}. Um ein $N$ mit der gewünschten Eigenschaft zu konstruieren, stellen wir uns zunächst folgende Frage: \emph{Wie können wir verhindern, dass eine ganze Zahl eine Primpotenz ist?} Die Antwort ist simpel: \emph{Indem wir dafür sorgen, dass die Zahl durch mindestens zwei Primzahlen teilbar ist!} Wir wollen also ein $N$ konstruieren, sodass jede der Zahlen $N+1,N+2,\dotsc,N+n$ durch mindestens zwei Primzahlen teilbar ist. Dafür wählen wir paarweise verschiedene Primzahlen $p_1,p_2,\dotsc,p_{2n}$. Nach dem Chinesischen Restsatz hat das System
	\begin{equation*}
		\left\{\begin{aligned}
			N&\equiv -1\mod p_1p_2\,,\\
			N&\equiv -2\mod p_3p_4\,,\\
			&\mathrel{\tikz[inner sep=0,outer sep=0]{\node at (0,-0.5ex) {$\phantom{=}$};\node at (0,0) {$\vdots$};}}\\
			N&\equiv -n\mod p_{2n-1}p_{2n}
		\end{aligned}\right.
	\end{equation*}
	eine Lösung. Für alle $i=1,2,\dotsc,n$ ist dann $N+i$ durch $p_{2i-1}p_{2i}$ teilbar und wir sind fertig.
	
	Für~\ref{teilaufgabe:EchtePotenzen} stellen wir uns analog zu~\ref{teilaufgabe:Primpotenzen} die Frage: \emph{Wie können wir verhindern, dass eine ganze Zahl eine echte Potenz ist?} Diese Frage ist nicht so offensichtlich zu beantworten, aber nach einigem Nachdenken fällt uns folgendes auf: Wenn eine echte Potenz $m^k$ durch eine Primzahl $p$ teilbar ist, dann ist $m^k$ auch durch $p^k$ teilbar, also auf jeden Fall durch $p^2$. Insbesondere kann es nicht passieren, dass $m^k\equiv p\mod p^2$ gilt. Daraus können wir eine ähnliche Konstruktion wie in~\ref{teilaufgabe:Primpotenzen} basteln: Wähle paarweise verschiedene Primzahlen $p_1,p_2,\dotsc,p_n$. Nach dem Chinesischen Restsatz hat das System
	\begin{equation*}
		\left\{\begin{aligned}
			N&\equiv p_1-1\mod p_1^2\,,\\
			N&\equiv p_2-2\mod p_2^2\,,\\
			&\mathrel{\tikz[inner sep=0,outer sep=0]{\node at (0,-0.5ex) {$\phantom{=}$};\node at (0,0) {$\vdots$};}}\\
			N&\equiv p_n-n\mod p_n^2
		\end{aligned}\right.
	\end{equation*}
	eine Lösung. Für alle $i=1,2,\dotsc,n$ ist dann $N+i\equiv p_i\mod p_i^2$, sodass $N+i$ keine echte Potenz sein kann. Damit sind wir fertig.
\end{proof}

\subsection*{Die Eulersche $\boldsymbol{\varphi}$-Funktion}
\begin{definition}
	Für jede positive ganze Zahl $n$ sei $\varphi(n)$ die Anzahl der positiven ganzen Zahlen $1\leqslant m\leqslant n$, die zu $n$ teilerfremd sind. Die Funktion $\varphi\colon \mathbb Z_{>0}\rightarrow \mathbb Z_{>0}$ wird \emph{Eulersche $\varphi$-Funktion} genannt.
\end{definition}
Im Fall $n=1$ gilt zum Beispiel $\varphi(1)=1$, denn $1$ ist zu sich selbst teilerfremd (auch wenn es durch sich selbst teilbar ist). Wenn $n=p$ eine Primzahl ist, dann sind alle kleineren positiven ganzen Zahlen zu $p$ teilerfremd, sodass $\varphi(p)=p-1$ ist. Wenn $n=p^\alpha$ eine Primpotenz ist, dann gilt $\varphi(p^\alpha)=p^\alpha-p^{\alpha-1}=(p-1)p^{\alpha-1}$. Eine positive ganze Zahl $1\leqslant m\leqslant p^\alpha$ ist nämlich genau dann \emph{nicht} zu $p^\alpha$ teilerfremd, wenn $m$ durch $p$ teilbar ist, und es gibt genau $p^{\alpha-1}$ durch $p$ teilbare positive ganze Zahlen $\leqslant p^\alpha$.

Durch ähnliche Überlegungen können wir eine allgemeine Formel für $\varphi(n)$ angeben. Seien $p_1,p_2,\dotsc,p_r$ die Primfaktoren von $n$. Eine positive ganze Zahl $1\leqslant m\leqslant n$ ist genau dann \emph{nicht} zu $n$ teilerfremd, wenn $k$ durch eine der Primzahlen $p_1,p_2,\dotsc,p_r$ teilbar ist. Es gibt genau $\frac{n}{p_i}$ durch $p_i$ teilbare Wahlen von $m$. Also müssen wir $\frac{n}{p_1}+\frac{n}{p_2}+\dotsb+\frac{n}{p_r}$ von $n$ subtrahieren. Dabei haben wir aber alle $m$ doppelt gezählt, die durch mindestens zwei der Primzahlen $p_1,p_2,\dotsc,p_r$ teilbar sind. Wir müssen also $\frac{n}{p_ip_j}$ für alle $i<j$ addieren. Dabei haben wir aber wiederum alle $m$ doppelt gezählt, die durch mindestens drei der Primzahlen $p_1,p_2,\dotsc,p_r$ teilbar sind. Also müssen wir $\frac{n}{p_ip_jp_k}$ für alle $i<j<k$ subtrahieren und so weiter. Es folgt
\begin{align*}
	\varphi(n)&=n+\sum_{\ell=1}^r\sum_{1\leqslant i_1<\dotsb<i_\ell\leqslant r}(-1)^\ell\frac{n}{p_{i_1}\dotsm p_{i_\ell}}\\
	&=n\parens*{1-\frac 1{p_1}}\parens*{1-\frac 1{p_2}}\dotsm \parens*{1-\frac 1{p_r}}\,.
\end{align*}
\begin{satzmitnamen}[Lemma]
	Wenn $m$ und $n$ teilerfremde positive ganze Zahlen sind, dann ist $\varphi(mn)=\varphi(m)\varphi(n)$.
\end{satzmitnamen}
\begin{proof}[Erster Beweis]
	Das folgt direkt aus der obigen Formel.
\end{proof}

Umgekehrt lässt sich das Lemma benutzen, um die obige Formel auf den Fall von Primpotenzen zurückzuführen. Damit würden wir einen weiteren Beweis der obigen Formel erhalten, vorausgesetzt, wir könnten das Lemma ohne die Formel beweisen. Einen solchen Beweis werden wir nun vorstellen.

\begin{proof}[Zweiter Beweis]
	Die Gleichung $\varphi(mn)=\varphi(m)\varphi(n)$ ist trivial wenn $m=1$ oder $n=1$. Für $m>1$ ist $m$ nicht teilerfremd zu $m$. Also ist $\varphi(m)$ auch die Anzahl aller Restklassen modulo $m$, die teilerfremd zu $m$ sind. Selbiges gilt für $\varphi(n)$. Eine Restklasse $r$ modulo $mn$ ist genau dann teilerfremd zu $mn$, wenn $r$ teilerfremd zu $m$ und $n$ ist. Wenn $s$ eine teilerfremde Restklasse modulo $m$ und $t$ eine teilerfremde Restklasse modulo $n$ ist, dann hat das System von Kongruenzen
	\begin{equation*}
		\left\{\begin{alignedat}{2}
			r&\equiv s&&\mod m\,,\\
			r&\equiv t&&\mod n
		\end{alignedat}\right.
	\end{equation*}
	laut dem Chinesischen Restsatz genau eine Lösung modulo $mn$. Weil es $\varphi(m)\varphi(n)$ Wahlen für~$s$ und~$t$ gibt, gibt es $\varphi(m)\varphi(n)$ Restklassen modulo $mn$, die zu $mn$ teilerfremd sind. Es folgt $\varphi(mn)=\varphi(m)\varphi(n)$, wie behauptet.
\end{proof}

Die wichtigste Anwendung der Eulerschen $\varphi$-Funktion im Kontext von Olympiade-Mathematik ist der \emph{Satz von Euler-Fermat:}
\begin{satzmitnamen}[Satz von Euler-Fermat]
	Gegeben sei eine positive ganze Zahl $m$ sowie eine ganze Zahl $a$, die zu $m$ teilerfremd ist. Dann gilt
	\begin{equation*}
		a^{\varphi(m)}\equiv 1\mod m\,.
	\end{equation*}
\end{satzmitnamen}
In dem Spezialfall, dass $m=p$ eine Primzahl ist, erhalten wir $a^{p-1}\equiv 1\mod p$ für alle ganzen Zahlen $a$, die nicht durch $p$ teilbar sind. Diese Kongruenz ist auch als der \emph{kleine Satz von Fermat} bekannt.

\begin{proof}
	Für $m=1$ ist der Satz trivial. Für $m>1$ ist, wie wir weiter oben festgestellt haben, $\varphi(m)$ die Anzahl der Restklassen modulo $m$, die teilerfremd zu $m$ sind. Seien $r_1,r_2,\dotsc,r_{\varphi(m)}$ diese Restklassen. Dann sind auch $ar_1,ar_2,\dotsc,ar_{\varphi(m)}$ zu $m$ teilerfremde Restklassen modulo $m$. Außerdem sind $ar_1,ar_2,\dotsc,ar_{\varphi(m)}$ paarweise verschieden. Aus $ar_i\equiv ar_j\mod m$ folgt nämlich $r_i\equiv r_j\mod m$, denn wenn $a$ teilerfremd zu $m$ ist, dann können modulo $m$ durch $a$ dividieren, wie wir am Anfang des Kapitels gesehen haben. Wenn aber $ar_1,ar_2,\dotsc,ar_{\varphi(m)}$ paarweise verschiedene zu $m$ teilerfremde Restklassen modulo $m$ sind, dann muss $ar_1,ar_2,\dotsc,ar_{\varphi(m)}$ eine Permutation von $r_1,r_2,\dotsc,r_{\varphi(m)}$ sein. Es folgt
	\begin{equation*}
		r_1r_2\dotsb r_{\varphi(m)}\equiv ar_1\cdot ar_2\dotsm ar_{\varphi(m)}\equiv a^{\varphi(m)}r_1r_2\dotsm r_m\mod m\,.
	\end{equation*}
	Weil wir modulo $m$ durch die zu $m$ teilerfremde Restklasse $r_1r_2\dotsm r_{\varphi(m)}$ teilen dürfen, erhalten wir $1\equiv a^{\varphi(m)}\mod m$, wie behauptet.
\end{proof}

Im Kapitel zu Diophantischen Gleichungen werdet ihr sehen, wie der Satz von Euler-Fermat ein unverzichtbares Hilfsmittel bei solchen Gleichungen ist. Eine weitere typische Anwendung des Satzes sind Aufgaben, bei denen ihr zeigen sollt, dass eine natürliche Zahl $n$ ein Vielfaches hat, dessen Dezimaldarstellung eine bestimmte Form hat. Wenn $n$ teilerfremd zu $10$ ist, dann gilt $10^{k\varphi(n)}\equiv 1\mod n$ für alle nichtnegativen ganzen Zahlen $k$. Durch geeignete Summen von Zahlen dieser Form kann dann häufig ein Vielfaches von $n$ mit den gewünschten Eigenschaften gebastelt werden. Die folgende Aufgabe ist dafür ein perfektes Beispiel:
\begin{aufgabe*}
	Sei $n$ eine positive ganze Zahl, die nicht durch $10$ teilbar ist. Zeige, dass $n$ ein Vielfaches hat, das eine Palindromzahl ist. (\emph{Eine Palindromzahl ist eine Zahl, deren Dezimaldarstellung vorwärts und rückwärts gelesen gleich ist.})
\end{aufgabe*}
\begin{proof}[Lösung]
	Betrachte zuerst den Fall, dass $n$ weder durch $2$ noch durch $5$ teilbar ist. Dann muss $n$ teilerfremd zu $10$ sein. Betrachte die Zahl $A\coloneqq 1+10^{\varphi(n)}+10^{2\varphi(n)}+\dotsb+10^{(n-1)\varphi(n)}$. Die Dezimaldarstellung von $A$ besteht aus $n$ Einsen, zwischen denen jeweils $\varphi(n)-1$ Nullen liegen. Also ist $A$ ein Palindrom. Andererseits gilt nach dem Satz von Euler-Fermat
	\begin{equation*}
		A\equiv \underbrace{1+1+\dotsb+1}_{\text{$n$ Summanden}}\equiv n\equiv 0\mod n\,.
	\end{equation*}
	Folglich ist $A$ durch $n$ teilbar.
	
	Betrachte als nächstes den Fall, dass $n$ durch $2$ teilbar ist. Dann kann $n$ nicht durch $5$ teilbar sein. Also ist $n$ von der Form $n=2^k m$, wobei $m$ zu $10$ teilerfremd ist. Sei $B\coloneqq 2^k$ und sei $\overline{B}$ die Zahl, die wir erhalten, wenn wir die Dezimaldarstellung von $B$ umdrehen. Betrachte nun zuerst die Zahl $B+10^{k\varphi(n)}$. Wegen $10^{k\varphi(n)}>2^k=B$ besteht die Dezimaldarstellung von $B+10^{k\varphi(n)}$, gelesen von links nach rechts, aus einer $1$, gefolgt von einer Reihe Nullen und schließlich der Dezimaldarstellung von $B$. Sei $a$ die Anzahl der Nullen. Betrachte nun die Zahl
	\begin{equation*}
		C\coloneqq B+10^{k\varphi(m)}+10^{2k\varphi(m)}+\dotsb+10^{(2m-1)k\varphi(m)}+10^{(2m-1)k\varphi(m)+a+1}\overline{B}\,.
	\end{equation*}
	Die Dezimaldarstellung von $C$, gelesen von links nach rechts, besteht zunächst aus der Dezimaldarstellung von $\overline{B}$, gefolgt von $a$ Nullen. Danach folgen $2m-1$ Einsen, die jeweils durch $k\varphi(m)-1$ Nullen getrennt sind. Zuletzt folgen $a$ Nullen und die Dezimaldarstellung von $B$. Damit ist $C$ eine Palindromzahl. Außerdem ist $C$ durch~$2^k$ teilbar, denn alle Summanden sind durch $2^k$ teilbar. Allerdings muss $C$ nicht unbedingt durch $m$ teilbar sein. Sei~$r$ der Rest von $C$ modulo~$m$. Um $C$ durch $m$ teilbar zu machen, wollen wir~$r$ der $2m-1$ Einsen durch Nullen ersetzen. Wegen $10^{ik\varphi(m)}\equiv 1\mod m$ verringert sich der Rest bei jeder Ersetzung um~$1$, sodass wir zum Schluss eine durch $m$ teilbare Zahl $C'$ erhalten. Wenn~$r$ gerade ist, wählen wir die~$r$ zu ersetzenden Einsen symmetrisch zur mittleren der $2m-1$ Einsen. Wenn~$r$ ungerade ist, wählen wir die mittlere Eins und den Rest wieder symmetrisch. Damit können wir sicherstellen, dass $C'$ immer noch eine Palindromzahl ist. Außerdem ist $C'$ immer noch durch $2^k$ teilbar, denn alle Summanden, die wir gelöscht haben indem wir eine $1$ zu einer~$0$ gemacht haben, waren durch~$2^k$ teilbar. Folglich ist $C$ durch $2^km=n$ teilbar und wir sind fertig.
	
	Der Fall, dass $n$ durch~$5$ teilbar ist, geht völlig analog.
\end{proof}\newpage
	\section{Diophantische Gleichungen}\label{kapitel:Diophantastisch}

In vielen Olympiadeaufgaben ist nach allen ganzzahligen Lösungen einer Gleichung (meistens in mehreren Variablen) gefragt. Solche Gleichungen werden \emph{Diophantische Gleichungen} genannt. Für diophantische Gleichungen gibt es (beweisbar) keine Lösungsmethoden, die immer zum Ziel führen, oder gar Lösungsformeln. Trotzdem gibt es einige Tricks und Strategien, mit denen sich die meisten dieser Gleichungen, denen ihr in Olympiaden begegnen werdet, lösen lassen.

Wir werden nun die wichtigsten dieser Strategien zusammenfassen

\textbf{1.~Errate die Lösungen.} Olympiadegleichungen haben in den allermeisten Fällen keine absurden Lösungen. Wenn ihr alle \enquote{kleinen} Fälle durchprobiert habt, dann habt ihr meistens auch alle Lösungen gefunden.

\enquote{Kleine} Fälle auszuprobieren sollte immer eines der ersten Dinge sein, die ihr versucht. Einerseits bekommt ihr dadurch ein Gefühl für die Aufgabe. Andererseits zielen viele der Methoden, die wir weiter unten besprechen werden, darauf ab, zu zeigen, dass die Gleichung keine bzw.\ keine weiteren Lösungen hat. Das kann natürlich nur klappen, wenn ihr die Lösungen schon kennt. Zum Beispiel kann es vorkommen, dass in einer Diophantischen Gleichung der Term $3^n$ vorkommt. Nachdem ihr die Fälle $n=0$ und $n=1$ durchprobiert habt (und möglicherweise Lösungen gefunden habt), dürft ihr $n\geqslant 2$ annehmen. Ab dann ist $3^n$ durch $9$ teilbar und ihr könnt die Gleichung zum Beispiel modulo~$9$ betrachten.

Das bringt uns zu unserer nächsten Strategie.

\textbf{2.~Betrachte die Gleichung modulo einer geeigneten Zahl.} Diese Methode ist vor allem dann geeignet, wenn ihr zeigen wollt, dass eine Diophantische Gleichung keine Lösungen hat. Denn dafür reicht es aus, ein~$m$ zu finden, sodass die Gleichung keine Lösungen modulo~$m$ hat. Oft habt ihr nicht ganz so viel Glück, dass die Gleichung keine Lösungen modulo~$m$ hat, aber die Betrachtung modulo~$m$ liefert euch trotzdem Einschränkungen, die ihr nutzen könnt, um weitere Strategien anzuwenden (zum Beispiel Faktorisierungen oder unendlichen Abstieg).

Wie findet ihr heraus, modulo welcher Zahl sich die Gleichung zu betrachten lohnt? Um eine Chance auf einen Widerspruch modulo~$m$ zu haben, sollten die Terme in eurer Gleichung möglichst wenige Reste modulo~$m$ annehmen. Das kann zum Beispiel passieren, indem
\begin{itemize}
	\item eine oder besser einige der Zahlen in der Gleichung durch $m$ teilbar sind.
	\item eine oder besser einige der Zahlen in der Gleichung den Rest $\pm1$ modulo $m$ lassen.
\end{itemize}
In vielen Diophantischen Gleichungen kommen polynomielle Terme wie $x^2,x^3,\dotsc$ vor. Für solche Terme gibt es folgende Faustregel, mit der ihr geeignete~$m$ gezielt erraten könnt:
\begin{itemize}\itshape
	\item[$(*)$] Für eine gegebene positive ganze Zahl~$n$ nimmt der Term $x^n$ möglichst wenige verschiedene Werte modulo~$m$ an, wenn~$n$ ein Teiler von $\varphi(m)$ ist.
\end{itemize}
Der Grund hierfür ist der Satz von Euler-Fermat (siehe Kapitel~\ref{kapitel:Teilerfremdheit}: \emph{Teiler und Teilerfremdheit}): Wenn $x$ teilerfremd zu $m$ ist, dann ist $x^{\varphi(m)}\equiv 1\mod m$. Im Fall $n=\varphi(m)$ ist also nur der Rest~$1$ möglich. Wenn~$n$ nur ein Teiler von $\varphi(m)$ ist, können wir zumindest hoffen, dass nicht zu viele Reste möglich sind.\footnote{Umgekehrt ist es aussichtslos, auf wenige Reste zu hoffen, wenn $n$ zu $\varphi(m)$ teilerfremd ist. Nach dem Lemma von Bézout gibt es dann positive ganze Zahlen $a$ und $b$ mit $an=b\varphi(m)+1$. Der Term $x^n$ nimmt, solange $x$ teilerfremd zu $m$ ist, mindestens so viele Reste an, wie der Term $(x^a)^n\equiv x^{b\varphi(m)+1}\equiv x\mod m$.} Es kann natürlich auch passieren, dass $x$ nicht teilerfremd zu~$m$ ist, aber auch in diesem Fall können wir darauf hoffen, dass nicht zu viele Reste auftreten können (falls $m$ eine Primzahl ist, kann zum Beispiel nur der Rest~$0$ auftreten).

Als konkrete Anwendungen dieser Faustregel erhalten wir die folgenden gereimten Weisheiten:
\begin{itemize}\itshape
	\item[$(**)$] Sind Kuben niedergeschrieben -- rechne modulo neun oder sieben!\\
	Stehen Quadrate auf dem Papier -- rechne modulo acht oder vier \embrace{oder drei}.
\end{itemize}
Kubikzahlen nehmen modulo~$7$ oder~$9$ nur die Werte $-1$, $0$ und $1$ an. Das stimmt perfekt mit Faustregel~$(*)$ überein, denn $\varphi(7)=6=\varphi(9)$ ist durch~$3$ teilbar. Quadratzahlen nehmen modulo~$3$ und~$4$ nur die Werte $0$ und $1$ an sowie modulo~$8$ nur die Werte~$0$,~$1$ und~$4$. Auch das passt mit Faustregel~$(*)$. Allerdings ist $\varphi(m)$ für alle $m\neq 1,2$ durch~$2$ teilbar und somit gibt es für Quadratzahlen viele potentiell geeignete~$m$. Die Wahlen $m=3,4,8$ kommen jedoch am häufigsten zum Einsatz.

\textbf{3.~Faktorisiere die Gleichung.} Die wichtigste Faktorisierung, die ihr kennen solltet, ist die dritte binomische Formel $a^2-b^2=(a-b)(a+b)$ sowie ihr großer Bruder
\begin{equation*}
	a^n-b^n=(a-b)\parens*{a^{n-1}+a^{n-2}b+\dotsb+ab^{n-2}+b^{n-1}}\,.
\end{equation*}
Indem wir in der obigen Formel $b$ durch $-b$ ersetzen, erhalten wir ferner die Faktorisierungen
\begin{align*}
	a^n-b^n&=(a+b)\parens*{a^{n-1}-a^{n-2}b\pm\dotsb+ab^{n-2}-b^{n-1}}\quad\text{falls $n$ gerade ist}\,,\\
	a^n+b^n&=(a+b)\parens*{a^{n-1}-a^{n-2}b\pm\dotsb-ab^{n-2}+b^{n-1}}\quad\text{falls $n$ ungerade ist}\,.
\end{align*}
Schließlich solltet ihr die \emph{Sophie-Germain-Faktorisierung} kennen:
\begin{equation*}
	a^4+4b^4=\parens*{a^2+2ab+2b^2}\parens*{a^2-2ab+2b^2}
\end{equation*}
Manchmal sind Faktorisierungen erst anwendbar, nachdem ihr einige der anderen Strategien verwendet habt. Zum Beispiel könnte in einer Aufgabe der Term $2^n$ vorkommen und durch geeignete Modulo-Betrachtungen (etwa modulo~$3$) könnt ihr zwar keinen Widerspruch erzeugen, aber zumindest zeigen, dass $n$ gerade sein muss. Dann ist $n=2k$ und $2^n=(2^{k})^2$ ist also eine Quadratzahl, sodass wir zum Beispiel versuchen können, eine Faktorisierung mit der dritten binomischen Formel zu finden.

Manchmal sind Faktorisierungen auch nicht offensichtlich zu sehen. Zum Beispiel können wir die Gleichung
\begin{equation*}
	\frac 1x+\frac 1y=\frac1{\the\year}
\end{equation*}
für ganze Zahlen $x,y\neq 0$ umformen zu $\the\year(x+y)=xy$. Diese Gleichung wiederum lässt sich umformen zu
\begin{equation*}
	\the\year^2=xy-\the\year(x+y)+\the\year^2=(x-\the\year)(y-\the\year)
\end{equation*} 
Um diese Gleichung zu lösen, müssen wir nun nur noch alle Teiler von $\the\year^2$ durchprobieren. Die möglichen Teiler lassen sich anhand der Primfaktorzerlegung von $\the\year$ ablesen. Für solche Zwecke solltet ihr übrigens die Primfaktorzerlegung des aktuellen Jahres auswendig wissen, um sie euch in der Olympiade nicht erst mühselig herleiten zu müssen!

\textbf{4.~Benutze unendlichen Abstieg/das Extremalprinzip.} Um zu zeigen, dass eine Diophantische Gleichung keine oder nur triviale Lösungen hat, könnt ihr indirekt argumentieren: Ihr nehmt an, dass eine Lösung existiert und führt diese Annahme zum Widerspruch. Eine Möglichkeit, einen Widerspruch zu erzeugen, ist aus eurer angenommenen Lösung eine kleinere Lösung zu konstruieren. Indem ihr diesen Schritt wiederholt, könnt ihr eine unendliche Folge von immer kleiner werdenden Lösungen konstruieren, was offensichtlich nicht sein kann (diese Art des Widerspruches ist als \emph{unendlicher Abstieg} bekannt). Alternativ könnt ihr mit dem Extremalprinzip argumentieren und eine in geeignetem Sinne minimale Lösung betrachten. Wenn ihr eine kleinere Lösung konstruieren könnt, habt ihr euren Widerspruch erzeugt.

Um diese kleineren Lösungen zu konstruieren, sind häufig Modulo-Betrachtungen oder Faktorisierungen nützlich. Eine weitere Möglichkeit ist das sogenannte \emph{Vieta-Jumping:} Hierbei wird die gegebene Diophantische Gleichung in eine quadratische Gleichung in einer Variablen umgeformt. Mithilfe des Satzes von Vieta wird sodann gezeigt, dass die zweite Lösung dieser quadratischen Gleichung ebenfalls ganzzahlig und kleiner als die erste Lösung ist.

\textbf{5.~Schätze ab.} Die wichtigsten Abschätzungen in der Zahlentheorie sind die folgenden beiden Trivialitäten:
\begin{itemize}
	\item Wenn $m$ und~$n$ ganze Zahlen sind und $m>n$ gilt, dann folgt schon $m\geqslant n+1$.
	\item Wenn $a$ und~$n$ positive ganze Zahlen sind und~$a$ durch~$n$ teilbar ist, dann gilt $a\geqslant n$.
\end{itemize}
Obwohl beide Abschätzungen vollkommen offensichtlich sind, lassen sie sich erstaunlich oft anwenden. Und selbst manche tiefe Resultate in der modernen Forschungsmathematik beruhen am Ende auf diesen beiden Beobachtungen.

Besonders dann, wenn in einer Aufgabe Potenzen auftreten, kann es vorkommen, dass eine Seite schnell viel größer als die andere Seite wird. Oft wird auch folgender Trick verwendet: Wenn ihr (zum Beispiel durch Modulo-Betrachtungen) zeigen könnt, dass eine Potenz $a^n$ durch eine Primzahl $p$ teilbar ist, dann ist $a^n$ sogar durch $p^n$ teilbar (und mithin $a^n\geqslant p^n$, falls $a$ eine positive ganze Zahl ist).

\textbf{6.~Benutze Quadratschachtelung.} Wenn in einer Diophantischen Gleichung ein quadratischer Term vorkommt, könnt ihr versuchen, die Gleichung in der Form $x^2=R$ zu schreiben. Wenn ihr Glück habt, sieht $R$ selbst schon fast wie ein Quadrat aus und durch geeignete quadratische Ergänzung könnt ihr einen Term~$r$ finden, sodass $r^2<R<(r+1)^2$ gilt (außer vielleicht für kleine Werte von $r$). Dann habt ihr einen Widerspruch erzeugt, denn zwischen $r^2$ und $(r+1)^2$ kann keine weitere Quadratzahl $x^2=R$ liegen. Selbiges geht natürlich auch für höhere Potenzen.

Quadratschachtelung ist übrigens eine Anwendung der trivialen Beobachtung, dass für ganze Zahlen aus $m>n$ schon $m\geqslant n+1$ folgt!

Ansonsten gilt für Diophantische Gleichungen der gleiche Hinweis wie für Gleichungssysteme: \emph{Macht eine Probe!!!} Eine fehlende Probe ist einer der häufigsten Gründe für ärgerliche und vermeidbare Punktabzüge.

\subsection*{Beispielaufgaben}
Ihr sollt nun die beschriebenen Methoden selbstständig auf einige Beispielaufgaben anwenden. Am Ende dieses Kapitels (nach den weiteren Übungsaufgaben) findet ihr erst Tipps und dann Lösungen zu den Beispielaufgaben. Aufgabe~\ref{aufgabe:VAIMO2010} ist so schwer, dass es fast unmöglich ist, von selbst auf den Trick zu kommen. Dafür ist die Aufgabe aber sehr instruktiv! Wenn ihr nicht weiterkommt, holt euch einen Tipp oder lest euch die Lösung durch.

\begin{aufgabe*}\label{aufgabe:521235}
	Ermittle alle ganzzahligen Lösungen $(x,y)$ der Gleichung $19x^3-17y^3=50$.
\end{aufgabe*}
\begin{aufgabe*}\label{aufgabe:Modulo11}
	Ermittle alle ganzzahligen Lösungen $(a,b)$ der Gleichung $a^2+4=b^5$.
\end{aufgabe*}
\begin{aufgabe*}\label{aufgabe:UnendlicherAbstieg}
	Ermittle alle ganzzahligen Lösungen $(a,b,c,d)$ der Gleichung $a^4+2b^4+4c^4=8d^4$.
\end{aufgabe*}
\begin{aufgabe*}\label{aufgabe:Modulo+Faktorisierung}
	Ermittle alle Tripel $(a,b,n)$ von nichtnegativen ganzen Zahlen, die die Gleichung $2^a+3^b=n^2$ erfüllen.
\end{aufgabe*}
\begin{aufgabe*}[*]\label{aufgabe:471046}
	Ermittle alle reellen Zahlen $x$, für die $4x^3-7$ und $4x^5-7$ Quadratzahlen sind.
\end{aufgabe*}
\begin{aufgabe*}[***]\label{aufgabe:VAIMO2010}
	Ermittle alle Paare $(m,n)$ von nichtnegativen ganzen Zahlen, die die Gleichung $3^m-7^n=2$ erfüllen.
\end{aufgabe*}

\subsection*{Weitere Übungsaufgaben}
\begin{aufgabe*}\label{aufgabe:501044}
	Ermittle alle ganzzahligen Lösungen der Gleichung $n^2=2^k+5$.
\end{aufgabe*}
\begin{aufgabe*}
	Gibt es positive ganze Zahlen $a$ und $b$, sodass sowohl $a^2+4b$ als auch $b^2+4a$ Quadratzahlen sind?
\end{aufgabe*}
\begin{aufgabe*}
	Gibt es ganze Zahlen $x$ und $y$, sodass $x^3+y^4=19^{19}$?
\end{aufgabe*}
\begin{aufgabe*}
	Sei $n$ eine ungerade positive ganze Zahl und seien $x$, $y$ rationale Zahlen, sodass $x^n+2y=y^n+2x$ gilt. Zeige, dass $x=y$ sein muss.
\end{aufgabe*}
\begin{aufgabe*}[*]
	Finde alle Paare $(n,k)$ nichtnegativer ganzer Zahlen, sodass $(n+1)^k=n!+1$.
\end{aufgabe*}
\begin{aufgabe*}[*]
	Finde alle Paare $(x,y)$ nichtnegativer ganzer Zahlen, sodass $1+2^x+2^{2x+1}=y^2$.
\end{aufgabe*}
\begin{aufgabe*}[**]
	Finde alle Paare $(a,b)$ nichtnegativer ganzer Zahlen, sodass $2^a=3^b+5$.
\end{aufgabe*}
\begin{aufgabe*}[**]
	Finde alle Tripel $(a,b,c)$ nichtnegativer ganzer Zahlen, sodass $3^a+4^b=5^c$.
\end{aufgabe*}

\vfill\hrule\vspace{-1em}

\subsection*{Tipps zu den Beispielaufgaben}
\textbf{Tipp zu Aufgabe~\ref{aufgabe:521235}.} Finde eine geeignete Zahl~$m$, sodass Kubikzahlen möglichst wenige Reste modulo~$m$ annehmen, und betrachte die Gleichung modulo~$m$.

\textbf{Tipp zu Aufgabe~\ref{aufgabe:Modulo11}.} Finde eine geeignete Zahl~$m$, sodass Quadratzahlen und fünfte Potenzen möglichst wenige Reste modulo~$m$ annehmen, und betrachte die Gleichung modulo~$m$.

\textbf{Tipp zu Aufgabe~\ref{aufgabe:UnendlicherAbstieg}.} Betrachte die Gleichung modulo~$2$ und benutze unendlichen Abstieg.

\textbf{Tipps zu Aufgabe~\ref{aufgabe:Modulo+Faktorisierung}.} Betrachte die Gleichung modulo~$3$. Faktorisiere sie danach.

Um die faktorisierte Gleichung zu lösen, betrachte die Faktoren modulo $3$ und modulo $4$ und faktorisiere erneut.


\textbf{Tipp zu Aufgabe~\ref{aufgabe:471046}.} Zeige zuerst, dass $x$ rational ist und dann, dass $x$ ganzzahlig ist.

Benutze danach Quadratschachtelung.


\textbf{Tipp zu Aufgabe~\ref{aufgabe:VAIMO2010}.} Betrachte die Gleichung erst modulo~$9$, danach modulo~$49$ und schließlich modulo~$43$ (wenn du wissen willst, wie zum Geier du auf die~43 hättest kommen sollen, dann schau einmal in die Musterlösung).\newpage
	\section{Quersummen}\label{kapitel:Quersummen}
Manchmal kommen in Olympiade-Aufgaben Quersummen vor. So zum Beispiel in den beiden folgenden Aufgaben:
\begin{aufgabe*}\label{aufgabe:441244}
	Wir bezeichnen die Quersumme einer positiven ganzen Zahl $n$ mit $Q(n)$. Was ist
	\begin{equation*}
		Q\parens*{Q\parens*{Q\parens*{2005^{2005}}}}\,?
	\end{equation*}
\end{aufgabe*}
\begin{aufgabe*}\label{aufgabe:531042}
	Für eine positive ganze Zahl sei $Q(n)$ die Quersumme von $n$ und $P(n)$ das Querprodukt von $n$, also das Produkt aller Ziffern von $n$. Ferner sei $R(n)\coloneqq n+P(n)Q(n)$. Für ein gegebenes $n$ definieren wir nun eine Folge $(a_i)_{i\geqslant 0}$ rekursiv durch $a_0\coloneqq n$, $a_{i+1}\coloneqq R(a_i)$. Zeige, dass die Folge $(a_i)_{i\geqslant 0}$ ab irgendeinem Punkt konstant sein muss.
\end{aufgabe*}

Bei solchen Aufgaben gibt es zwei Tricks, deren Kombination fast immer zum Erfolg führt.

\textbf{1. ~Abschätzungen.} Wenn $n$ eine $k$-stellige Zahl ist, dann gilt offensichtlich $n\geqslant 10^{k-1}$ und andererseits $Q(n)\leqslant 9k$. Weil lineare Funktionen viel langsamer wachsen als Exponentialfunktionen, ist $10^{k-1}$ für große $k$ viel größer als $9k$. Somit ist $n$ im Allgemeinen viel größer als~$Q(n)$.

\textbf{2.~Modulo~9.} Wenn $n=a_0+10a_1+10^2a_2+\dotsb+10^ka_k$ die Dezimaldarstellung einer positiven ganzen Zahl $n$ ist, dann gilt
\begin{equation*}
	n\equiv a_0+1\cdot a_1+1^2\cdot a_2+\dotsb+1^k\cdot a_k\equiv Q(n)\mod 9
\end{equation*}
(das ist eine Verschärfung der bekannten Teilbarkeitsregel für~$9$).

Ihr sollt nun versuchen, die beiden Aufgaben selbstständig zu lösen. Wenn ihr nicht weiterkommt, helfen euch die folgenden Tipps weiter. Am Ende dieses Heftes findet ihr außerdem die Lösungen.

\vfill\hrule\vspace{-1em}

\subsection*{Tipps zu den Beispielaufgaben}

\textbf{Tipp zu Aufgabe~\ref{aufgabe:441244}.} Schätze ab, wie groß $Q(Q(Q(2005^{2005})))$ höchstens sein kann. Wie kannst du herausfinden, welche der (sehr wenigen) verbleibenden Möglichkeiten die richtige ist?

\textbf{Tipp zu Aufgabe~\ref{aufgabe:531042}.} Wenn die Behauptung falsch wäre, dann müsste die Folge $(a_i)_{i\geqslant 0}$ streng monoton steigend sein. Um einen Widerspruch zu erzeugen, konstruiere ein $a_i$, welches an geeigneter Stelle eine Ziffer~$0$ enthält.\newpage
	
	\phantomsection
	\cftaddtitleline{toc}{part}{Lösungen zu den Beispielaufgaben}{\thepage}% Damit in der PDF-Navigationsleiste auch der Abschnitt "Lösungen" auftaucht, muss ein zusätzliches Bookmark gesetzt werden. Irrelevant für die Druckversion.
	\pdfbookmark{Lösungen zu den Beispielaufgaben}{Loesungen}
	\section*{Lösungen zu den Beispielaufgaben}
	Die Lösungen sind nicht immer so formuliert, wie ihr das in der Olympiade tun solltet. Zum Teil sind sie sehr knapp -- zum Beispiel überspringen wir triviale Umformungsschritte oder lassen die Probe weg. In der Olympiade solltet ihr etwas ausführlicher sein und immer die Probe machen. Umgekehrt erklären wir gelegentlich (vor allem bei besonders schweren Aufgaben), wie wir auf die Lösung gekommen sind. In der Olympiade müsst ihr solche Überlegungen natürlich nicht aufschreiben, sondern könnt eure ausgefuchste Lösung einfach vom Himmel fallen lassen.
	
	Soweit bekannt ist außerdem angegeben, aus welchem Wettbewerb die betreffende Aufgabe stammt, damit ihr (zum Beispiel in einschlägigen Foren) nach Alternativlösungen suchen könnt.
	\subsection*{Lösungen zu Kapitel~\ref{kapitel:Gleichungssysteme}: \emph{Nichtlineare Gleichungssysteme}}

\begin{proof}[Lösung zu Aufgabe~\ref{aufgabe:520943}]
	Es ist klar, dass $x,y,z\neq 0$, sonst wären die Brüche nicht definiert. Wir zeigen zuerst, dass $x$,~$y$ und~$z$ das gleiche Vorzeichen haben müssen. Angenommen, das wäre nicht der Fall. Indem wir gegebenenfalls $(x,y,z)$ durch $(-x,-y,-z)$ ersetzen (wodurch wir immer noch eine Lösung bekommen), dürfen wir annehmen, dass genau eine Variable negativ ist und die anderen beiden positiv sind. Nach zyklischer Vertauschung der Variablen dürfen wir dann annehmen, dass $x$ negativ und $y$, $z$ positiv sind. Dann ist $z-\frac 1x$ positiv, aber $x-\frac 1y$ negativ, Widerspruch!
	
	Also haben $x$, $y$, $z$ das gleiche Vorzeichen. Indem wir gegebenenfalls $(x,y,z)$ durch $(-x,-y,-z)$ ersetzen, dürfen wir annehmen, dass alle Variablen positiv sind. Das Gleichungssystem ist zyklisch in $x$, $y$ und $z$, also dürfen wir ohne Beschränkung der Allgemeinheit annehmen, dass $x=\max\{x,y,z\}$. Aus $x\geqslant y$ und $x-\frac 1y=y-\frac 1z$ folgt $\frac 1y\geqslant \frac 1z$. Aus $x\geqslant z$ und $x-\frac 1y=z-\frac 1x$ folgt $\frac 1y\geqslant \frac 1z$. Folglich ist $\frac 1y=\max\braces[\big]{\frac 1x,\frac 1y,\frac 1z}$. Weil die Funktion $f\colon \mathbb R_{>0}\rightarrow \mathbb R_{>0}$, $f(t)=\frac 1t$ streng monoton fallend ist, muss deshalb $y=\min\{x,y,z\}$ gelten. Mit einem analogen Argument folgt aus $y=\min\{x,y,z\}$ zuerst $\frac 1z=\min\braces[\big]{\frac 1x,\frac 1y,\frac 1z}$ und dann $z=\max\{x,y,z\}$. Also ist $x=z$. Indem wir das gleiche Argument noch einmal mit~$z$ durchführen, erhalten wir $x=\min\{x,y,z\}$. Also ist auch $x=y$. Somit muss $x=y=z$ sein. Es ist unmittelbar klar, dass jedes Tripel $(x,y,z)$ mit $x=y=z\neq 0$ auch tatsächlich das Gleichungssystem löst.
\end{proof}
\begin{proof}[Lösung zu Aufgabe~\ref{aufgabe:521043}.]
	Es ist klar, dass $x,y,z\neq 0$, sonst wären die Brüche nicht definiert. Sei $a$ der gemeinsame Wert von $x+\frac 1y$, $y+\frac 1z$ und $z+\frac 1x$. Aus $x+\frac 1y=a$ folgt $ay=xy+1$ und aus $y+\frac 1z=a$ folgt $az=yz+1$. Nach Multiplikation mit~$a$ und Einsetzen folgt
	\begin{equation*}
		a^2z=ayz+a=(xy+1)z+a=xyz+z+a\quad\Longleftrightarrow \quad\parens*{a^2-1}z=xyz+a\,.
	\end{equation*}
	Analog gilt auch $(a^2-1)x=xyz+a$ und $(a^2-1)y=xyz+a$. Für $a^2-1\neq 0$ folgt aus den Gleichungen $(a^2-1)x=(a^2-1)y=(a^2-1)z$ schon $x=y=z$. Offensichtlich ist jedes Tripel $(x,y,z)$ mit $x=y=z\neq 0$ auch tatsächlich eine Lösung des Gleichungssystems.
	
	Übrig bleibt der Fall $a^2-1=0$, also $a=\pm 1$. Wir machen nun eine Fallunterscheidung.
	
	\emph{Fall~1: Es gilt $a=1$.} Die Gleichung $ay=xy+1$ wird in diesem Fall zu $y=xy+1$, was sich zu $y=\frac{1}{1-x}$ umformen lässt (im Fall $x=1$ bekommen wir keine Lösung). Analog gilt $z=\frac{1}{1-y}$ und $x=\frac{1}{1-z}$. Indem wir diese Gleichungen nacheinander ineinander einsetzen, bekommen wir zuerst $z=-\frac{1-x}{x}$ und dann die wahre Aussage $x=x$. Das lässt uns vermuten, dass für $x\neq 0,1$ das Tripel $(x,y,z)=\parens[\big]{x,\frac{1}{1-x},-\frac{1-x}{x}}$ eine Lösung des Gleichungssystems ist. Das lässt sich durch Einsetzen unmittelbar verifizieren.
	
	
	\emph{Fall~2: Es gilt $a=-1$.} Völlig analog zu Fall~1 bekommen wir für jede reelle Zahl $x\neq 0,-1$ das Lösungstripel $(x,y,z)=\parens[\big]{x,-\frac{1}{1+x},-\frac{1+x}{x}}$.
	
	Damit sind alle Fälle abgearbeitet und wir haben alle Lösungen gefunden.
\end{proof}
\begin{proof}[Lösung zu Aufgabe~\ref{aufgabe:380943}]
	Aus der AM-GM-Ungleichung folgt $t+\frac 1t\geqslant 2\sqrt{t\cdot\frac 1t}=2$ für alle $t>0$. Gleichheit gilt nur für $t=\frac 1t$, was auf $t^2=1$ und damit $t=1$ führt (der Fall $t=-1$ ist wegen $t>0$ ausgeschlossen). Insbesondere erhalten wir
	\begin{equation*}
		x+3y^3+5z^5+\frac 1x+\frac3{y^3}+\frac5{z^5}=\parens*{x+\frac 1x}+3\parens*{y^3+\frac1{y^3}}+5\parens*{z^5+\frac1{z^5}}\geqslant 2+3\cdot 2+5\cdot 2=18\,.
	\end{equation*}
	Gleichheit gilt nur für $x=y^3=z^5=1$, was auf $(x,y,z)=(1,1,1)$ führt. Eine Probe zeigt, dass dieses Tripel tatsächlich eine (und folglich die einzige) Lösung ist.
\end{proof}
\begin{proof}[Lösung zu Aufgabe~\ref{aufgabe:451046}.]
	Offenbar muss $x,y,z\neq 0$ sein, sonst wären die Brüche nicht definiert. Indem wir die ersten beiden Gleichungen voneinander subtrahieren, erhalten wir
	\begin{equation*}
		0=\parens*{x+y+\frac1z}-\parens*{y+z+\frac 1x}=(x-z)\parens*{1-\frac{1}{xz}}\,.
	\end{equation*}
	Aus dieser Gleichung folgt $x=z$ oder $xz=1$. Analog folgt $x=y$ oder $xy=1$ sowie $y=z$ oder $yz=1$. Wenn $x=y$ oder $x=z$ gilt, dann sind zwei Variablen gleich. Wenn nicht, dann muss $xy=1$ und $yz=1$ gelten. Wegen $x\neq 0$ folgt $y=z$ und auch in diesem Fall sind zwei Variablen gleich. Es müssen also in jedem Fall mindestens zwei der drei Variablen gleich sein. Bis auf zyklische Vertauschung der Variablen ergeben sich dann die folgenden beiden Fälle:
	
	\emph{Fall~1: Es gilt $x=y=z$.} In diesem Fall erhalten wir durch Einsetzen $2x+\frac 1x=3$, was sich zu $0=2x^2-3x+1=(2x-1)(x-1)$ umformen lässt. Wir können die Lösungen $x=\frac12$ und $x=1$ ablesen. Es ergeben sich die Lösungstripel $(x,y,z)=\parens[\big]{\frac12,\frac12,\frac12}$ und $(x,y,z)=(1,1,1)$. Durch Einsetzen lässt sich unmittelbar nachprüfen, dass es sich hierbei tatsächlich um Lösungen des Gleichungssystems handelt.
	
	\emph{Fall~2: Es gilt $x=y$ und $xz=1$.} Dann gilt $x=\frac 1z$. In der Gleichung $x+y+\frac 1z=3$ sind folglich alle drei Summanden gleich und es muss $x=y=\frac 1z=1$ gelten. Das führt auf das Lösungstripel $(x,y,z)=(1,1,1)$, welches wir bereits in Fall~1 behandelt haben.
	
	Damit sind alle Fälle abgehandelt und wir haben alle Lösungen gefunden.
\end{proof}
\begin{proof}[Lösung zu Aufgabe~\ref{aufgabe:461041}]
	Offenbar gilt $x,y,z\neq 0$, sonst wären die Brüche nicht definiert. Aus der Bedingung $\frac 1x+\frac 1y+\frac 1z=1$ folgt, dass eine reelle Zahl $a$ mit $a=xy+yz+zx=xyz$ existiert. Nun sind $x$, $y$ und $z$ die drei Nullstellen des Polynoms
	\begin{align*}
		P(X)\coloneqq (X-x)(X-y)(X-z)&=X^3-(x+y+z)X^2+(xy+yz+zx)X-xyz\\
		&=X^3-X^2+aX-a\\
		&=(X^2+a)(X-1)\,.
	\end{align*}
	Wir können direkt ablesen, dass $P(X)$ die drei Nullstellen $X=1$ und $X=\pm\sqrt{-a}$ hat (insbesondere muss $a\leqslant 0$ sein). Indem wir $t\coloneqq \sqrt{-a}$ setzen, erhalten wir $(x,y,z)=(1,t,-t)$ sowie Permutationen davon. Durch Einsetzen ist klar, dass Tripel von dieser Form mit $t\neq 0$ tatsächlich das gegebene Gleichungssystem lösen.
\end{proof}
\begin{proof}[Lösung zu Aufgabe~\ref{aufgabe:541241}]
	Indem wir die zweite Gleichung mit $27$ multiplizieren und zur ersten Gleichung addieren, erhalten wir
	\begin{equation*}
		64=10+27\cdot 2=x^3+9x^2y+27xy^2+27y^3=\parens*{x+3y}^3\,.
	\end{equation*}
	Es folgt $x+3y=4$. Indem wir $3y=4-x$ in die erste Gleichung einsetzen, erhalten wir
	\begin{equation*}
		10=x^3+9x^2y=x^3+3x^2(4-x)=-2x^3+12x^2\quad\Longleftrightarrow\quad 0=2(x-1)\parens*{x^2-5x-5}\,.
	\end{equation*}
	Diese Faktorisierung haben wir gefunden, indem uns aufgefallen ist, dass $(x,y)=(1,1)$ eine Lösung des ursprünglichen Gleichungssystems ist, sodass sich ein Faktor $x-1$ ausklammern lassen muss. Es bleiben folglich zwei Fälle:
	
	\emph{Fall~1: Es gilt $x=1$.} Dieser Fall führt auf $(x,y)=(1,1)$, was offensichtlich eine Lösung des Gleichungssystems ist.
	
	
	\emph{Fall~2: Es gilt $x^2-5x-5=0$.} Die Lösungsformel für quadratische Gleichungen liefert uns $x=\frac{5\pm3\sqrt{5}}{2}$ und $y=\frac{4-x}{3}=\frac{1\mp\sqrt{5}}{2}$. Durch Einsetzen erhalten wir nach kurzer Rechnung, dass die Paare $(x,y)=\parens[\big]{\frac{5+3\sqrt{5}}{2},\frac{1-\sqrt{5}}{2}}$ und $(x,y)=\parens[\big]{\frac{5-3\sqrt{5}}{2},\frac{1+\sqrt{5}}{2}}$ tatsächlich Lösungen sind.
	
	Damit sind alle Fälle behandelt und wir haben alle Lösungen gefunden.
\end{proof}
\begin{proof}[Lösung zu Aufgabe~\ref{aufgabe:Sayda2013}]
	Indem wir die erste Gleichung von der zweiten subtrahieren und faktorisieren, erhalten wir $(u-v)(u+v)=2(u-v)$. Es muss folglich $u=v$ oder $u+v=2$ gelten. Selbiges gilt für jedes andere Paar von Variablen. Wir behaupten, dass dann von je drei Variablen mindestens zwei gleich sein müssen. Wir werden diese Behauptung nur für $u$, $v$ und $w$ beweisen, alle anderen Fälle sind völlig analog. Falls $u=v$ oder $u=w$ gilt, dann ist die Behauptung offensichtlich erfüllt. Ansonsten muss $u+v=2$ und $u+w=2$ gelten, woraus aber $v=w$ und damit ebenfalls die Behauptung folgt. Damit ist gezeigt, dass es unter je drei Variablen stets mindestens zwei gleiche geben muss.
	
	Es folgt, dass die Variablen $u$, $v$, $w$, $x$ und $y$ nur höchstens zwei verschiedene Werte annehmen können. Bis auf Vertauschung der Variablen ergeben sich also folgende Fälle:
	
	\emph{Fall~1: Es gilt $u=v=w=x=y$.} In diesem Fall erhalten wir die Gleichung $4u^2=6-2u$, welche sich zu $0=2(u-1)(2u+3)$ umformen lässt. Die Lösungen $u=1$ und $u=-\frac23$ lassen sich ablesen und wir erhalten die beiden Lösungstupel $(u,v,w,x,y)=(1,1,1,1,1)$ und $(u,v,w,x,y)=\parens[\big]{-\frac23,-\frac23,-\frac23,-\frac23,-\frac23}$. Durch Einsetzen sehen wir, dass diese tatsächlich das Gleichungssystem lösen.
	
	\emph{Fall~2: Es gilt $u=v=w=x$ und $u\neq y$.} Aus $u\neq y$ folgt, wie wir gesehen haben, $u+y=2$. Indem wir $y=2-u$ in die letzte Gleichung einsetzen, folgt $4u^2=6-2(2-u)=2u+2$, was sich zu $0=2(u-1)(2u+1)$ umformen lässt. Das führt auf die beiden Lösungstupel $(u,v,w,x,y)=(1,1,1,1,1)$ und $(u,v,w,x,y)=\parens[\big]{-\frac12,-\frac12,-\frac12,-\frac12,\frac52}$ sowie alle Permutationen dieser Tupel. Durch Einsetzen sehen wir, dass diese tatsächlich das Gleichungssystem lösen.
	
	\emph{Fall~3: Es gilt $u=v=w$, $x=y$ und $u\neq x$.} Aus $u\neq x$ folgt $u+x=2$. Indem wir $x=y=2-u$ in die letzte Gleichung einsetzen, erhalten wir $3u^2+(2-u)^2=6-2(2-u)=2u+2$, was sich zu $0=2(u-1)(2u-1)$ umformen lässt. Das führt auf die beiden Lösungstupel $(u,v,w,x,y)=(1,1,1,1,1)$ und $(u,v,w,x,y)=\parens[\big]{\frac12,\frac12,\frac12,\frac32,\frac32}$ sowie alle Permutationen dieser Tupel. Durch Einsetzen sehen wir, dass diese tatsächlich das Gleichungssystem lösen.
	
	Damit haben wir alle Fälle untersucht und alle Lösungen gefunden.
\end{proof}
\begin{proof}[Lösung zu Aufgabe~\ref{aufgabe:IMOSL1993VNM}]
	Weil das Gleichungssystem zyklisch ist, dürfen wir annehmen, dass $x_1$ den maximalen Betrag unter~$x_1,x_2,\dotsc,x_{42}$ hat. Wir unterscheiden zwei Fälle:
	
	\emph{Fall~1: Es gilt $x_1\geqslant 0$.} Weil $x_1$ den maximalen Betrag hat, ist $x_1^2$ das Maximum von $x_1^2,x_2^2,\dotsc,x_{42}^2$. Wegen $x_1^2=ax_2+1$ muss $ax_2+1$ das Maximum von $ax_1+1,ax_2+1,\dotsc,ax_{42}+1$ sein. Wegen $a>0$ folgt, dass $x_2$ das Maximum von $x_1,x_2,\dotsc,x_{42}$ ist. Weil $x_1$ nichtnegativ ist und unter den Zahlen $x_1,x_2,\dotsc,x_{42}$ den maximalen Betrag hat, muss $x_1$ auch das Maximum von $x_1,x_2,\dotsc,x_{42}$ sein. Es folgt $x_1=x_2$. Indem wir das gleiche Argument mit $x_2,x_3,\dotsc$ wiederholen, erhalten wir $x_1=x_2=\dotsb=x_{42}$. Dann ist $x_1$ eine Lösung von $x_1^2-ax_1-1=0$. Aus der üblichen quadratischen Lösungsformel erhalten wir
	\begin{equation*}
		x_1=\frac{a+\sqrt{a^2+4}}2=x_2=x_3=\dotsb=x_{42}
	\end{equation*}
	(die andere Lösung der quadratischen Gleichung ist negativ und kommt deshalb im Falle $x_1\geqslant 0$ nicht in Frage). Aus der Konstruktion ist klar, dass dies tatsächlich eine Lösung ist.
	
	\emph{Fall~2: Es gilt $x_1<0$.} Damit die Gleichung $x_{42}^2=ax_1+1$ überhaupt eine reelle Lösung hat, muss $x_1\geqslant -\frac 1a>-1$ sein, denn nach Annahme gilt $a>1$. Also ist $x_1^2<1$. Nach dem gleichen Argument wie in Fall~1 ist $x_2$ wieder das Maximum von $x_1,x_2,\dotsc,x_{42}$. Andererseits ist $ax_2+1=x_1^2<1$. Aus $a>1$ folgt nun $x_2<0$. Somit sind $x_1,x_2,\dotsc,x_{42}$ allesamt negativ, denn ihr Maximum $x_2$ ist negativ. Als Maximum von lauter negativen Zahlen hat $x_2$ zwangsläufig den minimalen Betrag. Wegen $x_2^2=ax_3+1$ muss $x_3$ minimal unter $x_1,x_2,\dotsc,x_{42}$ sein. Weil diese alle negativ sind, muss $x_3$ den maximalen Betrag haben. Also ist $x_1=x_3$. Indem wir diese Argumente iterieren, erhalten wir $x_1=x_3=\dotsb=x_{41}$ und $x_2=x_4=\dotsb=x_{42}$. Das ursprüngliche Gleichungssystem lässt sich also zu folgendem Gleichungssystem vereinfachen:
	\begin{equation*}
		\left\{\begin{aligned}
			x_1^2&=ax_2+1\,,\\
			x_2^2&=ax_1+1\,.
		\end{aligned}\right.
	\end{equation*}
	Indem wir die beiden Gleichungen subtrahieren, erhalten wir $(x_1-x_2)(x_1+x_2)=a(x_2-x_1)$. Es gibt also nur die Möglichkeiten $x_1=x_2$ und $x_1+x_2=-a$. 
	
	\emph{Fall~2.1: Es gilt $x_1=x_2$.} Dieser Fall führt auf die quadratische Gleichung $x_1^2-ax_1-1=0$ und damit auf
	\begin{equation*}
		x_1=\frac{a-\sqrt{a^2+4}}2=x_2=x_3=\dotsb=x_{42}\,.
	\end{equation*}
	(die andere Lösung der quadratischen Gleichung ist positiv und kommt daher im Fall $x_1^2<0$ nicht in Frage). Aus der Konstruktion ist klar, dass dies tat tatsächlich eine Lösung ist.
	
	\emph{Fall~2.2: Es gilt $x_1+x_2+a=0$} Wir setzen $x_2=-a-x_1$ in die erste Gleichung ein und erhalten die quadratische Gleichung $x_1^2+ax_1+(a^2-1)=0$. Die Diskriminante dieser quadratischen Gleichung ist $4-3a^2$. Für $a>{2}/{\sqrt{3}}$ ist dieser Ausdruck negativ und wir erhalten keine reellen Lösungen. Für $a\leqslant {2}/{\sqrt{3}}$ führt die übliche Lösungsformel auf 
	\begin{equation*}
		x_1=\frac{-a\pm\sqrt{4-3a^2}}2=x_3=x_5=\dotsb=x_{41}\,,\quad x_2=\frac{-a\mp\sqrt{4-3a^2}}2=x_4=x_6=\dotsb=x_{42}\,.
	\end{equation*}
	Durch Einsetzen lässt sich überprüfen, dass dies wiederum eine Lösung des ursprünglichen Gleichungssystems ist.
	
	Damit haben wir alle Fälle abgefrühstückt und alle Lösungen gefunden.
\end{proof}
	\subsection*{Lösungen zu Kapitel~\ref{kapitel:Umordnung}: \emph{Die Umordnungs-Ungleichung}}

\begin{proof}[Lösung zu Aufgabe~\ref{aufgabe:EasyUmordnung} \textmd{(mit der Umordnungs-Ungleichung)}]
	Egal, wie die Folge $a,b,c$ geordnet ist, können wir immer feststellen, dass die Folgen $a^2,b^2,c^2$ und $a,b,c$ gleich geordnet sind. Aus der Umordnungs-Ungleichung folgt also sofort $a^2\cdot a+b^2\cdot b+c^2\cdot c\geqslant a^2b+b^2c+c^2a$.
\end{proof}
Beachte, dass wir bei Aufgabe~\ref{aufgabe:EasyUmordnung} \emph{nicht} ohne Einschränkung der Allgemeinheit $a\geqslant b\geqslant c$ annehmen dürfen. Das wäre nur erlaubt, wenn die Ungleichung \emph{symmetrisch} in $a$, $b$ und $c$ ist, aber die rechte Seite ist lediglich \emph{zyklisch} in den Variablen. Solche Fehler passieren in der Olympiade leider häufig und werden oftmals mit harten Punktabzügen bestraft.

Aufgabe~\ref{aufgabe:EasyUmordnung} (genau wie viele ähnliche Aufgaben) lässt sich auch mit der gewichteten AM-GM-Ungleichung lösen. Weil das eine sehr wichtige Technik ist, werden wir auch diese Lösung besprechen.

\begin{proof}[Zweite Lösung zu Aufgabe~\ref{aufgabe:EasyUmordnung} \textmd{(mit der gewichteten AM-GM-Ungleichung)}]
	Aus der gewichteten AM-GM-Ungleichung folgt $\frac23a^3+\frac13b^3\geqslant (a^3)^{2/3}(b^3)^{1/3}=a^2b$. Analog gilt $\frac23b^3+\frac13c^3\geqslant b^2c$ und $\frac23c^3+\frac13a^3\geqslant c^2a$. Addition dieser Ungleichung liefert schon das Gewünschte.
\end{proof}

\begin{proof}[Lösung zu Aufgabe~\ref{aufgabe:AM-GM-MitUmordnung}]
	 Egal, wie eine Folge $a_1,a_2,\dotsc,a_n>0$ von positiven reellen Zahlen geordnet ist, sind die Folgen $a_1,a_2,\dotsc,a_n$ und $1/a_1,1/a_2,\dotsc,1/a_n$ stets gegensinnig geordnet. Aus der Umordnungs-Ungleichung folgt also insbesondere
	\begin{equation*}
		\frac{a_2}{a_1}+\frac{a_3}{a_2}+\dotsb+\frac{a_n}{a_{n-1}}+\frac{a_1}{a_n}\geqslant a_1\cdot \frac{1}{a_1}+a_2\cdot \frac1{a_2}+\dotsb+a_n\cdot \frac{1}{a_n}= n\,.
	\end{equation*}
	Jetzt betrachten wir positive reelle Zahlen $x_1,x_2,\dotsc,x_n>0$ sowie ihr geometrisches Mittel $G\coloneqq \sqrt[n]{x_1x_2\dotsm x_n}$ und setzen
	\begin{equation*}
		a_1\coloneqq \frac{x_1}{G}\,,\quad a_2\coloneqq \frac{x_1x_2}{G^2}\,,\quad\dotsc\,,\quad a_n\coloneqq \frac{x_1x_2\dotsm x_n}{G^n}\,.
	\end{equation*}
	Indem wir diese Werte in die obige Ungleichung einsetzen, erhalten wir
	\begin{equation*}
		\frac{x_1+x_2+\dotsb+x_n}{G}=\frac{a_2}{a_1}+\frac{a_3}{a_2}+\dotsb+\frac{a_n}{a_{n-1}}+\frac{a_1}{a_n}\geqslant n\,.
	\end{equation*}
	Daraus folgt sofort die AM-GM-Ungleichung für positive reelle Zahlen $x_1,x_2,\dotsc,x_n>0$. Falls ein $x_i=0$ ist, ist die AM-GM-Ungleichung trivial, denn dann gilt $\sqrt[n]{x_1x_2\dotsm x_n}=0$.
\end{proof}
	\subsection*{Lösungen zu Kapitel~\ref{kapitel:Inversion}: \emph{Inversion am Kreis}}

\begin{figure}[ht]
	\centering
	\begin{tabularx}{\textwidth}{X c X c X}
		& \begin{tikzpicture}[x=1.25cm,y=1.25cm]
			\draw [line width=0.3, shift={(-0.811,0)}] (0:0.811) arc (0:180:0.811);
			\draw [line width=0.3, shift={(1.339,0)}] (0:1.339) arc (0:180:1.339);
			\draw [dashed,line width=0.3] (0,0.76) circle (0.76);
			\coordinate (A) at (-1.622,0);
			\coordinate (B) at (2.678,0);
			\coordinate (C) at (0,2.856);
			\coordinate (H) at (0,0);
			\coordinate (P) at (-1.226,0.696);
			\coordinate (Q) at (1.425,1.336);
			\coordinate (S) at(-0.759,0.809);
			\coordinate (T) at (0.653,1.15);
			\draw (C) to (A) to (B) to (C) to (H);
			\draw [line width=0.3] (H) to (P) to (Q) to cycle;
			\draw [line width=0.3,shift={(P)}] (240.412:0.32cm) arc (240.412:330.412:0.32cm);
			\fill [shift={(P)}] (285.412:0.18cm) circle (1pt);
			\draw [line width=0.3,shift={(P)}] (240.412:0.32cm) arc (240.412:330.412:0.32cm);
			\fill [shift={(P)}] (285.412:0.18cm) circle (1pt);
			\draw [line width=0.3,shift={(Q)}] (223.16:0.32cm) arc (223.16:313.16:0.32cm);
			\fill [shift={(Q)}] (268.16:0.18cm) circle (1pt);
			\draw[fill=black] (A) circle (2pt) node[shift={(220:2ex)}] {$A$};
			\draw[fill=black] (B) circle (2pt) node[shift={(-40:2ex)}] {$B$};
			\draw[fill=black] (C) circle (2pt) node[shift={(90:2ex)}] {$C$};
			\draw[fill=white] (H) circle (2pt) node[shift={(-90:2ex)}] {$H$};
			\draw[fill=black] (P) circle (2pt) node[shift={(140:2ex)}] {$P$};
			\draw[fill=black] (Q) circle (2pt) node[shift={(70:2ex)}] {$Q$};
			\draw[fill=black] (S) circle (2pt) node[shift={(120:1.5ex)}] {$S$};
			\draw[fill=black] (T) circle (2pt) node[shift={(70:2ex)}] {$T$};
		\end{tikzpicture} & & 
		\begin{tikzpicture}[x=1.25cm,y=1.25cm]
			\draw [line width=0.3, shift={(-0.419,1.737)}] (145:1.787) arc (145:395:1.787);
			\coordinate (A) at (-2.126,0);
			\coordinate (B) at (1.287,0);
			\coordinate (C) at (0,1.207);
			\coordinate (H) at (0,0);
			\coordinate (P) at (-2.126,1.207);
			\coordinate (Q) at (1.287,1.207);
			\coordinate (S) at(-2.126,2.267);
			\coordinate (T) at (1.287,2.267);
			\draw (S) to (A) to (B) to (T);
			\draw (P) to (Q);
			\draw [line width=0.3,dashed] (S) to (T);
			\draw (C) to (H);
			\draw [line width=0.3,shift={(H)}] (90:0.32cm) arc (90:180:0.32cm);
			\fill [shift={(H)}] (135:0.18cm) circle (1pt);
			\draw [line width=0.3,shift={(C)}] (180:0.32cm) arc (180:270:0.32cm);
			\fill [shift={(C)}] (225:0.18cm) circle (1pt);
			\draw [line width=0.3,shift={(A)}] (0:0.32cm) arc (0:90:0.32cm);
			\fill [shift={(A)}] (45:0.18cm) circle (1pt);
			\draw [line width=0.3,shift={(B)}] (90:0.32cm) arc (90:180:0.32cm);
			\fill [shift={(B)}] (135:0.18cm) circle (1pt);
			\draw[fill=black] (A) circle (2pt) node[shift={(220:2ex)}] {$A'$};
			\draw[fill=black] (B) circle (2pt) node[shift={(-40:2ex)}] {$B'$};
			\draw[fill=black] (C) circle (2pt) node[shift={(90:2ex)}] {$C'$};
			\draw[fill=white] (H) circle (2pt) node[shift={(-90:2ex)}] {$H$};
			\draw[fill=black] (P) circle (2pt) node[shift={(180:2ex)}] {$P'$};
			\draw[fill=black] (Q) circle (2pt) node[shift={(0:2ex)}] {$Q'$};
			\draw[fill=black] (S) circle (2pt) node[shift={(150:1.5ex)}] {$S'$};
			\draw[fill=black] (T) circle (2pt) node[shift={(20:2ex)}] {$T'$};
		\end{tikzpicture} & \\
		& vor Inversion & & nach Inversion & 
	\end{tabularx}
\end{figure}

\begin{proof}[Lösung zu Aufgabe~\ref{aufgabe:450943} \textmd{(\href{https://www.mathematik-olympiaden.de/moev/index.php?option=com_download&thema=a&format=raw&datei=A45094a.pdf}{MO 450943})}]
	Wir invertieren an~$H$. Die Bildpunkte werden wie üblich mit $A'$,~$B'$,~\ldots\ bezeichnet. Wegen $\winkel APH=90^\circ=\winkel HPC$ gilt $\winkel HA'P'=90^\circ= \winkel P'C'H$ nach Eigenschaft~\ref{itm:Winkel}. Weil die Geraden $CH$ und~$AB$ durch~$H$ gehen, werden sie auf sich selbst abgebildet. Also steht $C'H$ immer noch senkrecht auf~$A'B'$. Folglich ist $A'HC'P'$ ein Rechteck. Analog ist $HB'Q'C'$ ein Rechteck. Also ist auch $A'B'Q'P'$ ein Rechteck. Weil $S$ auf dem Umkreis $\odot AHP$ liegt, muss $S'$ auf der Geraden~$A'P'$ liegen. Analog liegt $T'$ auf~$B'Q'$. Weil schließlich $S$~und~$T$ auf~$PQ$ liegen, müssen $S'$~und~$T'$ auf dem Umkreis $\odot P'HQ'$ liegen. Nach Satz vom Sehnenviereck gilt dann $\winkel P'S'T'=180^\circ-\winkel T'Q'P'=180^\circ-90^\circ=90^\circ$. Also ist auch $P'Q'T'S'$ ein Rechteck. Folglich ist $S'T'$ parallel zu~$A'B'$. Das bedeutet aber genau, dass vor der Inversion der Umkreis $\odot HTS$ die Gerade~$AB$ in~$H$ berührt haben muss.
\end{proof}

\begin{proof}[Lösung zu Aufgabe~\ref{aufgabe:IMO1996} \textmd{(\href{https://artofproblemsolving.com/community/c3823_1996_imo}{IMO 1996/2})}]
	Seien $X$~und~$Y$ die Schnittpunkte der Winkelhalbierenden von $\winkel PBA$ und $\winkel ACP$ mit~$AP$.
	% Tien: Vielleicht ein "bzw.\" oder "respektive".
	Wir wollen $X=Y$ zeigen. Dazu invertieren wir an~$A$. Wie üblich werden die Bildpunkte unter der Inversion mit $B'$,~$C'$,~\ldots\ bezeichnet. Aus Eigenschaft~\ref{itm:Winkel} folgt dann $\winkel P'B'C'=\winkel P'B'A'-\winkel C'B'A'=\winkel APB-\winkel ACB$ und analog $\winkel B'C'P=\winkel CPA-\winkel CBA$. Die seltsame Winkelbedingung aus der Aufgabenstellung sagt uns also genau, dass das Dreieck $C'B'P'$ gleichschenklig ist.
	\begin{figure}[ht]
		\centering
		\begin{tabularx}{\textwidth}{X c X c X}
			& \begin{tikzpicture}
				\coordinate (A) at (0,0);
				\coordinate (B) at (3.069,0);
				\coordinate (C) at (3.788,3.506);
				\coordinate (P) at (2.637,1.374);
				\coordinate (X) at (1.794,0.935);
				\coordinate (Q) at (1.425,1.336);
				\draw (C) to (A) to (B) to (P) to (C) to (B);
				\draw [line width=0.3, shorten >=-4em] (A) to (P);
				\draw [line width=0.3, dashed, shorten <=-2ex] (X) to (B);
				\draw [line width=0.3, dashed, shorten <=-2ex] (X) to (C);
				\draw[fill=black] (A) circle (2pt) node[shift={(220:2ex)}] {$A$};
				\draw[fill=black] (B) circle (2pt) node[shift={(-40:2ex)}] {$B$};
				\draw[fill=black] (C) circle (2pt) node[shift={(0:2ex)}] {$C$};
				\draw[fill=black] (P) circle (2pt) node[shift={(-112:2.25ex)}] {$P$};
				\draw[fill=white] (X) circle (2pt) node[shift={(270:2.5ex)}] {$X$};
			\end{tikzpicture} & & \begin{tikzpicture}
				\coordinate (A) at (0,0);
				\coordinate (B) at (3.864,0);
				\coordinate (C) at (1.686,1.561);
				\coordinate (P) at (3.537,1.843);
				\coordinate (X) at (5.197,2.709);
				\draw (A) to (B) to (C) to cycle;
				\draw [line width=0.3,shorten >=-4ex] (P) to node[pos=0.5, sloped] {$\scriptscriptstyle|$} (X);
				\draw [line width=0.3] (A) to (P);
				\draw [line width=0.3] (C) to node[pos=0.5, sloped] {$\scriptscriptstyle|$} (P);
				\draw [line width=0.3] (P) to node[pos=0.5, sloped] {$\scriptscriptstyle|$} (B);
				\draw [line width=0.3] (C) to (X) to (B);
				\draw [line width=0.3,shift={(B)}] (100.056:0.32cm) arc (100.056:144.372:0.32cm);
				\draw [line width=0.3,shift={(C)}] (324.372:0.32cm) arc (324.372:368.688:0.32cm);
				\draw [line width=0.3,shift={(B)}] (63.793:0.42cm) arc (63.793:100.056:0.42cm);
				\draw [line width=0.3,shift={(B)}] (63.793:0.37cm) arc (63.793:100.056:0.37cm);
				\draw [line width=0.3,shift={(X)}] (207.53:0.42cm) arc (207.53:243.793:0.42cm);
				\draw [line width=0.3,shift={(X)}] (207.53:0.37cm) arc (207.53:243.793:0.37cm);
				\draw[fill=black] (A) circle (2pt) node[shift={(220:2ex)}] {$A$};
				\draw[fill=black] (B) circle (2pt) node[shift={(-40:2ex)}] {$B'$};
				\draw[fill=black] (C) circle (2pt) node[shift={(120:2ex)}] {$C'$};
				\draw[fill=black] (P) circle (2pt) node[shift={(-12:2.25ex)}] {$P'$};
				\draw[fill=white] (X) circle (2pt) node[shift={(90:2ex)}] {$X'$};
			\end{tikzpicture}
			& \\
			& vor Inversion & & nach Inversion & 
		\end{tabularx}
	\end{figure}
	
	Als nächstes untersuchen wir, wohin $X$ unter der Inversion geschickt wird. Weil $X$ auf~$AP$ liegt, liegt $X'$ auf der Geraden~$AP'$. Wir können sogar genauer sagen, dass $P'$ zwischen $A$~und~$X'$ liegt, weil vor der Inversion $X$ zwischen $A$~und~$P$ lag. Vor der Inversion galt $\winkel XBA=\frac12\winkel PBA$. Aus Eigenschaft~\ref{itm:Winkel} folgt also $\winkel AX'B'=\frac 12\winkel AP'B'$. Nach dem Außenwinkelsatz im Dreieck $P'B'X'$ gilt zudem $\winkel AP'B'=\winkel AX'B'+\winkel X'B'P'$. Also muss auch $\winkel X'B'P'=\frac 12\winkel AP'B'$ gelten. Folglich ist auch das Dreieck $P'B'X'$ gleichschenklig und wir erhalten $\abs{X'P'}=\abs{B'P'}=\abs{C'P'}$. Mit der gleichen Argumentation folgt auch $\abs{Y'P'}=\abs{B'P'}=\abs{C'P'}$, also $\abs{X'P'}=\abs{Y'P'}$. Wie wir bereits gesehen haben, liegen $X'$~und~$Y'$ auf der gleichen Seite von~$P'$. Also muss $X'=Y'$ sein. Es folgt $X=Y$, wie gewünscht.
\end{proof}

\begin{proof}[Lösung zu Aufgabe~\ref{aufgabe:521243} \textmd{(\href{https://www.mathematik-olympiaden.de/moev/index.php?option=com_download&thema=a&format=raw&datei=A52124a.pdf}{MO 521243})}]
	Bei dieser Aufgabe bringt es wenig, am gewünschten Berührpunkt zu invertieren, denn wir wissen ja (noch) gar nicht, wo der liegt. Stattdessen invertieren wir an~$X$. Überlegt euch zunächst selbst, wie die Skizze nach der Inversion aussieht (bzw.\ warum sie so wie unten aussieht)~-- ihr werdet feststellen, dass sich durch die Inversion eigentlich nichts geändert hat. Auf den ersten Blick wirkt es also, als hätten wir nichts erreicht. Auf den zweiten Blick haben wir jedoch durch diese Feststellung die Aufgabe schon fast gelöst!
	\begin{figure}[ht]
		\centering
		\begin{tabularx}{\textwidth}{X c X c X}
			& \begin{tikzpicture}[x=1.1cm,y=1.1cm]
				\clip (-1.77,-2.33) rectangle (4.49,1.65);
				\draw (0,0) circle (1);
				\draw (2,0) circle (1.62);
				\draw (1.852,-0.827) circle (1.028);
				\coordinate (Q) at (0.594,-0.805);
				\coordinate (X) at (-0.176,-0.984);
				\coordinate (P) at (3.617,-0.099);
				\coordinate (A) at (0.884,-1.174);
				\coordinate (B) at (2.64,-1.488);
				\coordinate (T) at (1.402,-1.752);
				\coordinate (D) at (2.033,0.185);
				\coordinate (E) at (0.913,-0.408);
				\draw [line width=0.3,shorten <=-4ex,shorten >=-4ex] (X) to (P);
				\draw [line width=0.3,shorten <=-4em,shorten >=-4em] (X) to node [pos=-0.42,sloped] {$\scriptscriptstyle/\!/$} (B);
				\draw [line width=0.3,shorten <=-5ex,shorten >=-10em] (P) to node [pos=3.14,sloped] {$\scriptscriptstyle/\!/$} (D);
				\draw [line width=0.3,dashed,shorten <=-7ex,shorten >=-6ex] (X) to (T);
				\draw[fill=black] (A) circle (2pt) node[shift={(50:2ex)}] {$A$};
				\draw[fill=black] (B) circle (2pt) node[shift={(112:2ex)}] {$B$};
				\draw[fill=black] (P) circle (2pt) node[shift={(50:2.25ex)}] {$P$};
				\draw[fill=black] (Q) circle (2pt) node[shift={(155:2.5ex)}] {$Q$};
				\draw[fill=black] (X) circle (2pt) node[shift={(265:2ex)}] {$X$};
				\draw[fill=white] (T) circle (2pt) node[shift={(270:2.5ex)}] {$T$};
				\draw[fill=white] (D) circle (2pt);
				\node[shift={(270:2.5ex)}] at (-0.65,0.5) {$\omega_1$};
				\node[shift={(270:2.5ex)}] at (2.95,1.3) {$\omega_2$};
				\node[shift={(270:2.5ex)}] at (2.6,-0.25) {$\omega$};
				\node[shift={(270:2.5ex)}] at (4.2,-0.1) {$\ell$};
			\end{tikzpicture} & & \begin{tikzpicture}[x=1.1cm,y=1.1cm]
				\clip (-1.77,-2.33) rectangle (4.49,1.65);
				\draw (0,0) circle (1);
				\draw (2,0) circle (1.62);
				\draw (1.852,-0.827) circle (1.028);
				\coordinate (Q) at (0.594,-0.805);
				\coordinate (X) at (-0.176,-0.984);
				\coordinate (P) at (3.617,-0.099);
				\coordinate (A) at (0.884,-1.174);
				\coordinate (B) at (2.64,-1.488);
				\coordinate (D) at (2.033,0.185);
				\coordinate (E) at (0.913,-0.408);
				\coordinate (T) at (1.402,-1.752);
				\draw [line width=0.3,shorten <=-4ex,shorten >=-4ex] (X) to (P);
				\draw [line width=0.3,shorten <=-4em,shorten >=-4em] (X) to node [pos=-0.42,sloped] {$\scriptscriptstyle/\!/$} (B);
				\draw [line width=0.3,shorten <=-5ex,shorten >=-10em] (P) to node [pos=3.14,sloped] {$\scriptscriptstyle/\!/$} (D);
				\draw[fill=black] (A) circle (2pt) node[shift={(40:2.25ex)}] {$B'$};
				\draw[fill=black] (B) circle (2pt) node[shift={(112:2.25ex)}] {$A'$};
				\draw[fill=black] (P) circle (2pt) node[shift={(50:2.5ex)}] {$Q'$};
				\draw[fill=black] (Q) circle (2pt) node[shift={(160:2.5ex)}] {$P'$};
				\draw[fill=black] (X) circle (2pt) node[shift={(265:2ex)}] {$X$};
				\node[shift={(270:2.5ex)}] at (T) {$\phantom{T}$};
				\draw[fill=white] (E) circle (2pt);
				\node[shift={(270:2.5ex)}] at (-0.7,0.5) {$\ell'$};
				\node[shift={(270:2.5ex)}] at (2.95,1.3) {$\omega_2'$};
				\node[shift={(270:2.5ex)}] at (2.55,-0.2) {$\omega'$};
				\node[shift={(270:2.5ex)}] at (4.2,-0.1) {$\omega_1'$};
			\end{tikzpicture}
			& \\
			& vor Inversion & & nach Inversion & 
		\end{tabularx}
	\end{figure}
	
	Um das klarer zu machen, invertieren wir ausnahmsweise nicht an einem beliebigen Kreis um~$X$, sondern an dem Kreis mit Radius $r=\sqrt{\abs{XA}\cdot\abs{XB}}$. Dann behaupten wir:
	\begin{enumerate}[label={$(\arabic*)$},ref={$(\arabic*)$}]\itshape
		\item \label{claim:AB}
		Der Punkt~$A$ wird unter der Inversion auf den Punkt~$B$ abgebildet und umgekehrt. Der Kreis~$\omega$ wird auf sich selbst abgebildet.
		\item \label{claim:PQ}
		Der Punkt~$P$ wird unter der Inversion auf den Punkt~$Q$ abgebildet und umgekehrt. Der Kreis~$\omega_2$ wird auf sich selbst abgebildet.
		\item \label{claim:omegaell}
		Der Kreis~$\omega_1$ wird unter der Inversion auf die Gerade~$\ell$ abgebildet und umgekehrt.
	\end{enumerate}
	Wegen $r^2=\abs{XA}\cdot\abs{XB}$ werden $A$~und~$B$ in der Tat aufeinander abgebildet. Um zu zeigen, dass $\omega$ auf sich selbst abgebildet wird, legen wir durch~$X$ eine Tangente an~$\omega$; der Berührpunkt sei~$T$. Nach dem Sehnen-Tangentensatz gilt $\abs{XT}^2=\abs{XA}\cdot\abs{XB}$, also $\abs{XT}=r$.
	% Tien: Das folgende habe ich etwas umgeschrieben
	Also liegt $T$ auf dem Inversionskreis und wird folglich auf sich selbst abgebildet. Dann wird auch der Umkreis~$\omega$ von $ABT$ auf sich selbst abgebildet, wie behauptet. Alternativ könnt ihr euch überlegen, dass sich $\omega$ und der Kreis um~$X$ mit Radius~$\overline{XT}$ senkrecht schneiden, sodass Eigenschaft~\ref{itm:Schnitt90} anwendbar ist. Das zeigt Behauptung~\ref{claim:AB}. Nach dem Sekantensatz gilt $\abs{XQ}\cdot \abs{XP}=\abs{XA}\cdot\abs{XB}=r^2$, also können wir Behauptung~\ref{claim:PQ} völlig analog zeigen. Weil $\omega_1$ ein Kreis durch $X$~und~$Q$ ist, ist sein Bild eine Gerade durch den Bildpunkt von~$Q$, also durch~$P$. Weil $\omega_1$ die Gerade~$AX$ in~$X$ berührt, muss sein Bild parallel zu~$A'X$ sein, also wird $\omega_1$ in der Tat auf~$\ell$ abgebildet. Nach Eigenschaft~\ref{itm:Involution} muss dann auch~$\ell$ auf~$\omega_1$ abgebildet werden. Das zeigt Behauptung~\ref{claim:omegaell}.
	
	Nach Voraussetzung berühren sich $\ell$~und~$\omega$. Also berühren sich auch ihre Bilder unter der Inversion. Nach \ref{claim:AB}~und~\ref{claim:omegaell} folgt also, dass sich $\omega_1$~und~$\omega$ berühren und wir sind fertig.
\end{proof}

\begin{proof}[Lösung zu Aufgabe~\ref{aufgabe:Ptolemaeus}]
	Wir invertieren an einem Kreis mit Radius~$r$ um~$A$ und bezeichnen wie üblich die Bildpunkte mit $B'$,~$C'$ und~$D'$. Nach der Dreiecksungleichung gilt $\abs{B'C'}+\abs{C'D'}\geqslant \abs{B'D'}$. Indem wir die Längenformel (siehe den Unterabschnitt \emph{Strategien, Tipps und Tricks für Inversionslösungen}) erhalten wir
	\begin{equation*}
		\frac{r^2}{\abs{AB}\cdot\abs{AC}}\cdot \abs{BC}+\frac{r^2}{\abs{AC}\cdot\abs{AD}}\cdot \abs{CD}\geqslant \frac{r^2}{\abs{AB}\cdot\abs{AD}}\cdot\abs{BD}\,.
	\end{equation*}
	Indem wir diese Ungleichung mit $\abs{AB}\cdot\abs{AC}\cdot \abs{AD}/r^2$ multiplizieren erhalten wir sofort die Ungleichung von Ptolemäus. Gleichheit gilt genau dann, wenn in der Dreiecksungleichung Gleichheit gilt, also genau dann, wenn $C'$ auf der Strecke~$\overline{B'D'}$ liegt. Aber $C'$ liegt genau dann auf der Geraden~$B'D'$, wenn $ABCD$ ein Sehnenviereck ist, und die Bedingung, dass $C'$ zwischen $B'$~und~$D'$ liegt, ist genau dazu äquivalent, dass das Sehnenviereck $ABCD$ nicht-überschlagen (also konvex) ist.
\end{proof}
	
\subsection*{Lösungen zu Kapitel~\ref{kapitel:Graphentheorie}: \emph{Graphentheorie}}
\begin{proof}[Lösung zu Aufgabe~\ref{aufgabe:Handschlagslemma}]
	Wir starten mit dem gerichteten Graphen: Weil jede Kante genau einen Ausgangsknoten hat, gilt $\sum_{v\in V}d^+(v)=\abs{E}$. Genauso hat jede Kante genau einen Eingangsknoten, also $\sum_{v\in V}d^-(v)=\abs{E}$. Für einen ungerichteten Graphen können wir eine beliebige Ausrichtung der Kanten festlegen und folgern $\sum_{v\in V}d(v)=\sum_{v\in V}\parens*{d^+(v)+d^-(v)}=2\abs{E}$.
\end{proof}
\begin{proof}[Lösung zu Aufgabe~\ref{aufgabe:Blatt}]
	Angenommen, es gäbe einen Baum ohne Blätter. Falls ein Knoten $v\in V$ mit $d(v)=0$ existiert, dann muss $\{v\}$ eine eigene Zusammenhangskomponente bilden. Bäume sind aber per Definition zusammenhängend, also kann in diesem Fall nur $V=\{v\}$ gelten. Wenn für $\abs*{V}\geqslant 2$ keine Blätter existieren, dann muss somit $d(v)\geqslant 2$ für alle Knoten $v\in V$ gelten. Wir können also an einem beliebigen Knoten $v_1\in V$ starten und einen Weg $v_1v_2\dots v_n$ formen, indem wir von $v_n$ immer zu einem benachbarten Knoten $v_{n+1}\neq v_{n-1}$ gehen. Allerdings müssen wir in unserem endlichen Graphen irgendwann an einem Knoten ankommen, den wir schon einmal besucht haben Dann haben wir aber einen Kreis gefunden, Widerspruch! Damit ist~\ref{teilaufgabe:BaumHatBlaetter} gezeigt.
	
	Für~\ref{teilaufgabe:BaumKnotenKanten} machen wir eine Induktion über die Anzahl der Knoten. Für $\abs*{V}=1$ ist die Aussage trivial. Für~$\abs*{V}\geqslant 2$ finden wir nach~\ref{teilaufgabe:BaumHatBlaetter} ein Blatt~$v$. Sei~$G'=(V',E')$ der Graph, der entsteht, indem wir das Blatt~$v$ abschneiden (also~$v$ nebst seiner angrenzenden Kante löschen). Wir behaupten, dass~$G'$ immer noch ein Baum ist. Durch das Löschen von Kanten können keine Kreise entstehen, also ist $G'$ kreisfrei. Jeder Weg in~$G$, der durch~$v$ läuft aber nicht in~$v$ beginnt oder endet, muss die Folge $\ldots uvu\ldots$ enthalten, wobei $u$ der eindeutige Nachbarknoten von $v$ ist. Indem wir für jedes Auftreten von~$v$ diese Folge durch $\ldots u\ldots $ ersetzen, erhalten wir einen Weg, der komplett in $G'$ verläuft. Also ist $G'$ immer noch zusammenhängend. Somit ist $G'$ tatsächlich ein Baum.
	
	Beim Abschneiden von~$v$ verringern sich $\abs*{V}$ und $\abs{E}$ jeweils um~$1$, also ist $\abs*{V}=\abs{V'}+1$ und $\abs*{E}=\abs{E'}+1$. Nach Induktionsvoraussetzung ist $\abs{E'}=\abs{V'}-1$, also auch $\abs*{E}=\abs{V}-1$. Damit ist die Induktion beendet.
	
	Um zu zeigen, dass~$G$ mindestens zwei Blätter enthält, benutzen wir das Handschlagslemma: Wäre das nicht der Fall, so wäre $2\abs*{E}=\sum_{v\in V}d(v)\geqslant 2\abs*{V}-1$ somit $\abs*{E}\geqslant \abs*{V}-\frac12$. Das ist ein Widerspruch zu $\abs*{E}=\abs*{V}-1$.
\end{proof}
\begin{proof}[Lösung zu Aufgabe~\ref{aufgabe:Bipartit}]
	Es ist klar, dass für jeden Kreis in~$G$ die Knoten abwechselnd zu~$A$ oder~$B$ gehören müssten. Für ungerade Kreise geht das jedoch nicht auf, also können bipartite Graphen keine ungeraden Kreise enthalten.
	
	Nehmen wir umgekehrt an, dass~$G$ keinen ungeraden Kreis enthält. Es genügt, den Fall zu betrachten, dass~$G$ zusammenhängend ist, denn im allgemeinen Fall können wir in jeder Zusammenhangskomponente einzeln eine Bipartition konstruieren. Für zwei Knoten $u,v\in V$ definieren wir den \emph{Abstand} zwischen~$u$ und~$v$ als die minimale Länge eines Weges von~$u$ nach~$v$. Wähle einen beliebigen Knoten $v_0\in V$ definiere~$A$ als die Menge aller Knoten $v\in V$, die geraden Abstand von~$v_0$ haben und~$B$ als die Menge aller Knoten $v\in V$, die ungeraden Abstand von~$v_0$ haben. Wir behaupten, dass es dann keine Kanten innerhalb von~$A$ oder~$B$ geben kann, sodass wir eine Bipartition gefunden hätten.
	
	Nehmen wir umgekehrt an, es gäbe eine Kante $v_1v_2$ zwischen zwei Knoten $v_1,v_2\in A$. Betrachte minimale Wege $W_1$ und $W_2$ von~$v_0$ nach~$v_1$ und $v_2$. Dann bilden $W_1$, $v_1v_2$ und $W_2$ einen geschlossenen Weg von ungerader Länge. Wenn $W_1$ und $W_2$ keinen Knoten außer $v_0$ gemeinsam haben, ist dieser geschlossene Weg auch ein Kreis und wir haben unseren Widerspruch. Sonst gehen wir wie folgt vor: Betrachte den ersten Knoten~$u$, den $W_1$ und $W_2$ von~$v_0$ aus gesehen gemeinsam haben. Dann bilden die beiden Wegabschnitte von $v_0$ nach $u$ einen geschlossenen Weg. Dieser muss sogar ein Kreis sein. Denn nach Minimalität von $u$ haben die beiden Wegabschnitte keinen Knoten außer $v_0$ und~$u$ gemeinsam. Innerhalb der Wegabschnitte können auch keine Knoten mehrfach besucht werden, denn $W_1$ und $W_2$ sind Wege von minimaler Länge. Also bilden die beiden Wegabschnitte von $v_0$ nach $u$ in der Tat einen Kreis. Dieser muss gerade Länge haben. Entferne nun diesen Kreis und wiederhole das Argument mit $u$ statt $v_0$. In jedem Schritt wird ein gerader Kreis entfernt, also hat der übrigbleibende geschlossene Weg immer noch ungerade Länge. Nach endlich vielen Schritten sind wir in einer Situation angelangt, in der die übrigbleibenden Wegabschnitte keinen Knoten außer dem Anfangsknoten gemeinsam haben. Dann erhalten wir also einen ungeraden Kreis und somit einen Widerspruch.
	
	Völlig analog lässt sich zeigen, dass innerhalb von~$B$ keine Kanten verlaufen können.
\end{proof}
\begin{proof}[Lösung zu Aufgabe~\ref{aufgabe:Schlicht}]
	Sei $n=\abs*{V}$ die Anzahl der Knoten von~$G$. Weil~$G$ keine Schleifen und keine Mehrfachkanten hat, muss $d(v)\leqslant n-1$ sein. Wenn alle Knotengrade verschieden wären, dann müsste jeder Grad $0,1,\dotsc,n-1$ genau einmal vertreten sein. Sei $v_0$ der Knoten mit $d(v_0)=0$ und $v_{n-1}$ der Knoten mit $d(v_{n-1})=v_{n-1}$. Dann ist $v_0$ mit keinem anderen Knoten verbunden, aber $v_{n-1}$ ist mit allen anderen Knoten verbunden. Widerspruch!
\end{proof}
\begin{proof}[Lösung zu Aufgabe~\ref{aufgabe:Euler-Hierholzer}]
	Angenommen, $G$ enthält einen Eulerweg $W$. Für jedes $v\in V$, bis auf den Anfangs- und den Endknoten von $W$, läuft $W$ genauso oft nach~$v$ rein wie aus~$v$ raus. Also ist $d(v)$ gerade. Folglich können höchstens der Anfangs- und der Endknoten von~$W$ ungeraden Grad haben. Wenn der Eulerweg geschlossen ist, dann muss auch der Anfangs($=$End)knoten von~$W$ geraden Grad haben.
	
	Betrachte nun einen zusammenhängenden Graphen~$G$, in dem alle Knoten geraden Grad haben. Wir zeigen per Induktion über $\abs*{E}$, dass $G$ einen geschlossenen Eulerweg besitzt. Der Fall $\abs*{E}=0$ ist trivial, denn dann enthält~$G$ höchstens einen Knoten (sonst wäre der Graph nicht zusammenhängend). Für den Induktionsschritt sei $\abs*{E}\geqslant 1$ und wir nehmen an, dass die Aussage für zusammenhängende Graphen mit $\leqslant\abs*{E}-1$ Kanten gilt. Sei $W$ ein geschlossener Weg in~$G$, der jede Kante höchstens einmal durchläuft (so einen Weg finden wir, indem wir von einem Knoten immer weiter laufen, bis wir das erste Mal an einem Knoten ankommen, den wir schon mal besucht haben). Sei $G'$ der Graph, der entsteht, wenn wir die Kanten von~$W$ löschen. Für jeden Knoten löschen wir eine gerade Anzahl angrenzender Kanten, also haben die Knoten von~$G'$ immer noch geraden Grad. Seien $G_1,G_2,\dotsc,G_k$ die Zusammenhangskomponenten von $G'$. Nach Induktionsvoraussetzung hat jedes $G_i$ einen geschlossenen Eulerweg $W_i$. Nun gehen wir~$W$ entlang. Wenn wir das erste Mal auf einen Knoten aus $W_i$ für irgendein $i=1,2,\dotsc,k$ treffen, dann laufen wir einmal durch~$W_i$. Danach laufen wir auf~$W$ weiter. Auf diese Weise erhalten wir einen geschlossenen Eulerweg für~$G$.
	
	Betrachte zuletzt den Fall, dass~$G$ zusammenhängend ist und höchstens zwei Knoten ungeraden Grades enthält. Nach dem Handschlagslemma enthält~$G$ gerade viele solche Knoten, also entweder~$0$ oder~$2$. Den ersten Fall haben wir gerade abgehandelt. Im zweiten Fall seien~$u$ und~$v$ die beiden ungeraden Knoten. Wenn wir eine zusätzliche Kante $uv$ einfügen, haben alle Knoten geraden Grad und ein geschlossener Eulerweg existiert. Nachdem wir $uv$ wieder löschen, erhalten wir einen (nicht mehr geschlossenen) Eulerweg in~$G$.
\end{proof}
\begin{proof}[Lösung zu Aufgabe~\ref{aufgabe:Polyeder}]
	Wir machen eine Induktion über die Anzahl der Kreise in $G$. Wenn $G$ keinen Kreis hat, dann ist jede Zusammenhangskomponente $G_i=(V_i,E_i)$ von $G$ ein Baum. Damit ist $\abs*{F}=1$ (nur die unendlich große äußere Fläche existiert) und wegen Aufgabe~\ref{aufgabe:Blatt} gilt $\abs{V_i}-1=\abs{E_i}$ für jede Zusammenhangskomponente $G_i$. Dann folgt
	\begin{equation*}
		\abs*{V}-\abs*{E}+\abs*{F}=\sum_{G_i\in Z}\parens[\big]{\abs{V_i}-\abs{E_i}}+1=\sum_{G_i\in Z}1+1=\abs*{Z}+1
	\end{equation*}
	Jetzt nehmen wir an, dass~$G$ mindestens einen Kreis enthält und dass die Aussage für Graphen mit weniger Kreisen wahr ist. Wähle eine Kante $e$ aus einem Kreis~$C$ aus und lösche sie. Der resultierende Graph $G'=(V',E')$ ist immer noch planar, hat weniger Kreise als $G$, aber immer noch die gleiche Anzahl von Zusammenhangskomponenten. Offenbar gilt $\abs{E'}=\abs{E}-1$. Außerdem gilt $ F' = F-1$, weil zwei Flächen die vorher durch $e$ getrennt waren, zu einer Fläche geworden sind. Nach Induktionsvoraussetzung ist $\abs{V}-\abs{E}+\abs*{F}=\abs{V'}-\abs{E'}+\abs*{F'}=\abs*{Z}+1$ und wir sind fertig.
\end{proof}
\begin{proof}[Lösung zu Aufgabe~\ref{aufgabe:Unplanar}]
	Angenommen, $K_5$ wäre planar. Nach dem Eulerschen Polyedersatz wäre dann $2=\abs{V}-\abs{E}+\abs*{F}=5-\binom{5}{2}+\abs*{F}$, also $\abs*{F}=7$. Jede Fläche wird von mindestens~$3$ Kanten begrenzt, denn alle Kreise in~$K_5$ haben mindestens Länge~$3$. Andererseits grenzt jede Kante an genau~$2$ Flächen. Es folgt $3\abs*{F}\leqslant 2\abs*{E}$. Durch Einsetzen erhalten wir dann aber $21=3\abs*{F}\leqslant 2\abs{E}=20$, Widerspruch!
	
	Angenommen, $K_{3,3}$ wäre planar. Dann müsste $2=\abs{V}-\abs{E}+\abs*{F}=6-9+\abs*{F}$ gelten, also $\abs*{F}=5$. Jeder Kreis in~$K_{3,3}$ hat mindestens die Länge~$4$, also wird jede Fläche von mindestens~$4$ Kanten begrenzt. Andererseits grenzt jede Kante an genau zwei Flächen. Es folgt $4\abs*{F}\leqslant 2\abs{E}$. Durch Einsetzen erhalten wir aber $20=4\abs*{F}\leqslant 2\abs{E}=18$, Widerspruch!
\end{proof}
\begin{proof}[Lösung zu Aufgabe~\ref{aufgabe:Dirac}]
	Angenommen, die Aussage ist falsch. Dann gibt es ein auch Gegenbeispiel~$G=(V,E)$ mit maximaler Anzahl Kanten. Wähle zwei Knoten $u,v\in V$ die nicht benachbart sind (wären alle Knoten benachbart, gäbe es sicher einen Hamiltonkreis) Nach Maximalität von~$G$ existiert ein Hamiltonkreis $C$, sobald wir die Kante $uv$ hinzufügen. Dann muss $C$ die Kante $uv$ enthalten, denn sonst wäre~$C$ schon in~$G$ enthalten. Wir können also annehmen, dass $C$ von der Form $C=uv_1v_2\dotso v_{n-2}vu$ ist. Insbesondere ist $uv_1v_2\dots v_{n-2}v$ immer noch ein Pfad, der jeden Knoten genau einmal durchläuft. 
	\begin{figure}[ht]
		\centering
		\begin{tikzpicture}
			\coordinate (a) at (-5,0);
			\coordinate (b) at (-4,0);
			\coordinate (c) at (-3,0);
			\coordinate (d) at (-2,0);
			\coordinate (e) at (-1,0);
			\coordinate (f) at (0,0);
			\coordinate (g) at (1,0);
			\coordinate (h) at (2,0);
			\coordinate (i) at (3,0);
			\coordinate (j) at (4,0);
			\coordinate (k) at (5,0);
			\coordinate (l) at (6,0);
			\draw (a) to (b) to (c) to (d) to (e) to (f) to (g) to (h) to (i) to (j) to (k) to (l);
			\draw[dashed] (a) to[bend left=45]  (g);
			\draw[dashed] (f) to[bend left=45] (l);
			\draw[fill=white] (a) circle (2pt) node[shift={(270:2ex)}] {$u\vphantom{_1}$};
			\draw[fill=black] (b) circle (2pt) node[shift={(270:2ex)}] {$v_1$};
			\draw[fill=black] (c) circle (2pt);
			\draw[fill=black] (d) circle (2pt);
			\draw[fill=black] (e) circle (2pt);
			\draw[fill=black] (f) circle (2pt) node[shift={(270:2ex)}] {$v_{k-1}$};
			\draw[fill=black] (g) circle (2pt) node[shift={(270:2ex)}] {$v_k$};
			\draw[fill=black] (h) circle (2pt);
			\draw[fill=black] (i) circle (2pt);
			\draw[fill=black] (j) circle (2pt);
			\draw[fill=black] (k) circle (2pt) node[shift={(270:2ex)}] {$v_{n-2}$};
			\draw[fill=white] (l) circle (2pt) node[shift={(270:2ex)}] {$v\vphantom{_1}$};
		\end{tikzpicture}
	\end{figure}
	
	Wenn wir ein Paar von Kanten $uv_k$ und $vv_{k-1}$ finden, dann sehen wir, dass $G$ den Hamiltonkreis $uv_kv_{k+1}\dots v_{n-2}vv_{k-1}v_{k-2}\dots v_1u$ enthält, Widerspruch! Allerdings hat $u$ in $v_2,v_3,\dots,v_{n-2}$ mindestens $\frac{n}{2}-1$ benachbarte Knoten. Das heißt, es gibt mindestens $\frac{n}{2}-1$ Knoten in $v_1,v_2,\dots, v_{n-3}$, die nicht mit $v$ benachbart sein dürfen. Dan kann $v$ jedoch höchstens $(n-2)-\parens[\big]{\frac{n}{2}-1}=\frac{n}{2}-1$ Nachbarn in $v_1,v_2,\dots, v_{n-2}$ haben. Das steht im Widerspruch zur Annahme $d(v)\geqslant \frac n2$.
\end{proof}
Das gleiche Argument zeigt, dass statt der Annahme $d(v)\geqslant \frac n2$ schon die schwächere Voraussetzung $d(u)+d(v)\geqslant n$ für alle $u,v\in V$ ausreicht. Diese Verschärfung ist als \emph{Satz von Ore} bekannt.
	\subsection*{Lösungen zu den Kapitel~\ref{kapitel:Algorithmen}: \emph{Algorithmen in der Kombinatorik}}

\begin{proof}[Lösung zu Aufgabe~\ref{aufgabe:510846}]
	Im Folgenden bezeichnen wir das Feld in der $i$-ten Zeile und $j$-ten Spalte mit $(i,j)$. Um die Aufgabe zu lösen, werden wir schrittweise Zahlen in die Tabelle eintragen. Dazu durchlaufen wir der Reihe nach alle Felder (die Reihenfolge ist egal, es kommt nur darauf an, dass jedes Feld genau einmal durchlaufen wird). Wenn das Feld $(i,j)$ an der Reihe ist, tragen wir dort die größtmögliche Zahl ein. Das bedeutet: Wenn $z_i'$ und $s_j'$ die Summen der Zahlen sind, die zu diesem Zeitpunkt schon in der $i$-ten Zeile und in der $j$-ten Spalte eingetragen wurden, tragen wir die Zahl $\min\{z_i-z_i',s_j-s_j'\}$ ins Feld $(i,j)$ ein.
	
	Offensichtlich ist jeder dieser Schritte durchführbar und der Algorithmus endet, nachdem alle $mn$ Felder durchlaufen wurden.
	
	Wir werden nun beweisen, dass wir am Ende eine korrekt ausgefüllte Tabelle erhalten. Zuerst zeigen wir, dass alle Einträge nichtnegativ sind. Dazu bemerken wir:
	\begin{enumerate}[label={$(\arabic*)$},ref={$(\arabic*)$}]\itshape
		\item Nach jedem Schritt gilt: Wenn $z_k'$ die Summe der Einträge in der $k$-ten Zeile und $s_\ell'$ die Summe der Einträge in der $\ell$-ten Spalte bezeichnet, dann ist $z_k'\leqslant z_k$ und $s_\ell'\leqslant s_\ell$ für alle  $k=1,2,\dotsc,m$ und alle $\ell=1,2,\dotsc,n$.\label{eigenschaft:NichtnegativeEintraege}
	\end{enumerate}
	Zu Beginn ist das offensichtlich erfüllt. In dem Schritt, in dem das Feld~$(i,j)$ durchlaufen wird, ändern sich nur potentiell $z_i'$ und $s_j'$. Aber unsere Konstruktion ist genau so beschaffen, dass auch nach diesem Schritt noch $z_i'\leqslant z_i$ und $s_j'\leqslant s_j$ gilt (und in mindestens einer der Ungleichungen gilt dann sogar Gleichheit). Damit ist~\ref{eigenschaft:NichtnegativeEintraege} gezeigt. Aus~\ref{eigenschaft:NichtnegativeEintraege} folgt sofort, dass wir in jedem Schritt eine nichtnegative Zahl in die Tabelle eintragen.
	
	Als nächstes behaupten wir, dass zum Schluss alle Zeilen- und Spaltensummen stimmen:
	\begin{enumerate}[resume,label={$(\arabic*)$},ref={$(\arabic*)$}]\itshape
		\item Zum Schluss gilt $z_k'=z_k$ und $s_\ell'=s_\ell$ für alle $k=1,2,\dotsc,m$ und alle $\ell=1,2,\dotsc,n$.\label{eigenschaft:RichtigeZeilenSpaltenSummen}
	\end{enumerate}
	Betrachte die $k$-te Zeile. Aus unserer Konstruktion folgt: Nachdem das Feld~$(k,j)$ durchlaufen wurde, muss in mindestens einer der Ungleichungen $z_k'\leqslant z_k$ und $s_j'\leqslant s_j$ Gleichheit gelten. Wenn ersteres für irgendein $j=1,2,\dotsc,n$ der Fall ist, dann muss auch am Ende $z_k'=z_k$ gelten. Wenn nicht, dann gilt am Ende $s_j'=s_j$ für alle $j$. Mit~\ref{eigenschaft:NichtnegativeEintraege} folgt, dass am Ende 
	\begin{equation*}
		z_1'+z_2'+\dotsb+z_m'\leqslant z_1+z_2+\dotsb+z_m=s_1+s_2+\dotsb+s_n=s_1'+s_2'+\dotsb+s_n'
	\end{equation*}
	gilt. Am Ende gilt aber auch $z_1'+z_2'+\dotsb+z_m'=s_1'+s_2'+\dotsb+s_n'$, denn beide Seiten sind genau die Summe aller Einträge in der Tabelle. Also muss auch in diesem Fall in der Ungleichung $z_k'\leqslant z_k$ Gleichheit gelten. Damit ist gezeigt, dass am Ende $z_k'=z_k$ -- mit anderen Worten, alle Zeilensummen stimmen. Ein analoges Argument kann für die Spalten durchgeführt werden.
	
	Es bleibt zu zeigen:
	\begin{enumerate}[resume,label={$(\arabic*)$},ref={$(\arabic*)$}]\itshape
		\item In maximal $m+n-1$ Schritten wurde eine positive Zahl in die Tabelle eingetragen.\label{eigenschaft:PositiveEintraege}
	\end{enumerate}
	Wenn unser Algorithmus in das Feld~$(i,j)$ eine positive Zahl eingetragen hat, dann muss vor diesem Schritt $z_i'<z_i$ und $s_j'<s_j$ gewesen sein. Nach dem Schrit gilt hingegen mindestens eine der Gleichheiten $z_i'=z_i$ oder $s_j'=s_j$. Nach $m+n-1$ Schritten von dieser Sorte muss nach Schubfachprinzip $z_i'=z_i$ für alle $i=1,2,\dotsc,m$ oder $s_j'=s_j$ für alle $j=1,2,\dotsc,n$ gelten. Die Summe aller Zahlen, die zu diesem Zeitpunkt in der Tabelle eingetragen sind, ist dann aber genauso groß wie am Schluss. Also können keine weiteren positiven Zahlen eingetragen werden.
\end{proof}

\begin{proof}[Lösung zu Aufgabe~\ref{aufgabe:520945}]
	Wenn es zwei Zettel $Z_1$ und $Z_2$ gibt, sodass alle Zahlen von~$Z_1$ auch auf~$Z_2$ stehen, dann kann Basti den Zettel~$Z_2$ getrost schreddern. Denn egal, welche Zahl von~$Z_1$ er auf seinen eigenen Zettel schreibt, diese Zahl wird auch auf~$Z_2$ stehen. Dadurch bleibt Bedingung~\ref{bedingung:EineZahlVonJedemZettel} auf jeden Fall erhalten. Ferner genügt es offensichtlich, wenn~\ref{bedingung:KleinsteZahlVonEinemZettel} für einen anderen Zettel außer~$Z_2$ erfüllt ist.
	
	Basti schreddert nun so lange Zettel, bis kein schredderbarer Zettel mehr übrig ist. Dann wählt er einen Zettel~$Z$, auf dem das Minimum~$x$ aller verbleibenden Zettel steht, und schreibt dieses Minimum auf seinen Zettel. Für jeden verbleibenden Zettel löscht Basti alle Zahlen, die auch auf~$Z$ stehen. Dabei kann es nicht passieren, dass alle Zahlen gelöscht werden, denn sonst hätte der Zettel~$Z$ geschreddert werden können. Nun schreibt Basti von jedem verbleibenden Zettel eine beliebige verbleibende Zahl auf seinen Zettel. Diese Zahlen sind nicht in~$Z$ enthalten, also auf jeden Fall größer als~$x$. Damit ist~\ref{bedingung:KleinsteZahlVonEinemZettel} erfüllt. Nach Konstruktion gilt~\ref{bedingung:EineZahlVonJedemZettel} ebenfalls, also hat Basti seine Aufgabe gelöst.
\end{proof}
\begin{proof}[Lösung zu Aufgabe~\ref{aufgabe:531046}]
	Wir zeigen, dass eine solche Zerlegung stets möglich ist, indem wir sie schrittweise konstruieren. In jedem Schritt betrachten wir die kleinste Zahl $n$, die noch nicht in einer $ab$-normalen Teilmenge enthalten ist. Dann fügen wir, wenn möglich, die $ab$-normale Teilmenge $\{n,n+a,n+a+b\}$ hinzu. Wenn das nicht möglich ist, fügen wir stattdessen die $ab$-normale Teilmenge $\{n,n+b,n+a+b\}$ hinzu.
	
	Zeigen wir zunächst, dass diese Schritte stets durchführbar sind (bei den bisherigen Aufgaben war das quasi trivial, hier ist es der schwierigste Teil der Lösung). Angenommen, das wäre nicht so. Betrachte den ersten Schritt, der schief geht. Dann kann $n+a+b$ zu diesem Zeitpunkt noch in keiner $ab$-normalen Teilmenge enthalten sein, der Schritt kann also nur scheitern, indem sowohl $n+a$ als auch $n+b$ schon in einer $ab$-normalen Teilmenge enthalten sind. Denn angenommen, $n+a+b$ wäre zu diesem Zeitpunkt schon abgedeckt. Dann müsste $n+a+b$ in einem vorherigen Schritt von einer $ab$-normalen Teilmenge $A$ abgedeckt worden sein. Es ist klar, dass $n+a+b$ nicht die größte Zahl in $A$ sein kann, denn sonst wäre $n$ zwangsläufig die kleinste Zahl in $A$, aber $n$ ist noch nicht abgedeckt. Wenn aber $n+a+b$ nicht die größte Zahl in $A$ ist, dann muss $\min A> n$ sein. Das widerspricht aber unserer Vorgehensweise, nach der wir in jedem Schritt die kleinste nicht-abgedeckte Zahl auswählen.
	
	Somit muss es zum Zeitpunkt des Scheiterns zwei $ab$-normale Teilmengen $B$ und $C$ geben, die $n+a$ und $n+b$ abdecken (der Fall $B=C$ kann eintreten, wirkt sich aber nicht auf unser Argument aus). Dann muss $n+b$ die größte Zahl in $C$ sein. Denn wäre $n+b$ die zweitgrößte oder gar die kleinste Zahl in $C$, dann müsste $\min C>n$ sein, was wiederum unserer Vorgehensweise widerspricht. Aus $n+b=\max C$ folgt nun $C=\{n-a,n-a+b,n+b\}$. Zum Zeitpunkt des Scheiterns ist $n$ noch nicht abgedeckt, also war $n$ auch noch nicht abgedeckt, als $C$ hinzugefügt wurde. Gemäß unserer Vorgehensweise hätten wir dann aber statt $C$ die $ab$-normale Teilmenge $C'=\{n-a,n,n+b\}$ hinzugefügt, Widerspruch! Es folgt, dass alle Schritte durchführbar sind.
	
	In diesem Fall terminiert unser Algorithmus nicht nach endlicher Zeit (denn wir wollen ja $\mathbb Z_{>0}$ in unendlich viele $ab$-normale Teilmengen zerlegen). Aus der Vorgehensweise ist aber trotzdem klar, dass jede positive ganze Zahl irgendwann abgedeckt wird und dass der Algorithmus schrittweise eine Zerlegung mit den gewünschten Eigenschaften konstruiert.
\end{proof}
\begin{proof}[Lösung zu Aufgabe~\ref{aufgabe:541143}]
	Wir beginnen mit einer beliebigen Verteilung auf die beiden Busse. Dann führen wir eine Reihe von Schritten durch, um die Verteilung unseren Wünschen entsprechend umzubauen. In jedem Schritt wählen wir eine Person, die in ihrem jetzigen Bus mehr Bekanntschaften hat als in dem anderen Bus, und setzen diese Person stattdessen in den anderen Bus. Sobald keine solche Person mehr existiert, hören wir auf.
	
	Offensichtlich sind diese Schritte durchführbar. Weil in jedem Schritt die Summe der Bekanntschaften innerhalb der beiden Busse kleiner wird, können nur endlich viele Schritte durchgeführt werden. Danach sind wir in einer Situation, in der jede Person in ihrem Bus höchstens so viele Bekanntschaften hat wie in dem anderen Bus.
	
	Statt mit einem Algorithmus hätten wir auch eleganter mit dem Extremalprinzip argumentieren können: Wir betrachten einfach eine Situation, in der die Summe der Bekanntschaften innerhalb der beiden Busse minimal ist. Dann ist ebenfalls klar, dass jede Person in ihrem Bus höchstens so viele Bekanntschaften hat wie in dem anderen Bus, sonst könnte sie einfach den Bus wechseln.
	
	Es bleibt zu zeigen, dass in der beschriebenen Situation die gewünschte Bedingung erfüllt ist. Dazu stellen wir uns die Bekanntschaften als Graph $G=(V,E)$ vor, wobei $V$ die Menge aller Teilnehmenden ist und $E$ die Menge aller paarweisen Bekanntschaften. Nach Annahme gilt dann $k=\abs{E}$. Die Verteilung auf die Busse definiert eine disjunkte Zerlegung $V=A\cup B$ der Teilnehmenden sowie eine disjunkte Zerlegung $E=E_A\cup E_B\cup E_{AB}$, wobei $E_A$ die Menge der Bekanntschaften innerhalb von $A$, $E_B$ die Menge der Bekanntschaften innerhalb von $B$ und $E_{AB}$ die Menge der Bekanntschaften zwischen $A$ und $B$ bezeichnet. Wie üblich bezeiche $d(v)$ den Grad von $v\in V$, also die Anzahl der Bekanntschaften der Person. Ferner bezeichnen wir mit $d_A(v)$ und $d_B(v)$ die Anzahl der Bekanntschaften von $v$ innerhalb von $A$ bzw.\ $B$. Nach Konstruktion gilt dann $d_A(a)\leqslant d_B(a)$ für alle $a\in A$. Es folgt
	\begin{equation*}
		\sum_{a\in A}d_A(a)\leqslant \sum_{a\in A}d_B(a)\,.
	\end{equation*}
	Nach dem Handschlagslemma (siehe Aufgabe~\ref{aufgabe:Handschlagslemma} im Kapitel zu Graphentheorie) ist die Summe auf der linken Seite genau $2\abs{E_A}$, weil jede Kante innerhalb von $A$ doppelt gezählt wird. Die Summe auf der rechten Seite ist genau $\abs{E_{AB}}$, weil jede Kante zwischen $A$ und $B$ genau einmal gezählt wird. Also erhalten wir die Ungleichung $2\abs{E_A}\leqslant \abs{E_{AB}}$ Zusammen mit der Ungleichung $\abs{E_A}+\abs{E_{AB}}\leqslant \abs{E_A}+\abs{E_{AB}}+\abs{E_B}=\abs{E}=k$ folgt daraus $\abs{E_A}\leqslant \frac k3$, wie gewünscht. Mit einem analogen Argument können wir $\abs{E_B}\leqslant \frac k3$ zeigen.
\end{proof}
	\subsection*{Lösungen zu Kapitel~\ref{kapitel:Diophantastisch}}


\begin{proof}[Lösung zu Aufgabe~\ref{aufgabe:521235}]
	Wir betrachten die Gleichung modulo~$9$. Weil Kubikzahlen modulo~$9$ nur die Reste $-1$, $0$ und $1$ annehmen, kann $19x^3-17y^3\equiv x^3+y^3\mod 9$ nur die Reste $-2$, $-1$, $0$, $1$ oder $2$ annehmen. Es gilt jedoch $50\equiv 5\mod 9$. Also hat die Gleichung keine Lösungen.
\end{proof}

\begin{proof}[Lösung zu Aufgabe~\ref{aufgabe:Modulo11}]
	Auch diese Gleichung hat keine Lösungen. Um das zu zeigen, suchen wir eine Zahl~$m$, sodass sowohl Quadratzahlen als auch fünfte Potenzen möglichst wenige Reste modulo~$m$ annehmen. Wie wir im Theorieteil des Kapitels gesehen haben, sollte dafür $\varphi(m)$ durch $2$ und durch $5$ teilbar sein. Also ist $m=11$ ein geeigneter Kandidat, denn $\varphi(11)=10$. Durch Ausprobieren finden wir heraus, dass Quadratzahlen nur die Reste $0$, $1$, $3$, $4$, $5$ oder $9$ modulo $11$ annehmen. Somit kann $a^2+4$ nur die Reste $4$, $5$, $7$, $8$, $9$ oder $2$ modulo $11$ annehmen. Fünfte Potenzen hingegen nehmen modulo~$11$ nur die Reste $-1$, $0$ und $1$ an. Also hat die Gleichung keine Lösung, wie behauptet.
\end{proof}

\begin{proof}[Lösung zu Aufgabe~\ref{aufgabe:UnendlicherAbstieg}]
	Wir lösen die Gleichung mit unendlichem Abstieg. Angenommen, $(a,b,c,d)$ ist eine ganzzahlige Lösung der Gleichung. Dann muss $a^4$ gerade sein, denn alle anderen Terme sind gerade. Dann ist auch $a$ gerade. Schreibe also $a=2a_1$. Dann folgt $16a_1^4+2b^4+4c^4=8d^4$. Nach Division durch~$2$ erhalten wir $8a_1^4+b^4+2c^4=4d^4$. Dann muss $b^4$, also auch $b$, gerade sein, denn wiederum sind alle anderen Terme gerade. Indem wir $b=2b_1$ einsetzen und durch~$2$ dividieren, erhalten wir $4a_1^4+8b_1^4+c^4=2d^4$. Nach dem gleichen Argument muss nun $c$ gerade sein. Indem wir $c=2c_1$ einsetzen und durch~$2$ dividieren, erhalten wir $2a_1^4+4b_1^4+8c_1^4=d^4$. Nun muss also $d$ gerade sein. Indem wir $d=2d_1$ einsetzen und ein letztes Mal durch~$2$ dividieren, erhalten wir $a_1^4+2b_1^4+4c_1^4=8d_1^4$.
	
	Somit ist auch $(a_1,b_1,c_1,d_1)$ eine ganzzahlige Lösung der Gleichung. Indem wir das Argument iterieren, erhalten wir eine unendliche Folge $(a_i,b_i,c_i,d_i)_{i\geqslant 1}$ von ganzzahligen Lösungen der Gleichung, wobei $a_{i+1}=\frac{a_i}{2}$, $b_{i+1}=\frac{b_i}{2}$, $c_{i+1}=\frac{c_i}{2}$ und $d_{i+1}=\frac{d_i}{2}$. Da jede ganze Zahl $\neq 0$ nur endlich oft durch~$2$ teilbar sein kann, kommt somit nur $(a,b,c,d)=(0,0,0,0)$ als Lösung in Frage. Eine Probe bestätigt, dass dies tatsächlich eine Lösung und damit zwangsläufig die einzige Lösung der Gleichung ist.
\end{proof}

\begin{proof}[Lösung zu Aufgabe~\ref{aufgabe:Modulo+Faktorisierung}]
	Wir betrachten zuerst den Fall $b=0$. Dieser Fall führt auf die Gleichung $2^a=n^2-1=(n-1)(n+1)$. Dann müssen $n-1$ und $n+1$ ebenfalls Zweierpotenzen sein. Es gilt aber auch $(n+1)-(n-1)=2$. Die einzigen Zweierpotenzen mit Abstand 2 sind $2^1=2$ und $2^2=4$, sodass $n=3$ gelten muss. Tatsächlich ergibt sich für $b=0$ und $n=3$ die Gleichung $2^3+3^0=3^2$, sodass $(a,b,n)=(3,0,3)$ eine Lösung ist. Ebenso ist $(a,b,n)=(3,0,-3)$ eine Lösung.
	
	Für alle weiteren Lösungen dürfen wir also $b\geqslant 1$ annehmen, sodass $3^b$ durch~$3$ teilbar ist. Indem wir die Gleichung modulo~$3$ betrachten, erhalten wir $(-1)^a\equiv 2^a+3^b\equiv n^2\mod 3$. Weil $n^2$ nur die Reste $0$ oder $1$ annehmen kann, wohingegen $(-1)^a$ nur die Reste $-1$ und $1$ annimmt, kann diese Kongruenz nur für $(-1)^a\equiv 1\mod 3$ und $n^2\equiv 1\mod 3$ erfüllt sein. Insbesondere muss~$a$ gerade sein. Schreibe also $a=2a_1$. Dann können wir die Gleichung wie folgt faktorisieren:
	\begin{equation*}
		3^b=n^2-2^{2a_1}=\parens*{n-2^{a_1}}\parens*{n+2^{a_1}}\,.
	\end{equation*}
	Beide Faktoren auf der rechten Seite müssen Dreierpotenzen sein. Schreibe also $n-2^{a_1}=3^{b_1}$ und $n+2^{a_1}=3^{b_2}$; dabei gilt zwangsläufig $b_1<b_2$ und $b_1+b_2=b$. Für $b_1\geqslant 1$ wäre $3^{b_2}-3^{b_1}=(n+2^{a_1})-(n-2^{a_1})=2\cdot 2^{a_1}$ durch $3$ teilbar, das kann aber offensichtlich nicht sein. Also muss $b_1=0$ gelten, sodass $n=2^{a_1}+1$ folgt. Somit ist $3^{b_2}=n+2^{a_1}=2^{a_1+1}+1$. Der Fall $a_1=0$ führt auf $3^{b_2}=2+1=3$, also $b_2=1$ und somit $b=b_1+b_2=0+1=1$. Tatsächlich gilt $2^0+3^1=2^2$, sodass $(a,b,n)=(0,1,2)$ eine Lösung ist.
	
	Von nun an dürfen wir $a_1\geqslant 1$ annehmen. Dann ist $2^{a_1+1}$ durch~$4$ teilbar und wir erhalten $(-1)^{b_2}\equiv 3^{b_2}\equiv 2^{a_1+1}+1\equiv 1\mod 4$. Damit muss $b_2$ gerade sein und $3^{b_2}=m^2$ muss eine Quadratzahl sein. Dann gilt $2^{a_1+1}=m^2-1$. Eine Gleichung dieser Form haben wir aber im ersten Absatz bereits betrachtet und herausgefunden, dass $(a_1+1,m)=(3,3)$ die einzige Lösung ist. Das führt auf $a_1=2$, also $a=4$ und ferner auf $3^{b_2}=m^2=9$, also $b_2=2$, also $b=b_1+b_2=0+2=2$. Tatsächlich gilt $2^4+3^2=5^2$, also ist $(a,b,n)=(4,3,5)$ eine Lösung. Da unsere Fallunterscheidung vollständig ist, haben wir somit alle Lösungen gefunden.
\end{proof}

\begin{proof}[Lösung zu Aufgabe~\ref{aufgabe:471046}]
	Schreibe $4x^3-7=A^2$ und $4x^5-7=B^2$. Wir bemerken zuerst, dass $x$ eine rationale Zahl sein muss, denn
	\begin{equation*}
		x=\frac{x^6}{x^5}=\frac{\parens*{\frac14\parens*{A^2+7}}^2}{\frac14\parens*{B^2+7}}\,.
	\end{equation*}
	Als nächstes bemerken wir, dass $x$ sogar ganzzahlig sein muss. Denn wäre (nach vollständiger Kürzung) der Nenner von $x$ durch~$2$ teilbar, dann wäre der Nenner von $x^5$ durch~$32$ teilbar, sodass sich der Nenner in $4x^5-7$ nicht vollständig wegkürzen könnte. Und wäre der Nenner von $x$ durch eine Primzahl $p\neq 2$ teilbar, so würde sich der Nenner in $4x^5-7$ erst recht nicht kürzen.
	
	Es ist klar, dass $x=0$ und $x=1$ keine Lösungen sind. Für $x\geqslant 2$ gilt $x^2-1>0$ und somit einerseits $4x^5-7=A^2x^2+7(x^2-1)>(Ax)^2$. Andererseits ist $(Ax+1)^2=(Ax)^2+2Ax+1$. Wenn also $7(x^2-1)<2Ax+1$ gilt, dann haben wir $4x^5-7$ zwischen zwei aufeinanderfolgenden Quadratzahlen eingeschachtelt und erhalten einen Widerspruch. Die fragliche Ungleichung ist zu $7x^2-8<2Ax$ äquivalent. Für $x\geqslant 2$ sind beide Seiten positiv, also ist Quadrieren eine Äquivalenzumformung. Wir erhalten somit $49x^4-102x^2+64<4A^2x^2=16x^5-28x^2$. Für $x\geqslant 4$ ist $49x^4<16\cdot 4x^4\leqslant 16x^5$ und außerdem $64<(102-28)x^2=74x^2$, somit gilt die Ungleichung in diesem Fall und es kann keine Lösung geben. Es verbleiben die Fälle $x=2$ und $x=3$. Für $x=2$ erhalten wir die Quadratzahlen $4\cdot 2^3-7=5^2$ und $4\cdot 2^5-7=11^2$, also ist $x=2$ eine Lösung. Für $x=3$ gilt hingegen $4\cdot 3^3-7=101$, was keine Quadratzahl ist.
\end{proof}

Nun kommen wir zur schwersten der sechs Beispielaufgaben. Wir werden nicht nur die Lösung erklären, sondern auch die Überlegungen, die in diese Lösung geflossen sind.

\begin{proof}[Lösung zu Aufgabe~\ref{aufgabe:VAIMO2010}]
	Durch Ausprobieren kleiner Fälle finden wir zuerst die beiden Lösungspaare $(m,n)=(1,0)$ und $(m,n)=(2,1)$ und es scheint, dass das die einzigen sind. Das wollen wir nun beweisen.
	
	Nachdem wir alle kleineren Fälle durchprobiert haben, dürfen wir $m\geqslant 3$ und $n\geqslant 2$ annehmen. Wir wollen die Gleichung modulo geeigneten Zahlen betrachten. Als erstes kommen uns hier natürlich die Zahlen~$3$ und~$7$ in den Sinn, aber dadurch können wir unmöglich einen Widerspruch bekommen, denn die Gleichung besitzt ja Lösungen. Um die gefundenen Lösungen auszuschließen, müssen wir die Gleichung statt modulo~$3$ mindestens modulo $3^3=27$ und statt modulo~$7$ mindestens modulo $7^2=49$ betrachten. Das werden wir nun nacheinander tun.
	
	\emph{Betrachtung modulo~49.} Nach dem Satz von Euler-Fermat ist $3^{42}\equiv 3^{\varphi(49)}\equiv 1\mod 49$. Die Folge der Dreierpotenzen ist also periodisch modulo~$49$ mit Periodenlänge (höchstens) $42$. Wir müssen folglich nur die ersten~$42$ Fälle durchprobieren. Das sind immer noch ziemlich viele, zumal die Rechnungen aufwendig werden. Deswegen benutzen wir einen Trick: Wir betrachten die Gleichung zuerst modulo~$7$ (obwohl hier noch kein Widerspruch zu erwarten ist). Hier ist die Periodenlänge höchstens $\varphi(7)=6$, was eine machbare Anzahl Fälle ergibt. Wir erhalten:
	\begin{center}
		\begin{tabular}{r | c c c c c c}\toprule
			$m$ & $0$ & $1$ & $2$ & $3$ & $4$ & $5$ \\%\midrule
			$3^m\mod 7$ & $1$ & $3$ & $2$ & $-1$ & $-3$ & $-2$\\\bottomrule
		\end{tabular}
	\end{center}
	Also kommt nur $m=2$ in Frage, bzw.\ allgemein $m=6r+2$. Damit haben wir die ursprünglich $42$ Fälle auf nur noch sieben Fälle reduziert. Diese probieren wir nun nacheinander durch. Dabei benutzen wir $3^6\equiv -6\mod 49$, um die Rechnungen zu vereinfachen (wir müssen dann immer nur jedes vorherige Ergebnis mit $-6$ multiplizieren):
	\begin{center}
		\begin{tabular}{r | c c c c c c c}\toprule
			$n$ & $2$ & $8$ & $14$ & $20$ & $26$ & $32$ & $38$ \\%\midrule
			$3^n\mod 49$ & $9$ & $-5$ & $30$ & $16$ & $2$ & $-12$ & 23\\\bottomrule
		\end{tabular}
	\end{center}
	Leider gibt es immer noch eine Lösung, nämlich $m=26$, bzw.\ allgemein $m=42s+26$. Vielleicht haben wir ja modulo~$27$ mehr Glück.
	
	
	\emph{Betrachtung modulo~27.} Analog zum vorherigen Fall liefert der Satz von Euler-Fermat, dass die Folge der Siebenerpotenzen periodisch modulo $27$ mit Periodenlänge (höchstens) $\varphi(27)=18$ ist. Das sind wieder ziemlich viele Fälle, sodass sich der gleiche Trick wie vorher lohnt. Wir betrachten die Gleichung also zuerst modulo~$9$, wo die Periodenlänge höchstens $\varphi(9)=6$ beträgt. Wir erhalten:
	\begin{center}
		\begin{tabular}{r | c c c c c c}\toprule
			$n$ & $0$ & $1$ & $2$ & $3$ & $4$ & $5$ \\%\midrule
			$-7^n\mod 9$ & $-1$ & $2$ & $5$ & $-1$ & $2$ & $5$\\\bottomrule
		\end{tabular}
	\end{center}
	Es kommen nur $n=1$ oder $n=4$ in Frage, bzw.\ allgemein $n=3t+1$. Damit haben wir die ursprünglich $18$ Fälle auf nur noch sechs Fälle reduziert. Mit $7^3\equiv -8\mod 27$ ergibt sich
	\begin{center}
		\begin{tabular}{r | c c c c c c}\toprule
			$n$ & $1$ & $4$ & $7$ & $10$ & $13$ & $16$ \\%\midrule
			$-7^n\mod 27$ & $-7$ & $2$ & $-16$ & $-7$ & $2$ & $-16$\\\bottomrule
		\end{tabular}
	\end{center}
	Leider gibt es auch hier noch Lösungen, nämlich $n=4$ und $n=13$, bzw.\ allgemein $n=18u+4$ und $n=18u+13$.
	
	Nachdem beide Versuche nicht geklappt haben, könnten wir als nächstes nach Faktorisierungen suchen oder weitere Modulo-Betrachtungen durchführen. Da uns keine Faktorisierungen für die Exponenten $42s+26$ und $18u+4$ oder $18u+13$ ins Auge springen, entscheiden wir uns für letzteres. Um eine Fallunterscheidung zu vermeiden, arbeiten wir außerdem zunächst mit $n=3t+1$ statt $n=18u+4$ oder $n=18u+13$ sind. Wir wollen nun nach Zahlen~$M$ suchen, sodass sowohl $3^{42s+26}$ als auch $7^{3t+1}$ möglichst wenige Reste modulo $M$ annehmen. Wenn wir diese Zahlen in der Form $3^{26}\cdot (3^s)^{42}$ und $7\cdot (7^t)^3$ schreiben, würde es ausreichen, wenn $42$-te Potenzen und Kubikzahlen nur wenige Reste modulo~$M$ annehmen. Wie wir im Theorieteil gesehen haben, sollte dafür $\varphi(M)$ durch $42$ (und damit automatisch durch $3$) teilbar sein. Ein offensichtlicher Kandidat ist $M=43$, denn $\varphi(43)=42$. Nach dem Satz von Euler-Fermat und einer kurzen Rechnung ist $3^{42s+26}\equiv 3^{26}\equiv 15\mod 43$ (natürlich berechnen wir $3^{26}$ nicht durch $26$ Multiplikationen; stattdessen berechnen wir durch sukzessives Quadrieren die Reste von $3$, $3^2$, $3^4$, $3^8$ und $3^{16}$ modulo $43$ und benutzen dann $3^{26}=3^{16}\cdot 3^8\cdot 3^2$). Andererseits gilt $7^3=343=8\cdot 43-1$, also $7^{3t+1}\equiv (-1)^t\cdot 7\mod 43$. Somit kann $3^{42s+26}-7^{3t+1}$ modulo $43$ nur die Reste $15+7=22$ und $15-7=8$ annehmen und wir haben den gewünschten Widerspruch herbeigeführt!
\end{proof}
	\subsection*{Lösungen zu Kapitel~\ref{kapitel:Quersummen}: \emph{Quersummen}}

\begin{proof}[Lösung zu Aufgabe~\ref{aufgabe:441244} \textmd{(\href{https://www.mathematik-olympiaden.de/moev/index.php?option=com_download&thema=a&format=raw&datei=A44134b.pdf}{MO 441344})}]
	Wegen $2005^{2005}<10000^{2005}=10^{8020}$ hat $2005^{2005}$ höchstens $8020$ Stellen. Folglich ist $Q(2005^{2005})\leqslant 9\cdot 8020=72180$. Die Zahl in $\{1,2,\dotsc,72180\}$ mit der größten Quersumme ist $69999$. Also ist $Q(Q(2005^{2005}))\leqslant Q(69999)=42$. Die Zahl in $\{1,2,\dotsc,42\}$ mit der größten Quersumme ist $39$. Also ist
	\begin{equation*}
		Q\parens*{Q\parens*{Q\parens*{2005^{2005}}}}\leqslant Q(39)=12\,.
	\end{equation*}
	
	Andererseits $2005\equiv -2\mod 9$ und somit
	\begin{equation*}
		2005^{2005}\equiv (-2)^{3\cdot 668+1}\equiv (-8)^{668}\cdot (-2)\equiv 1^{668}\cdot 7\equiv 7\mod 9\,.
	\end{equation*}
	Es folgt $Q(Q(Q(2005^{2005})))\equiv 2005^{2005}\equiv 7\mod 9$. Weil $7$ die einzige Zahl in $\{1,2,\dotsc,12\}$ ist, die modulo $9$ den Rest $7$ lässt, folgt $Q(Q(Q(2005^{2005})))=7$.
\end{proof}
\begin{proof}[Lösung zu Aufgabe~\ref{aufgabe:531042} \textmd{(\href{https://www.mathematik-olympiaden.de/moev/index.php?option=com_download&thema=a&format=raw&datei=A53104a.pdf}{MO 531042})}]
	Die Folge $(a_i)_{i\geqslant 0}$ ist offenbar monoton steigend. Wenn es einen Index~$i$ mit $a_{i+1}=R(a_i)=a_i$ gibt, dann ist auch $a_{i+2}=R(a_{i+1})=R(a_i)=a_i$ und so weiter. Ab diesem Punkt muss die Folge also konstant sein.
	
	Angenommen, es gibt keinen solchen Index. Dann muss die Folge $(a_i)_{i\geqslant 0}$ streng monoton steigend sein. Weil Exponentialfunktionen schneller steigen als lineare Funktionen, gibt es eine positive ganze Zahl $k$, sodass folgendes gilt:
	\begin{equation*}
		\parens*{\frac{10}{9}}^k>90k\,,\quad \text{oder äquivalent}\quad 10^{k-1}>9^k\cdot 9k\,.
	\end{equation*}
	Wenn die Folge $(a_i)_{i\geqslant 0}$ streng monoton steigend ist, muss es einen Index $i$ geben, sodass $a_i<10^k$ und $a_{i+1}\geqslant 10^k$. Weil $a_i$ höchstens $k$-stellig ist, gilt $P(a_i)\leqslant 9^k$ und $Q(a_i)\leqslant 9k$. Somit ist
	\begin{equation*}
		a_{i+1}=a_i+P(a_i)Q(a_i)\leqslant a_i+9^k\cdot 9k<10^{k}+10^{k-1}\,.
	\end{equation*}
	Also ist $a_{i+1}$ eine Zahl zwischen $10^k$ und $10^k+10^{k-1}$. Somit ist die von links gelesen zweite Dezimalstelle von $a_{i+1}$ eine Null. Folglich ist das Querprodukt $P(a_{i+1})=0$. Daraus folgt nun $a_{i+2}=a_{i+1}+P(a_{i+1})Q(a_{i+1})=a_{i+1}$, was unserer Annahme widerspricht, dass $(a_i)_{i\geqslant 0}$ streng monoton steigend ist.
\end{proof}
	
	\newpage\phantomsection
	\cftaddtitleline{toc}{part}{MatBoj-Regeln}{\thepage}
	% Damit in der PDF-Navigationsleiste auch der Abschnitt "MatBoj-Regeln" auftaucht, muss ein zusätzliches Bookmark gesetzt werden. Irrelevant für die Druckversion.
	\pdfbookmark{MatBoj-Regeln}{MatBoj}
	\section*{MatBoj-Regeln} 

MatBoj -- abgeleitet aus dem Russischen -- steht für \enquote{mathematischer Kampf}.

Zwei Teams lösen Aufgaben und präsentieren anschließend ihre Lösungen.

\subsection*{Phase 1: Das Lösen der Aufgaben}
Jede Mannschaft gibt sich einen Namen und wählt einen Mannschaftskapitän und einen Stellvertreter. Diese vertreten die Mannschaft als Sprecher. Nur sie können für die Mannschaft verbindliche Entscheidungen verkünden.

Beide Teams erhalten den gleichen Satz von Aufgaben. Ihnen steht eine vorher bekanntgegebene Zeit zur Verfügung, um die Aufgaben getrennt voneinander zu lösen. 

Sollte einem Teammitglied eine Aufgabe bereits bekannt sein, so ist es aus Fairnessgründen dazu aufgefordert, dies der Jury bekanntzumachen (eventuell wird die betreffende Aufgabe durch eine neue ersetzt).


\subsection*{Phase 2: Das Vorstellen der Lösungen}
\begin{itemize}
	\item Den beiden Kapitänen wird gleichzeitig eine leichte Einstiegsaufgabe gestellt, die sie ohne Hilfsmittel lösen müssen. Keines der anderen Teammitglieder darf ihnen dabei helfen. Wer die richtige Antwort gibt, gewinnt für sein Team das Recht zu entscheiden, welches Team als erstes herausfordert. Gibt einer der Kapitäne eine falsche Antwort, erhält das Team des anderen Kapitäns dieses Recht.  
	\item \textbf{Herausfordern:} Das entsprechende Team fordert vom gegnerischen Team eine Aufgabe. Das herausgeforderte Team kann die Herausforderung annehmen oder ablehnen:
	\begin{itemize}
		\item Die \textit{Herausforderung wird angenommen}: Das herausgeforderte Team entsendet ein Teammitglied als \textit{Referenten}, der eine Lösung der Aufgabe vorstellt, das herausfordernde Team entsendet einen \textit{Kritiker}, der Lücken in der Lösung zu finden versucht. Nach der Vorstellung der Lösung darf der Kritiker erst Verständnisfragen stellen und dann die vorgetragene Lösung kritisieren und die von ihm aufgedeckten Lücken füllen. Hilfe aus dem Team ist unzulässig.
		\item Die \textit{Herausforderung wird abgelehnt}: Das herausfordernde Team entsendet ein Teammitglied, das eine Lösung der Aufgabe vorstellt, das herausgeforderte  Team entsendet einen Kritiker, der Lücken in der Lösung zu finden versucht. Nach der Vorstellung der Lösung darf der Kritiker erst Verständnisfragen stellen und dann die vorgetragene Lösung kritisieren. Er darf jedoch keine von ihm aufgedeckten Lücken füllen. Hilfe aus dem Team ist unzulässig.
	\end{itemize}
	\item \textbf{Bewertung:} Jede Aufgabe ist 12 Punkte wert. Der Referent erhält eine der Punktzahlen $0, 2, 4, 6, 8, 10, 12$, je nachdem, wie richtig und vollständig die von ihm vorgetragene Lösung ist. Der Kritiker erhält für das Aufdecken der Lücken in der vorgetragenen Lösung und für das Füllen dieser Lücken jeweils die Hälfte der noch nicht vergebenen Punkte. Wie weit Referent und Kritiker ihren Aufgaben im Einzelnen gerecht wurden, liegt im Ermessen der Jury.
	\item \textbf{Invalid challenge:} Wird die Herausforderung abgelehnt und kann das herausfordernde Team keine Lösung präsentieren, liegt ein \emph{invalid challenge} vor. Die Einschätzung, ob es sich um eine Lösung handelt, liegt im Ermessen der Jury.
	
	In diesem Fall erhält das herausgeforderte Team $6$ Punkte.
	\item Die \textit{nächste Herausforderung}: Es wird abwechselnd herausgefordert. Liegt ein invalid challenge vor, muss das herausfordernde Team erneut eine Aufgabe fordern.
	\item Die Endphase des Wettbewerbs: Zu einem beliebigen Zeitpunkt kann jedes der beiden Teams beschließen, keine Lösungen mehr zu präsentieren. Das betreffende Team muss aber weiter Kritiker entsenden, da das andere Team solange weiter Lösungen vorstellen kann, wie es will. Die Kritiker dürfen in dieser Endphase des Wettbewerbs nur noch Lücken in den vorgetragenen Lösungen aufzeigen, aber nicht mehr füllen.
	\item Am Ende des Wettbewerbs muss jedes Teammitglied mindestens einmal als Referent bzw. als Kritiker entsandt worden sein.
	\item \textbf{Time-Out:} Jedes Team hat dreimal im ganzen Wettbewerb die Möglichkeit, ein Time-Out (1 Minute) zu fordern. In dieser Zeit dürfen sich die Repräsentanten beider Teams  mit ihren Teammitgliedern absprechen und auch ausgewechselt werden.
	\item Am Ende des MatBojs gewinnt das Team mit der größeren Punktsumme. 
\end{itemize}
%\end{document}\newpage
	
	\section*{Notizen:}
\end{document}