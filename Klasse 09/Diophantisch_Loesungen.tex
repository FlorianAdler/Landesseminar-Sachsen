\subsection*{Lösungen zu Kapitel~\ref{kapitel:Diophantastisch}}


\begin{proof}[Lösung zu Aufgabe~\ref{aufgabe:521235} \textmd{(\href{https://www.mathematik-olympiaden.de/moev/index.php?option=com_download&thema=a&format=raw&datei=A52123b.pdf}{MO 521235})}]
	Wir betrachten die Gleichung modulo~$9$. Weil Kubikzahlen modulo~$9$ nur die Reste $-1$, $0$ und $1$ annehmen, kann $19x^3-17y^3\equiv x^3+y^3\mod 9$ nur die Reste $-2$, $-1$, $0$, $1$ oder $2$ annehmen. Es gilt jedoch $50\equiv 5\mod 9$. Also hat die Gleichung keine Lösungen.
\end{proof}

\begin{proof}[Lösung zu Aufgabe~\ref{aufgabe:Modulo11}]
	Auch diese Gleichung hat keine Lösungen. Um das zu zeigen, suchen wir eine Zahl~$m$, sodass sowohl Quadratzahlen als auch fünfte Potenzen möglichst wenige Reste modulo~$m$ annehmen. Wie wir im Theorieteil des Kapitels gesehen haben, sollte dafür $\varphi(m)$ durch $2$ und durch $5$ teilbar sein. Also ist $m=11$ ein geeigneter Kandidat, denn $\varphi(11)=10$. Durch Ausprobieren finden wir heraus, dass Quadratzahlen nur die Reste $0$, $1$, $3$, $4$, $5$ oder $9$ modulo $11$ annehmen. Somit kann $a^2+4$ nur die Reste $4$, $5$, $7$, $8$, $9$ oder $2$ modulo $11$ annehmen. Fünfte Potenzen hingegen nehmen modulo~$11$ nur die Reste $-1$, $0$ und $1$ an. Also hat die Gleichung keine Lösung, wie behauptet.
\end{proof}

\begin{proof}[Lösung zu Aufgabe~\ref{aufgabe:UnendlicherAbstieg}]
	Wir lösen die Gleichung mit unendlichem Abstieg. Angenommen, $(a,b,c,d)$ ist eine ganzzahlige Lösung der Gleichung. Dann muss $a^4$ gerade sein, denn alle anderen Terme sind gerade. Also ist auch $a$ gerade. Wir können folglich $a=2a_1$ schreiben. Dann folgt $16a_1^4+2b^4+4c^4=8d^4$. Nach Division durch~$2$ erhalten wir $8a_1^4+b^4+2c^4=4d^4$. Dann muss $b^4$, also auch $b$, gerade sein, denn wiederum sind alle anderen Terme gerade. Indem wir $b=2b_1$ einsetzen und durch~$2$ dividieren, erhalten wir $4a_1^4+8b_1^4+c^4=2d^4$. Nach dem gleichen Argument muss nun $c$ gerade sein. Indem wir $c=2c_1$ einsetzen und durch~$2$ dividieren, erhalten wir $2a_1^4+4b_1^4+8c_1^4=d^4$. Nun muss also $d$ gerade sein. Indem wir $d=2d_1$ einsetzen und ein letztes Mal durch~$2$ dividieren, erhalten wir $a_1^4+2b_1^4+4c_1^4=8d_1^4$.
	
	Somit ist auch $(a_1,b_1,c_1,d_1)$ eine ganzzahlige Lösung der Gleichung. Indem wir das Argument iterieren, erhalten wir eine unendliche Folge $(a_i,b_i,c_i,d_i)_{i\geqslant 1}$ von ganzzahligen Lösungen der Gleichung, wobei $a_{i+1}=\frac{a_i}{2}$, $b_{i+1}=\frac{b_i}{2}$, $c_{i+1}=\frac{c_i}{2}$ und $d_{i+1}=\frac{d_i}{2}$. Da jede ganze Zahl $\neq 0$ nur endlich oft durch~$2$ teilbar sein kann, kommt somit nur $(a,b,c,d)=(0,0,0,0)$ als Lösung in Frage. Eine Probe bestätigt, dass dies tatsächlich eine Lösung und damit zwangsläufig die einzige Lösung der Gleichung ist.
\end{proof}

\begin{proof}[Lösung zu Aufgabe~\ref{aufgabe:Modulo+Faktorisierung}]
	Wir betrachten zuerst den Fall $b=0$. Dieser Fall führt auf die Gleichung $2^a=n^2-1=(n-1)(n+1)$. Dann müssen $n-1$ und $n+1$ ebenfalls Zweierpotenzen sein. Es gilt aber auch $(n+1)-(n-1)=2$. Die einzigen Zweierpotenzen mit Abstand 2 sind $2^1=2$ und $2^2=4$, sodass $n=3$ gelten muss. Tatsächlich ergibt sich für $b=0$ und $n=3$ die Gleichung $2^3+3^0=3^2$, sodass $(a,b,n)=(3,0,3)$ eine Lösung ist. Ebenso ist $(a,b,n)=(3,0,-3)$ eine Lösung.
	
	Für alle weiteren Lösungen dürfen wir also $b\geqslant 1$ annehmen, sodass $3^b$ durch~$3$ teilbar ist. Indem wir die Gleichung modulo~$3$ betrachten, erhalten wir $(-1)^a\equiv 2^a+3^b\equiv n^2\mod 3$. Weil $n^2$ nur die Reste $0$ oder $1$ annehmen kann, wohingegen $(-1)^a$ nur die Reste $-1$ und $1$ annimmt, kann diese Kongruenz nur für $(-1)^a\equiv 1\mod 3$ und $n^2\equiv 1\mod 3$ erfüllt sein. Insbesondere muss~$a$ gerade sein. Schreibe also $a=2a_1$. Dann können wir die Gleichung wie folgt faktorisieren:
	\begin{equation*}
		3^b=n^2-2^{2a_1}=\parens*{n-2^{a_1}}\parens*{n+2^{a_1}}\,.
	\end{equation*}
	Beide Faktoren auf der rechten Seite müssen Dreierpotenzen sein. Wir können also $n-2^{a_1}=3^{b_1}$ und $n+2^{a_1}=3^{b_2}$ schreiben; dabei gilt zwangsläufig $b_1<b_2$ und $b_1+b_2=b$. Für $b_1\geqslant 1$ wäre $3^{b_2}-3^{b_1}=(n+2^{a_1})-(n-2^{a_1})=2\cdot 2^{a_1}$ durch $3$ teilbar, das kann aber offensichtlich nicht sein. Also muss $b_1=0$ gelten, sodass $n=2^{a_1}+1$ folgt. Somit ist $3^{b_2}=n+2^{a_1}=2^{a_1+1}+1$. Der Fall $a_1=0$ führt auf $3^{b_2}=2+1=3$, also $b_2=1$ und somit $b=b_1+b_2=0+1=1$. Tatsächlich gilt $2^0+3^1=2^2$, sodass $(a,b,n)=(0,1,2)$ eine Lösung ist.
	
	Von nun an dürfen wir $a_1\geqslant 1$ annehmen. Dann ist $2^{a_1+1}$ durch~$4$ teilbar und wir erhalten $(-1)^{b_2}\equiv 3^{b_2}\equiv 2^{a_1+1}+1\equiv 1\mod 4$. Damit muss $b_2$ gerade sein und $3^{b_2}=m^2$ muss eine Quadratzahl sein. Dann gilt $2^{a_1+1}=m^2-1$. Eine Gleichung dieser Form haben wir aber im ersten Absatz bereits betrachtet und herausgefunden, dass $(a_1+1,m)=(3,3)$ die einzige Lösung ist. Das führt auf $a_1=2$, also $a=4$ und ferner auf $3^{b_2}=m^2=9$, also $b_2=2$ und damit $b=b_1+b_2=0+2=2$. Tatsächlich gilt $2^4+3^2=5^2$, also ist $(a,b,n)=(4,3,5)$ eine Lösung. Da unsere Fallunterscheidung vollständig ist, haben wir somit alle Lösungen gefunden.
\end{proof}

\begin{proof}[Lösung zu Aufgabe~\ref{aufgabe:471046} \textmd{(\href{https://www.mathematik-olympiaden.de/moev/index.php?option=com_download&thema=a&format=raw&datei=A47104b.pdf}{MO 471046})}]
	Schreibe $4x^3-7=A^2$ und $4x^5-7=B^2$. Wir bemerken zuerst, dass $x$ eine rationale Zahl sein muss, denn
	\begin{equation*}
		x=\frac{x^6}{x^5}=\frac{\parens*{\frac14\parens*{A^2+7}}^2}{\frac14\parens*{B^2+7}}\,.
	\end{equation*}
	Als nächstes bemerken wir, dass $x$ sogar ganzzahlig sein muss. Denn wäre (nach vollständiger Kürzung) der Nenner von $x$ durch~$2$ teilbar, dann wäre der Nenner von $x^5$ durch~$32$ teilbar, sodass sich der Nenner in $4x^5-7$ nicht vollständig wegkürzen könnte. Und wäre der Nenner von $x$ durch eine Primzahl $p\neq 2$ teilbar, so würde sich der Nenner in $4x^5-7$ erst recht nicht kürzen.
	
	Es ist klar, dass $x=0$ und $x=1$ keine Lösungen sind. Für $x\geqslant 2$ gilt $x^2-1>0$ und somit einerseits $4x^5-7=A^2x^2+7(x^2-1)>(Ax)^2$. Andererseits ist $(Ax+1)^2=(Ax)^2+2Ax+1$. Wenn also $7(x^2-1)<2Ax+1$ gilt, dann haben wir $4x^5-7$ zwischen zwei aufeinanderfolgenden Quadratzahlen eingeschachtelt und erhalten einen Widerspruch. Die fragliche Ungleichung ist zu $7x^2-8<2Ax$ äquivalent. Für $x\geqslant 2$ sind beide Seiten positiv, also ist Quadrieren eine Äquivalenzumformung. Wir erhalten somit $49x^4-102x^2+64<4A^2x^2=16x^5-28x^2$. Für $x\geqslant 4$ ist $49x^4<16\cdot 4x^4\leqslant 16x^5$ und außerdem $64<(102-28)x^2=74x^2$, somit gilt die Ungleichung in diesem Fall und es kann keine Lösung geben. Es verbleiben die Fälle $x=2$ und $x=3$. Für $x=2$ erhalten wir die Quadratzahlen $4\cdot 2^3-7=5^2$ und $4\cdot 2^5-7=11^2$, also ist $x=2$ eine Lösung. Für $x=3$ gilt hingegen $4\cdot 3^3-7=101$, was keine Quadratzahl ist.
\end{proof}

%Nun kommen wir zur schwersten der sechs Beispielaufgaben. Wir werden nicht nur die Lösung erklären, sondern auch die Überlegungen, die in diese Lösung geflossen sind.

\begin{proof}[Lösung zu Aufgabe~\ref{aufgabe:VAIMO2010} \textmd{(\href{https://www.mathe-wettbewerbe.de/fileadmin/Mathe-Wettbewerbe/AIMO/Aufgaben_und_Loesungen_AIMO/aufgaben_awb_10.pdf}{IMO-Vorauswahl 2010})}]
	Durch Ausprobieren kleiner Fälle finden wir zuerst die beiden Lösungspaare $(m,n)=(1,0)$ und $(m,n)=(2,1)$ und es scheint, dass das die einzigen sind. Das wollen wir nun beweisen.
	
	Nachdem wir alle kleineren Fälle durchprobiert haben, dürfen wir $m\geqslant 3$ und $n\geqslant 2$ annehmen. Wir wollen die Gleichung modulo geeigneten Zahlen betrachten. Als erstes kommen uns hier natürlich die Zahlen~$3$ und~$7$ in den Sinn, aber dadurch können wir unmöglich einen Widerspruch bekommen, denn die Gleichung besitzt ja Lösungen. Um die gefundenen Lösungen auszuschließen, müssen wir die Gleichung statt modulo~$3$ mindestens modulo $3^3=27$ und statt modulo~$7$ mindestens modulo $7^2=49$ betrachten. Das werden wir nun nacheinander tun.
	
	\emph{Betrachtung modulo~49.} Nach dem Satz von Euler-Fermat ist $3^{42}\equiv 3^{\varphi(49)}\equiv 1\mod 49$. Die Folge der Dreierpotenzen ist also periodisch modulo~$49$ mit Periodenlänge (höchstens) $42$. Wir müssen folglich nur die ersten~$42$ Fälle durchprobieren. Das sind immer noch ziemlich viele, zumal die Rechnungen aufwendig werden. Deswegen benutzen wir einen Trick: Wir betrachten die Gleichung zuerst modulo~$7$ (obwohl hier noch kein Widerspruch zu erwarten ist). Hier ist die Periodenlänge höchstens $\varphi(7)=6$, was eine machbare Anzahl Fälle ergibt. Wir erhalten:
	\begin{center}
		\begin{tabular}{r | c c c c c c}\toprule
			$m$ & $0$ & $1$ & $2$ & $3$ & $4$ & $5$ \\%\midrule
			$3^m\mod 7$ & $1$ & $3$ & $2$ & $-1$ & $-3$ & $-2$\\\bottomrule
		\end{tabular}
	\end{center}
	Also kommt nur $m=2$ in Frage, bzw.\ allgemein $m=6r+2$. Damit haben wir die ursprünglich $42$ Fälle auf nur noch sieben Fälle reduziert. Diese probieren wir nun nacheinander durch. Dabei benutzen wir $3^6\equiv -6\mod 49$, um die Rechnungen zu vereinfachen (wir müssen dann immer nur jedes vorherige Ergebnis mit $-6$ multiplizieren):
	\begin{center}
		\begin{tabular}{r | c c c c c c c}\toprule
			$n$ & $2$ & $8$ & $14$ & $20$ & $26$ & $32$ & $38$ \\%\midrule
			$3^n\mod 49$ & $9$ & $-5$ & $30$ & $16$ & $2$ & $-12$ & 23\\\bottomrule
		\end{tabular}
	\end{center}
	Leider gibt es immer noch eine Lösung, nämlich $m=26$, bzw.\ allgemein $m=42s+26$. Vielleicht haben wir ja modulo~$27$ mehr Glück.
	
	
	\emph{Betrachtung modulo~27.} Analog zum vorherigen Fall liefert der Satz von Euler-Fermat, dass die Folge der Siebenerpotenzen periodisch modulo $27$ mit Periodenlänge (höchstens) $\varphi(27)=18$ ist. Das sind wieder ziemlich viele Fälle, sodass sich der gleiche Trick wie vorher lohnt. Wir betrachten die Gleichung also zuerst modulo~$9$, wo die Periodenlänge höchstens $\varphi(9)=6$ beträgt. Wir erhalten:
	\begin{center}
		\begin{tabular}{r | c c c c c c}\toprule
			$n$ & $0$ & $1$ & $2$ & $3$ & $4$ & $5$ \\%\midrule
			$-7^n\mod 9$ & $-1$ & $2$ & $5$ & $-1$ & $2$ & $5$\\\bottomrule
		\end{tabular}
	\end{center}
	Es kommen nur $n=1$ oder $n=4$ in Frage, bzw.\ allgemein $n=3t+1$. Damit haben wir die ursprünglich $18$ Fälle auf nur noch sechs Fälle reduziert. Mit $7^3\equiv -8\mod 27$ ergibt sich
	\begin{center}
		\begin{tabular}{r | c c c c c c}\toprule
			$n$ & $1$ & $4$ & $7$ & $10$ & $13$ & $16$ \\%\midrule
			$-7^n\mod 27$ & $-7$ & $2$ & $-16$ & $-7$ & $2$ & $-16$\\\bottomrule
		\end{tabular}
	\end{center}
	Leider gibt es auch hier noch Lösungen, nämlich $n=4$ und $n=13$, bzw.\ allgemein $n=18u+4$ und $n=18u+13$.
	
	Nachdem beide Versuche nicht geklappt haben, könnten wir als nächstes nach Faktorisierungen suchen oder weitere Modulo-Betrachtungen durchführen. Da uns keine Faktorisierungen für die Exponenten $42s+26$ und $18u+4$ oder $18u+13$ ins Auge springen, entscheiden wir uns für letzteres. Wir wollen also eine Zahl $M$ finden, sodass $3^{42s+26}$, $7^{18u+4}$ und $7^{18u+13}$ jeweils möglichst wenige Reste modulo~$M$ annehmen, damit die Kongruenzen
	\begin{equation*}
		3^{42s+26}-7^{18u+4}\equiv 2\mod M\quad\text{und}\quad 3^{42s+26}-7^{18u+13}\equiv 2\mod M
	\end{equation*}
	unmöglich sind. Dazu schreiben wir die Zahlen in der Form $3^{42s+26}=3^{26}\cdot (3^s)^{42}$, $7^{18u+4}=7^4\cdot (7^u)^{18}$ und $7^{18u+13}=7^{13}\cdot (7^u)^{18}$. Es würde also ausreichen, wenn $42$-te Potenzen und $18$-te Potenzen möglichst wenig Reste modulo~$M$ annehmen. Wie wir im Theorieteil gesehen haben, sollte $\varphi(M)$ dafür durch $42$ und durch $18$ teilbar sein, also auch durch ihr kleinstes gemeinsames Vielfaches $126$. Weil $127$ eine Primzahl ist, haben wir $\varphi(127)=126$, somit ist $M=127$ eine Möglichkeit. Tatsächlich stellt sich (durch Ausprobieren) heraus, dass die obigen Kongruenzen für $M=127$ unmöglich sind. Allerdings benötigt das Ausprobieren einige umständliche Rechnungen, die in der Olympiadezeit wohl nicht zu bewerkstelligen sind.
	
	Um auf einfachere Weise einen Widerspruch herbeizuführen, erinnern wir uns, dass wir $n=3t+1$ schreiben können (sowohl $18u+4$ als auch $18u+13$ sind von dieser Form). Vielleicht lässt sich schon ein $M$ finden, für welches
	\begin{equation*}
		3^{42s+26}-7^{3t+1}\equiv 2\mod M
	\end{equation*}
	unmöglich ist. Analog zu oben sollte $\varphi(M)$ dafür durch $42$ und durch $3$ teilbar sein. Ein offensichtlicher Kandidat ist $M=43$, denn $43$ ist eine Primzahl und somit $\varphi(43)=42$. Nach dem Satz von Euler-Fermat und einer kurzen Rechnung ist $3^{42s+26}\equiv 3^{26}\equiv 15\mod 43$ (natürlich berechnen wir $3^{26}$ nicht durch $26$ Multiplikationen; stattdessen berechnen wir durch sukzessives Quadrieren die Reste von $3$, $3^2$, $3^4$, $3^8$ und $3^{16}$ modulo $43$ und benutzen dann $3^{26}=3^{16}\cdot 3^8\cdot 3^2$). Andererseits gilt $7^3=343=8\cdot 43-1$, also $7^{3t+1}\equiv (-1)^t\cdot 7\mod 43$. Somit kann $3^{42s+26}-7^{3t+1}$ modulo $43$ nur die Reste $15+7=22$ und $15-7=8$ annehmen und wir haben den gewünschten Widerspruch herbeigeführt!
\end{proof}