\subsection*{Lösungen zu Kapitel~\ref{kapitel:Umordnung}: \emph{Die Umordnungs-Ungleichung}}

\begin{proof}[Lösung zu Aufgabe~\ref{aufgabe:EasyUmordnung} \textmd{(mit der Umordnungs-Ungleichung)}]
	Egal, wie die Folge $a,b,c$ geordnet ist, können wir immer feststellen, dass die Folgen $a^2,b^2,c^2$ und $a,b,c$ gleich geordnet sind. Aus der Umordnungs-Ungleichung folgt also sofort $a^2\cdot a+b^2\cdot b+c^2\cdot c\geqslant a^2b+b^2c+c^2a$.
\end{proof}
Beachte, dass wir bei Aufgabe~\ref{aufgabe:EasyUmordnung} \emph{nicht} ohne Einschränkung der Allgemeinheit $a\geqslant b\geqslant c$ annehmen dürfen. Das wäre nur erlaubt, wenn die Ungleichung \emph{symmetrisch} in $a$, $b$ und $c$ ist, aber die rechte Seite ist lediglich \emph{zyklisch} in den Variablen. Solche Fehler passieren in der Olympiade leider häufig und werden oftmals mit harten Punktabzügen bestraft.

Aufgabe~\ref{aufgabe:EasyUmordnung} (genau wie viele ähnliche Aufgaben) lässt sich auch mit der gewichteten AM-GM-Ungleichung lösen. Weil das eine sehr wichtige Technik ist, werden wir auch diese Lösung besprechen.

\begin{proof}[Zweite Lösung zu Aufgabe~\ref{aufgabe:EasyUmordnung} \textmd{(mit der gewichteten AM-GM-Ungleichung)}]
	Aus der gewichteten AM-GM-Ungleichung folgt $\frac23a^3+\frac13b^3\geqslant (a^3)^{2/3}(b^3)^{1/3}=a^2b$. Analog gilt $\frac23b^3+\frac13c^3\geqslant b^2c$ und $\frac23c^3+\frac13a^3\geqslant c^2a$. Addition dieser Ungleichung liefert schon das Gewünschte.
\end{proof}

\begin{proof}[Lösung zu Aufgabe~\ref{aufgabe:AM-GM-MitUmordnung}]
	 Egal, wie eine Folge $a_1,a_2,\dotsc,a_n>0$ von positiven reellen Zahlen geordnet ist, sind die Folgen $a_1,a_2,\dotsc,a_n$ und $1/a_1,1/a_2,\dotsc,1/a_n$ stets gegensinnig geordnet. Aus der Umordnungs-Ungleichung folgt also insbesondere
	\begin{equation*}
		\frac{a_2}{a_1}+\frac{a_3}{a_2}+\dotsb+\frac{a_n}{a_{n-1}}+\frac{a_1}{a_n}\geqslant a_1\cdot \frac{1}{a_1}+a_2\cdot \frac1{a_2}+\dotsb+a_n\cdot \frac{1}{a_n}= n\,.
	\end{equation*}
	Jetzt betrachten wir positive reelle Zahlen $x_1,x_2,\dotsc,x_n>0$ sowie ihr geometrisches Mittel $G\coloneqq \sqrt[n]{x_1x_2\dotsm x_n}$. Setze nun
	\begin{equation*}
		a_1\coloneqq \frac{x_1}{G}\,,\quad a_2\coloneqq \frac{x_1x_2}{G^2}\,,\quad\dotsc\,,\quad a_n\coloneqq \frac{x_1x_2\dotsm x_n}{G^n}\,.
	\end{equation*}
	Indem wir diese Werte in die obige Ungleichung einsetzen, erhalten wir
	\begin{equation*}
		\frac{x_1+x_2+\dotsb+x_n}{G}=\frac{a_2}{a_1}+\frac{a_3}{a_2}+\dotsb+\frac{a_n}{a_{n-1}}+\frac{a_1}{a_n}\geqslant n\,.
	\end{equation*}
	Daraus folgt sofort die AM-GM-Ungleichung für positive reelle Zahlen $x_1,x_2,\dotsc,x_n>0$. Falls ein $x_i=0$ ist, ist die AM-GM-Ungleichung trivial, denn dann gilt $\sqrt[n]{x_1x_2\dotsm x_n}=0$.
\end{proof}