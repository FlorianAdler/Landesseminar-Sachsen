\subsection*{Lösungen zu Kapitel~\ref{kapitel:Catalan}: \emph{Die Catalan-Zahlen}}

\begin{proof}[Lösung zu Aufgabe~\ref{aufgabe:FibonacciErzeugendeFunktion}]
	Aus der Rekursionsgleichung der Fibonacci-Zahlen folgt
	\begin{equation*}
		(1-x-x^2)F(x)=F_0+(F_1-F_0)x+\sum_{n=2}^\infty (\underbrace{F_n-F_{n-1}-F_{n-2}}_{0})x^n=0+1\cdot x=x\,.
	\end{equation*}
	Daraus folgt bereits~\ref{teilaufgabe:FibonacciRekursion}. Um~\ref{teilaufgabe:FibonacciExplizit} zu zeigen, faktorisieren wir $1-x-x^2=-(1-\phi x)(1-\overline{\phi}x)$. Damit schreiben wir
	\begin{equation*}
		F(x)=\frac{x}{1-x-x^2}=-\frac{x}{(1-\phi x)(1-\overline{\phi}x)}=\frac{1}{\sqrt{5}}\left(\frac{1}{1-\phi x}-\frac{1}{1-\overline{\phi}x}\right)\,.
	\end{equation*}
	Dieser Trick wird als \emph{Partialbruchzerlegung} bezeichnet und er funktioniert nicht nur für quadratische Polynome, sondern allgemein für Polynome von beliebigem Grad. Partialbruchzerlegung kann immer wieder ziemlich nützlich sein und ihr solltet sie im Hinterkopf behalten.
	
	Mit der geometrischen Summenformel können wir weiter umformen:
	\begin{equation*}
		\frac{1}{\sqrt{5}}\left(\frac{1}{1-\phi x}-\frac{1}{1-\overline{\phi}x}\right)=\frac{1}{\sqrt{5}}\left(\sum_{n=0}^\infty(\phi x)^n-\sum_{n=0}^\infty (\overline{\phi}x)^n\right)=\sum_{n=0}^\infty \frac{1}{\sqrt{5}}\left(\phi^n-\overline{\phi}^n\right)x^n\,.
	\end{equation*}
	Durch Koeffizientenvergleich folgt $F_n=1/\sqrt{5}(\phi^n-\overline{\phi}^n)$, wie gewünscht.
\end{proof}
\begin{proof}[Lösung zu Aufgabe~\ref{aufgabe:Partitionszahlen}]
	Wir benutzen die geometrische Summenformel. Wenn wir das Produkt
	\begin{equation*}
		\prod_{n=1}^{\infty}\frac{1}{1-x^n}=\prod_{n=1}^{\infty}\left(1+x^n+x^{2n}+x^{3n}+\dotsb\right)
	\end{equation*}
	ausmultiplizieren, tritt der Term $x^m$ genau so oft auf, wie wir $m$ als Summe $m=\sum_{n=1}^\infty a_nn$ mit ganzen Zahlen $a_n\geqslant n$ darstellen können (dabei gilt zwangsläufig $a_n=0$ für alle bis auf endlich viele $n$). So eine Darstellung ist aber genau eine Partition von $m$; die Zahl $a_n$ gibt dabei an, wie oft $n$ in der Partition vorkommt. Also tritt der Term $x^m$ beim Ausmultiplizieren genau $p_m$ mal vor. Das zeigt die Behauptung.
\end{proof}
\begin{proof}[Lösung zu Aufgabe~\ref{aufgabe:IMC2018} \textmd{(\href{https://imc-math.org.uk/?year=2018&section=problems&item=prob8q}{IMC 2018/8})}]
	Betrachte die Abbildung $\pi\colon \mathbb Z^3\rightarrow \mathbb Z^2$, $(x,y,z)\mapsto (x+y,z)$. Diese bildet die Menge $\Omega$ bijektiv auf die Menge $\Omega'=\left\{(i,j)\in \mathbb Z^2\ \middle|\ j\geqslant 0,\,i\geqslant 2j\right\}$ ab, denn für jedes $i\geqslant 0$ gibt es genau ein Paar $(x,y)$ mit $y+1\geqslant x\geqslant y$ und $x+y=i$ (je nachdem, ob $x$ gerade oder ungerade ist, gilt $x=y$ oder $x=y+1$).
	
	Für suchen also äquivalent die Anzahl aller Wege der Länge $3n$ von $(0,0)$ nach $(2n,n)$, die zu keinem Zeitpunkt \enquote{oberhalb der Diagonalen} verlaufen, wobei die Diagonale diesmal von $(0,0)$ nach $(2n,n)$ verläuft. Dazu benutzen wir eine zweidimensionale Rekursion wie im dritten Beweis für die expliziten Darstellung der Catalan-Zahlen. Sei allgemein $w(i,j)$ die Anzahl solcher Wege von $(0,0)$ nach $(i,j)$. Wiederum gilt die Rekursion aus dem Pascalschen Dreieck,
	\begin{equation*}
		w(i,j)=w(i-1,j)+w(i,j-1)\,,
	\end{equation*}
	also liegt es nahe, dass sich $w(i,j)$ als gewichtete Summe $\sum_{s,t\in\mathbb Z}\lambda_{s,t}\binom{i+j+s}{i+t}$ schreiben lässt. Um die richtigen Gewichte $\lambda_{s,t}$ zu raten, betrachten wir wieder die \enquote{Randfälle}. Wenn $i=2j-1$, dann gibt es keinen solchen Weg, denn $(2j-1,j)$ liegt oberhalb der Diagonale. Folglich $w(2j-1,j)=0$. Wenn $j=0$, dann gibt es genau einen solchen Weg. Folglich $w(i,0)=1$.
	
	Nach etwas Herumprobieren fällt uns auf, dass $\binom{3j-1}{j}=2\binom{3j-1}{j-1}$ gilt. Davon inspiriert raten wir
	\begin{equation*}
		w(i,j)=\binom{i+j}{j}-2\binom{i+j}{j-1}\,.
	\end{equation*}
	Im Fall $i=2j-1$ gilt dann $w(2j-1,j)=0$, wie gewünscht. Für $j=0$ gilt ebenfalls wie gewünscht $w(i,j)=\binom{i}{0}-2\binom{i}{-1}=1$, denn der Binomialkoeffizient $\binom{i}{-1}$ verschwindet per Definition. Die obige Formel liefert somit am \enquote{Rand} die richtigen Werte. Weil sie außerdem die richtige Rekursion erfüllt, muss $w(i,j)$ auch für allgemeine $(i,j)$ mit $j\geqslant 0$ und $i\geqslant 2j$ den richtigen Wert liefern. Im Fall $(i,j)=(2n,n)$ erhalten wir
	\begin{equation*}
		w(2n,n)=\binom{3n}{n}-2\binom{3n}{n-1}=\frac{1}{2n+1}\binom{3n}{n}\,.
	\end{equation*}
	Das war gesucht.
\end{proof}