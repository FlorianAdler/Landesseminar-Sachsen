\documentclass[a4paper, 12pt]{article}

\usepackage{../landesseminarheader}
% ************************************************************************************
% Alternativ kann auch die bisherige Makro-Datei verwendet werden. Dann müssen die folgenden Zeilen entkommentiert werden.
%\usepackage{yhmath}
\usepackage{ngerman} % a4wide,, latexsym
\usepackage[T1]{fontenc}
\usepackage{lmodern}
\usepackage[utf8]{inputenc}
\usepackage{times}
\usepackage[slantedGreek]{mathptmx}
\usepackage{amsmath}
\usepackage{amssymb}
%\usepackage{amscd}
\usepackage{exscale}
\usepackage{enumerate}
\usepackage{amsthm}
\usepackage{graphics}
\usepackage{graphicx}	
\usepackage{longtable}
\usepackage{color}
\usepackage{dsfont} 
\usepackage{bbm}
%\usepackage{wasysym}
\usepackage{ifpdf}
%\usepackage{pst-all}
%\usepackage{pstricks,pstricks-add,pst-math,pst-xkey}
\usepackage{lscape}
\usepackage{eurosym, url, hyperref}
%\usepackage{fourier}


\frenchspacing

%--------------------------------------------------------------------

\setlength\parskip{\medskipamount}
\setlength\parindent{0pt}
\setlength{\voffset}{-3cm} %-1	%-1.5 %bis -3
%%\setlength{\topmargin}{0.625cm}		% oberer Rand bis Oberkante Kopfzeile
\setlength{\oddsidemargin}{0.0cm} \setlength{\evensidemargin}{0.0cm}%	Linker Rand 
%%\setlength{\headheight}{1.25cm}		% Höhe der Kopfzeile
%%\setlength{\headsep}{0.625cm}			% Abstand zw. Kopfzeile und 
\setlength{\topskip}{0cm}
%%\setlength{\footskip}{1cm}
\setlength{\textheight}{26cm} %24 %23.5; voffset ausblenden % bis 27
\setlength{\textwidth}{16cm} %16


\setcounter{secnumdepth}{3}								% Nummerierungstiefe
\setcounter{tocdepth}{1}									% Inhaltsverzeichnistiefe
%\numberwithin{equation}{section}					% Formeln abschnittsweise nummerieren

\flushbottom
\renewcommand{\baselinestretch}{1.0}



%Abkürzungen-------------------------------------------------------------
%Zahlenbereiche----------------------------------
\newcommand{\N}{{\mathbb{N}}}
\newcommand{\Z}{{\mathbb{Z}}}
\newcommand{\Q}{{\mathbb{Q}}}
\newcommand{\R}{{\mathbb{R}}}
\newcommand{\C}{{\mathbb{C}}}	
%(Komplexe) Zahlen
\newcommand{\I}{{\mathrm{i}}}					% Imaginäre Einheit
\newcommand{\real}{{\mathrm{Re}}}			% Realteil
\newcommand{\imag}{{\mathrm{Im}}}			% Imaginärteil
\newcommand{\dual}{{\mathrm{Du}}}			% Dualteil
%Abkürzung griechischer Buchstaben & Abbildungen etc.
\newcommand{\id}{\mathrm{id}}
\newcommand{\ve}{\varepsilon}
\newcommand{\vp}{\varphi}
\newcommand{\eul}{{\mathrm{e}}} 			% Eulersche Zahl e
\newcommand{\ld}{{\mathrm{ld}}} 			% duadischer Logarithmus
%Matrizen und Vektorrechnung---------------------
\newcommand{\T}{^{\mathrm{T}}}				% Transponiertzeichen
\newcommand{\rg}{{\mathrm{rg}}}				% Rang
\newcommand{\bild}{{\mathrm{bild}}}		% Bild
\newcommand{\Kern}{{\mathrm{kern}}}		% Kern
\newcommand{\lin}{{\mathrm{lin}}}			% Kern
\newcommand{\ul}[1]{\underline{#1}} 	% Vektorunterstrich
%Algebra----------------------
\newcommand{\ggT}{{\mathrm{ggT}}}				% ggT
\newcommand{\kgV}{{\mathrm{kgV}}}				% kgV
%Geometrie
\newcommand{\ol}[1]{\overline{#1}} 	  % Strecke
\newcommand{\TV}{{\mathrm{TV}}}				% Teilverhältnis
\newcommand{\DV}{{\mathrm{DV}}}				% Doppelverhältnis
\newcommand{\mbb}[1]{\mathbb{#1}}			% \mathbb	
%Einheiten
\newcommand{\mm}{{\mbox{\,} \mathrm{mm}}}
\newcommand{\cm}{{\mbox{\,} \mathrm{cm}}}				% Einheit cm
\newcommand{\dm}{{\mbox{\,} \mathrm{dm}}}	
\newcommand{\m}{{\mbox{\,} \mathrm{m}}}	
\newcommand{\LE}{{\mbox{\,} \mathrm{LE}}}	
\newcommand{\fe}{{\mbox{\,} \mathrm{FE}}}	
\newcommand{\km}{{\mbox{\,} \mathrm{km}}}
\newcommand{\gpkcm}{{\mbox{\,} \mathrm{g}/\mathrm{cm}^3}}
\newcommand{\kmh}{{\mbox{\,} \frac{\mathrm{km}}{\mathrm{h}}}}
\newcommand{\komma}{{,}}
\newcommand{\entspricht}{\mathrel{\widehat{=}}}   		% Entspricht

\newcommand{\h}{{\mbox{\,} \mathrm{h}}}
\newcommand{\s}{{\mbox{\,} \mathrm{s}}}
\newcommand{\g}{{\mbox{\,} \mathrm{g}}}
\newcommand{\kg}{{\mbox{\,} \mathrm{kg}}}	
\newcommand{\gc}{{\grad\mathrm{C}}}	
\newcommand{\grad}{^{\circ}}				% Einheit °
\newcommand{\mbm}[1]{\mathbbm{#1}}	
\newcommand{\arc}{{\mathrm{arc}}}	
%Schüler
\newcommand{\ds}{\displaystyle}
% shortcuts
\def\ol#1{\overline{#1}}
\def\ul#1{\underline{#1}}
\def\br#1{\left(#1\right)}             % brackets
\def\sbr#1{\left[#1\right]}            % square brackets
\def\cbr#1{\left\{#1\right\}}          % curly brackets
\def\iff{\Leftrightarrow\ }
\def\yields{\Rightarrow\ }
\newcommand{\rf}[1][]{\textup{\eqref{#1}}}
\newcommand{\half}{\frac{1}{2}}
\newcommand{\third}{\frac{1}{3}}
\newcommand{\ov}{\overline}
\newcommand{\nn}{\nonumber}
\newcommand{\RRA}{{\,\,\Longrightarrow}\,\,}
\newcommand{\LRA}{{\,\,\Leftrightarrow}\,\,}
%\newcommand{\qed}[1][\rule{1ex}{1ex}]{\nopagebreak\hspace*{2em}\hspace*{\fill}{$#1$}}


%Analysis --------------------
\newcommand{\inte}[4]{\int\limits_{#1}^{#2} {#3}\mathrm{d}{#4} } 

%---------------------------------------------------------------------------------------------
\newenvironment{description*}[2]
   {\begin{list}{}{%
      \settowidth{\labelwidth}{#2{#1}}
      \setlength{\leftmargin}{\labelwidth}
         \addtolength{\leftmargin}{\labelsep}
      \setlength{\parsep}{0.5ex plus0.2ex minus0.2ex}
      \setlength{\itemsep}{0.3ex}
      \renewcommand{\makelabel}[1]{#2{##1}\hfill}}}
   {\end{list}}
%*********************************************************************************************

\newcommand{\bsp}[1]{\begin{Bsp}\hspace*{1cm}\newline#1\end{Bsp}}
\newcommand{\kartauf}[3]{
\newpage
\normalsize
\begin{tabular}{p{13cm}}
\textbf{#1 \hfill #2}\\\hline
\end{tabular}
\begin{enumerate}\small
#3
\end{enumerate}}

\newcommand{\bem}[1]{\begin{Bem}\normalfont \hspace*{1cm}\newline#1\end{Bem}}

\newcommand{\defi}[2]{\begin{Def}[#1]\hspace*{1cm}\newline#2\end{Def}}

\newcommand{\theo}[2]{\begin{Theo}[#1]\hspace*{1cm} #2\end{Theo}} %\newline

\newcommand{\satz}[2]{\begin{Satz}[#1]\hspace*{1cm}\newline#2\end{Satz}}

\newcommand{\folg}[2]{\begin{Folg}[#1]\hspace*{1cm}\newline#2\end{Folg}}

\newcommand{\auf}[1]{\begin{Auf} \normalfont#1\end{Auf}} % \hspace*{1cm}\newline

\newcommand{\lem}[1]{\begin{Lem}\hspace*{1cm}\newline#1\end{Lem}}

\newcommand{\cor}[1]{\begin{Cor}\hspace*{1cm}\newline#1\end{Cor}}

\newcommand{\tipp}[2]{\begin{Tipp}[#1]\hspace*{1cm}\newline#2\end{Tipp}}

\newcommand{\bew}[1]{\textsl{Beweis: }#1 \hfill $\Box$}  %ge{\"a}ndert (\!)

\newcommand{\loes}[1]{\textit{Lösung: }#1 \hfill $\Box$}  %ge{\"a}ndert (\!)

\newcommand{\lema}[2]{\begin{Lem}[#1]\hspace*{1cm}\newline#2\end{Lem}}

\newtheoremstyle{definition}% name
     {3pt}%      Space above
     {5pt}%      Space below
     {\itshape}%         Body font % evtl 
     {0ex}%         Indent amount (empty = no indent, \parindent = para indent)
     {\bfseries}% Thm head font
     {:}%        Punctuation after thm head
     {.5em}%     Space after thm head: " " = normal interword space;
           %       \newline = linebreak
     {}%         Thm head spec (can be left empty, meaning `normal')

\newtheoremstyle{Beweis}% name
     {30pt}%      Space above
     {3pt}%      Space below
     {}%         Body font
     {}%         Indent amount (empty = no indent, \parindent = para indent)
     {\itshape}% Thm head font
     {:}%        Punctuation after thm head
     {.5em}%     Space after thm head: " " = normal interword space;
           %       \newline = linebreak
     {}%         Thm head spec (can be left empty, meaning `normal')


\newtheoremstyle{break}% name
     {3pt}%      Space above
     {7pt}%      Space below
     {\itshape}%         Body font
     {}%         Indent amount (empty = no indent, \parindent = para indent)
     {\bfseries}% Thm head font
     {:}%        Punctuation after thm head
     {.5em}%     Space after thm head: " " = normal interword space;
           %       \newline = linebreak
     {}%         Thm head spec (can be left empty, meaning `normal')

\theoremstyle{definition}

\newtheorem{Def}{Definition}[section]


\theoremstyle{break}
\newtheorem{Satz}[Def]{Satz}
\newtheorem{Lem}[Def]{Lemma}
\newtheorem{Cor}[Def]{Korollar}
\newtheorem{Tipp}[Def]{Tipp}

\theoremstyle{definition}
\newtheorem{Theo}[Def]{Theorem}
\newtheorem{Auf}[Def]{Aufgabe}
\newtheorem{Bem}[Def]{Bemerkung}
\newtheorem{Bsp}[Def]{Beispiel}
\newtheorem{beispiel}[Def]{Beispiel}
\newtheorem{aufgabe}[Def]{Aufgabe}
\newtheorem{Folg}[Def]{Folgerung}
\newtheorem{obs}[Def]{Beobachtung}

\theoremstyle{Beweis}
\newtheorem{Bew}{Beweis}



\makeindex

\endinput

%\usepackage{mathtools} % für DeclarePairedDelimiter
%\usepackage{wasysym} % für das Winkel-Symbol
%\usepackage{tikz} % für die Skizzen
%\usetikzlibrary{positioning,calc,arrows.meta,shapes,decorations.pathmorphing,decorations.markings,hobby}
%\tikzset{every picture/.style={line width=0.6, line cap=round,line join=round}}
%\usepackage{csquotes} % für bessere Anführungszeichen
%\usepackage[shortlabels]{enumitem} % um Aufzählungen automatisch in der Form (a), (b), ... zu setzen
%\setlist[enumerate]{label={$(\alph*)$}, ref={$(\alph*)$},topsep=0pt}
%\setlist[itemize]{topsep=0pt,itemsep=0pt,parsep=\parskip}
%\usepackage{booktabs} % schönere Tabellen
%\usepackage{tabularx} % Ich benutze tabularx, um mehrere Bilder so nebeneinander anzuordnen, dass die Abstände alle gleich groß sind
%\usepackage{wrapfig} % für Bilder, die von Text umflossen werden sollen
%\usepackage{microtype} % typographische Mikrooptimierungen (verhindert overfull hboxes)
%\usepackage{xurl} % damit URLs automatisch umgebrochen werden (verhindert over/underfull hboxes).
% ************************************************************************************

% ************************* Kosmetik fürs Inhaltsverzeichnis *************************
% Ich finde es optisch ansprechend und inhaltlich sinnvoll, wenn die Beiträge im Inhaltsverzeichnis nach Theme sortiert auftauchen. Dazu wird das Erscheinungsbild des Inhaltsverzeichnisses ein wenig umgebaut. Wird das nicht gewünscht, dann können die folgenden Zeilen sowie alle \cftaddtitleline-Befehle im Dokument entfernt werden.
\usepackage{tocloft}

% passe Abstände und Erscheinungsbild für Sections an
\cftsetindents{section}{1.5em}{2.3em} % Einrückung
\setlength{\cftbeforesecskip}{0.25em} % Zeilenabstand
\renewcommand{\cftsecleader}{\cftdotfill{\cftdotsep}} % Punkte zwischen Section-Titel und Seitenzahl
\renewcommand{\cftsecfont}{} % Section-Titel nicht fett
\renewcommand{\cftsecpagefont}{} % Seitenzahl nicht fett

% passe Abstände und Erscheinungsbild für Parts an
\setlength{\cftbeforepartskip}{1.0em} % Zeilenabstand
\renewcommand{\cftpartfont}{\bfseries} % Part-Titel nur fett, aber nicht größer
\renewcommand{\cftpartpagefont}{\bfseries} % Seitenzahl nur fett, aber nicht größer
% ************************************************************************************


% **************************** Eine praktische Umgebungen ****************************
% Die amsthm-Umgebungen erlauben es nicht, Bilder in Form von Floats einzufügen. Deswegen wird hier die proof-Umgebungen neu definiert. Außerdem definieren wir neue Umgebungen für Aufgaben, Definitionen und Sätze mit Namen.

% Die proof-Umgebung wird neu definiert
\RenewDocumentEnvironment{proof}{ O{\proofname} }{
	\par\pushQED{\qed}
	\noindent\textbf{#1.}\ \ignorespaces
}{%
	\popQED\par
}

% Aufgaben-Umgebung. Das optionale Argument wird meistens benutzt, um schwere Aufgaben durch Asteriske zu kennzeichnen.
\newcounter{caufgabe}[section]
\NewDocumentEnvironment{aufgabe*}{ O{} }{
	\par\refstepcounter{caufgabe}
	\noindent\textbf{Aufgabe~\thecaufgabe#1.}\ \ignorespaces
}{
	\par
}

% Eine Umgebung für benannte Sätze. Der Name kommt in das optionale Argument. Zum Beispiel liefert "\begin{satzmitnamen}[Satz von Euler-Fermat] ..." im Text "Satz von Euler-Fermat. ..."
\NewDocumentEnvironment{satzmitnamen}{ O{Satz} }{
	\par\begingroup
	\noindent\textbf{#1.}\ \ignorespaces\itshape
}{
	\endgroup\par
}

% Eine Umgebung für Definitionen
\NewDocumentEnvironment{definition}{ O{Definition}}{
	\par
	\noindent\textbf{#1.}\ \ignorespaces
}{
	\par
}
% ************************************************************************************

% ****************************** Eine praktische Makros ******************************
% bessere Klammern in kursiven Umgebungen
\newcommand{\embrace}[1]{\textup{(}#1\textup{)}}

% Das \varangle-Symbol wird in \itshape-Umgebungen kursiv dargestellt; dieses Kommando behebt das Problem.
\newcommand{\winkel}{\textup{\varangle}}

% Ein Verkehrszeichen
\DeclareRobustCommand{\Warnung}{\smash{\tikz[baseline, anchor=center]\node[draw, regular polygon, regular polygon sides=3, rounded corners=2, thick, inner sep=-0.25pt] at (0,0) {\textbf{!}};}}

% Gepaarte Klammern
\DeclarePairedDelimiter{\parens}{\lparen}{\rparen}
\DeclarePairedDelimiter{\braces}{\lbrace}{\rbrace}
\DeclarePairedDelimiter{\brackets}{[}{]}
\DeclarePairedDelimiter{\abs}{\lvert}{\rvert}
% ************************************************************************************


% Wenn $\boldsymbol{...}$ in einem Titel verwendet wird, kommt es manchmal zu einer Warnung (aber alles sieht ok aus). Dieser Befehl unterdrückt die Warnung.
\SetSymbolFont{wasy}{bold}{U}{wasy}{m}{n}
\begin{document}
	\sffamily
	\title{\Huge \textbf{Sächsisches Landesseminar} \\ \textbf{Mathematik 2024}\\ in Sayda}
	\author{}
	\date{11.\,03. -- 15.\,03.\,2024}
	\maketitle
	
	\renewcommand{\baselinestretch}{1.5} \LARGE
	%zur Vorbereitung auf die Bundesrunde der
	
	%52. Mathematik-Olympiade
	
	%in Hamburg (5. bis 8. Mai 2013)
	
	\vspace{3cm}
	
	\begin{center}\Huge
		\textbf{Begleitmaterial zum Seminarprogramm\\der Klassenstufe 12}
	\end{center}
	
	\thispagestyle{empty}
	
	\renewcommand{\baselinestretch}{1.0}\normalsize\rmfamily
	
	\vfill
	
	\author{Herausgegeben im Auftrag des Sächsischen Landeskomitees zur Förderung\\ mathematisch-naturwissenschaftlich begabter und interessierter Schüler}
	%:\\ Sebastian Banert, Ingolf Busch, Pierre Landrock, Jens Reinhold, Lisa Sauermann
	
	\newpage
	
	\section*{Vorwort}
	
	Liebe Schülerinnen,\\
	liebe Schüler,
	
	ich freue mich, Euch im Sächsischen Landesseminar Mathematik begrüßen zu können.
	
	In den nächsten drei Tagen werdet Ihr zehn mathematische Seminare haben. Ich hoffe, dass Ihr während dieser Seminare nicht nur neue Lösungsmethoden oder mathematische Sachverhalte kennen lernt, sondern dass Ihr dabei auch Freude an der Mathematik und am Lösen von Aufgaben haben werdet.
	
	Am Donnerstag wird dann die Auswahlklausur geschrieben, für die ich Euch jetzt schon alles Gute und viel Erfolg wünsche. Während Ihr am vorbereiteten Freizeitprogramm teilnehmt, werden die Klausuren korrigiert. Am Freitag wird dann die Mannschaft, die Sachsen auf der Bundesrunde der Mathematik-Olympiade vertreten darf, feierlich bekannt gegeben.
	
	Ich hoffe, dass Ihr es in den nächsten Tagen neben der Mathematik auch genießen könnt, Euch mit Gleichgesinnten zu unterhalten, Euch auszutauschen und um Lösungsansätze gemeinsam zu ringen. Dazu soll insbesondere auch der MatBoj beitragen. Aber natürlich denke ich dabei auch an die vielen Spiele, die eine lange Tradition im Landesseminar haben.
	
	Ich wünsche Euch viel Erfolg und eine gute Woche.
	
	Joachim Lippert
	
	\section*{Über dieses Heft}
	Dieses Heft behandelt einige der wichtigsten Themen für die Mathematik-Olympiade in der Klassenstufe~12. Einige dieser Themen werden auch in den Seminaren besprochen, einige werdet ihr bestimmt schon kennen und andere werden euch neu sein. Es wird empfohlen, dass ihr das Heft zur Vorbereitung auf die Bundesrunde oder die nächste Olympiade durcharbeitet.
	
	In diesem Heft gibt es zwei Typen von Aufgaben: \emph{Beispielaufgaben} und \emph{Übungsaufgaben}. An Beispielaufgaben lassen sich die vorgestellten Methoden besonders gut vorführen. Manchmal lösen wir Beispielaufgaben direkt auf, aber meistens findet ihr am Ende des jeweiligen Kapitels Tipps und erst ganz am Ende des Heftes die Lösungen für die Beispielaufgaben. Die Beispielaufgaben solltet ihr zuerst bearbeiten, wenn ihr euch mit einer neuen Methode vertraut machen wollt. Bei den Übungsaufgaben hingegen seid ihr auf euch allein gestellt. Sie dienen zur weiteren Vertiefung der Inhalte.
	
	Schwere Aufgaben sind mit einem (*) bis drei (***) Sternen gekennzeichnet. Ein Stern bedeutet dabei, dass die Aufgabe schwerer als die durchschnittliche Bundesrunden-Aufgabe ist. Besonders bei solchen Aufgaben gilt: Wenn ihr nicht weiterkommt, dann holt euch einen Tipp und wenn ihr dann immer noch feststeckt, dann lest euch auch gern die Lösung durch -- dafür sind die Tipps und die Lösungen schließlich da.
	
	
	\vfill
	
	\scriptsize
	
	\emph{Texte:} Ferdinand Wagner. Mit tatkräftiger Unterstützung von Sebastian Bürger, Leo Gitin, Cara Hobohm, Tien Nguyen und Arne Wolf. Aufbauend auf früheren Texten von Ingolf Busch, Frank Göhring, Maximilian Keitel, Eric Müller, Jens Reinhold und Lisa Sauermann. Herzlichen Dank auch an alle weiteren, die in früheren Ausgaben dieses Begleitheftes Beiträge erstellt haben.
	
	\emph{Textsatz:} Joachim Lippert, Tien Nguyen, Ferdinand Wagner.
	
	\emph{Redaktion}: Joachim Lippert (\href{mailto:lippert@landesseminar-sachsen.de}{\texttt{lippert@landesseminar-sachsen.de}}).
	\normalsize
	
	\newpage
	\renewcommand{\baselinestretch}{1}\normalsize
	\tableofcontents
	
	\vspace{1cm}
	
	
	
	\newpage\cftaddtitleline{toc}{part}{Gleichungen und Ungleichungen}{\thepage}
	\section{Trigonometrische Substitutionen}\label{kapitel:TrigSub}

Manchmal lassen sich Aufgaben lösen, indem ihr auf sehr clevere Art und Weise mit Sinus, Cosinus oder Tangens substituiert. Solche Lösungen sind sehr schwer zu finden, aber wenn ihr sie findet, dann sind sie unglaublich elegant. 

Um herauszufinden, wann sich eine trigonometrische Substitution lohnt, müsst ihr lernen, versteckte Additionstheoreme zu erkennen. Wir werden in diesem Kapitel anhand von Beispielaufgaben drei \enquote{Standardverstecke} besprechen. Am Ende des Kapitels findet ihr Tipps zu den Beispielaufgaben und am Ende des Heftes könnt ihr die Lösungen nachrechnen.

\textbf{1.~Das Tangens-Additionstheorem.} Das Additionstheorem für den Tangens nimmt bekanntlich folgende Form an:
\begin{equation*}
	\tan(\alpha\pm\beta)=\frac{\tan\alpha\pm\tan\beta}{1\mp\tan\alpha\tan\beta}\,.
\end{equation*}
Wenn in einer Aufgabe die Ausdrücke $1-xy$ und $x+y$ vorkommen, dann ist Tangens-Substitu-tionszeit. So zum Beispiel bei folgender Aufgabe:
\begin{aufgabe*}\label{aufgabe:Tangenssubstitution}
	Gegeben seien sechs reelle Zahlen $a_1,a_2,\dotsc,a_6$. Zeige: Unter diesen sechs Zahlen gibt es stets zwei Zahlen~$x$ und~$y$ mit der Eigenschaft
	\begin{equation*}
		\sqrt{3}(x-y)\leqslant 1+xy\,.
	\end{equation*}
\end{aufgabe*}


\textbf{2.~Das Sinus-Additionstheorem.} Das Additionstheorem für den Sinus nimmt bekanntlich folgende Form an:
\begin{equation*}
	\sin(\alpha\pm\beta)=\sin\alpha\cos\beta\pm\cos\alpha\sin\beta\,.
\end{equation*}
Auch dieses Additionstheorem lässt sich gelegentlich in Aufgaben entdecken, kann aber deutlich komplizierter zu erkennen sein. Hier sind zwei Beispielaufgaben. In Aufgabe~\ref{aufgabe:SinusVersteckt} ist das Sinus-Additionstheorem ganz besonders gut versteckt.
\begin{aufgabe*}\label{aufgabe:SinusWurzelAdditionstheorem}
	Vier reelle Zahlen $a_1,a_2,a_3,a_4$ werden aus dem Intervall $\brackets[\big]{\frac{\sqrt{2}-\sqrt{6}}{2},\frac{\sqrt{2}+\sqrt{6}}{2}}$ ausgewählt. Zeige: Unter $a_1,a_2,a_3,a_4$ gibt es stets zwei Zahlen $x$ und $y$ mit
	\begin{equation*}
		\abs*{x\sqrt{4-y^2}-y\sqrt{4-x^2}}\leqslant 2\,.
	\end{equation*}
\end{aufgabe*}
\begin{aufgabe*}\label{aufgabe:SinusVersteckt}
	Für eine reelle Zahl~$x$ definieren wir eine Folge $(a_n)_{n\geqslant 0}$ rekursiv durch $a_0=x$ und $a_{n+1}=4a_n(1-a_n)$ für alle $n\geqslant 0$. Finde alle Werte von $x$, für die $a_{\the\year}=0$ gilt!
\end{aufgabe*}

\textbf{3.~Trigonometrische Identitäten im Dreieck.} Für $\alpha+\beta+\gamma=180^\circ$ ergeben sich zusätzliche Identitäten, die ihr ebenfalls in einigen Aufgaben entdecken könnt. Dafür beweisen wir zunächst das folgende Lemma.
\begin{satzmitnamen}[Lemma]\leavevmode
	\begin{enumerate}[label={$(\alph*)$},ref={$(\alph*)$}]
		\item \label{behauptung:DreieckTangensIdentitaet}Für reelle Zahlen $x$, $y$ und $z$ mit
		\begin{equation*}
			x+y+z=xyz
		\end{equation*}
		können wir stets $x=\tan \alpha$, $y=\tan\beta$ und $z=\tan\gamma$ substituieren, wobei $\alpha$, $\beta$, $\gamma$ \embrace{nicht notwendigerweise positive} Winkel mit $\alpha+\beta+\gamma=180^\circ$ sind. Wenn zusätzlich $x,y,z\geqslant 0$ gilt, dann können $\alpha$, $\beta$, $\gamma$ als die Innenwinkel eines spitzwinkligen Dreiecks gewählt werden.
		\item \label{behauptung:DreieckCotangensIdentitaet}Für reelle Zahlen $x$, $y$ und $z$ mit 
		\begin{equation*}
			xy+yz+zx=1
		\end{equation*}
		können wir stets $x=\cot\alpha$, $y=\cot\beta$ und $z=\cot\gamma$ substituieren, wobei $\alpha$, $\beta$, $\gamma$ \embrace{nicht notwendigerweise positive} Winkel  mit $\alpha+\beta+\gamma=180^\circ$ sind. Wenn zusätzlich $x,y,z\geqslant 0$ gilt, dann können $\alpha$, $\beta$, $\gamma$ als die Innenwinkel eines spitzwinkligen oder rechtwinkligen Dreiecks gewählt werden.
		\item \label{behauptung:DreieckCosinusIdentitaet} Für nichtnegative reelle Zahlen $x,y,z\geqslant 0$ mit
		\begin{equation*}
			x^2+y^2+z^2+2xyz=1
		\end{equation*}
		können wir stets $x=\cos\alpha$, $y=\cos\beta$ und $z=\cos\gamma$ substituieren, wobei $\alpha$, $\beta$, $\gamma$ die Innenwinkel eines spitzwinkligen oder rechtwinkligen Dreiecks sind.
	\end{enumerate}
\end{satzmitnamen}

\begin{proof}
	Wir zeigen zuerst~\ref{behauptung:DreieckTangensIdentitaet}. Wähle Winkel $\alpha$, $\beta$ und $\gamma$ mit $x=\tan\alpha$, $y=\tan\beta$ und $z=\tan\gamma$. Indem wir ganzzahlige Vielfache von $180^\circ$ addieren, können wir $0^\circ<\alpha+\beta+\gamma< 360^\circ$ annehmen. Falls $x,y,z\geqslant 0$,  können wir direkt $0^\circ\leqslant \alpha,\beta,\gamma<90^\circ$ annehmen und die Bedingung $0^\circ<\alpha+\beta+\gamma< 360^\circ$ ist automatisch erfüllt. Nun gilt
	\begin{equation*}
		\tan(\alpha+\beta+\gamma)=\frac{\tan\alpha+\tan\beta+\tan\gamma-\tan\alpha\tan\beta\tan\gamma}{1-\parens*{\tan\alpha\tan\beta+\tan\beta\tan\gamma+\tan\gamma\tan\alpha}}\,.
	\end{equation*}
	Der Zähler des Bruches auf der rechten Seite ist genau $x+y+z-xyz$. Folglich gilt $x+y+z=xyz$ genau dann, wenn $\tan(\alpha+\beta+\gamma)=0$. Wegen $0^\circ<\alpha+\beta+\gamma< 360^\circ$ ist das genau für $\alpha+\beta+\gamma=180^\circ$ möglich.
	
	Der Beweis für~\ref{behauptung:DreieckCotangensIdentitaet} ist ähnlich. Analog zu~\ref{behauptung:DreieckTangensIdentitaet} wählen wir Winkel $\alpha$, $\beta$ und $\gamma$ mit $x=\tan\alpha$, $y=\tan\beta$, $z=\tan\gamma$ und außerdem $0^\circ<\alpha+\beta+\gamma< 180^\circ$. Im Fall $x,y,z\geqslant 0$ nehmen wir außerdem $0^\circ <\alpha,\beta,\gamma\leqslant 90^\circ$ an. Nun gilt
	\begin{multline*}
		\cot\alpha\cot\beta+\cot\beta\cot\gamma+\cot\gamma\cot\alpha-1\\
		\begin{aligned}
			&=\frac{\cos\alpha\cos\beta\sin\gamma+\cos\beta\cos\gamma\sin\alpha+\cos\gamma\cos\alpha\sin\beta-\sin\alpha\sin\beta\sin\gamma}{\sin\alpha\sin\beta\sin\gamma}\\
			&=\frac{\sin(\alpha+\beta+\gamma)}{\sin\alpha\sin\beta\sin\gamma}\,.
		\end{aligned}
	\end{multline*}
	Also gilt $xy+yz+zx=1$ genau dann, wenn $\sin(\alpha+\beta+\gamma)=0$. Wegen $0^\circ<\alpha+\beta+\gamma< 360^\circ$ ist das genau für $\alpha+\beta+\gamma=180^\circ$ möglich.
	
	Für~\ref{behauptung:DreieckCosinusIdentitaet} zeigen wir zuerst, dass die Identität $x^2+y^2+z^2+2xyz=1$ tatsächlich erfüllt ist, wenn $x=\cos\alpha$, $y=\cos\beta$ und $z=\cos\gamma$ gilt, wobei $\alpha$, $\beta$ und $\gamma$ die Innenwinkel eines Dreiecks sind. In diesem Fall müssen zwei der drei Winkel $\leqslant 90^\circ$ sein. Ohne Beschränkung der Allgemeinheit seien das $\alpha$ und $\beta$. Betrachte nun ein Dreieck $ABC$ mit den  Innenwinkeln $\winkel BAC=90^\circ-\alpha$, $\winkel CBA=90^\circ-\beta$ und $\winkel ACB=180^\circ-\gamma$. Außerdem sei $\abs*{AB}=\sin\gamma$. Aus dem Sinussatz lassen sich die anderen Seitenlängen bestimmen:
	\begin{equation*}
		\abs*{BC}=\abs{AB}\cdot \frac{\sin(90^\circ-\beta)}{\sin(180^\circ-\gamma)}=\cos \beta\quad\text{und}\quad \abs*{CA}=\abs{AB}\cdot \frac{\sin(90^\circ-\alpha)}{\sin(180^\circ-\gamma)}=\cos \alpha\,.
	\end{equation*}
	Der Cosinussatz im Dreieck $ABC$ liefert nun $\abs*{AB}^2=\abs*{CA}^2+\abs*{BC}^2-2\abs*{CA}\cdot\abs*{BC}\cos(180^\circ-\gamma)$. Durch Einsetzen der Seitenlängen sowie $\cos(180^\circ-\gamma)=-\cos\gamma$ folgt dann
	\begin{equation*}
		\sin^2\gamma=\cos^2\beta+\cos^2\alpha+2\cos\alpha\cos\beta \cos\gamma\,.
	\end{equation*}
	Durch Einsetzen von $\sin^2\gamma=1-\cos^2\gamma$ folgt sofort die gewünschte Identität.
	
	Nun zeigen wir, dass die gewünschten Identität nur in dem behaupteten Fall erfüllt ist. Zunächst ist klar, dass aus $x^2+y^2+z^2+2xyz=1$ und $x,y,z\geqslant 0$ folgt, dass $0\leqslant x,y,z\leqslant 1$ gilt. Wir können also $y=\cos\beta$ und $z=\cos\gamma$ mit $0^\circ\leqslant \beta,\gamma\leqslant 90^\circ$ substituieren. Dann muss $\beta+\gamma\geqslant 90^\circ$ gelten. Wäre das nämlich nicht der Fall, dann gälte $\cos\gamma>\cos(90^\circ-\beta)$ nach Monotonie des Cosinus auf dem Intervall $[0^\circ,90^\circ]$. Dann würde
	\begin{equation*}
		y^2+z^2=\cos^2\beta+\cos^2\gamma>\cos^2\beta+\cos^2(90^\circ-\beta)=\cos^2\beta+\sin^2\beta=1
	\end{equation*}
	folgen, was einen Widerspruch darstellt. Sei nun $x'\coloneqq \cos(180^\circ-\beta-\gamma)$. Nach dem obigen gilt dann $x'\geqslant 0$. Ferner erfüllt $x'$ ebenfalls die Gleichung $(x')^2+y^2+z^2+2x'yz=1$. Für $x'=x$ sind wir fertig. Ansonsten sind $X=x$ und $X=x'$ die beiden Lösungen der quadratischen Gleichung $X^2+2Xyz+y^2+z^2-1=0$. Nach dem Satz von Vieta gilt dann $x+x'=-2yz$. Wegen $x,x',y,z\geqslant 0$ kann diese Gleichung nur für $x=x'=0$ gelten und wir erhalten einen Widerspruch zu unserer Annahme $x\neq x'$.
\end{proof}

Häufig ist eine Variante dieser Substitutionen nützlich. Wenn $\alpha_0$, $\beta_0$ und $\gamma_0$ Winkel mit der Eigenschaft $\alpha_0+\beta_0+\gamma_0=180^\circ$ sind und Winkel $\alpha$, $\beta$ und $\gamma$ mit $\alpha_0=90^\circ-\frac\alpha2$, $\beta_0=90^\circ-\frac\beta2$ und $\gamma_0=90^\circ-\frac\gamma2$ gewählt werden, dann gilt auch $\alpha+\beta+\gamma=180^\circ$. Daraus folgt:
\begin{enumerate}[label={$(\alph*)$},ref={$(\alph*)$}]\itshape
	\item Wenn $x+y+z=xyz$, dann können wir auch $x=\cot\parens[\big]{\frac{\alpha}2}$, $y=\cot\parens[\big]{\frac{\beta}2}$ und $z=\cot\parens[\big]{\frac{\gamma}2}$ mit $\alpha+\beta+\gamma=180^\circ$ substituieren. Falls $x,y,z\geqslant 0$, dann können $\alpha$, $\beta$ und $\gamma$ als die Innenwinkel eines Dreiecks gewählt werden \embrace{dieses Dreiecks muss aber nicht spitz- oder rechtwinklig sein}.
	\item Wenn $xy+yz+zx=1$, dann können wir auch $x=\tan\parens[\big]{\frac{\alpha}2}$, $y=\tan\parens[\big]{\frac{\beta}2}$ und $z=\tan\parens[\big]{\frac{\gamma}2}$ mit $\alpha+\beta+\gamma=180^\circ$ substituieren. Falls $x,y,z\geqslant 0$, dann können $\alpha$, $\beta$ und $\gamma$ als die Innenwinkel eines Dreiecks gewählt werden \embrace{dieses Dreiecks muss aber nicht spitz- oder rechtwinklig sein}.
	\item Wenn $x$, $y$ und $z$ nichtnegative reelle Zahlen mit $x^2+y^2+z^2+2xyz=1$ sind, dann können wir auch $x=\sin\parens[\big]{\frac{\alpha}2}$, $y=\sin\parens[\big]{\frac{\beta}2}$ und $z=\sin\parens[\big]{\frac{\gamma}2}$ substituieren, wobei $\alpha$, $\beta$ und $\gamma$ die Innenwinkel eines Dreiecks sind.
\end{enumerate}
Damit könnt ihr nun die folgenden beiden Aufgaben lösen, die ohne dieses Wissen sehr schwierig wären. Aufgabe~\ref{aufgabe:UngleichungInvertieren2} ist euch bereits im Kapitel \emph{Die $uvw$-Methode} im Heft für Klasse~11 begegnet. Hier seht ihr nun, wie sich diese Aufgabe wesentlich eleganter lösen lässt.
\begin{aufgabe*}\label{aufgabe:521236}
	Finde alle Tripel reeller Zahlen $(x,y,z)$ mit $xy+yz+zx=1$ und
	\begin{equation*}
		3\parens*{x+\frac 1x}=4\parens*{y+\frac 1y}=5\parens*{z+\frac 1z}\,.
	\end{equation*}
\end{aufgabe*}
\begin{aufgabe*}[*]\label{aufgabe:UngleichungInvertieren2}
	Gegeben seien nichtnegative reelle Zahlen $x,y,z\geqslant 0$ mit $x^2+y^2+z^2+2xyz=1$. Zeige, dass
	\begin{equation*}
		xy+yz+zx\leqslant \frac12+2xyz\,.
	\end{equation*}
\end{aufgabe*}
Aufgabe~\ref{aufgabe:UngleichungInvertieren2} haben wir euch schon als Beispielaufgabe im Kapitel \emph{Die $uvw$-Methode} im Heft für Klasse~11 gestellt. Die Lösung, die dort vorgestellt wurde, ist allerdings weder besonders einfach noch besonders schön. Hier habt ihr die Gelegenheit, euch eine elegantere Lösung zu überlegen.

\newpage\phantom{newpage}\vfill\hrule\vspace{-1em}



\subsection*{Tipps zu den Beispielaufgaben}

\textbf{Tipps zu Aufgabe~\ref{aufgabe:Tangenssubstitution}.} Substituiere $a_i=\tan \alpha_i$ und benutze das Schubfachprinzip. Jedoch aufgepasst: Das Schubfachprinzip lässt sich nicht völlig naiv anwenden. Du musst auch noch die Periodizität des Tangens ausnutzen!

\textbf{Tipps zu Aufgabe~\ref{aufgabe:SinusWurzelAdditionstheorem}.} Substituiere $a_i=2\sin\alpha_i$. Für welche Werte von $\alpha$ gilt $2\sin\alpha = \frac{\sqrt{2}\pm\sqrt{6}}{2}$? (Es kommen keine krummen Werte raus.)

\textbf{Tipps zu Aufgabe~\ref{aufgabe:SinusVersteckt}.} Es gilt $\sin^2(2\alpha)=(2\sin\alpha\cos\alpha)^2=4\sin^2\alpha(1-\sin^2\alpha)$. Um geeignet substituieren und diese Formel anwenden zu können, musst du natürlich einige Fälle für $x$ ausschließen.

\textbf{Tipps zu Aufgabe~\ref{aufgabe:521236}.} Zeige zuerst, dass $x,y,z\geqslant 0$ angenommen werden kann. Substituiere dann $x=\tan\parens[\big]{\frac{\alpha}2}$, $y=\tan\parens[\big]{\frac{\beta}2}$ und $z=\tan\parens[\big]{\frac{\gamma}2}$.

\textbf{Tipps zu Aufgabe~\ref{aufgabe:UngleichungInvertieren2}.} Substituiere $x=\sin\parens[\big]{\frac{\alpha}2}$, $y=\sin\parens[\big]{\frac{\beta}2}$ und $z=\sin\parens[\big]{\frac{\gamma}2}$.

Nimm $\alpha\leqslant \beta\leqslant \gamma$ an. Schreibe die Ungleichung in der Form
\begin{equation*}
	\sin\parens*{\frac{\alpha}2}\parens*{\sin\parens*{\frac{\beta}2}+\sin\parens*{\frac{\gamma}2}}+\parens*{1-2\sin\parens*{\frac{\alpha}2}}\sin\parens*{\frac{\beta}2}\sin\parens*{\frac{\gamma}2}\leqslant \frac12
\end{equation*}
und schätze die beiden Terme $\sin\parens[\big]{\frac{\beta}2}+\sin\parens[\big]{\frac{\gamma}2}$ und $\sin\parens[\big]{\frac{\beta}2}\sin\parens[\big]{\frac{\gamma}2}$ ab.\newpage
	\section{Ungleichungen mit Analysis beweisen}\label{kapitel:Analysis}
In der Theorie könnt ihr Ungleichungen beweisen, indem ihr sie als Extremwertaufgaben auffasst. Die lokalen Extrema lassen sich durch Ableiten bestimmen. Falls die Ungleichung eine Nebenbedingung hat, müsst ihr dafür die Lagrange-Multiplikator-Methode benutzen, die wir später in diesem Text erklären werden. Wenn ihr die Extrema bestimmt habt, könnt ihr durch Einsetzen überprüfen, dass die Ungleichung erfüllt ist.

In der Olympiade-Praxis wird das nicht funktionieren. Ihr werdet nichtlineare Gleichungssysteme bekommen, die so kompliziert sind, dass ihr sie niemals von Hand lösen könnt. Aber das müsst ihr auch gar nicht! Ihr wollt ja in Wirklichkeit gar nicht die lokalen Extrema bestimmen, sondern nur beweisen, dass die Ungleichung dort erfüllt ist. Das furchtbare nichtlineare Gleichungssystem, das ihr unmöglich lösen könnt, ist in Wirklichkeit \emph{Gratisinformation}, die euch beim Beweis der Ungleichung hilft, und die ihr einsetzen könnt, wie ihr wollt.

In diesem Kapitel werden wir an zahlreichen Beispielen vorführen, wie sich Analysis auf kreative Weise zum Beweis von schweren Olympiade-Ungleichungen benutzen lässt. Im Gegensatz zu den Beispielaufgaben in den anderen Kapiteln besprechen wir die Lösungen hier sofort und nicht erst am Ende des Heftes, denn ihr sollt einen Überblick bekommen, was es für Standardtricks in Analysislösungen gibt, damit ihr euch in der Olympiade eure eigenen Tricks überlegen könnt. Natürlich könnt ihr, wenn ihr Lust habt, zuerst versuchen, selbst auf die Tricks in den Beispielaufgaben zu kommen. In den meisten Fällen haben die Beispielaufgaben außerdem wesentlich elegantere Musterlösungen, die aber noch schwerer zu finden sind als unsere Analysistricks. 

Bevor wir zu den Beispielen kommen, müssen wir jedoch etwas Theorie einführen.%

\subsection*{Partielle Ableitungen}
\begin{definition}
	Sei $\Omega\subseteq \mathbb R^n$ eine offene Menge\footnote{Das bedeutet, dass es für jeden Punkt $x=(x_1,x_2,\dotsc,x_n)\in\Omega$ ein $\varepsilon >0$ gibt, sodass alle Punkte im Abstand $<\varepsilon$ von $x$ auch in $\Omega$ liegen.} und sei $f\colon \Omega\rightarrow \mathbb R$ eine stetige Funktion. Sei ferner $a=(a_1,a_2,\dotsc,a_n)\in\Omega$ ein Punkt. Die \emph{partielle Ableitung von $f$ nach $x_i$ in $a$} ist definiert als
	\begin{equation*}
		\frac{\partial f}{\partial x_i}(a)\coloneqq \lim_{h\rightarrow 0}\frac{f(a_1,\dotsc,a_{i-1},a_{i}+h,a_{i+1},\dotsc,a_n)-f(a_1,\dotsc,a_{i-1},a_{i},a_{i+1},\dotsc,a_n)}{h}\,,
	\end{equation*}
	sofern dieser Grenzwert existiert. Die partielle Ableitung nach $x_i$ ist also genau wie die Ableitung von Funktionen in einer Variablen definiert, nur dass wir die anderen Variablen $x_j$, $j\neq i$ wie Konstanten behandeln.
	
	Wenn alle partiellen Ableitungen von $f$ existieren, nennen wir $f$ \emph{partiell differenzierbar}. In diesem Fall definieren wir den \emph{Gradienten von $f$} als Funktion $\nabla f\colon \Omega\rightarrow \mathbb R^n$ gegeben durch
	\begin{equation*}
		\nabla f(a)\coloneqq \parens*{\frac{\partial f}{\partial x_1}(a),\frac{\partial f}{\partial x_2}(a),\dotsc,\frac{\partial f}{\partial x_n}(a)}\,.
	\end{equation*}
\end{definition}


Sei $f\colon \Omega\rightarrow \mathbb R$ eine partiell differenzierbare Funktion. Analog zum Fall von Funktionen in einer Variablen kann ein Punkt $a\in \Omega$ nur dann ein lokales Extremum von $f$ sein, wenn alle partiellen Ableitungen von $f$ in $a$ verschwinden.\footnote{Genau wir bei Funktionen in einer Variablen ist das natürlich nur eine notwendige, aber nicht unbedingt eine hinreichende Bedingung.} Diese Bedingung können wir sehr kompakt als $\nabla f(a)=0$ schreiben, wobei $0=(0,0,\dotsc,0)$ der Nullvektor in $\mathbb R^n$ ist.

Mit partiellen Ableitungen können wir Ungleichungen attackieren, die keine Nebenbedingung haben. Wie wir oben bereits beschrieben haben, ist es meistens hoffnungslos, die Gleichung (bzw.\ eigentlich das Gleichungssystem) $\nabla f(a)=0$ lösen zu wollen. Trotzdem lassen sich auf diese Weise Aufgaben lösen. Das werden wir nun an unserem ersten Beispiel demonstrieren.
\begin{aufgabe*}
	Gegeben seien positive reelle Zahlen $a,b,c>0$ mit $ab,bc,ca\geqslant 1$. Zeige, dass
	\begin{equation*}
		\sqrt[3]{\parens*{a^2+1}\parens*{b^2+1}\parens*{c^2+1}}\leqslant\parens*{\frac{a+b+c}{3}}^2+1\;.
	\end{equation*}
\end{aufgabe*}
(Diese Aufgabe hat eine wesentlich elegantere Lösung als die, die wir hier präsentieren werden. Findest du sie?)

\begin{proof}[Lösung]
	Wir erinnern uns, dass sich die AM-GM-Ungleichung beweisen lässt, indem die Jensensche Ungleichung auf die Logarithmusfunktion angewendet wird. Die Aufgabe sieht zunächst so aus, als wäre ein ähnlicher Trick möglich: Wir könnten versuchen, die Jensensche Ungleichung auf die Funktion $f(x)=\ln(x^2+1)$ anzuwenden \ldots\ wenn diese Funktion konkav wäre! Eine kurze Rechnung liefert aber
	\begin{equation*}
		f''(x)=\frac{2\parens*{1-x^2}}{\parens*{x^2+1}^2}\,,
	\end{equation*}
	was nur für $x\geqslant 1$ nichtpositiv ist. Also ist $f$ nur für $x\geqslant 1$ konkav. Die nächste Idee wäre, die Karamata-Schiebemethode zu verwenden (siehe das Kapitel \emph{Die Jensensche Ungleichung für nicht-konvexe Funktionen} im Heft für die Klasse~11). Mit dieser Methode lässt sich die Aufgabe auf den Fall reduzieren, dass zwei Variablen gleich sind oder dass in einer der Ungleichungen $ab,bc,ca\geqslant 1$ Gleichheit gilt. Aber damit lässt sich die Aufgabe nicht auf eine Variable reduzieren, denn die Nebenbedingung ist keine Gleichheit. Es ist also nicht klar, wie wir an dieser Stelle fortfahren würden.
	
	Stattdessen benutzen wir Analysis. Sei $\overline{\Omega}\coloneqq \braces*{(x,y,z)\ \middle|\ xy,yz,zx\geqslant 1}$ die Menge aller Tripel, für die die Nebenbedingung erfüllt ist, und sei $\Omega\coloneqq \braces*{(x,y,z)\ \middle|\ xy,yz,zx> 1}$. Dann ist $\Omega$ eine offene Menge, aber $\overline{\Omega}$ nicht. Betrachte die Funktion $g\colon\overline{\Omega}\rightarrow \mathbb R$ gegeben durch
	\begin{equation*}
		g(x,y,z)\coloneqq 3\ln\parens*{\parens*{\frac{x+y+z}{3}}^2+1}-\parens[\Big]{\ln\parens*{x^2+1}+\ln\parens*{y^2+1}+\ln\parens*{z^2+1}}\,.
	\end{equation*}
	Wir müssen zeigen, dass $g\geqslant 0$. Wenn $(a,b,c)\in\Omega$ ein Punkt ist, an dem $g$ ein lokales Extremum annimmt, dann muss $\nabla g(a,b,c)=0$ sein. Schreibe $s\coloneqq \frac{a+b+c}{3}$. Dann muss also
	\begin{equation*}
		0=\frac{\partial g}{\partial x}(a,b,c)=\frac{2s}{s^2+1}-\frac{2a}{a^2+1}
	\end{equation*}
	gelten. Indem wir diese Gleichung logarithmieren, erhalten wir
	\begin{equation*}
		\ln\parens*{s^2+1}-\ln\parens*{a^2+1}=\ln(s)-\ln(a)\,.
	\end{equation*}
	Analoge Gleichungen gelten auch für $b$ und $c$. Um die gewünschte Ungleichung
	\begin{equation*}
		\ln\parens*{s^2+1}-\parens*{\ln\parens*{a^2+1}+\ln\parens*{b^2+1}+\ln\parens*{c^2+1}}\geqslant 0
	\end{equation*}
	zu zeigen, genügt es also, die Ungleichung $3\ln(s)-(\ln(a)+\ln(b)+\ln(c))\geqslant 0$ zu zeigen. Diese Ungleichung folgt aber sofort aus Jensen! Folglich ist die gewünschte Ungleichung an allen lokalen Extrema von $g$ erfüllt!
	
	Damit haben wir den größten Teil der Aufgabe geschafft. Wir müssen nur noch das Verhalten von $g$ \enquote{im Unendlichen} sowie am Rand von $\overline{\Omega}$ (also für Punkte in $\overline{\Omega}\smallsetminus \Omega$) untersuchen. Das ist nicht schwer, aber etwas länglich.
	
	\emph{Verhalten von $g$ im Unendlichen.} Für $a,b,c\geqslant 1$ folgt die gewünschte Ungleichung einfach aus Jensen, denn $f(x)=\ln(x^2+1)$ ist konkav für $x\geqslant 1$. Ansonsten dürfen wir ohne Einschränkung $c<1$ und $a=\max\{a,b,c\}$ annehmen. Dann gilt $\sqrt[3]{(a^2+1)(b^2+1)(c^2+1)}<\sqrt[3]{2(a^2+1)^2}$ und $s^2+1> \frac19a^2+1$. Für hinreichend große $a$ ist aber $\sqrt[3]{2(a^2+1)^2}<\frac19a^2+1$. Folglich ist die Ungleichung für $a\rightarrow \infty$ erfüllt.
	
	\emph{Verhalten von $g$ am Rand von $\overline{\Omega}$.} Wenn $(a,b,c)$ auf dem Rand von $\Omega$ liegt, dürfen wir ohne Einschränkung $bc=1$ annehmen. Dann gilt $b\leqslant 1$ oder $c\leqslant 1$. In jedem Fall folgt aus $ab,ca\geqslant 1$, dass $a\geqslant 1$ sein muss. Zusammen mit $b+c=b+\frac1b\geqslant 2$ folgt auch $s\geqslant 1$. Nun betrachten wir zwei Fälle:
	
	\emph{Fall~1: Es gilt auch $ab=1$ oder $ca=1$.} Wir betrachten nur den Fall $ab=1$; der Fall $ac=1$ lässt sich völlig analog behandeln. Aus $ab=bc=1$ folgt $a=c=\frac1b$ und wir haben die Ungleichung auf eine Variable zurückgeführt. Durch Einsetzen erhalten wir die Ungleichung
	\begin{equation*}
		\frac{a^2+1}{a^{2/3}}=\sqrt[3]{\frac{(a^2+1)^3}{a^2}}\leqslant \parens*{\frac{2a+\frac1a}{3}}^2+1=\frac{4a^2+4+\frac1{a^2}}{9}+1\,.
	\end{equation*}
	Indem wir $t=a^2$ setzen, ausmultiplizieren und den Gleichheitsfall $t=1$ ausklammern, erhalten wir die äquivalente Ungleichung
	\begin{equation*}
		0\leqslant (t-1)^3\parens*{64t^3+87t^2-42t-1}\,,
	\end{equation*}
	welche für $t\geqslant 1$ offensichtlich erfüllt ist.
	
	\emph{Fall~2: Es gilt $ab,ca>1$.} Wir fixieren $b$ (und wegen $bc=1$ auch $c$) und betrachten die Funktion $h(x)=g(x,b,c)$. Wir zeigen zuerst, dass die gewünschte Ungleichung an den lokalen Extrema von $h$ erfüllt ist. Wenn $x=a$ ein solches lokales Extremum ist, folgt wie oben
	\begin{equation*}
		0=h'(a)=\frac{\partial g}{\partial x}(a,b,c)=\frac{2s}{s^2+1}-\frac{2a}{a^2+1}\,.
	\end{equation*}
	Diese Gleichung impliziert $s+\frac1s=a+\frac1a$. Wir haben weiter oben gesehen, dass $a,s\geqslant 1$ gilt. Also kann diese Gleichung nur für $a=s$ gelten. Aus $ab,ca>1$ und $bc=1$ folgt $a>b,c$. Also gilt
	\begin{equation*}
		\sqrt[3]{\parens*{a^2+1}\parens*{b^2+1}\parens*{c^2+1}}<\sqrt[3]{\parens*{a^2+1}^3}=a^2+1=s^2+1
	\end{equation*}
	und die gewünschte Ungleichung ist trivial. Damit gilt die Ungleichung an allen lokalen Extrema von $h$. Wir müssen nur noch das Verhalten von $h$ für $a\rightarrow \infty$ untersuchen. Das folgt aus unserer allgemeinen Analyse des Verhaltens von $g$ im Unendlichen.
\end{proof}

Da diese Lösung aus vielen Schritten bestand, lasst uns die entscheidende Idee hervorheben: Die Funktion $f(x)=\ln(x^2+1)$ ist zwar nicht überall konkav, aber die Bedingungen, die für ein lokales Extremum des Ausdruckes in der Aufgabenstellung gelten müssen, sind genau so beschaffen, dass wir die Jensensche Ungleichung trotzdem anwenden konnten! Damit haben wir die Aufgabe praktisch ausgetrickst. Alle weiteren Schritte in der obigen Lösung bestanden dann nur noch darin, das Verhalten im Unendlichen sowie am Rand des Definitionsbereiches zu untersuchen.

\subsection*{Die Langrange-Multiplikator-Methode}
Mit der Langrange-Multiplikator-Methode lassen sich mehrdimensionale Extremwertprobleme mit Nebenbedingungen lösen. Um die Methode zu beschreiben, betrachten wir eine Teilmenge $\overline{\Omega}\subseteq \mathbb R^n$. Sei $\Omega\subseteq \overline{\Omega}$ das \emph{Innere von $\overline{\Omega}$}, also die größte offene Teilmenge von $\mathbb R^n$, die in $\overline{\Omega}$ enthalten ist. Betrachte außerdem zwei stetige Funktionen $f,g\colon \overline{\Omega}\rightarrow \mathbb R$, die auf $\Omega$ \emph{stetig differenzierbar} sind. Das bedeutet, dass sie partiell differenzierbar sind und ihre partiellen Ableitungen stetig sind. Schließlich wollen wir $f(x_1,x_2,\dotsc,x_n)$ für $(x_1,x_2,\dotsc,x_n)\in\overline{\Omega}$ maximieren, wobei die Nebenbedingung $g(x_1,x_2,\dotsc,x_n)=0$ erfüllt sein soll.

\begin{satzmitnamen}[Lagrange-Multiplikator-Methode]
	Um das eben beschriebene Extremwertproblem zu lösen, können wir folgendermaßen vorgehen:
	\begin{enumerate}
		\item Finde alle $(n+1)$-Tupel $(a_1,a_2,\dotsc,a_n,\lambda)$ mit $a=(a_1,a_2,\dotsc,a_n)\in\Omega$ und $\lambda\in \mathbb R$, die Lösungen des folgenden Gleichungssystems sind:\label{schritt:Gleichungssystem}
		\begin{equation*}
			\left\{\begin{alignedat}{2}
				\frac{\partial f}{\partial x_1}(a)&-\lambda\frac{\partial g}{\partial x_1}(a)&&=0\,,\\
				&&&\mathrel{\tikz[inner sep=0,outer sep=0]{\node at (0,-0.5ex) {$\phantom{=}$};\node at (0,0) {$\vdots$};}}\\
				\frac{\partial f}{\partial x_n}(a)&-\lambda\frac{\partial g}{\partial x_n}(a)&&=0\,,\\
				&&\llap{$g(a_1,a_2,\dotsc,a_n)$}&=0\,.
			\end{alignedat}\right.
		\end{equation*}
		Alle diese Lösungen kommen als lokale Extrema von $f(x_1,x_2,\dotsc,x_n)$ unter der Nebenbedingung $g(x_1,x_2,\dotsc,x_n)=0$ in Frage. Die reelle Zahl $\lambda$ wird auch \emph{Lagrange-Multiplikator} genannt und gibt der Methode ihren Namen.
		\item Finde alle Lösungen von $\nabla g(a)=0$ mit $a=(a_1,a_2,\dotsc,a_n)\in\Omega$. Auch diese kommen als lokale Extrema von $f$ unter $g(x_1,x_2,\dotsc,x_n)=0$ in Frage.\label{schritt:SingulaerePunkte}
		\item Untersuche das Verhalten von $f$ am Rand des Definitionsbereiches, also auf $\overline{\Omega}\smallsetminus\Omega$, sowie \enquote{im Unendlichen} \embrace{falls $\overline{\Omega}$ eine unbeschränkte Menge ist}.\label{schritt:VerhaltenAmRand}
	\end{enumerate}
	Durch Einsetzen der erhaltenen Werte kann dann das Maximum von $f(x_1,x_2,\dotsc,x_n)$ unter der Nebenbedingung $g(x_1,x_2,\dotsc,x_n)=0$ ermittelt werden.
\end{satzmitnamen}
\begin{proof}
	Sei $a=(a_1,a_2,\dotsc,a_n)\in\Omega$ ein Punkt mit $\nabla g(a)\neq 0$. Bis auf Vertauschung der Variablen dürfen wir dann $\frac{\partial g}{\partial x_n}(a)\neq 0$ annehmen. Diese Bedingung garantiert, dass wir die Gleichung $g(x_1,x_2,\dotsc,x_n)=0$ nach $x_n$ umstellen können (zumindest in einer kleinen Umgebung von $a$). Es gibt also eine Funktion $h(x_1,x_2,\dotsc,x_{n-1})$, sodass
	\begin{equation*}
		g(x_1,x_2,\dotsc,x_n)=0\quad\Longleftrightarrow \quad x_n=h(x_1,x_2,\dotsc,x_{n-1})
	\end{equation*}
	(zumindest für $(x_1,x_2,\dotsc,x_n)$ in einer kleinen Umgebung von $a$). Um dieses Argument sauber durchführen zu können, brauchen wir den \emph{Satz von der impliziten Funktion}, den ihr im Studium kennenlernen werdet. Wenn ihr diesen Satz noch nicht kennt, könnt ihr euch trotzdem leicht überlegen, dass so eine Funktion $h$ für alle Nebenbedingungen existiert, die normalerweise in Olympiade-Aufgaben vorkommen (zum Beispiel für $x_1+x_2+\dotsb+x_n=1$ oder $x_1x_2\dotsm x_n=1$).
	
	Indem wir die Funktion $F(x_1,x_2,\dotsc,x_{n-1})\coloneqq f(x_1,x_2,\dotsc,x_{n-1},h(x_1,x_2,\dotsc,x_{n-1}))$ betrachten, können wir die Nebenbedingung loswerden. Betrachte nun den Fall, dass $a=(a_1,a_2,\dotsc,a_n)$ ein lokales Extremum von $f(x_1,x_2,\dotsc,x_{n-1})$ unter der Nebenbedingung $g(x_1,x_2,\dotsc,x_n)=0$ ist. Dann ist $a'\coloneqq (a_1,a_2,\dotsc,a_{n-1})$ ein lokales Extremum von $F$ und es gilt
	\begin{equation*}
		0=\frac{\partial F}{\partial x_i}(a')=\frac{\partial f}{\partial x_i}(a)+\frac{\partial f}{\partial x_n}(a)\frac{\partial h}{\partial x_i}(a')
	\end{equation*}
	(hier haben wir die mehrdimensionale Kettenregel verwendet; überlegt euch, dass sie in der Tat diese Form hat, indem ihr den Beweis der eindimensionalen Kettenregel anpasst). Nach Konstruktion von $h$ ist die Funktion $G(x_1,x_2,\dotsc,x_{n-1})\coloneqq g(x_1,x_2,\dotsc,x_{n-1},h(x_1,x_2,\dotsc,x_{n-1}))$ konstant $0$. Es gilt also auch
	\begin{equation*}
		0=\frac{\partial G}{\partial x_i}(a')=\frac{\partial g}{\partial x_i}(a)+\frac{\partial g}{\partial x_n}(a)\frac{\partial h}{\partial x_i}(a')\,.
	\end{equation*}
	Aus diesen beiden Gleichungen folgt: Wenn $\lambda=\frac{\partial f}{\partial x_n}(a)\mathop{\big/}\frac{\partial g}{\partial x_n}(a)$, dann erfüllt $(a_1,a_2,\dotsc,a_n,\lambda)$ das Gleichungssystem aus Schritt~\ref{schritt:Gleichungssystem}.
	
	Insgesamt haben wir also gesehen, dass alle lokalen Extrema $a=(a_1,a_2,\dotsc,a_n)\in\Omega$ von $f(x_1,x_2,\dotsc,x_{n-1})$ unter der Nebenbedingung $g(x_1,x_2,\dotsc,x_n)=0$ das Gleichungssystem aus Schritt~\ref{schritt:Gleichungssystem} oder die Gleichung $\nabla g(a)=0$ aus Schritt~\ref{schritt:SingulaerePunkte} erfüllen. Mit diesen beiden Schritten finden wir also alle lokalen Extrema. Mit Schritt~\ref{schritt:VerhaltenAmRand} finden wir dann auch alle möglichen Extrema auf dem Rand $\overline{\Omega}\smallsetminus \Omega$. Somit funktioniert die Lagrange-Multiplikator-Methode.
\end{proof}

Wir wenden die Lagrange-Multiplikator-Methode nun auf mehrere Olympiade-Aufgaben an. Ihr werdet sehen, dass wir nur in Aufgabe~\ref{aufgabe:MatBoj2014Lagrange} das Gleichungssystem aus Schritt~\ref{schritt:Gleichungssystem} tatsächlich lösen. In den anderen beiden Aufgaben werten wir das Gleichungssystem auf andere, nicht-triviale Weise aus.

Wir beginnen mit einer Aufgabe, die euch schon im Heft für die Klasse~11 begegnet ist. Dort haben wir eine elegante Lösung besprochen, hier kommt nun die Lösung mit der Brechstange.
\begin{aufgabe*}\label{aufgabe:AIMO2014}
	Gegeben seien positive reelle Zahlen $a,b,c,d>0$ mit $abcd=1$. Beweise die Ungleichung
	\begin{equation*}
		\frac{a^2}{a^3+1}+\frac{b^2}{b^3+1}+\frac{c^2}{c^3+1}+\frac{d^2}{d^3+1}\leqslant 2\,.
	\end{equation*}
\end{aufgabe*}
\begin{proof}[Lösung]
	Wir betrachten den gegebenen Ausdruck als Funktion in $a$, $b$, $c$ und $d$ und versuchen, diese unter der Nebenbedingung $abcd=1$ zu maximieren. Dazu benutzen wir die Lagrange-Multiplikator-Methode. Betrachte zuerst den Fall, dass $(a,b,c,d,\lambda)$ eine Lösung des Gleichungssystems aus Schritt~\ref{schritt:Gleichungssystem} ist. Dann erhalten wir
	\begin{equation*}
		0=\frac{\partial}{\partial a}\frac{a^2}{a^3+1}-\lambda\frac{\partial}{\partial a}(abcd)=\frac{2a-a^4}{\parens*{a^3+1}^2}-\lambda bcd\,.
	\end{equation*}
	Indem wir diese Gleichung mit $a$ multiplizieren und $abcd=1$ benutzen, erhalten wir
	\begin{equation*}
		\lambda=\lambda abcd=\frac{2a^2-a^5}{\parens*{a^3+1}^2}\eqqcolon g(a)\,.
	\end{equation*}
	Analog erhalten wir $\lambda=g(b)=g(c)=g(d)$. Ist das schon genug, um $a=b=c=d$ zu folgern? Nicht ganz! Wir rechnen nach, dass
	\begin{equation*}
		g'(x)=\frac{x}{\parens*{x^3+1}^3}\parens*{-2x^6-7x^3+4}
	\end{equation*}
	gilt. Der Bruch ist stets nichtnegativ. Mit der Substitution $t=x^3$ wird der zweite Faktor zu $-(2t^2+7t-4)$, was nur eine positive Nullstelle hat. Folglich hat auch $g'$ nur eine positive Nullstelle. Somit kann $g$ jeden Wert maximal zweifach annehmen und wir müssen uns in einem der folgenden Fälle befinden:
	
	\emph{Fall~1: $a=b=c=d=1$.} Dieser Fall ist trivial.
	
	\emph{Fall~2: $a=b=c$, $d=\frac1{a^3}$.} Durch Einsetzen können wir die Ungleichung nun auf eine Variable zurückführen. Indem wir alles auf einen Hauptnenner bringen und den Gleichheitsfall $a=1$ ausklammern, erhalten wir
	\begin{equation*}
		2-\frac{3a^2}{a^3+1}-\frac{\frac{1}{a^6}}{\frac1{a^9}+1}=\frac{(a-1)^2\parens*{2a^7+a^6-a^4+a^3+3a^2+4a+2}}{a^9+1}\geqslant 0\,.
	\end{equation*}
	Wegen $a^6-a^4+a^2\geqslant a^6-2a^4+a^2=a^4(a-1)^2$ ist der zweite Faktor im Zähler stets positiv und die Ungleichung gilt aus offensichtlichen Gründen.
	
	\emph{Fall~3: $a=b$, $c=d=\frac1a$.} Wir setzen ein und erhalten die offensichtliche Ungleichung
	\begin{equation*}
		\frac{2a^2}{a^3+1}+\frac{\frac{2}{a^2}}{\frac1{a^3}+1}=\frac{2\parens*{a^2+a}}{a^3+1}=\frac{2a}{(a-1)^2+a}\leqslant 2\,.
	\end{equation*}
	
	Damit haben wir Schritt~\ref{schritt:Gleichungssystem} erfolgreich abgeschlossen. Schritt~\ref{schritt:SingulaerePunkte} ist trivial, denn die partiellen Ableitungen von $g(a,b,c,d)\coloneqq abcd-1$ verschwinden nicht für $a,b,c,d>0$. Für Schritt~\ref{schritt:VerhaltenAmRand} müssen wir nur das Verhalten im Unendlichen untersuchen, weil wir keine Randpunkte vorliegen haben. Für $a\rightarrow \infty$ muss wegen $abcd=1$ mindestens eine Variable gegen $0$ gehen. Sagen wir, $b\rightarrow 0$. Damit gilt $\frac{a^2}{a^3+1}\rightarrow 0$ und $\frac{b^2}{b^3+1}\rightarrow 0$. Es genügt also, $\frac{x^2}{x^3+1}< 1$ für alle $x>0$ zu zeigen. Für $x\leqslant 1$ folgt diese Ungleichung aus $x^2\leqslant 1$ und für $x\geqslant 1$ folgt sie aus $x^3\geqslant x^2$.
	
	Damit haben wir alle Schritte geschafft und die Ungleichung ist bewiesen.
\end{proof}


\begin{aufgabe*}\label{aufgabe:MatBoj2014Lagrange}
	Finde die kleinste positive reelle Zahl $C>0$, sodass
	\begin{equation*}
		\frac{x^2}{yz}+\frac{y^2}{zx}+\frac{z^2}{xy}<C\parens*{\frac{x}{y}+\frac{y}{z}+\frac{z}{x}}
	\end{equation*}
	für alle $x,y,z>0$ gilt.
\end{aufgabe*}
\begin{proof}[Lösung]
	Sei $a=\frac xy$, $b=\frac yz$ und $c=\frac zx$. Dann gilt $abc=1$ und die linke Seite der gewünschten Ungleichung lässt sich als
	\begin{equation*}
		\frac{a}{c}+\frac{b}{a}+\frac{c}{b}=\frac{a^2b+b^2c+c^2a}{abc}=a^2b+b^2c+c^2a
	\end{equation*}
	schreiben. Wir suchen also das kleinste $C>0$ mit $a^2b+b^2c+c^2a<C(a+b+c)^3$ unter der Nebenbedingung $abc=1$. Weil die Ungleichung homogen in $a$, $b$ und $c$ ist, können wir die Nebenbedingung ignorieren und nur $a,b,c>0$ annehmen. Wie kann es nun sein, dass wir eine strikte Ungleichung bekommen? Wir vermuten, dass in Wirklichkeit die Ungleichung
	\begin{equation*}
		a^2b+b^2c+c^2a\leqslant C(a+b+c)^3
	\end{equation*}
	für alle $a,b,c\geqslant 0$ gilt, wobei Gleichheit nur eintritt, wenn eine der Variablen gleich $0$ ist. Um diese Vermutung zu beweisen und $C$ zu bestimmen, dürfen wir die neue Nebenbedingung $a+b+c=1$ einführen, denn die Ungleichung ist homogen in $a$, $b$ und $c$. Wir betrachten also die Menge $\overline{\Omega}=\mathbb R_{\geqslant 0}^3$, ihr Inneres $\Omega=\mathbb R_{>0}^3$ sowie die Funktionen $f,g\colon \overline{\Omega}\rightarrow \mathbb R$ gegeben durch $f(a,b,c)=a^2b+b^2c+c^2a$ und $g(a,b,c)=a+b+c-1$. Wir wollen $f(a,b,c)$ unter der Nebenbedingung $g(a,b,c)=0$ maximieren. Dafür benutzen wir die Lagrange-Multiplikator-Methode.
	
	Betrachten wir zuerst den Fall, dass $(a,b,c)$ ein lokales Extremum von $f$ unter der Nebenbedingung $g(a,b,c)=0$. Es gilt offenbar $\nabla g(a,b,c)=(1,1,1)\neq 0$.  Also muss eine reelle Zahl $\lambda$ existieren, sodass $(a,b,c,\lambda)$ Lösung des Gleichungssystems
	\begin{equation*}
		\left\{\begin{alignedat}{3}
			2ab&+c^2&&-\lambda&&=0\\
			2bc&+a^2&&-\lambda&&=0\\
			2ca&+b^2&&-\lambda&&=0\\
			&&&&\llap{$a+b+c$}&=1
		\end{alignedat}\right.
	\end{equation*}
	ist. Durch Subtraktion der ersten beiden Gleichungen folgt
	\begin{equation*}
		0=c^2-a^2+2ab-2bc=(c-a)(c+a-2b)
	\end{equation*}
	Also muss $c=a$ oder $c+a=2b$ gelten. Analog gilt $a=b$ oder $a+b=2c$ und $b=c$ oder $b+c=2a$. Wenn stets die erste Option gilt, muss $a=b=c$ sein. Wenn stets die zweite Option gilt, folgt auch $a=b=c$. Wenn mal die erste und mal die zweite Option erfüllt ist, also etwa $c=a$ und $a+b=2c$, so folgt ebenfalls $a=b=c$. Zusammen mit $a+b+c=1$ erhalten wir also $a=b=c=\frac13$ und somit $a^2b+b^2c+c^2a=\frac19$.
	
	Das Verhalten im Unendlichen müssen wir wegen $a,b,c\leqslant 1$ nicht untersuchen. Es bleibt das Verhalten auf dem Rand, also wenn eine der Variablen gleich $0$ ist. Ohne Einschränkung sei $c=0$. Dann müssen wir $a^2b$ unter der Nebenbedingung $a+b=1$ maximieren. Aus AM-GM folgt aber sofort
	\begin{equation*}
		a^2b=4\cdot\frac{a}{2}\cdot \frac{a}{2}\cdot b\leqslant 4\parens*{\frac{\frac a2+\frac a2+b}{3}}^3=\frac{4}{27}
	\end{equation*}
	mit Gleichheit genau für $a=\frac23$, $b=\frac13$. Damit haben wir unsere Antwort: Es gilt $C=\frac{4}{27}$. Gleichheit wird nur am Rand angenommen, weil, wie wir gesehen haben, alle lokalen Extrema stattdessen den Wert $\frac19$ haben. Somit gilt in der originalen Aufgabe die Ungleichung mit \enquote{$<$} statt \enquote{$\leqslant$}.
\end{proof}

\begin{aufgabe*}
	Gegeben sei der Ausdruck
	\begin{align*}
		T\coloneqq x_1x_2x_4+x_2x_3x_5+x_3x_4x_6+x_4x_5x_7+x_5x_6x_1+x_6x_7x_2+x_7x_1x_3
	\end{align*}
	für nichtnegative reelle Zahlen $x_1,\ldots,x_7\geqslant 0$ mit $x_1+x_2+\ldots+x_7=1$. Beweise, dass $T$ einen maximalen Wert annimmt und bestimme diesen.
\end{aufgabe*}
\begin{proof}[Lösung]
	Wir betrachten $T$ als Funktion in $x_1,x_2,\dotsc,x_7$ und wollen $T$ unter der Nebenbedingung $g(x_1,x_2,\dotsc,x_7)=0$ maximieren, wobei $g(x_1,x_2,\dotsc,x_7)\coloneqq x_1+x_2+\ldots+x_7=1$. Dafür benutzen wir die Lagrange-Multiplikator-Methode. Betrachte zuerst den Fall, dass $(x_1,x_2,\dotsc,x_7)$ ein lokales Extremum von $T$ unter der gegebenen Nebenbedingung ist. Der Gradient $\nabla g$ kann nicht verschwinden, denn $\nabla g(x_1,x_2,\dotsc,x_7)=(1,1,\dotsc,1)$. Also muss es eine reelle Zahl $\lambda$ geben, sodass das folgende Gleichungssystem erfüllt ist:
	\begin{equation*}
		\left\{\begin{alignedat}{3}
			x_2x_4&+x_5x_6&&+x_7x_3&&=\lambda\\
			x_1x_4&+x_3x_5&&+x_6x_7&&=\lambda\\
			&\mathrel{\tikz[inner sep=0,outer sep=0]{\node at (0,-0.5ex) {$\phantom{=}$};\node at (0,0) {$\vdots$};}}\\
			x_4x_5&+x_6x_2&&+x_1x_3&&=\lambda\\
			&&&&\llap{$x_1+x_2+\dotsb+x_7$}&=1
		\end{alignedat}\right.
	\end{equation*}
	Wenn wir die ersten sieben Gleichungen addieren, erhalten wir
	\begin{equation*}
		\sum_{i<j}x_ix_j=7\lambda\,.
	\end{equation*}
	Wenn wir die erste Gleichung mit $x_1$ multiplizieren, erhalten wir $x_1x_2x_4+x_5x_6x_1+x_7x_1x_3=\lambda x_1$. Die Terme auf der linken Seite treten genau so auch in $T$ auf.\footnote{Das sollte uns natürlich nicht verwundern, denn $T$ ist eine lineare Funktion in $x_1$. Wenn wir also $T$ nach $x_1$ ableiten und dann wieder mit $x_1$ multiplizieren, kriegen wir alles bis auf den \enquote{konstanten Term} (also den Teil von $T$, der kein $x_1$ enthält) zurück.} Das gleiche gilt, wenn wir die zweite Gleichung mit $x_2$ multiplizieren, die dritte Gleichung mit $x_3$ multiplizieren und so weiter. Wenn wir die entstehenden Gleichungen aufaddieren, erhalten wir unter Ausnutzung von $x_1+x_2+\dotsc+x_7=1$:
	\begin{equation*}
		3T=\lambda(x_1+x_2+\dotsb+x_7)=\lambda\,.
	\end{equation*}
	Nun können wir wie folgt abschätzen:
	\begin{equation*}
		\frac 17=\frac{x_1+\ldots+x_7}{7}\geqslant\sqrt{\frac{\sum_{i<j}x_ix_j}{21}}=\sqrt{\frac{\lambda}{3}}\,.
	\end{equation*}
	Die erste Gleichheit ist die Vorausetzung $x_1+\ldots+x_7=1$, die Abschätzung danach folgt aus der Maclaurin-Ungleichung (die wir weiter unten beweisen werden) und die letzte Gleichheit haben wir oben nachgeprüft. Es folgt
	\begin{equation*}
		T=\frac{\lambda}{3}\leqslant\frac{1}{49}\,.
	\end{equation*}
	
	Das Verhalten im Unendlichen müssen wir nicht untersuchen, weil offenbar $x_1,x_2,\dotsc,x_7\leqslant 1$ gilt. Es bleibt, das Verhalten am Rand zu untersuchen, also wenn mindestens eine der Variablen gleich $0$ ist. Ohne Einschränkung sei $x_1=0$. Die nun verbleibende Ungleichung 
	\begin{equation*}
		x_2x_3x_5+x_3x_4x_6+x_4x_5x_7+x_6x_7x_2=x_3(x_2x_5+x_4x_6)+x_7(x_4x_5+x_6x_2)\leqslant\frac{1}{27}
	\end{equation*}
	lässt sich mit der Schiebemethode bewerkstelligen. Halten wir $x_2,x_4,x_5,x_6$ sowie $x_3+x_7$ fest und variieren $x_3,x_7$, dann ist die linke Seite linear und nimmt daher ihr Maximum für $x_3=0$ oder $x_7=0$ an. Die nun verbleibende Ungleichung hat die Form 
	\begin{equation*}
		a(bc+de)\leqslant\frac{1}{27}\quad\text{für }a+b+c+d+e=1\;.
	\end{equation*}
	Mit AM-GM können wir ganz grob folgendermaßen abschätzen
	\begin{equation*}
		a(bc+de)\leqslant a\left(\frac{(b+c)^2+(d+e)^2}{4}\right)\leqslant a\left(\frac{b+c+d+e}{2}\right)^2 \leqslant\left(\frac{a+b+c+d+e}{3}\right)^3=\frac{1}{27}\,.
	\end{equation*}
	Insgesamt erhalten wir also $T\leqslant \frac1{27}$. Für $x_2=x_3=x_5=\frac13$ und $x_1=x_4=x_6=x_7=0$ wird dieser Wert auch tatsächlich angenommen. Damit haben wir gezeigt, dass ein minimaler Wert existiert und durch $\frac1{27}$ gegeben ist.
\end{proof}

In der obigen Lösung haben wir die \emph{Maclaurin-Ungleichung} verwendet. Weil sich diese ebenfalls auf sehr elegante und überraschende Art mit Analysis beweisen lässt, bildet sie den perfekten Abschluss dieses Kapitels.
\begin{satzmitnamen}[Maclaurin-Ungleichung]
	Gegeben seien nichtnegative reelle Zahlen $a_1,a_2,\dotsc,a_n\geqslant 0$. Für $k=1,2,\dotsc,n$ sei
	\begin{equation*}
		\sigma_k\coloneqq \sigma_k(a_1,a_2,\dotsc,a_n)= \sum_{1\leqslant i_1<i_2<\dotsb<i_k\leqslant n}a_{i_1}a_{i_2}\dotsm a_{i_k}
	\end{equation*}
	das $k$-te elementarsymmetrische Polynom in $a_1,a_2,\dotsc,a_n$. Dann gilt die folgende Kette von Ungleichungen:
	\begin{equation*}
		\frac{\sigma_1}{n}\geqslant\sqrt{\frac{\sigma_2}{\binom{n}{2}}}\geqslant \dotsb\geqslant \sqrt[k]{\frac{\sigma_k}{\binom{n}{k}}}\geqslant \dotsb\geqslant\sqrt[n]{\sigma_n}\,.
	\end{equation*}
\end{satzmitnamen}
Die Ungleichung $\sigma_1/n\geqslant \sqrt[n]{\sigma_n}$ ist offenbar genau die AM-GM-Ungleichung, sodass die Mac-laurin-Ungleichung eine Verschärfung von AM-GM darstellt. Im Fall $n=4$ erhalten wir die Ungleichung $\sqrt{\sigma_2/6}\geqslant \sqrt[3]{\sigma_3/4}$. Diese wurde bei der vierten Runde der 9.\ Mathematik-Olympiade 1970 gestellt und ist aufgrund ihrer Schwierigkeit als \emph{Pirlscher Hammer} in die Geschichte eingegangen.


\begin{proof}
	Wir werden für alle $k$ die Ungleichung $\sqrt[k-1]{\sigma_{k-1}/\binom{n}{k-1}}\geqslant \sqrt[k]{\sigma_k/\binom{n}{k}}$ zeigen. Betrachte dazu das Polynom
	\begin{equation*}
		P(X)=(X-a_1)(X-a_2)\dotsm (X-a_n)=X^n-\sigma_1 X^{n-1}+\sigma_2 X^{n-2}\mp\dotsb+(-1)^n\sigma_n\,.
	\end{equation*}
	Die Nullstellen von $P$ sind alle nichtnegativ. Zwischen je zwei Nullstellen muss $P$ einen Extremwert annehmen, also muss zwischen je zwei Nullstellen von $P$ auch eine Nullstelle der Ableitung $P'$ liegen. Folglich hat $P'$ mindestens $n-1$ nichtnegative Nullstellen. Weil $P'$ ein Polynom $(n-1)$-ten Grades ist, müssen somit alle Nullstellen von $P'$ nichtnegative reelle Zahlen sein. Dieser Schluss ist auch dann korrekt, wenn einige der Nullstellen von $P$ zusammenfallen, denn für jede $r$-fache Nullstelle von $P$ muss $P'$ an dieser Stelle eine $(r-1)$-fache Nullstelle haben.
	
	Betrachte nun die $(n-k)$-fache Ableitung
	\begin{equation*}
		P^{(n-k)}(X)=n(n-1)\dotsm (k+1)\parens*{X^k-\binom{k}{1}\sigma_1 X^{k-1}+\binom{k}{2}\sigma_2 X^{n-2}\mp\dotsb+(-1)^{k}\binom{k}{k}\sigma_k}\,.
	\end{equation*}
	Seien $b_1,b_2,\dotsc,b_k$ die Nullstellen von $P^{(n-k)}$. Dann gilt also
	\begin{align*}
		b_1b_2\dotsm b_{k-1}+b_1b_2\dotsm b_{k-2}b_k+\dotsb+b_2b_3\dotsm b_k=\binom{k}{k-1}\sigma_{k-1}\,,\quad b_1b_2\dotsm b_k= \binom{k}{k}\sigma_k\,.
	\end{align*}
	Nach dem obigen Argument sind die Nullstellen $b_1,b_2,\dotsc,b_k$ allesamt nichtnegative reelle Zahlen. Also können wir AM-GM anwenden und erhalten
	\begin{equation*}
		\frac{1}{k}\binom{k}{k-1}\sigma_{k-1}\geqslant \sqrt[k]{(b_1b_2\dotsm b_k)^{k-1}}=\sqrt[k]{\binom{k}{k}\sigma_k^{k-1}}\,.
	\end{equation*}
	Durch Umformungen erhalten wir die gewünschte Ungleichung
	\begin{equation*}
		\sqrt[k-1]{\frac{\sigma_{k-1}}{\binom{n}{k-1}}}\geqslant \sqrt[k]{\frac{\sigma_k}{\binom{n}{k}}}\,.\qedhere
	\end{equation*}
\end{proof}

\subsection*{Weitere Übungsaufgaben}
\begin{aufgabe*}
	Gegeben seien positive reelle Zahlen $a,b,c>0$ mit $abc=1$. Zeige die Ungleichung
	\begin{equation*}
		\sqrt{9+16a^2}+\sqrt{9+16b^2}+\sqrt{9+16c^2}\geqslant 3+4(a+b+c)\,.
	\end{equation*}
\end{aufgabe*}
\begin{aufgabe*}
	Gegeben seien positive reelle Zahlen $a,b,c,d>0$ mit $ac+bd=(a+c)(b+d)$. Was ist der minimale Wert, den der Ausdruck
	\begin{equation*}
		S\coloneqq\frac ab+\frac bc+\frac cd+\frac da
	\end{equation*}
	annehmen kann?
\end{aufgabe*}\newpage
	
	\cftaddtitleline{toc}{part}{Geometrie}{\thepage}
	\section{Das Sechs-Inkreise-Lemma}\label{kapitel:SechsInkreise}
In diesem Kapitel werdet ihr ein Lemma kennenlernen, das sich immer dann anwenden lässt, wenn in einer Aufgabe ein Tangentenviereck sowie mehrere weitere Inkreise vorkommen. Die Nützlichkeit dieses Lemmas werden wir an zwei Beispielaufgaben demonstrieren. Am Ende des Kapitels findet ihr Tipps zu den Aufgaben und am Ende des Heftes könnt ihr die Lösungen nachschlagen. Bevor ihr euch die Lösungen durchlest, solltet ihr aber auf jeden Fall versuchen, die Aufgaben (mit den Tipps) selbstständig zu lösen.
\begin{aufgabe*}\label{aufgabe:SechsInkreiseSehnenviereck}
	Sei $ABCD$ ein Tangentenviereck und $P$ ein innerer Punkt der Strecke $\overline{CD}$. Die Inkreismittelpunkte der Dreiecke $DAP$, $ABP$ und $BCP$ seien mit $I$, $J$ und $K$ bezeichnet. Beweise, dass $PIJK$ ein Sehnenviereck ist.
\end{aufgabe*}
\begin{aufgabe*}\label{aufgabe:531246}
	Sei $ABCD$ ein Tangentenviereck mit Inkreismittelpunkt $I$. Auf den Strecken $\overline{AI}$ und $\overline{CI}$ liegen Punkte $P$ und $Q$, sodass $\winkel QBP=\frac12\winkel CBA$ gilt. Zeige, dass dann auch $\winkel PDQ=\frac 12\winkel ADC$ gilt.
\end{aufgabe*}

Kommen wir nun zu dem Lemma, dem dieses Kapitel gewidmet ist. Bevor ihr euch den Beweis durchlest, solltet ihr versuchen, das Lemma selbst zu beweisen, denn das ist eine gute Übungsaufgabe.
\begin{satzmitnamen}[Sechs-Inkreise-Lemma]
	Wenn in der folgenden Skizze fünf der sechs eingezeichneten Inkreise $\omega_A$, $\omega_B$, $\omega_C$, $\omega_D$, $\omega$ und $\Omega$ existieren, dann existiert auch der sechste Inkreis.
\end{satzmitnamen}
\begin{figure}[ht]
	\centering
	\begin{tikzpicture}[x=0.85cm,y=0.85cm]
		\draw [line width=0.3] (-10.79,11.71) circle (2.66);
		\draw [line width=0.3] (-10.886,11.989) circle (1.045);
		\draw [line width=0.3] (-12.36,13.503) circle (0.55);
		\draw [line width=0.3] (-9.561,13.916) circle (0.803);
		\draw [line width=0.3] (-13.174,9.956) circle (1.066);
		\draw [line width=0.3] (-8.302,10.031) circle (0.822);
		\coordinate (A) at (-14.768,8.784);
		\coordinate (B) at (-7.19,9.28);
		\coordinate (C) at (-9.029,14.869);
		\coordinate (D) at (-12.77,13.97);
		\coordinate (Ba) at (-12.188,8.953);
		\coordinate (Ca) at (-12.018,10.983);
		\coordinate (Da) at (-13.895,11.048);
		\coordinate (Ab) at (-8.89,9.169);
		\coordinate (Cb) at (-7.701,10.832);
		\coordinate (Db) at (-9.439,10.893);
		\coordinate (Ac) at (-10.136,13.079);
		\coordinate (Bc) at (-8.471,13.174);
		\coordinate (Dc) at (-10.588,14.494);
		\coordinate (Ad) at (-13.18,12.904);
		\coordinate (Bd) at (-11.852,12.98);
		\coordinate (Cd) at (-11.749,14.215);
		\draw (A) to (B) to (C) to (D) to cycle;
		\draw (Ba) to (Cd);
		\draw (Ab) to (Dc);
		\draw (Da) to (Cb);
		\draw (Ad) to (Bc);
		\draw[fill=black] (A) circle (2pt) node[shift={(230:2ex)}] {$A$};
		\draw[fill=black] (Ab) circle (2pt) node[shift={(290:2ex)}] {$A_B$};
		\draw[fill=white] (Ac) circle (2pt) node[shift={(330:3.25ex)}] {$A_C$};
		\draw[fill=black] (Ad) circle (2pt) node[shift={(180:2.5ex)}] {$A_D$};
		\draw[fill=black] (B) circle (2pt) node[shift={(310:2ex)}] {$B$};
		\draw[fill=black] (Ba) circle (2pt) node[shift={(270:2ex)}] {$B_A$};
		\draw[fill=black] (Bc) circle (2pt) node[shift={(0:2.5ex)}] {$B_C$};
		\draw[fill=white] (Bd) circle (2pt) node[shift={(220:3ex)}] {$B_D$};
		\draw[fill=black] (C) circle (2pt) node[shift={(60:2ex)}] {$C$};
		\draw[fill=white] (Ca) circle (2pt) node[shift={(135:2.75ex)}] {$C_A$};
		\draw[fill=black] (Cb) circle (2pt) node[shift={(0:2.5ex)}] {$C_B$};
		\draw[fill=black] (Cd) circle (2pt) node[shift={(110:2.25ex)}] {$C_D$};
		\draw[fill=black] (D) circle (2pt) node[shift={(130:2ex)}] {$D$};
		\draw[fill=black] (Da) circle (2pt) node[shift={(180:2.5ex)}] {$D_A$};
		\draw[fill=white] (Db) circle (2pt) node[shift={(40:2.5ex)}] {$D_B$};
		\draw[fill=black] (Dc) circle (2pt) node[shift={(90:2ex)}] {$D_C$};
		\node at (-13.6,9.5) {$\omega_A$};
		\node at (-8.05,9.7) {$\omega_B$};
		\node at (-9.7,13.5) {$\omega_C$};
		\node at (-12.3,13.35) {$\omega_D$};
		\node at (-10.4,11.5) {$\omega$};
		\node at (-10.4,9.5) {$\Omega$};
	\end{tikzpicture}
\end{figure}

\begin{proof}
	Um nicht 16 verschiedene Berührpunkte benennen zu müssen, führen wir die folgende Nichtstandardnotation ein: Für zwei Kreise $\Omega_1$ und $\Omega_2$, die nicht ineinander liegen, sei $\abs{\Omega_1\Omega_2}$ die Länge der Strecke zwischen den Berührpunkten einer gemeinsamen äußeren Tangenten von $\Omega_1$ und $\Omega_2$. Da Punkte Kreise mit Radius $0$ sind, weitet sich die Notation auf den Fall aus, dass $\Omega_1=\{X\}$ ein Punkt ist. In diesem Fall ist $\abs{X\Omega_2}$ die Länge eines Tangentenabschnitts von $X$ an $\Omega_2$. Und falls $\Omega_2=\{Y\}$ selber ein Punkt ist, erhalten wir mit $\abs{XY}$ unsere Standardnotation für Streckenlängen zurück.
	
	Wir nehmen zunächst an, dass die Kreise $\omega_A$, $\omega_B$, $\omega_C$ und $\omega_D$ existieren. Wir wollen zeigen, dass $ABCD$ genau dann ein Tangentenviereck ist, wenn $A_CB_DC_AD_B$ ein Tangentenviereck ist. Mit der obigen Notation gilt $\abs{AB}=\abs{A\omega_A}+\abs{\omega_A\omega_B}+\abs{\omega_BB}$. Analoge Gleichungen gelten auch für die Seiten $\overline{BC}$, $\overline{CD}$ und $\overline{DA}$. Also ist
	\begin{equation*}
		\abs{AB}+\abs{CD}-\abs{BC}-\abs{DA}=\abs{\omega_A\omega_B}+\abs{\omega_C\omega_D}-\abs{\omega_B\omega_C}-\abs{\omega_D\omega_A}\,,
	\end{equation*}
	denn die Tangentenabschnitte $\abs{A\omega_A}$, $\abs{B\omega_B}$, $\abs{C\omega_C}$ und $\abs{D\omega_D}$ heben sich gerade weg. Völlig analog gilt $\abs{A_CB_D}=\abs{\omega_C\omega_D}-\abs{B_D\omega_D}-\abs{A_C\omega_C}$ und entsprechende Gleichungen für die anderen Seiten. Wir erhalten also 
	\begin{equation*}
		\abs{A_CB_D}+\abs{C_AD_B}-\abs{B_DC_A}-\abs{D_BA_C}=\abs{\omega_A\omega_B}+\abs{\omega_C\omega_D}-\abs{\omega_B\omega_C}-\abs{\omega_D\omega_A}\,.
	\end{equation*}
	Aus diesen Gleichungen folgt:
	\begin{align*}
		\abs{AB}+\abs{CD}=\abs{BC}+\abs{DA}&\Longleftrightarrow \abs{\omega_A\omega_B}+\abs{\omega_C\omega_D}=\abs{\omega_B\omega_C}+\abs{\omega_D\omega_A}\\
		&\Longleftrightarrow \abs{A_CB_D}+\abs{C_AD_B}=\abs{B_DC_A}+\abs{D_BA_C}\,.
	\end{align*}
	Mit anderen Worten: $ABCD$ ist ein Tangentenviereck genau dann, wenn $A_CB_DC_AD_B$ eines ist, sprich $\omega$ existiert genau dann, wenn $\Omega$ existiert.
	
	Jetzt nehmen wir an, dass $\omega$, $\Omega$ sowie drei der vier Kreise $\omega_A$, $\omega_B$, $\omega_C$ und $\omega_D$ existieren. Sei o.B.d.A.\ $\omega_A$ der fehlende. Sei $\omega_A'$ der Kreis, der die Strecken $\overline{AD}$, $\overline{AB_A}$ und $\overline{B_AC_D}$ berührt und sei $\ell$ die von $AB$ verschiedene gemeinsame äußere Tangente von $\omega_A'$ und $\omega_B$. Nach dem Teil, den wir schon bewiesen haben, muss $\ell$ auch den Kreis $\omega$ tangieren. Dann ist aber $\ell$ identisch mit der Geraden $D_AC_B$! Folglich ist $\omega_A'$ der Inkreis von $AB_AC_AD_A$, und insbesondere existiert der Inkreis $\omega_A=\omega_A'$.
\end{proof}

Das Sechs-Inkreise-Lemma ist insbesondere dann nützlich, wenn einige der kleinen Vierecke zu Punkten degenerieren (sodass ihre Inkreise aus trivialen Gründen existieren). Macht euch an Beispielen klar, wie solche degenerierten Spezialfälle aussehen und warum auch in diesen Fällen die Aussage des Lemmas alles andere als trivial ist!

\vfill\hrule\vspace{-1em}

\subsection*{Tipps zu den Beispielaufgaben}
\textbf{Tipps zu Aufgabe~\ref{aufgabe:SechsInkreiseSehnenviereck}.} Betrachte die von $CD$ verschiedene gemeinsame äußere Tangenten der Inkreise von $DAP$ und $BCP$. Was fällt dir auf? Kannst du deine Beobachtung auf einen Spezialfall des Sechs-Inkreise-Lemmas zurückführen? Kannst du mit deiner Beobachtung die Aufgabe lösen?

\textbf{Tipps zu Aufgabe~\ref{aufgabe:531246}.} Die Beobachtung $\winkel PBA=\winkel QBI$ drängt sich auf, aber bringt dich (vermutlich) nicht weiter.

Um bei dieser Aufgabe das Sechs-Inkreise-Lemma (in einem Spezialfall) anwenden zu können, musst du zusätzlich zum Inkreis von $ABCD$ noch zwei weitere Kreise einführen. Welche Kreise bieten sich an? Kannst du mithilfe dieser Kreise die Voraussetzung $\winkel QBP=\frac12\winkel CBA$ umformulieren?\newpage
	\section{Symmediane}\label{kapitel:Symmediane}
In diesem Kapitel werden wir zwei schwere Geometrieaufgaben besprechen. Wie üblich findet ihr am Ende des Kapitels Tipps und danach Lösungen zu beiden Aufgaben. Angesichts der Schwierigkeit ist es keine Schande, wenn ihr die Tipps verwendet oder euch die Lösungen durchlest.
\begin{aufgabe*}[**]\label{aufgabe:MEMO2014}
	Sei $ABC$ ein Dreieck und sei $D$ der Berührpunkt des Inkreises $\omega$ mit der Seite $\overline{BC}$. Sei $I_a$ der Mittelpunkt des Ankreises gegenüber $A$ und sei $L$ der von $D$ verschiedene Schnittpunkt von $AD$ mit $\omega$. Schließlich sei $M$ der Mittelpunkt von $\overline{BC}$ und $N$ der Mittelpunkt von $\overline{I_aM}$. Beweise, dass die Punkte $B$, $C$, $N$ und $L$ auf einem Kreis liegen.
\end{aufgabe*}
\begin{aufgabe*}[***]\label{aufgabe:PolenMO2019}
	Sei $ABC$ ein Dreieck mit Umkreis $\Omega$. Sei $S$ der Mittelpunkt des Bogens $\wideparen{BC}$, der $A$ nicht enthält, und sei $M$ der Mittelpunkt der Seite $\overline{BC}$. Sei $\omega$ ein Kreis mit Mittelpunkt $S$, der $\overline{BC}$ in $M$ berührt. Die Tangenten an $\omega$ durch $A$ schneiden $\overline{BC}$ in den Punkten $D$ und $E$, wobei $D$ zwischen $B$ und $E$ liegt. Sei $\omega_E$ ein Kreis, der die Strecken $\overline{AE}$, $\overline{BE}$ sowie den Kreis $\Omega$ in einem Punkt $P$ berührt. Sei $\omega_D$ ein Kreis, der die Strecken $\overline{AD}$, $\overline{CD}$ sowie den Kreis $\Omega$ in einem Punkt $Q$ berührt. Schließlich sei $R$ der Schnittpunkt von $DQ$ und $EP$. Beweise, dass $\winkel DAM=\winkel RAE$ gilt.
\end{aufgabe*}
Um bei diesen Aufgaben überhaupt eine Chance zu haben, müssen wir etwas Theorie einführen, die allgemein sehr nützlich zu wissen ist.	

\begin{definition}
	Sei $ABC$ ein Dreieck. Der \emph{Symmedian durch $A$} ist das Spiegelbild der Seitenhalbierenden von $\overline{BC}$ an der Winkelhalbierenden von $\winkel BAC$. Analog sind die Symmediane durch $B$ und $C$ definiert.
\end{definition}
Genauso wie sich die Seitenhalbierenden schneiden sich auch die drei Symmediane eines Dreiecks in einem Punkt (dem \emph{Lemoine-Punkt}). Das ist ein Spezialfall einer allgemeineren Aussage: Wenn sich drei Geraden durch $A$, $B$ und $C$ in einem Punkt $P$ schneiden, dann schneiden sich auch ihre Spiegelbilder an den entsprechenden Winkelhalbierenden in einem Punkt (dem \emph{isogonal konjugierten Punkt zu $P$}). Das folgt zum Beispiel trivial aus dem trigonometrischen Satz von Ceva.

Interessant werden Symmedianen durch das folgende Lemma, das uns in einigen Aufgaben eine ungemein praktische Winkelgleichheit liefert.
\begin{satzmitnamen}[Symmedian-Lemma]
	Sei $ABC$ ein Dreieck mit Umkreis $\Omega$ und sei $T$ der Schnittpunkt der Tangenten in $B$ und $C$ an $\Omega$. Dann ist $AT$ der Symmedian durch $A$. Insbesondere gilt $\winkel BAT=\winkel MAC$, wenn $M$ den Mittelpunkt von $\overline{BC}$ bezeichnet.
\end{satzmitnamen}
\begin{figure}[ht]
	\centering
	\begin{tabularx}{\textwidth}{X c X c X}
		& \begin{tikzpicture}[x=0.35cm,y=0.35cm]
			%\clip (-4.66,-6.97) rectangle (5.17,6.77);
			\draw [line width=0.3] (-0.186,1.372) circle (4.405);
			\coordinate (A) at (1.224,5.545);
			\coordinate (B) at (-2.984,-2.03);
			\coordinate (C) at (4,0);
			\coordinate (M) at (0.508,-1.015);
			\coordinate (T) at (1.993,-6.125);
			\draw (A) to (B) to (C) to cycle;
			\draw [line width=0.3] (M) to (A) to (T);
			\draw [line width=0.3,shorten <=-2em,shorten >=-2ex] (B) to (T);
			\draw [line width=0.3,shorten <=-2em,shorten >=-2ex] (C) to (T);
			\draw [line width=0.3,shift={(A)}] (240.945:0.57cm) arc (240.945:263.77:0.57cm);
			\draw [line width=0.3,shift={(A)}] (240.945:0.52cm) arc (240.945:263.77:0.52cm);
			\draw [line width=0.3,shift={(A)}] (273.77:0.57cm) arc (273.77:296.595:0.57cm);
			\draw [line width=0.3,shift={(A)}] (273.77:0.52cm) arc (273.77:296.595:0.52cm);
			\draw [fill=black] (A) circle (2pt) node[shift={(80:2ex)}] {$A$};
			\draw [fill=black] (B) circle (2pt) node[shift={(230:2ex)}] {$B$};
			\draw [fill=black] (C) circle (2pt) node[shift={(-20:2ex)}] {$C$};
			\draw [fill=black] (M) circle (2pt) node[shift={(280:2ex)}] {$M$};
			\draw [fill=black] (T) circle (2pt) node[shift={(200:2ex)}] {$T$};
			\node at (-3.8,2) {$\Omega$};
		\end{tikzpicture} & & \begin{tikzpicture}[x=0.35cm,y=0.35cm]
			%\clip (-4.96,-3.45) rectangle (12.98,9.74);
			\draw [line width=0.3] (-0.728,3.173) circle (3.072);
			\draw [line width=0.3] (6.924,3.677) circle (5.998);
			\coordinate (A) at (1.224,5.545);
			\coordinate (B) at (-1.711,0.263);
			\coordinate (C) at (5.004,-2.006);
			\coordinate (M) at (1.647,-0.872);
			\coordinate (T) at (1.518,1.077);
			\draw (B) to (A) to (C);
			\draw [line width=0.3,shorten <=-3.5em,shorten >=-4.5em] (B) to (C);
			\draw [line width=0.3] (M) to (A) to (T);
			\draw [line width=0.3,shift={(A)}] (273.77:0.52cm) arc (273.77:296.595:0.52cm);
			\draw [line width=0.3,shift={(A)}] (273.77:0.57cm) arc (273.77:296.595:0.57cm);
			\draw [fill=black] (A) circle (2pt) node[shift={(105:2.5ex)}] {$A$};
			\draw [fill=black] (B) circle (2pt) node[shift={(240:2.25ex)}] {$B'$};
			\draw [fill=black] (C) circle (2pt) node[shift={(270:2ex)}] {$C'$};
			\draw [fill=white] (M) circle (2pt);
			\draw [fill=black] (T) circle (2pt) node[shift={(-2:2.25ex)}] {$T'$};
			\node at (-4.4,0.3) {$\Omega'$};
		\end{tikzpicture} & \\
		& vor Inversion & & nach Inversion &
	\end{tabularx}
\end{figure}
\begin{proof}
	Wir invertieren an $A$; die Bildpunkte von $B$, $C$ und $T$ bezeichnen wir mit $B'$, $C'$ und $T'$. Der Kreis $\Omega$ durch $A$, $B$ und $C$ wird auf die Gerade $B'C'$ abgebildet. Die Tangenten an $\Omega$ in $B$ und $C$ werden auf Kreise durch $A$ abgebildet, die das Bild von $\Omega$, also die Gerade $B'C'$, berühren. Schließlich ist $T'$ der von $A$ verschiedene Schnittpunkt dieser beiden Kreise. Insbesondere muss $AT'$ die Potenzgerade der beiden Kreise sein. Folglich führt $AT'$ durch den Mittelpunkt ihrer gemeinsamen Tangente $\overline{B'C'}$, also ist $AT'$ die Seitenhalbierende von $\overline{B'C'}$. Nach einer bekannten Eigenschaft der Inversion sind $ABC$ und $AB'C'$ gegensinnig ähnlich sind. Weil die Seitenhalbierenden $AM$ und  $AT'$ einander entsprechen, erhalten wir $\winkel MAC=\winkel B'AT'$. Andererseits erhält Inversion an $A$ Winkel mit Scheitelpunkt $A$. Also gilt $\winkel BAT=\winkel B'AT'$. Es folgt $\winkel BAT=\winkel MAC$. Damit ist gezeigt, dass~$T$ auf dem Symmedian durch~$A$ liegt.
\end{proof}

Im Kapitel \emph{Projektive Geometrie} im Heft für Klasse~11 habt ihr gelernt, dass zwei Punktepaare $(A,B)$ und $(C,D)$ auf einem Kreis $\Omega$ genau dann harmonisch liegen, wenn der Pol von $AB$ auf $CD$ liegt. Aber der Pol von $AB$ ist genau der Schnittpunkt der Tangenten an $\Omega$ in $A$ und $B$. Wir erhalten also auch, dass $(A,B)$ und $(C,D)$ genau dann harmonische Punktepaare sind, wenn $D$ auf dem Symmedian von $ABC$ durch $C$ liegt.

\vfill\hrule\vspace{-1em}

\subsection*{Tipps zu den Beispielaufgaben}
\textbf{Tipps zu Aufgabe~\ref{aufgabe:MEMO2014}.} Was weißt du über die Gerade $AD$ sowie die In- und Ankreisberührpunkte mit $\overline{BC}$? Fällt dir in deiner Skizze etwas auf?

Findest du die Winkel $\winkel BLD$ und $\winkel DLC$ an anderer Stelle in deiner Skizze wieder? Kannst du damit zeigen, dass $BNCL$ ein Sehnenviereck ist?

\textbf{Tipps zu Aufgabe~\ref{aufgabe:PolenMO2019}.} Zeige, dass die Behauptung äquivalent dazu ist, dass $AR$ der Symmedian durch $A$ im Dreieck $ABC$ ist. Nach dem Symmedian-Lemma verläuft dieser Symmedian auch durch den Schnittpunkt $T$ der Tangenten an $\Omega$ in $B$ und $C$. Zeichne $T$ in deine Skizze ein. Was fällt dir dabei auf?

Die Aussage ist nun äquivalent dazu, dass sich $AT$, $DQ$ und $EP$ in einem Punkt schneiden. Formuliere diese Aussage mithilfe des Satzes von Desargues um. Um die neue Aussage zu zeigen, betrachte zum Beispiel die Streckung mit Zentrum $P$, die $\omega_E$ auf $\Omega$ abbildet.\newpage
	\section{Orientierte Winkel modulo \texorpdfstring{$\boldsymbol{180^\circ}$}{180°}}\label{kapitel:OrientierteWinkel}
Lagebeziehungsfallunterscheidungen sind lästig, rauben wertvolle Zeit beim Aufschreiben und führen zu unnötigen Punktabzügen, wenn ihr sie nicht ordentlich macht. In diesem Kapitel lernt ihr eine Technik kennen, mit der ihr euch diesen Ärger sparen könnt.

Wir beginnen mit einigen allgemeinen Betrachtungen. Seien $g$ und $h$ zwei Geraden, die sich in einem Punkt $S$ schneiden. Sei $K_{g,h}$ die Menge aller Winkel $\varphi$, sodass eine Drehung mit Zentrum $S$ und Winkel $\varphi$ die Gerade $g$ auf die Gerade $h$ abbildet. Wenn $\varphi_0$ in der Menge $K_{g,h}$ liegt, dann sind $\varphi_0 + 180^\circ,\varphi_0 + 360^\circ,\varphi_0 + 540^\circ,\dotsc$ offensichtlich ebenfalls Elemente von $K_{g,h}$. Das gleiche gilt für $\varphi_0 - 180^\circ,\varphi_0 - 360^\circ,\dotsc$. Umgekehrt muss aber auch jeder Winkel $\varphi\in K_{g,h}$ von der Form $\varphi=\varphi_0 + k\cdot 180^\circ$ für ein $k\in \mathbb Z$ sein. Denn wenn sowohl eine Drehung um $\varphi_0$ als auch eine Drehung um $\varphi$ die Gerade $g$ auf $h$ abbildet, dann muss eine Drehung um $\varphi - \varphi_0$ die Gerade $g$ auf sich selber abbilden, also muss $\varphi - \varphi_0$ ein ganzzahliges Vielfaches von $180^\circ$ sein. Wir erhalten folglich
\begin{equation*}
	K_{g,h}=\braces*{\varphi_0 + k\cdot 180^\circ\ \middle|\  k\in \mathbb Z}\,.
\end{equation*}
Mit anderen Worten, $K_{g,h}$ ist eine \emph{Restklasse modulo $180^\circ$}. Das soll Folgendes heißen: Genau wie für ganze Zahlen schreiben wir $\alpha\equiv\beta \mod 180^\circ$ für zwei Winkel $\alpha$ und $\beta$, wenn $\alpha - \beta$ ein ganzzahliges Vielfaches von $180^\circ$ ist. In diesem Sinne ist dann
\begin{equation*}
	K_{g,h}=\braces*{\varphi\ \middle|\ \varphi\equiv\varphi_0 \mod 180^\circ}\,,
\end{equation*}
wofür die Bezeichnung \emph{Restklasse modulo $180^\circ$} allemal gerechtfertigt ist.
\begin{definition}
	Mit $\winkel(g,h)$ bezeichnen wir einen beliebigen Repräsentanten $\varphi_0\in K_{g,h}$ der Restklasse $K_{g,h}$. Falls $g$ und $h$ parallele Geraden sind, definieren wir $\winkel(g,h)\coloneqq 0^\circ$.
\end{definition}

Genau wie bei ganzen Zahlen könnt ihr euch überlegen, dass Addition und Subtraktion von Winkeln sowie Multiplikation mit ganzen Zahlen auch modulo $180^\circ$ funktionieren. Solange wir also nur modulo $180^\circ$ rechnen, kommt es nicht darauf an, welchen konkreten Repräsentanten wir für $\winkel (g,h)$ gewählt haben.

Der Vorteil von orientierten Winkeln modulo $180^\circ$ liegt darin, dass sie sich unter Addition so verhalten, wie wir es intuitiv erwarten würden.
\begin{satzmitnamen}[Eigenschaften von orientierten Winkeln modulo $\boldsymbol{180^\circ}$]
	Sei $n\geqslant 2$ und seien $g_1,\dotsc,g_n$ Geraden in der Ebene.
	\begin{enumerate}
		\item Es gilt $\winkel(g_1,g_2) + \winkel(g_2,g_3)+ \dotsb + \winkel(g_{n-1},g_n) \equiv \winkel(g_1,g_n) \mod 180^\circ$.\label{eigenschaft:OrientierteWinkelAdditiv}
		\item Es gilt $\winkel(g_1,g_2) + \winkel(g_2,g_3) + \dotsb + \winkel(g_n,g_1) \equiv 0^\circ  \mod 180^\circ$.\label{eigenschaft:OrientierteWinkelAddierenZuNull}
		\item Wenn $g$ und $h$ beliebige Geraden sind, dann ist $\winkel(g,h) \equiv -\winkel(h,g) \mod 180^\circ$.\label{eigenschaft:OrientierungVorzeichen}
	\end{enumerate}
\end{satzmitnamen}
\begin{proof}
	Wir beweisen zunächst Eigenschaft~\ref{eigenschaft:OrientierteWinkelAddierenZuNull}. Dafür benutzen wir folgenden Fakt: Wenn $\sigma$ und $\tau$ zwei Drehungen oder Verschiebungen sind, dann ist ihre Hintereinanderausführung $\sigma\circ\tau$ wieder eine Drehung oder Verschiebung (das gilt insbesondere auch dann, wenn $\sigma$ und $\tau$ zwei Drehungen sind, die \emph{nicht} das gleiche Zentrum haben). Wenn ihr diesen Fakt noch nicht kanntet, könnt ihr ihn im Kapitel \emph{\embrace{Dreh-}Streckungen} im Heft für die Klasse~10 nachlesen.
	
	Für $i=1,2,\dotsc,n$ sei $\sigma_i$ eine Drehung um den Schnittpunkt von $g_i$ und $g_{i+1}$, die $g_i$ auf $g_{i+1}$ abbildet (dabei setzen wir $g_{n+1}\coloneqq g_1$). Nach Definition ist der Drehwinkel von $\sigma_i$ genau $\winkel (g_i,g_{i+1})$. Falls $g_i$ und $g_{i+1}$ parallel sind, definieren wir $\sigma_i$ stattdessen als eine Verschiebung, die $g_i$ auf $g_{i+1}$ abbildet. Auch in diesem Fall ist der \glqq Drehwinkel\grqq\ von $\sigma$ genau $\winkel (g_i,g_{i+1})=0^\circ$. Nach obigem Fakt ist die Hintereinanderausführung $\sigma\coloneqq \sigma_n\circ\sigma_{n-1}\circ\dotsb\circ\sigma_1$ wieder eine Drehung oder eine Verschiebung. Da sich Drehwinkel bei Hintereinanderausführung addieren, ist der Drehwinkel von $\sigma$ durch $\winkel(g_1,g_2) + \winkel(g_2,g_3) + \dotsb + \winkel(g_n,g_1)$ gegeben. Andererseits bildet $\sigma$ die Gerade $g_1$ auf sich selbst ab. Der Drehwinkel muss also $\equiv 0^\circ\mod 180^\circ$ sein. Genau das wollten wir zeigen.
	
	Nachdem Eigenschaft~\ref{eigenschaft:OrientierteWinkelAddierenZuNull} nun bewiesen ist, folgt Eigenschaft~\ref{eigenschaft:OrientierungVorzeichen} sofort als der Spezialfall $n=2$. Schließlich folgt Eigenschaft~\ref{eigenschaft:OrientierteWinkelAdditiv} aus einer Kombination von~\ref{eigenschaft:OrientierteWinkelAddierenZuNull} und~\ref{eigenschaft:OrientierungVorzeichen}.
\end{proof}

Bisher haben wir nur über Winkel zwischen Geraden geredet und nicht über solche, die von zwei Punkten $A$ und $B$ sowie einem Scheitelpunkt $S$ aufgespannt werden. Als Ersatz für den Winkel $\winkel ASB$ werden wir immer den orientierten Winkel $\winkel(AS,BS)$ modulo $180^\circ$ verwenden.

Die meisten Sätze über Winkel lassen sich zu Sätzen über orientierte Winkel modulo $180^\circ$ umformulieren. Üblicherweise wird die Aussage dabei eleganter. Zum Beispiel lassen sich der Peripheriewinkelsatz und der Satz über die gegenüberliegenden Winkel im Sehnenviereck zu folgendem Satz vereinigen -- eleganter geht es nicht:
\begin{satzmitnamen}[Orientierter Peripheriewinkelsatz]
	Vier Punkte $A$, $B$, $C$ und $D$ liegen genau dann auf einem Kreis, wenn Folgendes gilt:
	\begin{equation*}
		\winkel(AC,BC)\equiv\winkel(AD,BD) \mod 180^\circ\,.
	\end{equation*}
\end{satzmitnamen}
Wenn ihr in einer Aufgabe zeigen sollt, dass vier Punkte auf einem Kreis liegen, könnt ihr wie gewohnt eine Winkeljagd durchführen, ohne euch zunächst darüber Gedanken zu machen, ob es noch andere Lagebeziehungen als in eurer Skizze geben könnte. Beim Aufschreiben übersetzt ihr dann eure Argumente in orientierte Winkel modulo $180^\circ$ und formuliert alle Umformungen mithilfe der obigen Eigenschaften. Mit etwas Übung dauert das kaum länger als normalerweise. Dadurch habt ihr alle anderen Lagefälle automatisch mit abgedeckt, ihr spart euch jede Menge Zeit und ihr bekommt keine ärgerlichen Punktabzüge.


Ebenso elegant lässt sich der Sehnen-Tangentenwinkelsatz formulieren:
\begin{satzmitnamen}[Orientierter Sehnen-Tangentenwinkelsatz]
	Sei $ABC$ ein Dreieck. Eine Gerade $t$ durch $B$ ist genau dann Tangente an den Umkreis $\odot ABC$, wenn Folgendes gilt:
	\begin{equation*}
		\winkel(AC,BC)\equiv\winkel(AB,t) \mod 180^\circ\,.
	\end{equation*}
\end{satzmitnamen}
Beim Zentri-Peripheriewinkelsatz müssen wir ein wenig aufpassen:
\begin{satzmitnamen}[Orientierter Zentri-Peripheriewinkelsatz]\label{prop:ZPWS}
	Sei $\Omega$ ein Kreis mit Mittelpunkt $O$ und $A$, $B$ zwei gegebene Punkte auf $\Omega$.
	\begin{enumerate}
		\item Sei $M$ der Mittelpunkt von $\overline{AB}$. Dann liegt ein dritter Punkt $C$ genau dann auf $\Omega$, wenn
		\begin{equation*}
			\winkel(AC,BC)\equiv \winkel(AO,MO)\mod 180^\circ\,.
		\end{equation*}
		\item Wenn $C$ auf $\Omega$ liegt, dann gilt
		\begin{equation*}
			2\winkel(AC,BC)\equiv \winkel(AO,BO)\mod 180^\circ\,.
		\end{equation*}
	\end{enumerate}
\end{satzmitnamen}
\begin{itemize}
	\item[\Warnung] Beachte, dass die zweite Aussage im orientierten Zentri-Peripheriewinkelsatz \emph{keine} \glqq genau dann, wenn\grqq-Aussage ist: Die Bedingung $2\winkel(AC,BC)\equiv \winkel(AO,BO)\mod 180^\circ$ reicht noch nicht aus, um zu schließen, dass $C$ auf dem Kreis $\Omega$ um $O$ durch $A$ und $B$ liegt. Wenn wir nämlich $\alpha$ fixieren, wird die Gleichung $2\winkel(AC,BC)\equiv \alpha\mod180^\circ$ sowohl von $\winkel ACB=\alpha/2$, als auch von $\winkel ACB=90^\circ+\alpha/2$ gelöst, aber nur einer von beiden Fällen sorgt auch dafür, dass $C$ auf $\Omega$ liegt.
	
	Allgemein lassen sich orientierte Winkel modulo $180^\circ$ \emph{nicht} durch ganze Zahlen teilen. Zum Beispiel haben wir gerade gesehen, dass der Ausdruck \glqq$\alpha/2$\grqq\ modulo $180^\circ$ gar nicht wohldefiniert ist. Praktisch hat das aber so gut wie keine Auswirkungen, solange ihr eure Lösungen geschickt formuliert.
	\item[\Warnung] Orientierte Winkel modulo $180^\circ$ können wir auch nicht einfach so in Sinus und Cosinus einsetzen; zum Beispiel ist ja $\sin(\alpha + 180^\circ)=-\sin(\alpha)$, obwohl $\alpha + 180^\circ\equiv \alpha\mod180^\circ$ gilt. Andererseits ist das Vorzeichen aber auch das einzige Problem. Die Ausdrücke $\abs{\sin (\alpha)}$ und $\abs{\cos(\alpha)}$ sind also immer noch wohldefiniert, auch wenn $\alpha$ nur modulo $180^\circ$ bestimmt ist. Zudem haben Tangens und Cotangens die Periode $180^\circ$. Hier kommen wir also überhaupt nicht in Schwierigkeiten.
\end{itemize}
Mit orientierten Winkeln modulo $180^\circ$ lässt sich ein denkbar einfaches Kriterium formulieren, wann sich drei Kreise in einem Punkt schneiden. %Der bekannte \emph{Satz von Miquel} und seine Varianten sind einfache Folgerungen.
\begin{satzmitnamen}[Lemma]
	Gegeben seien sechs Punkte $A$, $B$, $C$ und $A'$, $B'$, $C'$. Dann schneiden sich sie Umkreise $\odot ABC'$, $\odot BCA'$, $\odot CAB'$ genau dann in einem Punkt, wenn die folgende Bedingung gilt:
	\begin{equation*}
		\winkel(AC',BC') + \winkel (BA',CA') + \winkel(CB', AB')\equiv 0 \mod 180^\circ\,.
	\end{equation*}
\end{satzmitnamen}

\begin{proof}
	Sei $T$ der von $B$ verschiedene Schnittpunkt der Umkreise $\odot ABC'$ und $\odot BCA'$. Nach dem orientierten Peripheriewinkelsatz liegt $T$ genau dann auf $\odot CAB'$, wenn die Kongruenz $\winkel(CT,AT)\equiv\winkel (CB',AB') \mod 180^\circ$ gilt. Nun ist aber
	\begin{alignat*}{2}
		0^\circ&\equiv \winkel(CT,AT) +\winkel(AT,BT) + \winkel(BT,CT) &&\mod 180^\circ\\
		&\equiv \winkel(CT,AT) + \winkel(AC',BC') + \winkel(BA',CA') &&\mod 180^\circ\,,
	\end{alignat*}
	wobei wir Eigenschaft~\ref{eigenschaft:OrientierteWinkelAddierenZuNull} sowie den orientierten Peripheriewinkelsatz für die Umkreise $\odot ABC'$ und $\odot BCA'$ verwendet haben. Die Behauptung folgt sofort.
\end{proof}
\begin{figure}[ht]
	\centering
	\begin{tabularx}{\textwidth}{X c X c X}
		& \begin{tikzpicture}[x=0.5cm,y=0.5cm]
			\clip (-1.96,-3.7) rectangle (5.47,3.67);
			\draw (3.68,3)-- (-1.76,-2);
			\draw (-1.76,-2)-- (5.06,-2);
			\draw (3.68,3)-- (5.06,-2);
			\draw [line width=0.3] (0.1,-1.78) circle (1.87);
			\draw [line width=0.3] (3.51,-0.88) circle (1.91);
			\draw [line width=0.3] (2.16,1.44) circle (2.18);
			\draw [fill=black] (3.68,3) circle (2pt);
			\draw [fill=black] (-1.76,-2) circle (2pt);
			\draw [fill=black] (5.06,-2) circle (2pt);
			\draw [fill=black] (0.47,0.05) circle (2pt);
			\draw [fill=black] (1.96,-2) circle (2pt);
			\draw [fill=black] (4.27,0.87) circle (2pt);
			\draw [fill=white] (1.61,-0.68) circle (2pt);
		\end{tikzpicture} & & \begin{tikzpicture}[x=0.5cm,y=0.5cm]
			\clip (-1.74,-3.55) rectangle (8.54,4.41);
			\draw  (-0.78,-1.66)-- (4.64,3.14);
			\draw  (4.64,3.14)-- (6.88,-2.3);
			\draw  (6.88,-2.3)-- (2.86,1.56);
			\draw  (-0.78,-1.66)-- (5.54,0.96);
			\draw [line width=0.3] (1.75,-0.85) circle (2.65);
			\draw [line width=0.3] (5.5,-0.97) circle (1.92);
			\draw [line width=0.3] (1.93,0.74) circle (3.62);
			\draw [line width=0.3] (5.54,0.33) circle (2.95);
			\draw [fill=black] (-0.78,-1.66) circle (2pt);
			\draw [fill=black] (4.64,3.14) circle (2pt);
			\draw [fill=black] (6.88,-2.3) circle (2pt);
			\draw [fill=black] (2.86,1.56) circle (2pt);
			\draw [fill=black] (5.54,0.96) circle (2pt);
			\draw [fill=black] (4.11,0.36) circle (2pt);
			\draw [fill=white] (4.03,-2.20) circle (2pt);
		\end{tikzpicture} & \\
		& Satz von Miquel & & Vier-Geraden-vier-Kreise-Satz & 
	\end{tabularx}
\end{figure}

Als direkte Folgerungen erhalten wir den bekannten \emph{Satz von Miquel} und den Vier-Geraden-vier-Kreise-Satz, die wir hier beide illustriert haben. (\emph{Übungsaufgabe: Trage die Punkte $A$, $B$, $C$ und $A'$, $B'$, $C'$ in die Skizzen ein, sodass ersichtlich wird, dass Spezialfälle des Lemmas vorliegen.})



Zuletzt wollen wir an einem Beispiel vorführen, wie eine Lösung mit orientierten Winkeln modulo $180^\circ$ aussehen kann. Die betreffende Aufgabe ist an sich nicht sonderlich schwierig und ihr werdet sie bestimmt schnell herausbekommen. Dennoch haben die meisten Teilnehmenden der damaligen Klausur höchsten 8 oder 9 von 10 Punkten bekommen, weil außerordentlich viele Lagefälle auftreten können.
\begin{aufgabe*}
	Zwei Kreise $\omega_1$ und $\omega_2$ mit den Mittelpunkten $O_1$ bzw.\ $O_2$ schneiden sich in den beiden verschiedenen Punkten $A$ und $B$. Eine durch $A$ verlaufende Gerade schneide $\omega_1$ erneut im Punkt $C\neq A$ und $\omega_2$ in $D\neq A$. Die Geraden $CO_1$ und $DO_2$ schneiden sich in $X$. Beweise, dass die vier Punkte $O_1$, $O_2$, $B$ und $X$ auf einem gemeinsamen Kreis liegen.
\end{aufgabe*}
\begin{figure}[ht]
	\centering		
	\begin{tikzpicture}[x=0.5cm,y=0.5cm]
		%\clip (-3.42,-5.86) rectangle (10.83,4.3);
		\draw [line width=0.3,shift={(1.34,-1.1)}] (30:4.71) arc (30:385:4.71);
		\draw [line width=0.3] (6.48,-1.64) coordinate (O2) circle (3.97);
		\draw [dashed,line width=0.3] (3.825,-2.179) circle (2.709);
		\coordinate (O1) at (1.34,-1.1);
		\coordinate (A) at (4.889,1.997);
		\coordinate (B) at (4.167,-4.867);
		\coordinate (C) at (0.23,3.477);
		\coordinate (D) at (9.879,0.412);
		\coordinate (X) at (2.111,-4.277);
		\draw (C) to (D) to (X) to cycle;
		\draw (O1) to (A) to (O2);
		\draw [fill=black] (O1) circle (2pt) node[shift={(180:2.25ex)}] {$O_1$};
		\draw [fill=black] (O2) circle (2pt) node[shift={(-25:2.25ex)}] {$O_2$};
		\draw [fill=black] (A) circle (2pt) node[shift={(80:2ex)}] {$A$};
		\draw [fill=black] (B) circle (2pt) node[shift={(270:2ex)}] {$B$};
		\draw [fill=black] (C) circle (2pt) node[shift={(130:2ex)}] {$C$};
		\draw [fill=black] (D) circle (2pt) node[shift={(20:2ex)}] {$D$};
		\draw [fill=white] (X) circle (2pt) node[shift={(240:2ex)}] {$X$};
		\draw [line width=0.3,shift={(C)}] (283.636:0.62cm) arc (283.636:342.375:0.62cm);
		\node [shift={(313.005:0.42cm)}] at (C) {$\alpha$};
		\draw [line width=0.3,shift={(A)}] (162.375:0.62cm) arc (162.375:221.113:0.62cm);
		\node [shift={(188.744:0.42cm)}] at (A) {$\alpha$};
		\draw [line width=0.3,shift={(A)}] (293.632:0.72cm) arc (293.632:342.375:0.72cm);
		\node [shift={(317:0.55cm)}] at (A) {$\beta$};
		\draw [line width=0.3,shift={(D)}] (162.375:0.72cm) arc (162.375:211.117:0.72cm);
		\node [shift={(191.764:0.52cm)}] at (D) {$\beta$};
	\end{tikzpicture}
\end{figure}

\begin{proof}
	Sei $\alpha \coloneqq \winkel(O_1C, CA)$. Weil das Dreieck $O_1AC$ gleichschenklig mit Spitze $O_1$ ist, gilt auch $\winkel(CA,O_1A)\equiv \alpha \mod 180^\circ$. Indem wir Eigenschaft~\ref{eigenschaft:OrientierteWinkelAddierenZuNull} auf die Seiten des Dreiecks $O_1AC$ anwenden, erhalten wir $\winkel (O_1A,O_1C)\equiv -2\alpha \mod 180^\circ$. Mit $\beta \coloneqq \winkel(AD,O_2D)$ erhalten wir analog $\winkel(O_2D,O_2A)\equiv -2\beta \mod 180^\circ$.
	
	Durch Eigenschaft~\ref{eigenschaft:OrientierteWinkelAddierenZuNull} angewendet auf die Geraden $CD$, $O_1A$ und $O_2A$ folgt 
	\begin{align*}
		\winkel(O_1A,O_2A)&\equiv-\winkel(CD,O_1A)-\winkel(O_2A,CD)\mod 180^\circ\\
		&\equiv-\winkel(CA,O_1A)-\winkel(O_2A,AD)\mod 180^\circ\\
		&\equiv-(\alpha + \beta) \mod 180^\circ\,.
	\end{align*}
	Weil $B$ das Spiegelbild von $A$ an $O_1O_2$ ist, gilt
	\begin{equation*}
		\winkel(O_1B,O_2B)\equiv -\winkel(O_1A,O_2A)\equiv \alpha + \beta \mod 180^\circ\,.
	\end{equation*}
	Im Viereck $O_1XO_2A$ erhalten wir nach Eigenschaft~\ref{eigenschaft:OrientierteWinkelAddierenZuNull} und den bisherigen Gleichungen
	\begin{align*}
		\winkel(O_1X,O_2X) &\equiv -\winkel(O_1A,O_1C) - \winkel(O_2A,O_1A) - \winkel(O_2D,O_2A) \mod 180^\circ\\
		&\equiv 2\alpha - (\alpha + \beta) + 2\beta \mod 180^\circ\\
		&\equiv \alpha + \beta \mod 180^\circ\,.
	\end{align*}
	Also ist $\winkel(O_1B,O_2B)\equiv \winkel(O_1X,O_2X) \mod 180^\circ$, was uns nach dem orientierten Peripheriewinkelsatz gerade die Behauptung liefert.
\end{proof}\newpage
	
	\cftaddtitleline{toc}{part}{Kombinatorik}{\thepage}
	\section{Die Raupe Nummersatt und die probabilistische Methode}
Eine besonders schwere Aufgabe wurde bei der Bundesrunde 2003 gestellt:

\begin{aufgabe*}[***]\label{aufgabe:RaupeNummersatt}
	Gegeben ist ein Quadratgitter aus $N \times N$ Kästchen; $N\geqslant 3$ sei eine ungerade Zahl.
	Die Raupe Nummersatt sitzt in dem Kästchen genau in der Mitte des Gitters. Jedes der übrigen
	Kästchen enthält eine positive ganze Zahl. Über die Verteilung der Zahlen ist nur bekannt, dass
	sich in keinen zwei Feldern die gleiche Zahl befindet.
	Nummersatt möchte durch dieses Zahlenmeer einen Weg nach draußen finden. Sie kann dabei
	von einem Kästchen stets nur zu einem entlang einer Seite angrenzenden Kästchen weiterwandern
	und muss jede Zahl fressen, durch deren Kästchen ihr Weg führt. Jede Zahl $n$ wiegt $\frac{1}{n}\operatorname{kg}$, und
	Nummersatt kann insgesamt nicht mehr als $2\operatorname{kg}$ Zahlen fressen.
	Untersuche
	\begin{enumerate}[label={$(\alph*)$},ref={$(\alph*)$}]
		\item für $N = 2003$,
		\item für alle ungeraden Zahlen $N\geqslant 3$,
	\end{enumerate}
	ob die Zahlen im Gitter so ungünstig verteilt sein können, dass Nummersatt keinen Weg nach
	draußen finden kann, auf dem höchstens $2\operatorname{kg}$ Zahlen liegen.
\end{aufgabe*}

In diesem Kapitel wollen wir eine Lösung dieser Aufgabe vorstellen. Wir werden zeigen, dass die Raupe Nummersatt in beiden Aufgabenteilen stets einen Weg nach draußen findet und dass die geforderten maximal $2\operatorname{kg}$ sogar durch $1\operatorname{kg}$ ersetzt werden können.

\subsection*{Die probabilistische Methode}

Der Lösung, die wir präsentieren werden, liegt eine der wichtigsten Methoden in der Kombinatorik zugrunde: Die \emph{probabilistische Methode:}
\begin{enumerate}\itshape
	\item[$(*)$] Wenn es sich als schwierig erweist, ein Objekt mit einer gewünschten Eigenschaft~$E$ direkt zu konstruieren, dann versuche stattdessen zu zeigen, dass ein \enquote{zufällig ausgewähltes Objekt} die Eigenschaft~$E$ \enquote{im Erwartungswert} erfüllt.
\end{enumerate}
Natürlich kann das nicht immer klappen. Die probabilistische Methode lohnt sich nur dann, wenn wir ohnehin vermuten, dass die Eigenschaft~$E$ in Wirklichkeit gar nicht so selten vorkommt. Aber wenn die Methode funktioniert, führt sie oftmals zu außerordentlich eleganten Lösungen. Auch in der modernen mathematischen Forschung ist die probabilistische Methode ein fester Bestandteil und es gibt viele phantastische Resultate, die sich nur auf diese Weise beweisen lassen.

Wir werden nun die probabilistische Methode anhand von Aufgabe~\ref{aufgabe:RaupeNummersatt} demonstrieren. Die Lösung ist so formuliert, dass nachvollziehbar ist, wie wir darauf gekommen sind. In der Olympiade würdet ihr diese Überlegungen natürlich weglassen und nur die relevanten Begründungen aufschreiben.

\begin{proof}[Lösung zu Aufgabe~\ref{aufgabe:RaupeNummersatt}]
	Wir beginnen mit einigen Bezeichnungen. Wenn $W$ ein Weg ist, dann identifizieren wir $W$ mit der Menge seiner Felder, sodass $f\in W$ bedeutet, $f$ ist ein Feld von $W$. Wenn~$f$ ein Feld ist, dann bezeichnen wir mit $n(f)$ die natürliche Zahl, die auf~$f$ steht. Unser Ziel ist, eine geeignete Menge $\mathcal W$ von nach draußen führenden Wegen sowie nichtnegative Gewichte $p(W)\geqslant 0$ für $W\in\mathcal W$ zu finden, sodass $\sum_{W\in \mathcal W}p(W)=1$ gilt und außerdem die folgende Ungleichung erfüllt ist:
	\begin{equation*}
		\sum_{W\in \mathcal{W}} p(W)\sum_{f\in W} \frac{1}{n(f)}<1\,.
	\end{equation*}
	Dann muss es nämlich einen Weg $W\in\mathcal W$ geben, sodass das Gewicht der Zahlen auf $W$ kleiner als $1$ ist. Wir können $p(W)$ als die \enquote{Wahrscheinlichkeit, dass $W$ ausgewählt wird} interpretieren und die obige Summe dann als den Erwartungswert des Gewichtes der Zahlen auf $W$.
	
	Bevor wir uns an den Beweis dieser Ungleichung machen können, müssen wir zwei entscheidende Fragen beantworten:
	\begin{enumerate}[label={$(\arabic*)$},ref={$(\arabic*)$}]\itshape
		\item Was ist eine geeignete Wahl für die Menge $\mathcal W$?\label{frage:WasIstW?}
		\item Was ist eine geeignete Wahl für die Gewichte $p(W)$, $W\in\mathcal W$?\label{frage:WasFuerGewichte?}
	\end{enumerate}
	Eine \enquote{geeignete Wahl} ist eine Wahl, für die die gewünschte Ungleichung nicht nur erfüllt ist, sondern sich auch möglichst einfach beweisen lässt. Unsere Erfahrung mit Olympiade-Aufgaben sagt uns, dass ein Beweis der gewünschten Ungleichung vermutlich mit folgender Umsortierung der fraglichen Summe beginnen würde:
	\begin{equation*}
		\sum_{W\in \mathcal{W}} p(W)\sum_{f\in W} \frac{1}{n(f)}=\sum_{f\in\mathcal F}\frac{1}{n(f)}\sum_{W\in \mathcal W_f}p(W)\,.
	\end{equation*}
	Hierbei bezeichnet $\mathcal F$ die Menge aller Felder, die die Wege aus $\mathcal W$ durchlaufen, und für $f\in \mathcal F$ ist $\mathcal W_f\subseteq \mathcal W$ die Teilmenge $\braces*{W\in\mathcal W\ \middle|\ f\in W}$. Für die innere Summe auf der rechten Seite schreiben wir kurz $p(f)$, also\footnote{Wenn $p(W)$ in der stochastischen Interpretation die Wahrscheinlichkeit angibt, mit der der Weg $W$ gewählt wurde, dann ist $p(f)$ die Wahrscheinlichkeit dafür, dass der gewählte Weg das Feld~$f$ enthält. Die obige Umsortierung der Summe lässt sich in der stochastischen Interpretation mithilfe von charakteristischen Funktionen und der Linearität des Erwartungswertes herleiten. Das ist aber deutlich umständlicher, als einfach die Summe umzusortieren.}
	\begin{equation*}
		p(f)\coloneqq\sum_{W\in\mathcal{W}_f}p(W)\,.
	\end{equation*}
	Unser Problem würde sich sicherlich deutlich vereinfachen, wenn wir eine einfache Beschreibung für $p(f)$ hätten. Wie einfach? Können wir erreichen, dass alle $p(f)$ gleich sind? Definitiv nicht. Denn je weiter $f$ vom Ursprung ist, desto weniger Wege führen durch $f$. Die nächstbeste Hoffnung wäre dann, dass $p(f)$ nur vom Abstand zwischen dem Ursprung und~$f$ abhängt. Der \enquote{Abstand} bezeichnet hier natürlich nicht den euklidischen Abstand, sondern die Anzahl der Felder auf einem minimalen Weg vom Ursprung nach~$f$.
	
	Für jedes $\ell\geqslant 0$ betrachten wir also die Menge $\mathcal R_\ell$ aller Felder, die vom Ursprung den Abstand $\ell$ haben. Ferner sei $\mathcal F_\ell=\bigcup_{k\leqslant \ell}\mathcal R_k$; dann ist $\mathcal F_\ell$ die Menge aller Felder, die sich vom Ursprung aus in maximal $\ell$ Schritten erreichen lassen. \enquote{Aus der Ferne betrachtet} sieht $\mathcal F_\ell$ aus wie ein auf der Ecke balanciertes Quadrat und $\mathcal R_\ell$ sieht aus wie dessen Rand. Wenn wir die Mittelpunkte der Kästchen in $\mathcal R_\ell$ verbinden, erhalten wir sogar ein tatsächliches Quadrat $\mathcal Q_\ell$.
	
	\begin{figure}[ht]
		\centering
		\begin{tabularx}{\textwidth}{X c X c X c X}
			& \begin{tikzpicture}[x=0.28cm,y=0.28cm]
				\draw (-7.5,-0.5)%
				% linke Ecke
				to ++(0,1) to ++ (1,0) to ++(0,1) to ++(1,0) to ++(0,1) to ++ (1,0) to ++(0,1) to ++ (1,0) to ++(0,1) to ++ (1,0) to ++(0,1) to ++ (1,0) to ++(0,1) to ++ (1,0) to ++(0,1) to ++ (1,0)%
				% obere Ecke
				to ++(0,-1) to ++(1,0) to ++(0,-1) to ++(1,0) to ++(0,-1) to ++(1,0) to ++(0,-1) to ++(1,0) to ++(0,-1) to ++(1,0) to ++(0,-1) to ++(1,0) to ++(0,-1) to ++(1,0) to ++(0,-1)%
				% rechte Ecke		
				to ++(-1,0) to ++(0,-1) to ++(-1,0) to ++(0,-1) to ++(-1,0) to ++(0,-1) to ++(-1,0) to ++(0,-1) to ++(-1,0) to ++(0,-1) to ++(-1,0) to ++(0,-1) to ++(-1,0) to ++(0,-1) to ++(-1,0)%
				% untere Ecke 
				to ++(0,1) to ++(-1,0) to ++(0,1) to ++(-1,0) to ++(0,1) to ++(-1,0) to ++(0,1) to ++(-1,0) to ++(0,1) to ++(-1,0) to ++(0,1) to ++(-1,0) to ++(0,1) to cycle;
				\draw (-6.5,-0.5) to ++(0,1) to ++ (1,0) to ++(0,1) to ++(1,0) to ++(0,1) to ++ (1,0) to ++(0,1) to ++ (1,0) to ++(0,1) to ++ (1,0) to ++(0,1) to ++ (1,0) to ++(0,1) to ++ (1,0) %
				% obere Ecke
				to ++(0,-1) to ++(1,0) to ++(0,-1) to ++(1,0) to ++(0,-1) to ++(1,0) to ++(0,-1) to ++(1,0) to ++(0,-1) to ++(1,0) to ++(0,-1) to ++(1,0) to ++(0,-1)%
				% rechte Ecke		
				to ++(-1,0) to ++(0,-1) to ++(-1,0) to ++(0,-1) to ++(-1,0) to ++(0,-1) to ++(-1,0) to ++(0,-1) to ++(-1,0) to ++(0,-1) to ++(-1,0) to ++(0,-1) to ++(-1,0)%
				% untere Ecke 
				to ++(0,1) to ++(-1,0) to ++(0,1) to ++(-1,0) to ++(0,1) to ++(-1,0) to ++(0,1) to ++(-1,0) to ++(0,1) to ++(-1,0) to ++(0,1) to cycle;
				%\draw [line width=0.3,dash pattern=on 0.07cm off 0.07cm,dash phase=0.035cm] (-0.5,-0.5) to ++(0,1) to ++(1,0) to ++(0,-1) to cycle;
				%\draw[fill=black] (0,0) circle (2pt) node[shift={(-65:2.5ex)}] {$O$};
			\end{tikzpicture} & & \begin{tikzpicture}[x=0.28cm,y=0.28cm]
				\draw (-7.5,-0.5)%
				% linke Ecke
				to ++(0,1) to ++ (1,0) to ++(0,1) to ++(1,0) to ++(0,1) to ++ (1,0) to ++(0,1) to ++ (1,0) to ++(0,1) to ++ (1,0) to ++(0,1) to ++ (1,0) to ++(0,1) to ++ (1,0) to ++(0,1) to ++ (1,0)%
				% obere Ecke
				to ++(0,-1) to ++(1,0) to ++(0,-1) to ++(1,0) to ++(0,-1) to ++(1,0) to ++(0,-1) to ++(1,0) to ++(0,-1) to ++(1,0) to ++(0,-1) to ++(1,0) to ++(0,-1) to ++(1,0) to ++(0,-1)%
				% rechte Ecke		
				to ++(-1,0) to ++(0,-1) to ++(-1,0) to ++(0,-1) to ++(-1,0) to ++(0,-1) to ++(-1,0) to ++(0,-1) to ++(-1,0) to ++(0,-1) to ++(-1,0) to ++(0,-1) to ++(-1,0) to ++(0,-1) to ++(-1,0)%
				% untere Ecke 
				to ++(0,1) to ++(-1,0) to ++(0,1) to ++(-1,0) to ++(0,1) to ++(-1,0) to ++(0,1) to ++(-1,0) to ++(0,1) to ++(-1,0) to ++(0,1) to ++(-1,0) to ++(0,1) to cycle;
				\begin{scope}[line width=0.3,dash pattern=on 0.07cm off 0.07cm,dash phase=0.035cm]
					\draw (-0.5,6.5) to (0.5,6.5);
					\draw (-1.5,5.5) to (1.5,5.5);
					\draw (-2.5,4.5) to (2.5,4.5);
					\draw (-3.5,3.5) to (3.5,3.5);
					\draw (-4.5,2.5) to (4.5,2.5);
					\draw (-5.5,1.5) to (5.5,1.5);
					\draw (-6.5,0.5) to (6.5,0.5);
					\draw (-0.5,-6.5) to (0.5,-6.5);
					\draw (-1.5,-5.5) to (1.5,-5.5);
					\draw (-2.5,-4.5) to (2.5,-4.5);
					\draw (-3.5,-3.5) to (3.5,-3.5);
					\draw (-4.5,-2.5) to (4.5,-2.5);
					\draw (-5.5,-1.5) to (5.5,-1.5);
					\draw (-6.5,-0.5) to (6.5,-0.5);
					\draw (6.5,-0.5) to (6.5,0.5);
					\draw (5.5,-1.5) to (5.5,1.5);
					\draw (4.5,-2.5) to (4.5,2.5);
					\draw (3.5,-3.5) to (3.5,3.5);
					\draw (2.5,-4.5) to (2.5,4.5);
					\draw (1.5,-5.5) to (1.5,5.5);
					\draw (0.5,-6.5) to (0.5,6.5);
					\draw (-6.5,-0.5) to (-6.5,0.5);
					\draw (-5.5,-1.5) to (-5.5,1.5);
					\draw (-4.5,-2.5) to (-4.5,2.5);
					\draw (-3.5,-3.5) to (-3.5,3.5);
					\draw (-2.5,-4.5) to (-2.5,4.5);
					\draw (-1.5,-5.5) to (-1.5,5.5);
					\draw (-0.5,-6.5) to (-0.5,6.5);
				\end{scope}
			\end{tikzpicture} & & \begin{tikzpicture}[x=0.28cm,y=0.28cm]
				\draw [line width=0.3,dash pattern=on 0.07cm off 0.07cm,dash phase=0.035cm] (-7.5,-0.5)%
				% linke Ecke
				to ++(0,1) to ++ (1,0) to ++(0,1) to ++(1,0) to ++(0,1) to ++ (1,0) to ++(0,1) to ++ (1,0) to ++(0,1) to ++ (1,0) to ++(0,1) to ++ (1,0) to ++(0,1) to ++ (1,0) to ++(0,1) to ++ (1,0)%
				% obere Ecke
				to ++(0,-1) to ++(1,0) to ++(0,-1) to ++(1,0) to ++(0,-1) to ++(1,0) to ++(0,-1) to ++(1,0) to ++(0,-1) to ++(1,0) to ++(0,-1) to ++(1,0) to ++(0,-1) to ++(1,0) to ++(0,-1)%
				% rechte Ecke		
				to ++(-1,0) to ++(0,-1) to ++(-1,0) to ++(0,-1) to ++(-1,0) to ++(0,-1) to ++(-1,0) to ++(0,-1) to ++(-1,0) to ++(0,-1) to ++(-1,0) to ++(0,-1) to ++(-1,0) to ++(0,-1) to ++(-1,0)%
				% untere Ecke 
				to ++(0,1) to ++(-1,0) to ++(0,1) to ++(-1,0) to ++(0,1) to ++(-1,0) to ++(0,1) to ++(-1,0) to ++(0,1) to ++(-1,0) to ++(0,1) to ++(-1,0) to ++(0,1) to cycle;
				\draw (-7,0) to (0,7) to (7,0) to (0,-7) to cycle;
				%\draw [line width=0.3,dash pattern=on 0.07cm off 0.07cm,dash phase=0.035cm] (-0.5,-0.5) to ++(0,1) to ++(1,0) to ++(0,-1) to cycle;
				%\draw[fill=black] (0,0) circle (2pt) node[shift={(-65:2.5ex)}] {$O$};
			\end{tikzpicture} & \\\addlinespace
			& die Menge $\mathcal R_\ell$ & & die Menge $\mathcal F_\ell$ & & das Quadrat $\mathcal Q_\ell$ & 
		\end{tabularx}
	\end{figure}
	
	Es genügt zu zeigen, dass sich die Raupe Nummersatt stets aus der Menge $\mathcal F_\ell$ herausfressen kann, denn für $\ell\geqslant N$ kann das ursprüngliche $N\times N$-Quadrat in $\mathcal F_\ell$ platziert werden. Gleichzeitig haben wir damit auch nichts verschenkt, denn wir wollen ja die Aussage für beliebiges $N$ zeigen und $\mathcal F_\ell$ kann in einem $(2\ell+1)\times (2\ell+1)$-Quadrat platziert werden.
	
	Wir müssen immer noch eine geeignete Menge $\mathcal W$ von Wegen sowie geeignete Gewichte $p(W)$ bestimmen. Es liegt nahe, nur \enquote{direkte} Wege aus $\mathcal F_\ell$ heraus zu betrachten, weil alle anderen Wege nur unnötig Gewicht verschwenden. Was ist aber ein \enquote{direkter} Weg? Hier benutzen wir eine sehr elegante Idee: Für einen beliebigen Punkt $X$ auf dem Rand des Quadrats $\mathcal Q_\ell$ betrachten wir die Strecke $\overline{OX}$, wobei $O$ der Mittelpunkt des Ursprungskästchens ist. Sei $W(X)$ die Menge aller Kästchen, durch die $\overline{OX}$ verläuft. Dann ist $W(X)$ ein direkter Weg nach draußen! Hierbei müssen wir allerdings ein wenig aufpassen: Es kann nämlich passieren, dass $\overline{OX}$ durch einen Eckpunkt von Kästchen in $\mathcal F_\ell$ verläuft. In diesem Fall ist nicht so klar, welche der vier angrenzenden Kästchen zu $W(X)$ zählen sollen. Dieses Problem lösen wir, indem wir einfach nur diejenigen $X$ betrachten, für die $\overline{OX}$ durch keinen Eckpunkt von Kästchen verläuft. Damit schließen wir lediglich endlich viele Möglichkeiten für $X$ aus (nämlich für jeden Eckpunkt $E$ die Zentralprojektion von $E$ auf den Rand von $\mathcal Q_\ell$, also den Schnittpunkt des Strahls $\overrightarrow{OE}$ mit dem Rand von $\mathcal Q_\ell$).
	
	\begin{figure}[ht]
		\centering
		\begin{tabularx}{\textwidth}{X c X c X}
			& \begin{tikzpicture}[x=0.28cm,y=0.28cm]
				\begin{scope}[line width=0.3,dash pattern=on 0.07cm off 0.07cm,dash phase=0.035cm]
					\draw (-7.5,-0.5)%
					% linke Ecke
					to ++(0,1) to ++ (1,0) to ++(0,1) to ++(1,0) to ++(0,1) to ++ (1,0) to ++(0,1) to ++ (1,0) to ++(0,1) to ++ (1,0) to ++(0,1) to ++ (1,0) to ++(0,1) to ++ (1,0) to ++(0,1) to ++ (1,0)%
					% obere Ecke
					to ++(0,-1) to ++(1,0) to ++(0,-1) to ++(1,0) to ++(0,-1) to ++(1,0) to ++(0,-1) to ++(1,0) to ++(0,-1) to ++(1,0) to ++(0,-1) to ++(1,0) to ++(0,-1) to ++(1,0) to ++(0,-1)%
					% rechte Ecke		
					to ++(-1,0) to ++(0,-1) to ++(-1,0) to ++(0,-1) to ++(-1,0) to ++(0,-1) to ++(-1,0) to ++(0,-1) to ++(-1,0) to ++(0,-1) to ++(-1,0) to ++(0,-1) to ++(-1,0) to ++(0,-1) to ++(-1,0)%
					% untere Ecke 
					to ++(0,1) to ++(-1,0) to ++(0,1) to ++(-1,0) to ++(0,1) to ++(-1,0) to ++(0,1) to ++(-1,0) to ++(0,1) to ++(-1,0) to ++(0,1) to ++(-1,0) to ++(0,1) to cycle;
					\draw[fill=black!15!white] (-0.5,-0.5) to ++(0,1) to ++(-1,0) to ++(0,2) to ++(-1,0) to ++(0,3) to ++(1,0) to ++(0,-2) to ++(1,0) to ++(0,-2) to ++(1,0) to ++(0,-2) to cycle;
					\draw (-2.5,4.5) to (-1.5,4.5);
					\draw (-2.5,3.5) to (-1.5,3.5) to (-1.5,2.5) to (-0.5,2.5);
					\draw (-1.5,1.5) to (-0.5,1.5) to (-0.5,0.5) to (0.5,0.5);
				\end{scope}
				\draw (-7,0) to (0,7) to (7,0) to (0,-7) to cycle;
				\coordinate (X) at (0.32*-7,0.68*7);
				\draw (0,0) to (X);
				\draw[fill=black] (0,0) circle (2pt) node[shift={(-65:2.5ex)}] {$O$};
				\draw[fill=black] (X) circle (2pt) node[shift={(135:2.5ex)}] {$X$};
			\end{tikzpicture} & & \begin{tikzpicture}[x=0.28cm,y=0.28cm]
				\begin{scope}[line width=0.3,dash pattern=on 0.07cm off 0.07cm,dash phase=0.035cm]
					\draw (-7.5,-0.5)%
					% linke Ecke
					to ++(0,1) to ++ (1,0) to ++(0,1) to ++(1,0) to ++(0,1) to ++ (1,0) to ++(0,1) to ++ (1,0) to ++(0,1) to ++ (1,0) to ++(0,1) to ++ (1,0) to ++(0,1) to ++ (1,0) to ++(0,1) to ++ (1,0)%
					% obere Ecke
					to ++(0,-1) to ++(1,0) to ++(0,-1) to ++(1,0) to ++(0,-1) to ++(1,0) to ++(0,-1) to ++(1,0) to ++(0,-1) to ++(1,0) to ++(0,-1) to ++(1,0) to ++(0,-1) to ++(1,0) to ++(0,-1)%
					% rechte Ecke		
					to ++(-1,0) to ++(0,-1) to ++(-1,0) to ++(0,-1) to ++(-1,0) to ++(0,-1) to ++(-1,0) to ++(0,-1) to ++(-1,0) to ++(0,-1) to ++(-1,0) to ++(0,-1) to ++(-1,0) to ++(0,-1) to ++(-1,0)%
					% untere Ecke 
					to ++(0,1) to ++(-1,0) to ++(0,1) to ++(-1,0) to ++(0,1) to ++(-1,0) to ++(0,1) to ++(-1,0) to ++(0,1) to ++(-1,0) to ++(0,1) to ++(-1,0) to ++(0,1) to cycle;
					%\draw (-0.5,2.5)  to ++(0,1) to ++ (1,0) %
					% obere Ecke
					%to ++(0,-1) to ++(1,0) to ++(0,-1) to ++(1,0) to ++(0,-1);
					%\draw (2.5,-0.5) %
					% rechte Ecke		
					%to ++(0,-1) to ++(-1,0) to ++(0,-1) to ++(-1,0) to ++(0,-1) to ++(-1,0)%
					% untere Ecke 
					%to ++(0,1) to ++(-1,0) to ++(0,1) to ++(-1,0) to ++(0,1) to ++(-1,0) to ++(0,1)%
					% linke Ecke
					%to ++ (1,0) to ++(0,1) to ++(1,0);
					\draw [fill=black!15!white] (-1.5,1.5) to ++(0,1) to ++(1,0) to ++(0,-1) to cycle;
					\draw [fill=black!15!white] (2.5,-0.5) to ++(0,1) to ++(1,0) to ++(0,-1) to cycle;
				\end{scope}
				\draw (-7,0) to (0,7) to (7,0) to (0,-7) to cycle;
				\draw (-3,0) to (0,3) to (3,0) to (0,-3) to cycle;
				\draw[fill=black] (0,0) circle (2pt) node[shift={(270:2ex)}] {$O$};
				\node[shift={(225:1.75ex)}] at (-1.5,-1.5) {$\mathcal Q_k$};
				\node[shift={(-45:2.75ex)}] at (3.5,-3.5) {$\mathcal Q_\ell$};
				\node[shift={(117:3.7)}] at (0,0) {$f$};
				\path (-3.5,3.5) to node[sloped,shift={(90:3ex)}] {$\overbrace{\hspace{0.924cm}}$} node[shift={(135:6.5ex)}] {$p(f)$} (-1.167,5.833);
				\draw[line width=0.3] (-3.5,3.5) to ++(135:2ex);
				\draw[line width=0.3] (-1.167,5.833) to ++(135:2ex);
				%\coordinate[shift={(45:2.75ex)}] (dummy) at (6.417,0.583);
				%\node [rotate=225] at (dummy) {$\boldsymbol{\big\lbrace}$};
				\path (5.833,1.167) to node[sloped,shift={(90:3ex)}] {$\overbrace{\hspace{0cm}}$} node[shift={(45:6ex)}] {$\frac{p(f')}{2}$} (7,0);
				\draw[line width=0.3] (5.833,1.167) to ++(45:2ex);
				\draw[line width=0.3] (7,0) to ++(45:2ex);
				\path (5.833,-1.167) to node[rotate=180,sloped,shift={(90:3ex)}] {$\overbrace{\hspace{0cm}}$} node[shift={(-45:6.5ex)}] {$\frac{p(f')}{2}$} (7,0);
				\draw[line width=0.3] (5.833,-1.167) to ++(-45:2ex);
				\draw[line width=0.3] (7,0) to ++(-45:2ex);
				\node[shift={(1:4.75)}] at (0,0) {$f'$};
				\draw [line width=0.3] (0,0) to (-3.5,3.5);
				\draw [line width=0.3] (0,0) to (-3.5,3.5);
				\draw [line width=0.3] (0,0) to (-3.5,3.5);
				\draw [line width=0.3] (0,0) to (-3.5,3.5);
				\draw [line width=0.3] (0,0) to (-1.167,5.833);
				\draw [line width=0.3] (0,0) to (5.833,1.167);
				\draw [line width=0.3] (0,0) to (5.833,-1.167);
			\end{tikzpicture} & \\\addlinespace
			& der Weg $W(X)$ & & geometrische Beschreibung von $p(f)$ & 
		\end{tabularx}
	\end{figure}
	
	Natürlich kann es Punkte $X\neq Y$ geben, für die $W(X)=W(Y)$ gilt. Wir können sogar genau sagen, wann das passiert: Die \enquote{verbotenen} Punkte, also die Zentralprojektionen der Ecken der Kästchen, teilen den Rand von $\mathcal Q_\ell$ in endlich viele Teilstrecken (bzw.\ Teilstrecken\emph{züge}: Die Ecken von $\mathcal Q_\ell$ sind \enquote{erlaubte Punkte}, also erhalten wir an den Ecken keine Strecken, sondern Streckenzüge mit einem Knick). Nun gilt $W(X)=W(Y)$ genau dann, wenn $X$ und $Y$ im Inneren der gleichen Teilstrecke (bzw.\ des gleichen Teilstreckenzuges) liegen. Ist $S$ eine solche Teilstrecke (bzw.\ ein solcher Teilstreckenzug), können wir
	also mit $W(S)$ den Weg bezeichnen, der sich für jedes $X\in S$ als $W(X)$ ergibt. Damit haben wir nun endlich einen Kandidaten für $\mathcal W$ gefunden: Wir wählen $\mathcal W$ als die Menge aller solcher $W(S)$. Gleichzeitig gibt es einen offensichtlichen Kandidaten für die Gewichte $p(W)$, $W\in \mathcal W$: Wir definieren $p(W(S))$ als die Länge der Teilstrecke (bzw.\ des Teilstreckenzuges) $S$. Damit $\sum_{W\in \mathcal W}p(W)=1$ gilt, müssen wir hierbei unsere Längeneinheit so wählen, dass der Umfang von $\mathcal Q_\ell$ genau $1$ beträgt.\footnote{Nach kurzer Rechnung ergibt sich, dass unsere Längeneinheit genau $4\sqrt{2}\ell$ mal die Seitenlänge eines Kästchens ist. Das ist aber für die weiteren Betrachtungen irrelevant.}
	
	Wir wollen nun zeigen, dass mit dieser Wahl auch unsere Forderung erfüllt ist, dass der Wert $p(f)\coloneqq \sum_{W\in\mathcal W_f}p(W)$ nur von der Entfernung zwischen dem Ursprung und $f$ abhängt. Anders ausgedrückt: Wenn $k\leqslant \ell$ fixiert ist, dann sind die Werte $p(f)$ für alle $f\in \mathcal R_k$ gleich. Wenn wir das Feld~$f$ durch $O$ auf den Rand von $\mathcal Q_\ell$ projizieren, erhalten wir eine Strecke $S_f$ (bzw.\ einen Streckenzug mit einem Knick). Ein Weg $W(X)$ enthält das Feld~$f$ genau dann, wenn die Strecke $\overline{OX}$ durch $f$ verläuft, was wiederum genau dann der Fall ist, wenn $X$ auf $S_f$ liegt. Folglich ist $p(f)$ genau die Länge von $S_f$. Nun liegt genau eine Diagonale von $f$ auf dem Rand des Quadrates $\mathcal Q_k$ (bzw.\ zwei Halbdiagonalen, wenn $f$ an einer der Ecken von $\mathcal R_k$ sitzt) und $S_f$ ist genau die Zentralprojektion dieser Diagonale (bzw.\ dieser zwei Halbdiagonalen) auf den Rand von $\mathcal Q_\ell$. Aus dem Strahlensatz folgt nun sofort, dass $p(f)$ tatsächlich für alle $f\in \mathcal R_k$ gleich ist. Siehe dazu die Abbildung auf der vorherigen Seite.
	
	Nun gilt $\abs{\mathcal R_k}=4k$ und somit
	\begin{equation*}
		p(f)=\frac{1}{\abs{\mathcal R_k}}=\frac{1}{4k}\,.
	\end{equation*}
	Wir können jetzt endlich damit beginnen, die gewünschte Ungleichung zu zeigen. Indem wir die fragliche Summe nach den Ringen $\mathcal R_k$ sortieren, erhalten wir
	\begin{equation*}
		\sum_{f\in \mathcal F_\ell}\frac{1}{n(f)}\sum_{W\in \mathcal W_f}p(W)=\sum_{k=1}^\ell\sum_{f\in \mathcal R_k}\frac{1}{n(f)}p(f)=\sum_{k=1}^\ell\frac{1}{4k}\sum_{f\in R_k}\frac{1}{n(f)}\,.
	\end{equation*}
	Nach der Umordnungsungleichung verkleinern wir die obige Summe nicht,
	wenn wir den Feldern $f$ in kleineren Ringen kleinere Zahlen $n(f)$ zuordnen. Jeder Ring $\mathcal R_k$ umschließt offenbar $k^2+(k-1)^2$ Felder
	(Schachbrettmuster hilft hier beim Zählen). Nummerieren wir also erst die Felder in $\mathcal R_1$, dann jene in $\mathcal R_2$ und so fort (lückenlos aufsteigend), so erhalten die Felder aus
	$\mathcal R_k$ die Nummern $k^2+(k-1)^2=2k(k-1)+1$ bis $k^2+(k+1)^2-1=2k(k+1)$. Somit erhalten wir
	\begin{equation*}
		\sum_{k=1}^\ell\frac{1}{4k}\sum_{f\in R_k}\frac{1}{n(f)}\leqslant
		\sum_{k=1}^\ell\frac{1}{4k}\sum_{j=2k(k-1)+1}^{2k(k+1)}\frac{1}{j}\,.
	\end{equation*}
	Wir bemerken, dass die innere Summe auf der rechten Seite das Arithmetische Mittel von $4k$ Reziproken aufeinanderfolgender natürlicher Zahlen darstellt. Nun ist das arithmetische Mittel dieser $4k$ Reziproken sicher kleiner als das arithmetische Mittel des kleinsten und des größten Reziproken:
	\begin{equation*}
		\frac{1}{1+b-a}\sum_{j=a}^b \frac{1}{j}\leqslant\frac{\frac{1}{a}+\frac{1}{b}}{2}\,.
	\end{equation*}
	Um das einzusehen, könnt ihr etwa jeweils das Mittel der Summanden im gleichen Abstand zu $\frac{a+b}{2}$ mit der rechten Seite vergleichen. Alternativ könnt ihr benutzen, dass $f(x)=\frac{1}{x}$ für $x>0$ konvex ist. Folglich liegt der Graph von $f$ im Intervall $[a,b]$ unterhalb der zugehörigen Sehne von $(a,f(a))$ nach $(b,f(b))$ -- das ist das selten genutzte Gegenstück zur Jensenschen Ungleichung. Mit dieser Abschätzung ergibt sich:
	\begin{equation*}
		\sum_{k=1}^\ell\frac{1}{4k}\sum_{j=2k(k-1)+1}^{2k(k+1)}\frac{1}{j}\leqslant \sum_{k=1}^\ell\frac{\frac{1}{2k(k-1)+1}+\frac{1}{2k(k+1)}}{2}
	\end{equation*}
	Indem wir die Summe auf der rechten Seite umordnen (wir spalten die erste Hälfte des ersten Summanden sowie die zweite Hälfte des $\ell$-ten Summanden ab und fassen den Rest zu neuen Zweiergrüppchen zusammen), erhalten wir 
	\begin{align*}
		\sum_{k=1}^\ell\frac{\frac{1}{2k(k-1)+1}+\frac{1}{2k(k+1)}}{2}&=\frac12+\frac1{2\ell(\ell+1)}+\sum_{k=1}^{\ell-1}\frac{\frac{1}{2k(k+1)}+\frac{1}{2k(k+1)+1}}{2}\\
		&\leqslant \frac{1}{2}+\frac{1}{2\ell(\ell+1)}+\sum_{k=1}^{\ell-1}\frac{1}{2k(k+1)}
	\end{align*}
	Nun bemerken wir, dass die Summe wegen $\frac{1}{2k(k+1)}=\frac{1}{2k}-\frac{1}{2(k+1)}$ zu einer schicken \enquote{Teleskopsumme} wird, was die Analyse abschließt:
	\begin{align*}
		\frac{1}{2}+\frac{1}{2\ell(\ell+1)}+\sum_{k=1}^{\ell-1}\frac{1}{2k(k+1)}&=\frac{1}{2}+\frac{1}{2\ell(\ell+1)}+\sum_{k=1}^{\ell-1}\parens*{\frac1{2k}-\frac1{2(k+1)}}\\
		&=\frac{1}{2}+\frac{1}{2\ell(\ell+1)}+\frac1{2}-\frac1{2\ell}\\
		&=1-\frac{1}{2(\ell+1)}\,.
	\end{align*}
	Der letzte Term ist offensichtlich kleiner als $1$. Damit sind wir fertig!
\end{proof}

\subsection*{Weitere Übungsaufgaben zur probabilistischen Methode}
\begin{aufgabe*}
	Ein \emph{Hamilton-Pfad} in einem gerichteten Graphen $G$ mit $n$-Knoten ist ein Pfad $v_1v_2\dotsb v_n$ der durch jeden der $n$-Knoten genau einmal führt. Beweise, dass sich die Kanten des vollständigen ungerichteten Graphen $K_n$ so ausrichten lassen, dass mindestens $\frac{n!}{2^{n-1}}$ Hamilton-Pfade existieren.
\end{aufgabe*}
\begin{aufgabe*}
	In einer Stadt leben $n$ Menschen, von denen jeder genau 1000 Freunde hat. Beweise: Es ist möglich, eine Menge $S$ von Menschen auszuwählen, sodass mindestens $\frac{n}{\the\year}$ Personen in $S$ genau zwei Freunde in $S$ haben.
\end{aufgabe*}
\begin{aufgabe*}[***]
	Beweise den folgenden Satz von Erd\H{o}s:
	\begin{satzmitnamen}[Satz]
		Eine Menge $A\subseteq\mathbb Z$ von ganzen Zahlen heiße summenfrei, wenn die Gleichung $x+y=z$ keine Lösung mit $x,y,z\in A$ besitzt. Sei $B\subseteq \mathbb Z$ eine endliche Menge von ganzen Zahlen mit $0\notin B$. Dann gibt es eine summenfreie Teilmenge $A\subseteq B$ mit $\abs{A}>\frac{\abs{B}}{3}$.
	\end{satzmitnamen}
\end{aufgabe*}\newpage
	
	\cftaddtitleline{toc}{part}{Zahlentheorie}{\thepage}
	\section{Das Lifting-The-Exponent-Lemma}\label{kapitel:LTE}
Das folgende Lemma kann unglaublich nützlich sein.
\begin{satzmitnamen}[Lifting-The-Exponent-Lemma (LTE)]
	Für jede Primzahl~$p$ und jede ganze Zahl $n\neq 0$ notieren wir den Exponenten von~$p$ in der Primfaktorzerlegung von~$n$ als $v_p(n)$.
	\begin{enumerate}
		\item Gegeben sei ein ungerade Primzahl $p\geqslant 3$ sowie ganze Zahlen $a$, $b$ mit $a\equiv b\not\equiv 0\mod p$. Dann gilt folgende Gleichung für alle positiven ganzen Zahlen $n\geqslant 1$:\label{behauptung:LTEpUngerade}
		\begin{equation*}
			v_p\parens*{a^n-b^n}=v_p(a-b)+v_p(n)\,.
		\end{equation*}
		\item Gegeben seien ganze Zahlen $a$, $b$ mit $a\equiv 0\mod 4$, aber $a,b\not\equiv 0\mod 2$. Dann gilt folgende Gleichung für alle positiven ganzen Zahlen $n\geqslant 1$:\label{behauptung:LTEp=2}
		\begin{equation*}
			v_2\parens*{a^n-b^n}=v_2(a-b)+v_2(n)\,.
		\end{equation*}
	\end{enumerate}
\end{satzmitnamen}
Es ist ein wenig schade, dass Aussage für $p=2$ ein wenig schwächer ist, aber in der Olympiade-Praxis sorgt das meistens nicht für Probleme. Wir können nämlich auch für $a\equiv b\mod 2$ konkrete Aussagen treffen. Wenn $n$ gerade ist, benutzen wir, dass aus $a\equiv b\mod 2$ auch $a^2\equiv b^2\mod 4$ folgt. Also können wir das LTE-Lemma auf $a^n-b^n=(a^2)^{n/2}-(b^2)^{n/2}$ anwenden und erhalten
\begin{equation*}
	v_2\parens*{a^n-b^n}=v_2\parens*{a^2-b^2}+v_2\parens*{\frac n2}=v_2(a-b)+v_2(a+b)+v_2(n)-1\,.
\end{equation*}
Wenn~$n$ ungerade ist, erhalten wir hingegen $v_2(a^n-b^n)=v_2(a-b)$ nach dem nun folgenden Hilfslemma. Dieses Hilfslemma ist auch der erste Schritt im Beweis des LTE-Lemmas.
\begin{satzmitnamen}[Lemma]
	Gegeben sei eine Primzahl~$p$ \embrace{der Fall~$p=2$ ist erlaubt} sowie ganze Zahlen $a$, $b$ mit $a\equiv b\not\equiv 0\mod p$. Sei ferner $n\geqslant 1$ eine positive ganze Zahl mit $n\not\equiv 0\mod p$. Dann gilt
	\begin{equation*}
		v_p\parens*{a^n-b^n}=v_p(a-b)\,.
	\end{equation*}
\end{satzmitnamen}
\begin{proof}
	Wir benutzen die Faktorisierung $a^n-b^n=(a-b)(a^{n-1}+a^{n-2}b+\dotsb+ab^{n-2}+b^{n-1})$. Wegen $a\equiv b\mod p$ und $a,n\not\equiv 0\mod p$ gilt
	\begin{equation*}
		a^{n-1}+a^{n-2}b+\dotsb+ab^{n-2}+b^{n-1}\equiv na^{n-1}\not\equiv 0\mod p\,.
	\end{equation*}
	Also ist der zweite Faktor in der Faktorisierung nicht durch~$p$ teilbar und die Behauptung $v_p(a^n-b^n)=v_p(a-b)$ folgt sofort.
\end{proof}

\begin{proof}[Beweis des LTE-Lemmas]
	Wir zeigen zuerst~\ref{behauptung:LTEpUngerade}. Es genügt, den Fall zu betrachten, dass $n=p^r$ eine Potenz von $p$ ist, denn der allgemeine Fall lässt sich mithilfe des vorherigen Lemmas auf diesen Spezialfall zurückführen. Indem wir Induktion nach~$r$ verwenden und $a$, $b$ durch $a^{p^{r-1}}$, $b^{p^{r-1}}$ ersetzen, können wir außerdem den Fall $n=p^r$ auf den Fall $n=p$ reduzieren. Betrachten wir also diesen Fall. Wegen $a^p-b^p=(a-b)(a^{p-1}+a^{p-2}b+\dotsb+ab^{p-2}+b^{p-1})$ sagt das LTE-Lemma im Fall $n=p$, dass der Faktor $a^{p-1}+a^{p-2}b+\dotsb+ab^{p-2}+b^{p-1}$ genau einmal durch~$p$ teilbar ist. Dazu schreiben wir $b=a+pm$ und betrachten den Ausdruck modulo~$p^2$: Wir erhalten zunächst
	\begin{equation*}
		b^i\equiv \parens*{a+pm}^i\equiv \sum_{k=0}^i\binom{i}{k}(pm)^ka^{i-k}\equiv a^i+ipm a^{i-1}\mod p^2\,,
	\end{equation*}
	denn für $k\geqslant 2$ sind alle Terme in der Summe durch~$p^2$ teilbar. Also ist
	\begin{alignat*}{2}
		a^{p-1}+a^{p-2}b+\dotsb+ab^{p-2}+b^{p-1}&\equiv \sum_{i=0}^{p-1}a^{p-1-i}\parens*{a^i+ipma^{i-1}} && \mod p^2\\
		&\equiv pa^{p-1}+\frac{(p-1)p}{2}pma^{p-1} &&  \mod p^2\\
		&\equiv pa^{p-1} && \mod p^2\,.
	\end{alignat*}
	Hier haben wir die Voraussetzung $p\geqslant 3$ benutzt, sodass $\frac{(p-1)p}{2}$ durch~$p$ teilbar ist. Aus der Rechnung folgt, dass $a^{p-1}+a^{p-2}b+\dotsb+ab^{p-2}+b^{p-1}$ durch~$p$, aber nicht durch $p^2$ teilbar ist. Das wollten wir zeigen.
	
	Die Behauptung in~\ref{behauptung:LTEp=2} lässt sich analog zu~\ref{behauptung:LTEpUngerade} auf den Fall $n=2$ reduzieren. Nach Voraussetzung gilt $a\equiv b\mod 4$, aber $a,b\not\equiv 0\mod 2$, sodass $a+b\equiv 2a\equiv 2\mod 4$. Es folgt $v_2(a+b)=1$ und damit $v_2(a^2-b^2)=v_2(a-b)+v_2(a+b)=v_2(a-b)+1$, wie behauptet.
\end{proof}

Es ist klar, dass sich das LTE-Lemma in vielen Diophantischen Gleichungen und ähnlichen Zahlentheorie-Aufgaben anwenden lässt. Ein häufiger Trick ist folgender: Wenn wir zeigen können, dass $a^n-b^n$ \enquote{wesentlich öfter} als $a-b$ durch~$p$ teilbar ist, dann muss $v_p(n)$ \enquote{groß} sein. Andererseits gilt aber auch $n\geqslant p^{v_p(n)}$, also muss $v_p(n)$ \enquote{wesentlich kleiner} als~$n$ sein. Konkret: $v_p(n)\leqslant \ln(n)/\ln(p)$. Oft genug lassen sich diese Beobachtungen formalisieren und wir erhalten einen Widerspruch.

\subsection*{Beispielaufgaben}
Ihr sollt nun die folgenden Olympiade-Aufgaben selbstständig mit dem LTE-Lemma lösen. Am Ende dieses Kapitels findet ihr Tipps zu den Aufgaben und am Ende dieses Heftes könnt ihr die Lösungen nachlesen.

\begin{aufgabe*}\label{aufgabe:NieQuadratfrei}
	Zeige: Für positive ganze Zahlen $a\geqslant 3$ ist $a^{a-1}-1$ nie quadratfrei. (\emph{Eine Zahl heißt quadratfrei, wenn sie durch keine Quadratzahl außer $1$ teilbar ist.})
\end{aufgabe*}
\begin{aufgabe*}\label{aufgabe:2p3pan}
	Gegeben sei die Diophantische Gleichung $2^p+3^p=a^n$, wobei $p$ eine Primzahl ist und $a,n\geqslant 1$ positive ganze Zahlen sind. Zeige, dass diese Gleichung nur die trivialen Lösungen mit $n=1$ besitzt.
\end{aufgabe*}
%	\begin{aufgabe*}\label{aufgabe:EndlicheMengeVonPrimzahlen}
	%		Sei $\Sigma$ eine endliche Menge von Primzahlen und sei $a>1$ eine positive ganze Zahl. Zeige, dass es nur endlich viele positive ganze Zahlen $n\geqslant 1$ gibt, für die alle Primfaktoren von $a^n-1$ in $\Sigma$ liegen.
	%	\end{aufgabe*}
\begin{aufgabe*}[**]\leavevmode\label{aufgabe:xn-yn}
	\begin{enumerate}[label={$(\alph*)$},ref={$(\alph*)$}]
		\item[$(a^*)$] Gegeben seien rationale Zahlen $x,y\in\mathbb Q$ mit $x>y>0$. Angenommen, für alle positiven ganzen Zahlen $n\geqslant 1$ ist $x^n-y^n$ eine positive ganze Zahl. Zeige, dass $x$ und $y$ selber positive ganze Zahlen sein müssen.\label{teilaufgabe:xn-yn}
		\item[$(b^{**})$] Zeige, dass die Schlussfolgerung aus~\ref{teilaufgabe:xn-yn} immer noch wahr ist, wenn wir nur wissen, dass $x^n-y^n$ für unendlich viele $n\geqslant 1$ eine positive ganze Zahl ist.\label{teilaufgabe:xn-ynEndlichViele}
	\end{enumerate}
\end{aufgabe*}
\begin{aufgabe*}[**]\label{aufgabe:IMOSL2014N5}
	Finde alle Tripel $(p,x,y)$, wobei~$p$ eine Primzahl ist und $x,y\geqslant 1$ positive ganze Zahlen sind, sodass $x^{p-1}+y$ und $x+y^{p-1}$ Potenzen von~$p$ sind.
\end{aufgabe*}


\newpage\phantom{newpage}\vfill\hrule\vspace{-1em}

\subsection*{Tipps zu den Beispielaufgaben}


\textbf{Tipp zu Aufgabe~\ref{aufgabe:NieQuadratfrei}.} Betrachte einen Primfaktor von $a-1$.

\textbf{Tipp zu Aufgabe~\ref{aufgabe:2p3pan}.} Untersuche, wie oft beide Seiten durch~$5$ teilbar sind.

%\textbf{Tipp zu Aufgabe~\ref{aufgabe:EndlicheMengeVonPrimzahlen}.} Betrachte die multiplikative Ordnung von~$a$ modulo jeder Primzahl $p\in\Sigma$. Schätze dann $v_p(a^n-1)$ in Abhängigkeit von $v_p(a^{\operatorname{ord}_p(a)}-1)$ ab.

\textbf{Tipps zu Aufgabe~\ref{aufgabe:xn-yn}.} Zeige zuerst, dass wir $x$ und $y$ als vollständig gekürzte Brüche $x=a/c$ und $y=b/c$ mit dem gleichen Nenner $c$ darstellen können.

Um~$(a)$ zu zeigen, nimm an, dass~$c$ einen Primteiler~$p$ besitzt und führe dies zum Widerspruch, indem du untersuchst, wie oft $a^n-b^n$ durch $p$ teilbar sein kann und wie oft es durch~$p$ teilbar sein müsste.

Für~$(b)$ wende das gleiche Argument auf $a^{\operatorname{ord}_p(a/b)}-b^{\operatorname{ord}_p(a/b)}$ statt $a-b$ an, wobei $a/b$ die Restklasse modulo~$p$ ist, die sich als Produkt von~$a$ mit dem multiplikativen Inversen von~$b$ modulo~$p$ ergibt.

\textbf{Tipps zu Aufgabe~\ref{aufgabe:IMOSL2014N5}.} Die Lösung besteht aus vielen kleinen Schritten und es sind zahlreiche Fälle zu unterscheiden. Aber es lohnt sich, eine Weile an dieser Aufgabe zu knobeln. Sei geduldig, schau, wie weit du kommst, und versuche, nicht den Überblick zu verlieren.

Behandle erst die trivialen Fälle $p=2$, $x=y$ und $x\equiv 0\mod p$. Zeige, dass in allen anderen Fällen $x\equiv y\mod p$ gilt. Betrachte sodann den Ausdruck $x(x^{p-1}+y)-y(x+y^{p-1})$.

Um die Bedingung, die das LTE-Lemma liefert, verwerten zu können, setze sie geschickt ein und betrachte die multiplikative Ordnung von $-y$ modulo~$p$.\newpage
	
	\cftaddtitleline{toc}{part}{Lösungen zu den Beispielaufgaben}{\thepage}
	\section*{Lösungen zu den Beispielaufgaben}
	Die Lösungen sind nicht immer so formuliert, wie ihr das in der Olympiade tun solltet. Zum Teil sind sie sehr knapp -- zum Beispiel überspringen wir triviale Umformungsschritte oder lassen die Probe weg. In der Olympiade solltet ihr etwas ausführlicher sein und immer die Probe machen. Umgekehrt erklären wir gelegentlich (vor allem bei besonders schweren Aufgaben), wie wir auf die Lösung gekommen sind. In der Olympiade müsst ihr solche Überlegungen natürlich nicht aufschreiben, sondern könnt eure ausgefuchste Lösung einfach vom Himmel fallen lassen.
	\subsection*{Lösungen zu Kapitel~\ref{kapitel:TrigSub}: \emph{Trigonometrische Substitutionen}}

\begin{proof}[Lösung zu Aufgabe~\ref{aufgabe:Tangenssubstitution}]
	Jede reelle Zahl~$a$ lässt sich als $a=\tan\alpha$ mit $-90^\circ<\alpha<90^\circ$ darstellen. Wir können also $\{a_1,a_2,\dotsc,a_6\}=\{\tan\alpha_1,\tan\alpha_2,\dotsc,\tan\alpha_6\}$ schreiben, wobei ohne Beschränkung der Allgemeinheit $-90^\circ<\alpha_1\leqslant\alpha_2\leqslant\dotsb\leqslant\alpha_6<90^\circ$ erfüllt sei. Dann gilt
	\begin{equation*}
		\frac{\tan\alpha_i-\tan\alpha_j}{1-\tan\alpha_i\tan\alpha_j}=\tan\parens*{\alpha_i-\alpha_j}
	\end{equation*}
	Es würde also genügen, $\alpha_i$ und $\alpha_j$ mit $\abs{\alpha_i-\alpha_j}\leqslant 30^\circ$ zu finden, denn $\tan(30^\circ)={1}/{\sqrt{3}}$. Das sollte nun aus dem Schubfachprinzip folgen. Dabei müssen wir aber etwas aufpassen: Wenn wir sechs Zahlen gleichmäßig auf ein offenes Intervall der Länge $180^\circ$ verteilen, dann können die Abstände zwischen je zwei aufeinanderfolgenden beliebig nah an $180^\circ/5$ herankommen. Insbesondere können die Abstände durchaus allesamt größer als $30^\circ$ sein. Der Grund, warum die Aufgabe trotzdem funktioniert, ist dass der Tangens periodisch ist, sodass wir auch dann fertig sind, wenn $\alpha_1$ nahe genug an $-90^\circ$ und $\alpha_6$ nahe genug an $90^\circ$ liegt (konkret: Wenn $(\alpha_1-(-90^\circ))+(90^\circ-\alpha_6)\leqslant 30^\circ$).
	
	Um diese Überlegung formal sauber durchzuführen, definieren wir $\alpha_7\coloneqq \alpha_1+180^\circ$. Dann sind $\alpha_1,\alpha_2,\dotsc,\alpha_7$ sieben Zahlen in einem Intervall der Länge~$180^\circ$. Somit können wir nun wirklich nach Schubfachprinzip folgern, dass $\alpha_i$ und $\alpha_j$ mit $\abs{\alpha_i-\alpha_j}\leqslant 30^\circ$ existieren. Wenn $\alpha_7$ eine der ausgewählten Zahlen ist, dann ersetzen wir $\alpha_7$ einfach durch $\alpha_1$, was wegen $\tan\alpha_1=\tan(\alpha_1+180^\circ)$ kein Problem darstellt.
\end{proof}

\begin{proof}[Lösung zu Aufgabe~\ref{aufgabe:SinusWurzelAdditionstheorem}]
	Der Audruck $\sqrt{4-y^2}$ erinnert uns an die Gleichung $\cos^2\alpha=1-\sin^2\alpha$. Wir sollten also $a_i=2\sin\alpha_i$ substituieren. Um zu sehen, dass das tatsächlich möglich ist, bemerken wir
	\begin{equation*}
		\frac{\sqrt{2}-\sqrt{6}}{2}=2\sin(-15^\circ)\quad\text{und}\quad \frac{2+\sqrt{6}}{2}=2\sin(75^\circ)\,.
	\end{equation*}
	Das lässt sich mithilfe der Halbwinkelformeln verifizieren. Hier ist die Rechnung für $\sin(-15^\circ)$, die Rechnung für $\sin(75^\circ)$ geht analog.
	\begin{equation*}
		\sin(-15^\circ)=-\sqrt{\frac{1-\cos(30^\circ)}{2}}=-\sqrt{\frac{2-\sqrt{3}}{4}}=-\sqrt{\frac{\parens*{\sqrt{2}-\sqrt{6}}^2}{16}}=\frac{\sqrt{2}-\sqrt{6}}{4}\,.
	\end{equation*}
	Wir sehen somit, dass die Substitution $a_i=2\sin\alpha_i$ tatsächlich möglich ist und wir dabei die Schranken $-15^\circ\leqslant \alpha_i\leqslant 75^\circ$ annehmen dürfen. Weil $\alpha_1$, $\alpha_2$, $\alpha_3$ und $\alpha_4$ vier Winkel in einem Intervall der Länge $90^\circ$ sind, gibt es unter diesen nach dem Schubfachprinzip vieren zwei Winkel $\alpha_i$ und $\alpha_j$ mit $\abs{\alpha_i-\alpha_j}\leqslant 30^\circ$. Im Intervall $[-15^\circ,75^\circ]$ ist der Cosinus stets positiv, also gilt $\cos \alpha_i=\sqrt{1-\sin^2\alpha_i}$ und analog $\cos \alpha_j=\sqrt{1-\sin^2\alpha_j}$. Es folgt
	\begin{equation*}
		\abs*{a_i\sqrt{4-a_j^2}-a_j\sqrt{4-a_i^2}}=4\abs*{\sin\alpha_i\cos\alpha_j-\sin\alpha_j\cos\alpha_i}=4\abs*{\sin(\alpha_i-\alpha_j)}\leqslant 4\sin(30^\circ)=2\,,
	\end{equation*}
	wie gewünscht. Damit sind wir fertig.
\end{proof}

\begin{proof}[Lösung zu Aufgabe~\ref{aufgabe:SinusVersteckt}]
	Wir sehen zuerst, dass der Fall $x<0$ unmöglich ist, denn daraus folgt induktiv $a_n<0$ für alle $n$. Ebenso ist $x>1$ unmöglich, denn dann wäre $a_1<0$, also würde die gleiche Induktion $a_n<0$ für alle $n\geqslant 1$ zeigen. Wir finden folglich einen Winkel $0^\circ\leqslant \alpha_0\leqslant 90^\circ$ mit $x=\sin^2\alpha_0$. Aus der Formel $\sin^2(2\alpha)=(2\sin\alpha\cos\alpha)^2=4\sin^2\alpha(1-\sin^2\alpha)$ und der gegebenen Rekursion folgt $a_n=\sin^2(2^n\alpha_0)$. Folglich gilt $a_{\the\year}=0$ genau dann, wenn $2^{\the\year}\alpha_0$ ein ganzzahliges Vielfaches von $90^\circ$ ist. Zusammen mit den Einschränkungen an $\alpha_0$ ergeben sich somit genau die $2^{\the\year}$ Lösungen
	\begin{equation*}
		x=\sin\parens*{\frac{i}{2^{\the\year}}\cdot 90^\circ}^2\,,\quad i=0,1,\dotsc,2^{\the\year}\,.\qedhere
	\end{equation*}
\end{proof}

\begin{proof}[Lösung zu Aufgabe~\ref{aufgabe:521236}]
	Aus den Gleichungen $3\parens[\big]{x+\frac1x}=4\parens[\big]{y+\frac1y}=5\parens[\big]{z+\frac1z}$ folgt dann, dass $x$, $y$ und $z$ das gleiche Vorzeichen haben müssen. Alle Gleichungen sind auch dann noch erfüllt, wenn wir $(x,y,z)$ durch $(-x,-y,-z)$ ersetzen. Also dürfen wir $x,y,z\geqslant 0$ annehmen. Aus der Bedingung $xy+yz+zx=1$ folgt, dass wir $x=\tan\parens[\big]{\frac{\alpha}2}$, $y=\tan\parens[\big]{\frac{\beta}2}$ und $z=\tan\parens[\big]{\frac{\gamma}2}$ substituieren dürfen, wobei $\alpha$, $\beta$ und $\gamma$ die Innenwinkel eines Dreiecks $ABC$ sind. Nun gilt
	\begin{equation*}
		x+\frac 1x=\frac{\sin \parens[\big]{\frac{\alpha}2}}{\cos\parens[\big]{\frac{\alpha}2}}+\frac{\cos\parens[\big]{\frac{\alpha}2}}{\sin\parens[\big]{\frac{\alpha}2}}=\frac{\sin^2\parens[\big]{\frac{\alpha}2}+\cos^2\parens[\big]{\frac{\alpha}2}}{\sin \parens[\big]{\frac{\alpha}2}\cos\parens[\big]{\frac{\alpha}2}}=\frac2{\sin\alpha}\,.
	\end{equation*}
	Die Gleichungen $3\parens[\big]{x+\frac1x}=4\parens[\big]{y+\frac1y}=5\parens[\big]{z+\frac1z}$ implizieren folglich die Verhältnisgleichung $\sin\alpha:\sin\beta:\sin\gamma=3:4:5$. Aus dem Sinussatz folgt dann, dass die Seiten des Dreiecks $ABC$ ebenfalls im Verhältnis $\abs*{BC}:\abs*{CA}:\abs*{AB}=3:4:5$ stehen. Somit muss $ABC$ ein $3$-$4$-$5$-Pythagoras-Dreieck sein. Es folgt $\tan \parens[\big]{\frac{\gamma}2}=\tan(45^\circ)=1$. Ferner ist $\sin\alpha=\cos\beta=\frac 35$ und $\cos\alpha=\sin\beta=\frac 45$. Aus den Halbwinkelformeln für den Tangens folgt nun
	\begin{equation*}
		\tan \parens*{\frac{\alpha}2}=\frac{1-\cos\alpha}{\sin\alpha}=\frac{1-\frac{4}{5}}{\frac{3}{5}}=\frac13\quad \text{und}\quad \tan \parens*{\frac{\beta}2}=\frac{1-\cos\beta}{\sin\beta}=\frac{1-\frac{3}{5}}{\frac{4}{5}}=\frac12\,.
	\end{equation*}
	Somit erhalten wir die Lösung $(x,y,z)=\parens[\big]{\frac13,\frac12,1}$. Eine Probe bestätigt, dass dieses Tripel tatsächlich eine Lösung ist. Da wir bisher $x,y,z\geqslant 0$ angenommen haben, erhalten auch noch die zweite Lösung $(x,y,z)=\parens[\big]{-\frac13,-\frac12,-1}$.
\end{proof}

\begin{proof}[Lösung zu Aufgabe~\ref{aufgabe:UngleichungInvertieren2}]
	Wie wir im Theorieteil des Kapitels gesehen haben, können wir $x=\sin\parens[\big]{\frac{\alpha}2}$, $y=\sin\parens[\big]{\frac{\beta}2}$ und $z=\sin\parens[\big]{\frac{\gamma}2}$ substituieren, wobei $\alpha$, $\beta$ und $\gamma$ die Innenwinkel eines Dreiecks sind. Weil die Ungleichung symmetrisch ist, dürfen wir ferner $\alpha\leqslant \beta\leqslant \gamma$ annehmen. Weil die Sinusfunktion im Intervall $[0^\circ,90^\circ]$ streng monoton steigend ist, folgt $\sin\parens[\big]{\frac{\alpha}2}\leqslant\sin\parens[\big]{\frac{\beta}2}\leqslant\sin\parens[\big]{\frac{\gamma}2}$ und $\sin\parens[\big]{\frac{\alpha}2}\leqslant \sin(30^\circ)=\frac12$. Schreibe die Ungleichung in der Form
	\begin{equation*}
		\sin\parens*{\frac{\alpha}2}\parens*{\sin\parens*{\frac{\beta}2}+\sin\parens*{\frac{\gamma}2}}+\parens*{1-2\sin\parens*{\frac{\alpha}2}}\sin\parens*{\frac{\beta}2}\sin\parens*{\frac{\gamma}2}\leqslant \frac12
	\end{equation*}
	Der Term $1-2\sin\parens[\big]{\frac{\alpha}2}\geqslant 1-2\cdot\frac12=0$ ist nichtnegativ, somit genügt es, die Terme $\sin\parens[\big]{\frac{\beta}2}+\sin\parens[\big]{\frac{\gamma}2}$ und $\sin\parens[\big]{\frac{\beta}2}\sin\parens[\big]{\frac{\gamma}2}$ nach oben abzuschätzen. Für den ersten Term benutzen wir die Jensensche Ungleichung: Die Sinusfunktion ist auf dem Intervall $[0^\circ,180^\circ]$ konkav, somit gilt
	\begin{equation*}
		\sin\parens*{\frac{\beta}2}+\sin\parens*{\frac{\gamma}2}\leqslant 2\sin\parens*{\frac{\beta+\gamma}{4}}\,.
	\end{equation*}
	Für den zweiten Term benutzen wir, dass die Cosinusfunktion auf dem Intervall $[0^\circ,90^\circ]$ streng monoton fallend ist. Somit gilt
	\begin{equation*}
		\sin\parens*{\frac{\beta}2}\sin\parens*{\frac{\gamma}2}=\frac12\parens*{\cos\parens*{\frac{\gamma-\beta}{2}}-\cos\parens*{\frac{\beta+\gamma}{2}}}\leqslant \frac12\parens*{1-\cos\parens*{\frac{\beta+\gamma}{2}}}\,.
	\end{equation*}
	Sei nun $t\coloneqq \sin\parens[\big]{\frac{\beta+\gamma}{4}}$. Aus den Doppelwinkelformeln und der Bedingung $\alpha+\beta+\gamma=180^\circ$ folgt dann $\sin\parens[\big]{\frac\alpha2}=\cos\parens[\big]{\frac{\beta+\gamma}{2}}=1-2t^2$. Durch Einsetzen dieser Abschätzungen und Substitutionen müssen wir nur noch die Ungleichung
	\begin{equation*}
		\parens*{1-2t^2}\cdot 2t+\parens[\Big]{1-2\parens*{1-2t^2}}\cdot\frac12\parens[\Big]{1-\parens*{1-2t^2}}\leqslant \frac12
	\end{equation*}
	beweisen, wobei $t=\sin\parens[\big]{\frac{\beta+\gamma}{4}}$ im Intervall $[0,\sin(45^\circ)]=\brackets[\big]{0,\frac{\sqrt{2}}{2}}$ liegt. In der ursprünglichen Ungleichung können wir die Gleichheitsfälle $x=y=z=\frac12$ und $x=0$, $y=z=\frac{\sqrt{2}}{2}$ erraten. Diese führen auf die Gleichheitsfälle $t=\frac12$ und $t=\frac{\sqrt{2}}{2}$ in der obigen Ungleichung. Indem wir diese Gleichheitsfälle ausklammern, erhalten wir die folgende Ungleichung, die offensichtlich für alle $t\in \brackets[\big]{0,\frac{\sqrt{2}}{2}}$ erfüllt ist:
	\begin{equation*}
		0\leqslant \frac12\parens*{1-2t^2}(2t-1)^2\,.\qedhere
	\end{equation*}
\end{proof}
	\subsection*{Lösungen zu Kapitel~\ref{kapitel:SechsInkreise}: \emph{Das Sechs-Inkreise-Lemma}}

\begin{proof}[Lösung zu Aufgabe~\ref{aufgabe:SechsInkreiseSehnenviereck}]
	Die Inkreise von $DAP$ und $BCP$ haben zwei gemeinsame äußere Tangenten. Eine davon ist offenbar $CD$; die andere bezeichnen wir mit $t$. Indem wir das Sechs-Inkreise-Lemma in dem Fall anwenden, dass $\omega_A$ und $\omega_B$ zu Punkten degeneriert sind und außerdem $C_D$ und $D_C$ im Punkt $P$ zusammenfallen, erhalten wir sofort, dass $t$ auch eine Tangente an den Inkreis von $ABP$ ist.	
	\begin{figure}[ht]
		\centering
		\begin{tikzpicture}[x=0.85cm,y=0.85cm]
			\draw [line width=0.3] (-10.79,11.71) circle (2.66);
			\draw [line width=0.3] (-12.187,13.304) coordinate (I) circle (0.783);
			\draw [line width=0.3] (-10.448,11.05) coordinate (J) circle (1.979);
			\coordinate (K) at (-9.374,14.25);
			\draw [line width=0.3,shift={(K)}] (85:0.521) arc (85:408:0.521);
			\draw [dashed,line width=0.3] (-10.464,12.836) circle (1.786);
			\coordinate (A) at (-14.768,8.784);
			\coordinate (B) at (-7.19,9.28);
			\coordinate (C) at (-9.029,14.869);
			\coordinate (D) at (-12.77,13.97);
			\coordinate (E) at (-11.694,12.665);
			\coordinate (F) at (-9.598,13.583);
			\coordinate (P) at (-10.165,14.596);
			\draw (A) to (B) to (C) to (D) to cycle;
			\draw (A) to (P) to (B);
			%\draw [line width=0.3] (B) to (I) to (D);
			\draw [shorten <=-8em,shorten >=-6em, dashed] (E) to (F);
			\draw [line width=0.3] (P) to (I) to (J) to (K) to cycle;
			\draw[fill=black] (A) circle (2pt) node[shift={(230:2ex)}] {$A$};
			\draw[fill=black] (B) circle (2pt) node[shift={(310:2ex)}] {$B$};
			\draw[fill=black] (C) circle (2pt) node[shift={(60:2ex)}] {$C$};
			\draw[fill=black] (D) circle (2pt) node[shift={(140:2ex)}] {$D$};
			\draw[fill=black] (E) circle (2pt) node[shift={(265:2.35ex)}] {$E$};
			\draw[fill=black] (F) circle (2pt) node[shift={(340:2.25ex)}] {$F$};
			\draw[fill=black] (I) circle (2pt) node[shift={(170:1.5ex)}] {$I$};
			\draw[fill=black] (J) circle (2pt) node[shift={(270:2ex)}] {$J$};
			\draw[fill=black] (K) circle (2pt) node[shift={(65:1.5ex)}] {$K$};
			\draw[fill=white] (P) circle (2pt) node[shift={(95:2ex)}] {$P$};
			\node at (-7.3,14.15) {$t$};
		\end{tikzpicture}
	\end{figure}
	
	Seien $E$ und $F$ die Schnittpunkte von $t$ mit $AP$ und $BP$ (in der Notation des Sechs-Inkreise-Lemmas wären das die Punkte $B_D$ und $A_C$). Nachdem $t$ nun auch eine Tangente an den Inkreis von $ABP$ ist, liegen $I$, $J$ und $E$ auf der Winkelhalbierenden von $\winkel(AP,t)$. Analog liegen $J$, $K$ und $F$ auf der Winkelhalbierenden von $\winkel(t,BP)$. Also erhalten wir die Gleichung $\winkel KJI=180^\circ-\winkel EFJ-\winkel JEF=180^\circ-\winkel PEI-\winkel KFP$. Andererseits gilt aber auch $\winkel IPK=180^\circ-\winkel DPI-\winkel KPC=180^\circ-\winkel IPE-\winkel FPK$. Nach den obigen Gleichungen und wegen der Innenwinkelsumme in $EPI$ und $FKP$ gilt nun
	\begin{align*}
		\winkel KJI+\winkel IPK&=360^\circ-\winkel PEI-\winkel IPE-\winkel KFP-\winkel FPK\,,\\
		\winkel JIP+\winkel PKJ&=360^\circ-\winkel PEI-\winkel IPE-\winkel KFP-\winkel FPK\,.
	\end{align*}
	Folglich $\winkel KJI+\winkel IPK=\winkel JIP+\winkel PKJ$. Dann muss aber $\winkel KJI+\winkel IPK=180^\circ$ sein, weshalb $PIJK$ in der Tat ein Sehnenviereck ist.
\end{proof}

\begin{proof}[Lösung zu Aufgabe~\ref{aufgabe:531246} \textmd{(\href{https://www.mathematik-olympiaden.de/moev/index.php?option=com_download&thema=a&datei=A53124b.pdf&format=raw}{MO 531246})}]
	Weil $P$ auf der Winkelhalbierenden $AI$ von $\winkel BAD$ liegt, gibt es einen Kreis $\omega_A$ mit Mittelpunkt $P$, der $\overline{AB}$ und $\overline{DA}$ berührt. Analog gibt es einen Kreis $\omega_C$ mit Mittelpunkt $Q$, der $\overline{BC}$ und $\overline{CD}$ berührt.
	\begin{figure}[ht]
		\centering
		\begin{tikzpicture}[x=0.85cm,y=0.85cm]
			\draw [line width=0.3] (-10.79,11.71) coordinate (I) circle (2.66);
			\draw [line width=0.3] (-12.61,10.371) coordinate (P) circle (1.443);
			\draw [line width=0.3] (-10.203,12.763) coordinate (Q) circle (1.773);
			\coordinate (A) at (-14.768,8.784);
			\coordinate (B) at (-7.19,9.28);
			\coordinate (C) at (-9.029,14.869);
			\coordinate (D) at (-12.77,13.97);
			\draw (A) to (B) to (C) to (D) to cycle;
			\draw [line width=0.3] (A) to (I) to (C);
			%\draw [line width=0.3] (B) to (I) to (D);
			\draw [shorten <=-2ex,shorten >=-2em, dashed] (D) to (-10.366,9.072);
			\draw [shorten <=-2ex,shorten >=1.5em] (B) to (-15.352,13.35);
			\draw [line width=0.3] (P) to (B) to (Q);
			\draw [dashed,line width=0.3] (P) to (D);
			\draw [dashed,line width=0.3] (Q) to (D);
			%\draw [shift={(B)}, line width=0.3] (108.215:0.52cm) arc (108.215:130.856:0.52cm);
			%\draw [shift={(B)}, line width=0.3] (130.856:0.57cm) arc (130.856:153.497:0.57cm);
			%\draw [shift={(B)}, line width=0.3] (153.497:0.52cm) arc (153.497:168.622:0.52cm);
			%\draw [shift={(B)}, line width=0.3] (153.497:0.47cm) arc (153.497:168.622:0.47cm);
			%\draw [shift={(B)}, line width=0.3] (168.622:0.57cm) arc (168.622:183.746:0.57cm);
			%\draw [shift={(B)}, line width=0.3] (168.622:0.62cm) arc (168.622:183.746:0.62cm);
			\draw[fill=black] (A) circle (2pt) node[shift={(230:2ex)}] {$A$};
			\draw[fill=black] (B) circle (2pt) node[shift={(260:2ex)}] {$B$};
			\draw[fill=black] (C) circle (2pt) node[shift={(60:2ex)}] {$C$};
			\draw[fill=black] (D) circle (2pt) node[shift={(180:2ex)}] {$D$};
			\draw[fill=black] (I) circle (2pt) node[shift={(130:2ex)}] {$I$};
			\draw[fill=white] (P) circle (2pt) node[shift={(138:2ex)}] {$P$};
			\draw[fill=white] (Q) circle (2pt) node[shift={(105:2ex)}] {$Q$};
			\node at (-13.1,9.4) {$\omega_A$};
			\node at (-8.95,12.3) {$\omega_C$};
			\node at (-14.45,12.45) {$t_B$};
			\node at (-10.52,8.4) {$t_D$};
		\end{tikzpicture}
	\end{figure}
	
	Sei nun $t_B$ die von $AB$ verschiedene Tangente durch $B$ an $\omega_A$. Dann ist $BP$ die Winkelhalbierende von $\winkel (t_B,AB)$. Analog sei $t_B'$ die von $BC$ verschiedene Tangente durch $B$ an $\omega_C$, sodass $BQ$ die Winkelhalbierende von $\winkel (BC,t_B')$ ist. Die Voraussetzung $\winkel QBP=\frac12\winkel CBA$ sagt uns nun genau, dass $t_B=t_B'$ gelten muss. Mit anderen Worten: Die Tangente $t_B$ an $\omega_A$ ist auch eine Tangente an den Kreis $\omega_C$.
	
	Mit einem völlig analogen Argument sehen wir: Um $\winkel PDQ=\frac12\winkel ADC$ zu beweisen, müssen wir nur zeigen, dass die von $DA$ verschiedene Tangente $t_D$ durch $D$ an $\omega_A$ auch eine Tangente an $\omega_C$ ist. Das folgt aber direkt aus dem Sechs-Inkreise-Lemma in dem Spezialfall, dass die Kreise $\omega_B$, $\omega_D$ und $\omega$ zu Punkten degeneriert sind.
\end{proof}
	\subsection*{Lösungen zu Kapitel~\ref{kapitel:Symmediane}: \emph{Symmediane}}

\begin{proof}[Lösung zu Aufgabe~\ref{aufgabe:MEMO2014} \textmd{(\href{https://artofproblemsolving.com/community/c3587_2014_middle_european_mathematical_olympiad}{MEMO 2014/T6})}]
	Wir beginnen mit einigen Überlegungen, die euch in Fleisch und Blut übergehen sollten, wenn in einer Aufgabe die Gerade durch einen Eckpunkt und den gegenüberliegenden Inkreisberührpunkt vorkommt. Sei $\omega_a$ der Ankreis gegenüber $A$, sei $D_a$ der Berührpunkt von $\omega_a$ mit $\overline{BC}$ und sei $S$ der gegenüberliegende Punkt zu $D_a$ auf $\omega_a$. Dann sind $A$, $D$ und $S$ kollinear, denn die Streckung mit Zentrum $A$, die den Inkreis $\omega$ auf den Ankreis $\omega_a$ abbildet, muss $D$ auf $S$ abbilden. Außerdem erinnern wir uns, dass die Tangentenabschnitte $\overline{CD}$ und $\overline{BD_a}$ am In- und Ankreis gleich lang sind, sodass $M$ auch der Mittelpunkt von $\overline{D_aD}$ ist. Die Gerade $I_aM$ verläuft also sowohl durch den Mittelpunkt $M$ von $\overline{D_aD}$, als auch durch den Mittelpunkt $I_a$ von $\overline{D_aS}$. Folglich muss $I_aM$ parallel zu $AD$ sein. Das sieht schon mal nach einem guten Anfang aus, aber warum $BNCL$ ein Sehnenviereck sein soll, ist daraus noch nicht ersichtlich.
	
	\begin{figure}[ht]
		\centering
		\begin{tikzpicture}[x=0.5cm,y=0.5cm]
			%\clip (-4.66,-6.97) rectangle (5.17,6.77);
			\draw [line width=0.3,shift={(-0.108,2.22)}] (-117:2.22) arc (-117:224:2.22);
			\draw [line width=0.3,shift={(-2.486,-5.813)}]  (65:5.813) arc (65:419:5.813);
			\draw [line width=0.3,dashed] (-1.297,0.949) circle (3.901);
			\coordinate (A) at (1.362,7.187);
			\coordinate (B) at (-5.081,0);
			\coordinate (C) at (2.487,0);
			\coordinate (D) at (-0.108,0);
			\coordinate (E) at (2.085,2.564);
			\coordinate (F) at (-1.76,3.702);
			\coordinate (Da) at (-2.486,0);
			\coordinate (Ea) at (-6.813,-1.932);
			\coordinate (Fa) at (3.257,-4.913);
			\coordinate (I) at (-0.108,2.22);
			\coordinate (Ia) at (-2.486,-5.813);
			\coordinate (L) at (0.764,4.262);
			\coordinate (M) at (-1.297,0);
			\coordinate (N) at (-1.891,-2.906);
			\coordinate (S) at (-2.486,-11.626);
			\draw (B) to (C);
			\draw [shorten >=-5em] (A) to (Ea);
			\draw [shorten >=-5em] (A) to (Fa);
			\draw [line width=0.3] (Da) to (S) to (A);
			\draw [line width=0.3] (Ia) to (B) to (N) to (C) to (Ia) to (M);
			\draw [line width=0.3] (B) to (L) to (E) to (D) to (F) to (L) to (C);
			\draw [line width=0.3,shift={(B)}] (-42.342:0.42cm) arc (-42.342:0:0.42cm);
			\draw [line width=0.3,shift={(L)}] (216.097:0.37cm) arc (216.097:258.439:0.37cm);
			\draw [line width=0.3,shift={(L)}] (258.439:0.42cm) arc (258.439:292.015:0.42cm);
			\draw [line width=0.3,shift={(L)}] (258.439:0.47cm) arc (258.439:292.015:0.47cm);
			\draw [line width=0.3,shift={(C)}] (180:0.42cm) arc (180:213.576:0.42cm);
			\draw [line width=0.3,shift={(C)}] (180:0.47cm) arc (180:213.576:0.47cm);
			\draw [fill=black] (A) circle (2pt) node[shift={(80:2ex)}] {$A$};
			\draw [fill=black] (B) circle (2pt) node[shift={(175:2ex)}] {$B$};
			\draw [fill=black] (C) circle (2pt) node[shift={(-10:2ex)}] {$C$};
			\draw [fill=white] (D) circle (2pt) node[shift={(305:2ex)}] {$D$};
			\draw [fill=white] (E) circle (2pt) node[shift={(30:2.125ex)}] {$E$};
			\draw [fill=white] (F) circle (2pt) node[shift={(150:1.75ex)}] {$F$};
			\draw [fill=black] (I) circle (2pt) node[shift={(170:1.5ex)}] {$I$};
			\draw [fill=black] (Ia) circle (2pt) node[shift={(210:2ex)}] {$I_a$};
			\draw [fill=white] (Da) circle (2pt) node[shift={(90:2ex)}] {$D_a$};
			\draw [fill=white] (Ea) circle (2pt);
			\draw [fill=white] (Fa) circle (2pt);
			\draw [fill=black] (L) circle (2pt) node[shift={(40:2ex)}] {$L$};
			\draw [fill=black] (M) circle (2pt) node[shift={(90:2ex)}] {$M$};
			\draw [fill=black] (N) circle (2pt) node[shift={(300:2ex)}] {$N$};
			\draw [fill=black] (S) circle (2pt) node[shift={(270:2ex)}] {$S$};
			\node at (-1.75,1.8) {$\omega$};
			\node at (-6.95,-7.95) {$\omega_a$};
		\end{tikzpicture}
	\end{figure}
	
	Um die Sehnenviereckseigenschaft nachzuprüfen, müssen wir irgendwie an die Winkel $\winkel BLC$ und $\winkel CNB$ herankommen. Auf den ersten Blick erscheint das hoffnungslos. Aber auf den zweiten Blick eröffnet das Symmedian-Lemma eine Möglichkeit: Wenn $E$ und $F$ die Berührpunkte des Inkreises $\omega$ mit den Seiten $\overline{CA}$ und $\overline{AB}$ sind, dann ist $BL$ der Symmedian durch $L$ im Dreieck $DLF$ und $CL$ ist der Symmedian durch $L$ im Dreieck $DEL$. Andererseits sind $BN$ und $CN$ die Seitenhalbierenden von $\overline{I_aM}$ in den Dreiecken $BI_aM$ und $CMI_a$. Wenn wir zeigen könnten, dass $DLF$ und $BI_aM$ (gegensinnig) ähnlich sind, dann würde also $\winkel BLD=\winkel NBM$ folgen. Analog wäre $\winkel DLC=\winkel MCN$. Mithilfe der Innenwinkelsumme im Dreieck $BNC$ würde also
	\begin{equation*}
		\winkel BLC=\winkel BLD+\winkel DLC=\winkel NBM+\winkel MCN=180^\circ-\winkel CNB
	\end{equation*}
	folgen und wir würden wie gewünscht erhalten, dass $BNCL$ ein Sehnenviereck ist. Die Ähnlichkeit $BI_aM\sim DLF$ ist nun eine einfache Winkeljagd: Weil $BI_a$ die Außenwinkelhalbierende von $\winkel CBA$ ist, gilt $\winkel I_aBM=90^\circ-\beta/2$, wobei wie üblich $\beta\coloneqq \winkel CBA$. Weil die Tangentenabschnitte $\overline{BD}$ und $\overline{BF}$ gleich lang sind, ist das Dreieck $BDF$ gleichschenklig und es gilt $\winkel FDB=90^\circ-\beta/2$. Nach dem Sehnen-Tangentenwinkelsatz ist also auch $\winkel FLD=90^\circ-\beta/2$. Andererseits haben wir schon $I_aM\parallel AD$ bemerkt, sodass $\winkel BMI_a=\winkel CDL$ gilt. Wiederum nach dem Sehnen-Tangentenwinkelsatz ist $\winkel CDL=\winkel DFL$. Also stimmen die Dreiecke $BI_aM$ und $DLF$ in zwei Winkeln überein und sind daher wie gewünscht (gegensinnig) ähnlich.
\end{proof}

\begin{proof}[Lösung zu Aufgabe~\ref{aufgabe:PolenMO2019} \textmd{(\href{https://artofproblemsolving.com/community/c904216_2019_polish_mo_finals}{Polnische MO 2019/6})}]
	Sich berührende Kreise riechen tendenziell nach Inversion, aber bei dieser Aufgabe führt das zu nichts. Der einzige Punkt, an dem wir potentiell invertieren könnten, ohne uns die Aufgabe komplett zu zerschießen, ist $A$. Nach der Inversion berührt $\omega_E'$ die Gerade $\Omega'$, aber auch den Umkreis $\odot AB'E'$, also haben wir es immer noch mit sich berührenden Kreisen zu tun und die Aufgabe ist nicht einfacher geworden (tatsächlich sogar schwerer, weil sich $\omega'$ schwerer beschreiben lässt). Allgemein gilt: Wenn sich in einer Aufgabe Kreise berühren und Inversion zu nichts führt, dann muss stattdessen am Berührpunkt gestreckt werden. Momentan ist noch nicht klar, was das bringt, aber wir behalten es im Hinterkopf.
	
	\begin{figure}[ht]
		\centering
		\begin{tikzpicture}[x=0.4cm,y=0.4cm]
			%\clip (-4.66,-6.97) rectangle (5.17,6.77);
			%\draw [line width=0.3,shift={(-0.108,2.22)}] (-117:2.22) arc (-117:224:2.22);
			%\draw [line width=0.3,shift={(-2.486,-5.813)}]  (65:5.813) arc (65:419:5.813);
			\coordinate (A) at (3.544,11.406);
			\coordinate (B) at (-4,0);
			\coordinate (C) at (6,0);
			\coordinate (D) at (-0.488,0);
			\coordinate (E) at (3.173,0);
			\coordinate (M) at (1,0);
			\coordinate (P) at (-5.958,4.178);
			\coordinate (Q) at (7.352,1.962);
			\coordinate (R) at (1.879,0.592);
			\coordinate (S) at (1,-2.104);
			\coordinate (T) at (1,-5.112);
			\coordinate (X) at (2.923,-7.68);
			\coordinate (Y) at (-4.434,-11.164);
			\coordinate (Z) at (19.139,0);
			\draw [line width=0.3] (1,4.891) circle (6.994);
			\draw [line width=0.3] (-1.33,4.652) circle (4.652);
			\draw [line width=0.3] (4.286,3.376) circle (3.376);
			\draw [line width=0.3] (S) circle (2.104);
			\draw (A) to (B) to (C) to cycle;
			\draw [line width=0.3,shorten <=-2ex,shorten >=-1.5em,dashed] (X) to (P);
			\draw [line width=0.3,shorten <=-2ex,shorten >=-1.5em,dashed] (Y) to (Q);
			\draw [line width=0.3,shorten >=-2ex] (A) to (X);
			\draw [line width=0.3,shorten >=-2ex] (A) to (Y);
			\draw [shorten <=-4em] (B) to (T);
			\draw [shorten <=-4em] (C) to (T);
			\draw [line width=0.3,shorten <=-2em,shorten >=-2ex] (B) to (Z);
			\draw [line width=0.3,shorten <=-2em,shorten >=-2ex,dashed] (P) to (Z);
			\draw [line width=0.3,dashed,shorten <=-2ex,shorten >=-2ex] (Y) to (Z);
			\draw [line width=0.3] (A) to (M);
			\draw [line width=0.3] (A) to (T);
			\draw [line width=0.3] (E) to (P);
			\draw [line width=0.3] (D) to (Q);
			\draw [line width=0.3,shift={(A)}] (250.532:0.87cm) arc (250.532:257.425:0.87cm);
			\draw [line width=0.3,shift={(A)}] (250.532:0.92cm) arc (250.532:257.425:0.92cm);
			\draw [line width=0.3,shift={(A)}] (261.243:0.87cm) arc (261.243:268.136:0.87cm);
			\draw [line width=0.3,shift={(A)}] (261.243:0.92cm) arc (261.243:268.136:0.92cm);
			\draw [fill=black] (A) circle (2pt) node[shift={(80:2ex)}] {$A$};
			\draw [fill=black] (B) circle (2pt) node[shift={(235:2ex)}] {$B$};
			\draw [fill=black] (C) circle (2pt) node[shift={(305:2ex)}] {$C$};
			\draw [fill=black] (D) circle (2pt) node[shift={(215:2.5ex)}] {$D$};
			\draw [fill=black] (E) circle (2pt) node[shift={(315:2.07ex)}] {$E$};
			\draw [fill=black] (M) circle (2pt) node[shift={(260:2ex)}] {$M$};
			\draw [fill=white] (P) circle (2pt) node[shift={(50:2.25ex)}] {$P$};
			\draw [fill=white] (Q) circle (2pt) node[shift={(120:2.25ex)}] {$Q$};
			\draw [fill=white] (R) circle (2pt) node[shift={(46:2.5ex)}] {$R$};
			\draw [fill=black] (S) circle (2pt) node[shift={(250:2ex)}] {$S$};
			\draw [fill=white] (T) circle (2pt) node[shift={(270:2.5ex)}] {$T$};
			\draw [fill=black] (X) circle (2pt) node[shift={(350:3ex)}] {$X$};
			\draw [fill=black] (Y) circle (2pt) node[shift={(315:2.25ex)}] {$Y$};
			\draw [fill=black] (Z) circle (2pt) node[shift={(270:2ex)}] {$Z$};
			\node at (2.1,-3.1) {$\omega$};
			\node at (-0.5,11) {$\Omega$};
			\node at (6.2,5) {$\omega_D$};
			\node at (-3,7.9) {$\omega_E$};
		\end{tikzpicture}
	\end{figure}
	
	Als nächstes erinnern wir uns, dass $S$ nach dem Südpolsatz auf der Winkelhalbierenden von $\winkel BAC$ liegt. Die Tangenten von $A$ an $\omega$ liegen symmetrisch bezüglich $AS$, also symmetrisch bezüglich der Winkelhalbierenden von $\winkel BAC$. Statt $\winkel DAM=\winkel RAE$ können wir also auch $\winkel BAM=\winkel RAC$ zeigen. Mit anderen Worten: Wir müssen zeigen, dass $R$ auf dem Symmedian durch $A$ liegt. Nach dem Symmedian-Lemma ist besagter Symmedian genau die Gerade $AT$, wobei $T$ der Schnittpunkt der Tangenten an $\Omega$ in $B$ und $C$ ist. Also zeichnen wir $T$ in unsere Skizze ein -- und siehe da: Es scheint, als ob $\omega$ der Inkreis des Dreiecks $BTC$ ist!
	
	Diese Beobachtung lässt sich durch eine einfache Winkeljagd bestätigen: Nach dem Sehnen-Tangentenwinkelsatz gilt $\winkel TBC=\alpha$, wobei wie üblich $\alpha\coloneqq \winkel BAC$. Nach dem Peripheriewinkelsatz gilt aber auch $\winkel SBC=\winkel SAC=\alpha/2$, weil $S$ auf der Winkelhalbierenden von $\winkel BAC$ liegt. Also liegt $S$ ebenfalls auf der Winkelhalbierenden von $\winkel TBC$ und analog auch auf der Winkelhalbierenden von $\winkel BCT$. Es handelt sich bei $S$ also in der Tat um den Inkreismittelpunkt von $BTC$ und daher bei $\omega$ um den zugehörigen Inkreis.
	
	Nun stecken wir erst einmal fest: Wir müssen zeigen, dass sich die Geraden $AT$, $DQ$ und $EP$ in einem Punkt (nämlich in $R$) schneiden, aber es ist schwer, an diese Geraden heranzukommen. Eine Idee wäre, $\omega$ als Ankreis von $ADE$ zu betrachten. Aus dem Satz von Ceva folgt leicht, dass sich in jedem Dreieck die drei Geraden durch einen Eckpunkt und den gegenüberliegenden Ankreisberührpunkt in einem Punkt schneiden (dem \emph{Nagel-Punkt}). Ihre Spiegelbilder an den Winkelhalbierenden schneiden sich dann ebenfalls in einem Punkt (dem isogonal konjugierten Punkt zum Nagel-Punkt). Mithilfe einer genauen Skizze werdet ihr aber feststellen, dass $DQ$ und $EP$ nicht mit diesen Spiegelbildern übereinstimmen. So kommen wir also nicht weiter.
	
	Wenn ihr feststeckt, dann könnt ihr immer versuchen, mit einem der vielen projektiven Hilfsmittel die Aufgabe umzuformulieren. Hier erinnern wir uns an den Satz von Desargues: Die Geraden $AT$, $DQ$ und $EP$ schneiden sich genau dann in einem Punkt, wenn der Schnittpunkt $X$ von $AE$ und $PT$, der Schnittpunkt $Y$ von $AD$ und $QT$ sowie der Schnittpunkt $Z$ von $BC$ und $PQ$ kollinear sind. Um das zu zeigen, betrachten wir zuerst den Punkt $X$ und erinnern uns, dass wir noch nicht an $P$ gestreckt haben. Das wollten wir tun, weil es eine Streckung an $P$ gibt, die $\omega_E$ auf $\Omega$ abbildet -- also weil $P$ das \emph{äußere Ähnlichkeitszentrum} der Kreise $\omega_E$ und $\Omega$ ist. Wir haben außerdem gesehen, dass $\omega$ der Inkreis von $BTC$ ist, also $T$ der Schnittpunkt der gemeinsamen äußeren Tangenten von $\omega$ und $\Omega$ und somit das äußere Ähnlichkeitszentrum von $\omega$ und $\Omega$. Nach dem Satz von Monge liegen die drei äußeren Ähnlichkeitszentren der Kreise $\omega$, $\omega_E$ und $\Omega$ auf einer Geraden, sodass $PT$ durch den Schnittpunkt der äußeren gemeinsamen Tangenten von $\omega$ und $\omega_E$ verläuft. Eine dieser äußeren gemeinsamen Tangenten ist $AE$. Also muss $X$, der Schnittpunkt von $PT$ und $AE$, genau das äußere Ähnlichkeitszentrum von $\omega$ und $\omega_E$ sein!
	
	Analog ist $Y$ das äußere Ähnlichkeitszentrum von $\omega$ und $\omega_D$. Eine ähnliche Überlegung mit $\omega_D$, $\omega_E$ und $\Omega$ führt schließlich darauf, dass $Z$, der Schnittpunkt von $PQ$ mit $BC$, das äußere Ähnlichkeitszentrum von $\omega_D$ und $\omega_E$ ist. Also sind $X$, $Y$ und $Z$ die äußeren Ähnlichkeitszentren der Kreise $\omega$, $\omega_D$ und $\omega_E$, welche nach dem Satz von Monge kollinear sind. Also sind wir fertig!
\end{proof}
	\subsection*{Lösungen zu Kapitel~\ref{kapitel:LTE}: \emph{Das Lifting-The-Exponent-Lemma}}

\begin{proof}[Lösung zu Aufgabe~\ref{aufgabe:NieQuadratfrei}]
	Für $a\geqslant 3$ besitzt $a-1$ mindestens einen Primfaktor $p$. Falls~$p$ ungerade ist, dann gilt $v_p(a^{a-1}-1)=v_p(a-1)+v_p(a-1)\geqslant 2$ nach dem LTE-Lemma. Also kann $a^{a-1}-1$ nicht quadratfrei sein. Es verbleibt der Fall, dass $a-1$ nur den Primfaktor $p=2$ besitzt, also eine Zweierpotenz ist. Für $a-1=2$ erhalten wir $a^{a-1}-1=8$, was nicht quadratfrei ist. Ansonsten muss $a\equiv 1\mod 4$ sein und es folgt wiederum $v_2(a^{a-1}-1)=v_2(a-1)+v_2(a-1)\geqslant 2$ nach dem LTE-Lemma. Auch in diesem Fall kann~$a$ nicht quadratfrei sein.
\end{proof}

\begin{proof}[Lösung zu Aufgabe~\ref{aufgabe:2p3pan} \textmd{(\href{https://artofproblemsolving.com/community/c583198_1996_irish_math_olympiad}{Irische MO 1996/8})}]
	Für $p=2$ erhalten wir $2^2+3^2=13$, was keine echte Potenz ist. In diesem Fall ist also nur $n=1$ möglich. Für ungerade Primzahlen $p\geqslant 3$ schreiben wir $2^p+3^p=2^p-(-3)^p$ und wenden das LTE-Lemma an:
	\begin{equation*}
		v_5\parens*{2^p-(-3)^p}=v_5\parens[\big]{2-(-3)}+v_5(p)=1+v_5(p)\,.
	\end{equation*}
	Für $p\neq 5$ folgt, dass $2^p+3^p$ genau einmal durch~$5$ teilbar ist. Wenn aber $a^n\equiv 0\mod 5$ gilt, dann muss $a$ durch~$5$ und somit $a^n$ durch $5^n$ teilbar sein. Folglich ist nur $n=1$ möglich. Im Fall $p=5$ sehen wir direkt, dass $2^5+3^5=275=5^2\cdot 11$ keine echte Potenz sein kann.
\end{proof}

%	\textbf{Lösung zu Aufgabe~\ref{aufgabe:EndlicheMengeVonPrimzahlen}.} Wir dürfen ohne Einschränkung annehmen, dass $a$ zu allen Primzahlen in $\Sigma$ teilerfremd ist, denn diejenigen, für die das nicht der Fall ist, können ohnehin nicht als Teiler von $a^n-1$ auftreten. Sei $p\in\Sigma$ eine ungerade Primzahl und sei $\operatorname{ord}_p(a)$ die multiplikative Ordnung\footnote{Die \emph{multiplikative Ordnung von $a$ modulo~$p$} ist die kleinste positive ganze Zahl, sodass $a^{\operatorname{ord}_p(a)}\equiv 1\mod p$ gilt. Siehe das Kapitel \emph{Multiplikative Ordnungen und Primitivwurzeln} im Heft für Klasse~10.} von~$a$ modulo~$p$. Wenn $n$ durch $\operatorname{ord}_p(a)$ teilbar ist, gilt nach dem LTE-Lemma
%	\begin{equation*}
	%		v_p\parens*{a^n-1}=v_p\parens*{a^{\operatorname{ord}_p(a)}-1}+v_p\parens*{\frac{n}{\operatorname{ord}_p(a)}}\,.
	%	\end{equation*}
%	Wenn $n$ nicht durch $\operatorname{ord}_p(a)$ teilbar ist, gilt $v_p(a^n-1)=0$. In jedem Fall erhalten wir die sehr großzügige Abschätzung
%	\begin{equation*}
	%		v_p\parens*{a^n-1}\leqslant C_p+v_p(n)\,,
	%	\end{equation*}
%	wobei $C_p\coloneqq v_p(a^{\operatorname{ord}_p(a)}-1)$ eine Konstante ist, die nicht von~$n$ abhängt. Falls $2\in\Sigma$, dann erhalten wir eine ähnliche Abschätzung für $p=2$. Wenn $n$ ein ungerades Vielfaches von $\operatorname{ord}_2(a)$ ist, dann gilt $v_2(a^n-1)=v_2(a^{\operatorname{ord}_2(a)}-1)$ nach dem Hilfslemma im Theorieteil des Kapitels. Falls $n$ ein gerades Vielfaches von $\operatorname{ord}_2(a)$ ist, dann haben wir im Theorieteil gezeigt, dass
%	\begin{equation*}
	%		v_2\parens*{a^n-1}=v_2{a^{\operatorname{ord}_2(a)}-1}+v_2\parens*{a^{\operatorname{ord}_2(a)}+1}+v_2\parens*{\frac{n}{\operatorname{ord}_2}}-1\,.
	%	\end{equation*}
%	Falls $n$ nicht durch $\operatorname{ord}_2(a)$ teilbar ist, gilt $v_2(a^n-1)$. Wir erhalten also auch hier eine großzügige Abschätzung der Form
%	\begin{equation*}
	%		v_2\parens*{a^n-1}\leqslant C_2+v_2(n)\,,
	%	\end{equation*}
%	wobei $C_1\coloneqq v_2(a^{\operatorname{ord}_2(a)}-1)+v_2(a^{\operatorname{ord}_2(a)}+1)$ eine Konstante ist, die nicht von~$n$ abhängt. Wenn $a^n-1$ nur durch Primzahlen aus $\Sigma$ teilbar ist, gilt
%	\begin{equation*}
	%		a^n-1=\prod_{p\in\Sigma}p^{v_p(a^n-1)}\leqslant \prod_{p\in\Sigma}p^{C_p+v_p(n)}\leqslant Cn\,,
	%	\end{equation*}
%	wobei $C\coloneqq \prod_{p\in \Sigma}C_p$ eine Konstante ist, die nicht von~$n$ abhängt. Aus dieser Ungleichungskette ist klar, dass es nur endlich viele Möglichkeiten für~$n$ geben kann.\qed
\begin{proof}[Lösung zu Aufgabe~\ref{aufgabe:xn-yn}]
	Schreibe $x=a/c$ und $y=b/c$, wobei $a$, $b$ und $c$ positive ganze Zahlen sind und $c$ minimal gewählt ist. Wir zeigen zuerst, dass $a/c$ und $b/c$ vollständig gekürzte Brüche sein müssen. Zu diesem Zweck gehen wir indirekt vor und nehmen an, $a/c$ ließe sich kürzen (der Fall für $b/c$ geht analog). Folglich gibt es eine Primzahl $p$ mit $a,c\equiv 0\mod p$. Dann muss $b\not\equiv 0\mod p$ sein, denn sonst ließen sich beide Brüche kürzen und~$c$ wäre nicht minimal. Dann ist aber $a-b\not\equiv 0\mod p$, was der Tatsache widerspricht, dass $x-y=(a-b)/c$ eine ganze Zahl ist. Das gleiche Argument lässt sich auch mit $x^n-y^n$ für jedes $n\geqslant 1$ durchführen, sodass wir auch in der Situation von~$(b)$ schlussfolgern können, dass  $a/c$ und $b/c$ vollständig gekürzt sind.
	
	Wir zeigen nun~$(a)$. Dazu genügt es, $c=1$ zu beweisen. Angenommen, das wäre nicht der Fall. Dann gibt es einen Primfaktor $p\mid c$. Weil $x-y$ ganzzahlig ist, muss $a\equiv b\mod p$ gelten und weil die Brüche  $a/c$ und $b/c$ vollständig gekürzt sind, muss $a,b\not\equiv 0\mod p$ sein. Die Bedingung, dass $x^n-y^n$ für alle $n\geqslant 1$ ganzzahlig ist, liefert uns $v_p(a^n-b^n)\geqslant nv_p(c)$. Wenn wir $n$ so wählen, dass $n\not\equiv 0\mod p$ gilt, dann ist also
	\begin{equation*}
		v_p\parens*{a^n-b^n}=v_p(a-b)
	\end{equation*}
	(das stimmt auch für $p=2$; siehe das Hilfslemma im Theorieteil des Kapitels). Somit ist $v_p(a-b)\geqslant nv_p(c)$. Die linke Seite ist aber eine Konstante, die nicht von~$n$ abhängt, also erhalten wir für hinreichend großes~$n$ einen Widerspruch.
	
	Für~$(b)$ genügt es wiederum, $c=1$ zu beweisen. Wie in~$(a)$ betrachten wir einen Primfaktor $p\mid c$ und folgern $a,b\not\equiv 0\mod p$. Allerdings erhalten wir nicht unbedingt $a\equiv b\mod p$. Betrachte stattdessen die Restklasse $a/b$ modulo~$p$ (also das Produkt von $a$ mit dem multiplikativen Inversen von~$b$). Dann ist $a^{\operatorname{ord}_p(a/b)}\equiv b^{\operatorname{ord}_p(a/b)}\mod p$. Nach Annahme gibt es unendlich viele $n\geqslant 1$, für die $x^n-y^n$ ganzzahlig ist. Für diese~$n$ muss wieder $v_p(a^n-b^n)\geqslant nv_p(c)$ gelten. Insbesondere ist $a^n\equiv b^n\mod p$, woraus
	\begin{equation*}
		\parens*{\frac{a}{b}}^n\equiv 1\mod p
	\end{equation*}
	folgt, sodass $n$ durch die multiplikative Ordnung $\operatorname{ord}_p(a/b)$ teilbar sein muss. Indem wir das LTE-Lemma auf $a^{\operatorname{ord}_p(a/b)}$ und $b^{\operatorname{ord}_p(a/b)}$ anwenden, erhalten wir
	\begin{equation*}
		v_p\parens*{a^n-b^n}=v_p\parens*{a^{\operatorname{ord}_p(a/b)}- b^{\operatorname{ord}_p(a/b)}}+v_p\parens*{\frac{n}{\operatorname{ord}_p(a/b)}}
	\end{equation*}
	falls~$p$ ungerade ist. Im Fall~$p=2$ erhalten wir
	\begin{equation*}
		v_2\parens*{a^n-b^n}=v_2\parens*{a^{\operatorname{ord}_p(a/b)}- b^{\operatorname{ord}_p(a/b)}}+v_2\parens*{a^{\operatorname{ord}_p(a/b)}+ b^{\operatorname{ord}_p(a/b)}}+v_2\parens*{\frac{n}{\operatorname{ord}_p(a/b)}}-1
	\end{equation*}
	falls $n$ ein gerades Vielfaches von $\operatorname{ord}_p(a/b)$ ist und $v_2(a^n-b^n)=v_2(a^{\operatorname{ord}_p(a/b)}- b^{\operatorname{ord}_p(a/b)})$ falls $n$ ein ungerades Vielfaches von $\operatorname{ord}_p(a/b)$ ist. In jedem dieser Fälle erhalten wir eine großzügige Abschätzung der Form $v_p(a^n-b^n)\leqslant C+v_p(n)$, wobei $C$ eine Konstante ist, die nicht von~$n$ abhängt. Andererseits haben wir bereits festgestellt, dass $v_p(a^n-b^n)\geqslant nv_p(c)$ gilt. Mit der Abschätzung $v_p(n)\leqslant \ln(n)/\ln(p)$, die aus $p^{v_p(n)}\leqslant n$ folgt, erhalten wir schlussendlich
	\begin{equation*}
		nv_p(n)\leqslant C+\frac{\ln(n)}{\ln(p)}\,.
	\end{equation*}
	Diese Ungleichung kann nur für endlich viele~$n$ gelten, denn links steht ein linearer Term in~$n$ (mit $v_p(c)>0$ nach Annahme), während der Term rechts logarithmisch in~$n$ ist. Das widerspricht der Annahme, dass $x^n-y^n$ für unendlich viele~$n$ ganzzahlig ist.
\end{proof}

\begin{proof}[Lösung zu Aufgabe~\ref{aufgabe:IMOSL2014N5} \textmd{(\href{https://artofproblemsolving.com/community/c107000_2014_imo_shortlist}{IMO-Shortlist 2014/N5})}]
	Der Fall $p=2$ führt auf die Bedingung, dass $x+y$ eine Zweierpotenz ist. Also erhalten wir die Lösungen $(p,x,y)=(2,x,2^n-x)$ für alle positiven ganzen Zahlen $n\geqslant 1$ und alle positiven ganzen Zahlen $1<x<2^n$.
	
	Von nun an nehmen wir an, dass $p\geqslant 3$ eine ungerade Primzahl ist. Im Fall $x=y$ müsste $x^{p-1}+x=x(1+x^{p-2})$ eine Potenz von $p$ sein. Da beide Faktoren teilerfremd sind, ist das nur für $x=1$ möglich. Aber für $p\geqslant 3$ ist $x=y=1$ keine Lösung. Von nun an nehmen wir ohne Einschränkung $x>y$ an. Dann gilt auch $x^{p-1}+y>x+y^{p-1}$. Wenn wir also $x^{p-1}+y=p^\alpha$ und $x+y^{p-1}=p^\beta$ schreiben, dann gilt $\alpha>\beta$.
	
	Als nächstes schließen wir $x\equiv 0\mod p$ aus. In diesem Fall wäre $p^\alpha =x^{p-1}+y>p^{p-1}$, also müsste $p^\beta$ durch $p^{p-1}$ teilbar sein. Dann wäre auch $y$ durch $p^{p-1}$ teilbar. Folglich gälte $p^\beta =x+y^{p-1}>p^{(p-1)^2}$ und wir sehen, dass $p^\beta$ und damit auch $x$ durch $p^{(p-1)^2}$ teilbar sein müssten. Indem wir dieses Argument iterieren, sehen wir, dass $x$ durch beliebig große Potenzen von $p$ teilbar wäre, was für $x>0$ nicht sein kann. Also ist $x\not\equiv 0\mod p$. Das gleiche Argument zeigt auch $y\not\equiv 0\mod p$. Nach dem kleinen Satz von Fermat gilt dann $x^{p-1}+y\equiv 1+y\mod p$. Andererseits ist $p^\alpha\equiv 0\mod p$, denn der Fall $p^\alpha=1$ ist wegen $x,y>0$ unmöglich. Es folgt dann $1+y\equiv 0\mod p$, also $y\equiv -1\mod p$. Ein analoges Argument zeigt $x\equiv -1\mod p$. Insbesondere ist $x\equiv y\mod p$ und das LTE-Lemma ist auf $x$ und $y$ anwendbar.
	
	Betrachte nun $x^p-y^p=x(x^{p-1}+y)-y(x+y^{p-1})=xp^\alpha-yp^\beta$. Nach dem LTE-Lemma gilt einerseits $v_p(x^p-y^p)=v_p(x-y)+1$. Andererseits gilt, wie wir oben festgestellt haben, $\alpha>\beta$, also $v_p(xp^\alpha)=\alpha >\beta=v_p(yp^\beta)$. Folglich ist $v_p(xp^\alpha-yp^\beta)=\beta$. Daraus folgt $x\equiv y\mod p^{\beta-1}$ (und $p^{\beta-1}$ ist sogar die größte Potenz von~$p$, die $x-y$ teilt). Das sieht nun endlich nach einer ziemlich starken Bedingung aus!
	
	Wir erhalten nun $0\equiv p^\beta\equiv x+y^{p-1}\equiv y+y^{p-1}\equiv y(1+y^{p-2})\mod p^{\beta-1}$. Wegen $y\not\equiv 0\mod p$ muss $1+y^{p-2}\equiv 0\mod p^{\beta-1}$ sein. Das können wir auch in der Form $(-y)^{p-2}\equiv 1\mod p^\beta$ schreiben. Insbesondere ist $\operatorname{ord}_{p^{\beta-1}}(-y)$ ein Teiler von $p-2$. Aber $\operatorname{ord}_{p^{\beta-1}}(-y)$ ist auch ein Teiler von $\varphi(p^{\beta-1})=(p-1)p^{\beta-2}$. Weil $p-2$ und $(p-1)p^{\beta-1}$ teilerfremd sind, muss zwangsläufig $\operatorname{ord}_{p^{\beta-1}}(-y)=1$ und somit $1+y\equiv 0\mod p^{\beta-1}$ gelten. Insbesondere ist $y\geqslant p^{\beta-1}-1$.
	
	Wegen $x>y$ und $p^{\beta-1}\mid x-y$ folgt $x>p^{\beta-1}$. Insgesamt erhalten wir die Ungleichungskette
	\begin{equation*}
		(p-1)p^{\beta-1}>p^\beta-x =y^{p-1}\geqslant \parens*{p^{\beta-1}-1}^{p-1}\,.
	\end{equation*}
	Damit sind wir nun definitiv auf dem richtigen Weg, denn die linke Seite ist $\approx p^\beta$, während die rechte Seite $\approx p^{(p-1)(\beta-1)}$ beträgt. Für $\beta\geqslant 2$ kann die Ungleichungskette also nicht gelten, außer vielleicht für einige kleine Werte von~$p$. Um diese Überlegung zu formalisieren, schätzen wir im Fall $\beta\geqslant 2$ folgendermaßen ab:
	\begin{align*}
		\parens*{p^{\beta-1}-1}^{p-1}=p^{(p-1)(\beta-1)}\parens*{1-\frac{1}{p^{\beta-1}}}^{p-1}&\geqslant p^{(p-1)(\beta-1)}\parens*{1-\frac{p-1}{p^{\beta-1}}}\\
		&\geqslant p^{(p-1)(\beta-1)}\cdot \frac 1p\\
		&=p^{(p-1)(\beta-1)-1}\,.
	\end{align*}
	In der ersten Abschätzung haben wir die Bernoulli-Ungleichung benutzt und in der zweiten Abschätzung haben wir die Annahme $\beta\geqslant 2$ verwendet. Andererseits haben wir die triviale Abschätzung $(p-1)p^{\beta-1}<p^\beta$. Für $p\geqslant 5$ und $\beta\geqslant 2$ gilt $(p-1)(\beta-1)-1\geqslant \beta$, also haben wir gezeigt, dass die behauptete Ungleichungskette in der Tat verletzt ist. Für $p=3$ erhalten wir erst für $\beta\geqslant 3$ einen Widerspruch. Übrig bleiben die folgenden beiden Fälle:
	
	\emph{Fall~1: Es gilt $\beta=1$.} Dann ist $x+y^{p-1}=p^\beta=p$. Wir haben aber weiter oben festgestellt, dass $x,y\equiv -1\mod p$ gilt. Also ist $x+y^{p-1}>x+y\geqslant 2(p-1)$ und wir erhalten einen Widerspruch.
	
	\emph{Fall~2: Es gilt $p=3$ und $\beta=2$.} Dieser Fall führt auf $x+y^2=p^\beta=9$. Zusammen mit unserer Feststellung $y\equiv -1\mod 3$ lässt das nur den Fall $x=5$, $y=2$ zu. Dieser Fall ist aber tatsächlich eine Lösung, denn $5^2+2=27=3^3$ ist ebenfalls eine Dreierpotenz. Wir erhalten somit die Lösung $(p,x,y)=(3,5,2)$. An dieser Stelle erinnern wir uns außerdem an unsere Annahme $x>y$. Wenn wir diese Annahme aufheben, erhalten wir die weitere Lösung $(p,x,y)=(3,2,5)$.
	
	Damit wurden alle Fälle betrachtet und die Aufgabe ist gelöst.
\end{proof}
	
	\newpage
	\cftaddtitleline{toc}{part}{MatBoj-Regeln}{\thepage}
	\section*{MatBoj-Regeln} 

MatBoj -- abgeleitet aus dem Russischen -- steht für \enquote{mathematischer Kampf}.

Zwei Teams lösen Aufgaben und präsentieren anschließend ihre Lösungen.

\subsection*{Phase 1: Das Lösen der Aufgaben}
Jede Mannschaft gibt sich einen Namen und wählt einen Mannschaftskapitän und einen Stellvertreter. Diese vertreten die Mannschaft als Sprecher. Nur sie können für die Mannschaft verbindliche Entscheidungen verkünden.

Beide Teams erhalten den gleichen Satz von Aufgaben. Ihnen steht eine vorher bekanntgegebene Zeit zur Verfügung, um die Aufgaben getrennt voneinander zu lösen. 

Sollte einem Teammitglied eine Aufgabe bereits bekannt sein, so ist es aus Fairnessgründen dazu aufgefordert, dies der Jury bekanntzumachen (eventuell wird die betreffende Aufgabe durch eine neue ersetzt).


\subsection*{Phase 2: Das Vorstellen der Lösungen}
\begin{itemize}
	\item Den beiden Kapitänen wird gleichzeitig eine leichte Einstiegsaufgabe gestellt, die sie ohne Hilfsmittel lösen müssen. Keines der anderen Teammitglieder darf ihnen dabei helfen. Wer die richtige Antwort gibt, gewinnt für sein Team das Recht zu entscheiden, welches Team als erstes herausfordert. Gibt einer der Kapitäne eine falsche Antwort, erhält das Team des anderen Kapitäns dieses Recht.  
	\item \textbf{Herausfordern:} Das entsprechende Team fordert vom gegnerischen Team eine Aufgabe. Das herausgeforderte Team kann die Herausforderung annehmen oder ablehnen:
	\begin{itemize}
		\item Die \textit{Herausforderung wird angenommen}: Das herausgeforderte Team entsendet ein Teammitglied als \textit{Referenten}, der eine Lösung der Aufgabe vorstellt, das herausfordernde Team entsendet einen \textit{Kritiker}, der Lücken in der Lösung zu finden versucht. Nach der Vorstellung der Lösung darf der Kritiker erst Verständnisfragen stellen und dann die vorgetragene Lösung kritisieren und die von ihm aufgedeckten Lücken füllen. Hilfe aus dem Team ist unzulässig.
		\item Die \textit{Herausforderung wird abgelehnt}: Das herausfordernde Team entsendet ein Teammitglied, das eine Lösung der Aufgabe vorstellt, das herausgeforderte  Team entsendet einen Kritiker, der Lücken in der Lösung zu finden versucht. Nach der Vorstellung der Lösung darf der Kritiker erst Verständnisfragen stellen und dann die vorgetragene Lösung kritisieren. Er darf jedoch keine von ihm aufgedeckten Lücken füllen. Hilfe aus dem Team ist unzulässig.
	\end{itemize}
	\item \textbf{Bewertung:} Jede Aufgabe ist 12 Punkte wert. Der Referent erhält eine der Punktzahlen $0, 2, 4, 6, 8, 10, 12$, je nachdem, wie richtig und vollständig die von ihm vorgetragene Lösung ist. Der Kritiker erhält für das Aufdecken der Lücken in der vorgetragenen Lösung und für das Füllen dieser Lücken jeweils die Hälfte der noch nicht vergebenen Punkte. Wie weit Referent und Kritiker ihren Aufgaben im Einzelnen gerecht wurden, liegt im Ermessen der Jury.
	\item \textbf{Invalid challenge:} Wird die Herausforderung abgelehnt und kann das herausfordernde Team keine Lösung präsentieren, liegt ein \emph{invalid challenge} vor. Die Einschätzung, ob es sich um eine Lösung handelt, liegt im Ermessen der Jury.
	
	In diesem Fall erhält das herausgeforderte Team $6$ Punkte.
	\item Die \textit{nächste Herausforderung}: Es wird abwechselnd herausgefordert. Liegt ein invalid challenge vor, muss das herausfordernde Team erneut eine Aufgabe fordern.
	\item Die Endphase des Wettbewerbs: Zu einem beliebigen Zeitpunkt kann jedes der beiden Teams beschließen, keine Lösungen mehr zu präsentieren. Das betreffende Team muss aber weiter Kritiker entsenden, da das andere Team solange weiter Lösungen vorstellen kann, wie es will. Die Kritiker dürfen in dieser Endphase des Wettbewerbs nur noch Lücken in den vorgetragenen Lösungen aufzeigen, aber nicht mehr füllen.
	\item Am Ende des Wettbewerbs muss jedes Teammitglied mindestens einmal als Referent bzw. als Kritiker entsandt worden sein.
	\item \textbf{Time-Out:} Jedes Team hat dreimal im ganzen Wettbewerb die Möglichkeit, ein Time-Out (1 Minute) zu fordern. In dieser Zeit dürfen sich die Repräsentanten beider Teams  mit ihren Teammitgliedern absprechen und auch ausgewechselt werden.
	\item Am Ende des MatBojs gewinnt das Team mit der größeren Punktsumme. 
\end{itemize}
%\end{document}\newpage
	
	\section*{Notizen:}
\end{document}