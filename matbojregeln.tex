\section*{MatBoj-Regeln} 

MatBoj -- abgeleitet aus dem Russischen -- steht für \enquote{mathematischer Kampf}.

Zwei Teams lösen Aufgaben und präsentieren anschließend ihre Lösungen.

\subsection*{Phase 1: Das Lösen der Aufgaben}
Jede Mannschaft gibt sich einen Namen und wählt einen Mannschaftskapitän und einen Stellvertreter. Diese vertreten die Mannschaft als Sprecher. Nur sie können für die Mannschaft verbindliche Entscheidungen verkünden.

Beide Teams erhalten den gleichen Satz von Aufgaben. Ihnen steht eine vorher bekanntgegebene Zeit zur Verfügung, um die Aufgaben getrennt voneinander zu lösen. 

Sollte einem Teammitglied eine Aufgabe bereits bekannt sein, so ist es aus Fairnessgründen dazu aufgefordert, dies der Jury bekanntzumachen (eventuell wird die betreffende Aufgabe durch eine neue ersetzt).


\subsection*{Phase 2: Das Vorstellen der Lösungen}
\begin{itemize}
	\item Den beiden Kapitänen wird gleichzeitig eine leichte Einstiegsaufgabe gestellt, die sie ohne Hilfsmittel lösen müssen. Keines der anderen Teammitglieder darf ihnen dabei helfen. Wer die richtige Antwort gibt, gewinnt für sein Team das Recht zu entscheiden, welches Team als erstes herausfordert. Gibt einer der Kapitäne eine falsche Antwort, erhält das Team des anderen Kapitäns dieses Recht.  
	\item \textbf{Herausfordern:} Das entsprechende Team fordert vom gegnerischen Team eine Aufgabe. Das herausgeforderte Team kann die Herausforderung annehmen oder ablehnen:
	\begin{itemize}
		\item Die \textit{Herausforderung wird angenommen}: Das herausgeforderte Team entsendet ein Teammitglied als \textit{Referenten}, der eine Lösung der Aufgabe vorstellt, das herausfordernde Team entsendet einen \textit{Kritiker}, der Lücken in der Lösung zu finden versucht. Nach der Vorstellung der Lösung darf der Kritiker erst Verständnisfragen stellen und dann die vorgetragene Lösung kritisieren und die von ihm aufgedeckten Lücken füllen. Hilfe aus dem Team ist unzulässig.
		\item Die \textit{Herausforderung wird abgelehnt}: Das herausfordernde Team entsendet ein Teammitglied, das eine Lösung der Aufgabe vorstellt, das herausgeforderte  Team entsendet einen Kritiker, der Lücken in der Lösung zu finden versucht. Nach der Vorstellung der Lösung darf der Kritiker erst Verständnisfragen stellen und dann die vorgetragene Lösung kritisieren. Er darf jedoch keine von ihm aufgedeckten Lücken füllen. Hilfe aus dem Team ist unzulässig.
	\end{itemize}
	\item \textbf{Bewertung:} Jede Aufgabe ist 12 Punkte wert. Der Referent erhält eine der Punktzahlen $0, 2, 4, 6, 8, 10, 12$, je nachdem, wie richtig und vollständig die von ihm vorgetragene Lösung ist. Der Kritiker erhält für das Aufdecken der Lücken in der vorgetragenen Lösung und für das Füllen dieser Lücken jeweils die Hälfte der noch nicht vergebenen Punkte. Wie weit Referent und Kritiker ihren Aufgaben im Einzelnen gerecht wurden, liegt im Ermessen der Jury.
	\item \textbf{Invalid challenge:} Wird die Herausforderung abgelehnt und kann das herausfordernde Team keine Lösung präsentieren, liegt ein \emph{invalid challenge} vor. Die Einschätzung, ob es sich um eine Lösung handelt, liegt im Ermessen der Jury.
	
	In diesem Fall erhält das herausgeforderte Team $6$ Punkte.
	\item Die \textit{nächste Herausforderung}: Es wird abwechselnd herausgefordert. Liegt ein invalid challenge vor, muss das herausfordernde Team erneut eine Aufgabe fordern.
	\item Die Endphase des Wettbewerbs: Zu einem beliebigen Zeitpunkt kann jedes der beiden Teams beschließen, keine Lösungen mehr zu präsentieren. Das betreffende Team muss aber weiter Kritiker entsenden, da das andere Team solange weiter Lösungen vorstellen kann, wie es will. Die Kritiker dürfen in dieser Endphase des Wettbewerbs nur noch Lücken in den vorgetragenen Lösungen aufzeigen, aber nicht mehr füllen.
	\item Am Ende des Wettbewerbs muss jedes Teammitglied mindestens einmal als Referent bzw. als Kritiker entsandt worden sein.
	\item \textbf{Time-Out:} Jedes Team hat dreimal im ganzen Wettbewerb die Möglichkeit, ein Time-Out (1 Minute) zu fordern. In dieser Zeit dürfen sich die Repräsentanten beider Teams  mit ihren Teammitgliedern absprechen und auch ausgewechselt werden.
	\item Am Ende des MatBojs gewinnt das Team mit der größeren Punktsumme. 
\end{itemize}
%\end{document}