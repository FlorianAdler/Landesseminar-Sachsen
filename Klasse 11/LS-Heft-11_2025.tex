\documentclass[a4paper, 12pt]{article}

\usepackage{../landesseminarheader}
% ************************************************************************************
% Alternativ kann auch die bisherige Makro-Datei verwendet werden. Dann müssen die folgenden Zeilen entkommentiert werden.
%\usepackage{yhmath}
\usepackage{ngerman} % a4wide,, latexsym
\usepackage[T1]{fontenc}
\usepackage{lmodern}
\usepackage[utf8]{inputenc}
\usepackage{times}
\usepackage[slantedGreek]{mathptmx}
\usepackage{amsmath}
\usepackage{amssymb}
%\usepackage{amscd}
\usepackage{exscale}
\usepackage{enumerate}
\usepackage{amsthm}
\usepackage{graphics}
\usepackage{graphicx}	
\usepackage{longtable}
\usepackage{color}
\usepackage{dsfont} 
\usepackage{bbm}
%\usepackage{wasysym}
\usepackage{ifpdf}
%\usepackage{pst-all}
%\usepackage{pstricks,pstricks-add,pst-math,pst-xkey}
\usepackage{lscape}
\usepackage{eurosym, url, hyperref}
%\usepackage{fourier}


\frenchspacing

%--------------------------------------------------------------------

\setlength\parskip{\medskipamount}
\setlength\parindent{0pt}
\setlength{\voffset}{-3cm} %-1	%-1.5 %bis -3
%%\setlength{\topmargin}{0.625cm}		% oberer Rand bis Oberkante Kopfzeile
\setlength{\oddsidemargin}{0.0cm} \setlength{\evensidemargin}{0.0cm}%	Linker Rand 
%%\setlength{\headheight}{1.25cm}		% Höhe der Kopfzeile
%%\setlength{\headsep}{0.625cm}			% Abstand zw. Kopfzeile und 
\setlength{\topskip}{0cm}
%%\setlength{\footskip}{1cm}
\setlength{\textheight}{26cm} %24 %23.5; voffset ausblenden % bis 27
\setlength{\textwidth}{16cm} %16


\setcounter{secnumdepth}{3}								% Nummerierungstiefe
\setcounter{tocdepth}{1}									% Inhaltsverzeichnistiefe
%\numberwithin{equation}{section}					% Formeln abschnittsweise nummerieren

\flushbottom
\renewcommand{\baselinestretch}{1.0}



%Abkürzungen-------------------------------------------------------------
%Zahlenbereiche----------------------------------
\newcommand{\N}{{\mathbb{N}}}
\newcommand{\Z}{{\mathbb{Z}}}
\newcommand{\Q}{{\mathbb{Q}}}
\newcommand{\R}{{\mathbb{R}}}
\newcommand{\C}{{\mathbb{C}}}	
%(Komplexe) Zahlen
\newcommand{\I}{{\mathrm{i}}}					% Imaginäre Einheit
\newcommand{\real}{{\mathrm{Re}}}			% Realteil
\newcommand{\imag}{{\mathrm{Im}}}			% Imaginärteil
\newcommand{\dual}{{\mathrm{Du}}}			% Dualteil
%Abkürzung griechischer Buchstaben & Abbildungen etc.
\newcommand{\id}{\mathrm{id}}
\newcommand{\ve}{\varepsilon}
\newcommand{\vp}{\varphi}
\newcommand{\eul}{{\mathrm{e}}} 			% Eulersche Zahl e
\newcommand{\ld}{{\mathrm{ld}}} 			% duadischer Logarithmus
%Matrizen und Vektorrechnung---------------------
\newcommand{\T}{^{\mathrm{T}}}				% Transponiertzeichen
\newcommand{\rg}{{\mathrm{rg}}}				% Rang
\newcommand{\bild}{{\mathrm{bild}}}		% Bild
\newcommand{\Kern}{{\mathrm{kern}}}		% Kern
\newcommand{\lin}{{\mathrm{lin}}}			% Kern
\newcommand{\ul}[1]{\underline{#1}} 	% Vektorunterstrich
%Algebra----------------------
\newcommand{\ggT}{{\mathrm{ggT}}}				% ggT
\newcommand{\kgV}{{\mathrm{kgV}}}				% kgV
%Geometrie
\newcommand{\ol}[1]{\overline{#1}} 	  % Strecke
\newcommand{\TV}{{\mathrm{TV}}}				% Teilverhältnis
\newcommand{\DV}{{\mathrm{DV}}}				% Doppelverhältnis
\newcommand{\mbb}[1]{\mathbb{#1}}			% \mathbb	
%Einheiten
\newcommand{\mm}{{\mbox{\,} \mathrm{mm}}}
\newcommand{\cm}{{\mbox{\,} \mathrm{cm}}}				% Einheit cm
\newcommand{\dm}{{\mbox{\,} \mathrm{dm}}}	
\newcommand{\m}{{\mbox{\,} \mathrm{m}}}	
\newcommand{\LE}{{\mbox{\,} \mathrm{LE}}}	
\newcommand{\fe}{{\mbox{\,} \mathrm{FE}}}	
\newcommand{\km}{{\mbox{\,} \mathrm{km}}}
\newcommand{\gpkcm}{{\mbox{\,} \mathrm{g}/\mathrm{cm}^3}}
\newcommand{\kmh}{{\mbox{\,} \frac{\mathrm{km}}{\mathrm{h}}}}
\newcommand{\komma}{{,}}
\newcommand{\entspricht}{\mathrel{\widehat{=}}}   		% Entspricht

\newcommand{\h}{{\mbox{\,} \mathrm{h}}}
\newcommand{\s}{{\mbox{\,} \mathrm{s}}}
\newcommand{\g}{{\mbox{\,} \mathrm{g}}}
\newcommand{\kg}{{\mbox{\,} \mathrm{kg}}}	
\newcommand{\gc}{{\grad\mathrm{C}}}	
\newcommand{\grad}{^{\circ}}				% Einheit °
\newcommand{\mbm}[1]{\mathbbm{#1}}	
\newcommand{\arc}{{\mathrm{arc}}}	
%Schüler
\newcommand{\ds}{\displaystyle}
% shortcuts
\def\ol#1{\overline{#1}}
\def\ul#1{\underline{#1}}
\def\br#1{\left(#1\right)}             % brackets
\def\sbr#1{\left[#1\right]}            % square brackets
\def\cbr#1{\left\{#1\right\}}          % curly brackets
\def\iff{\Leftrightarrow\ }
\def\yields{\Rightarrow\ }
\newcommand{\rf}[1][]{\textup{\eqref{#1}}}
\newcommand{\half}{\frac{1}{2}}
\newcommand{\third}{\frac{1}{3}}
\newcommand{\ov}{\overline}
\newcommand{\nn}{\nonumber}
\newcommand{\RRA}{{\,\,\Longrightarrow}\,\,}
\newcommand{\LRA}{{\,\,\Leftrightarrow}\,\,}
%\newcommand{\qed}[1][\rule{1ex}{1ex}]{\nopagebreak\hspace*{2em}\hspace*{\fill}{$#1$}}


%Analysis --------------------
\newcommand{\inte}[4]{\int\limits_{#1}^{#2} {#3}\mathrm{d}{#4} } 

%---------------------------------------------------------------------------------------------
\newenvironment{description*}[2]
   {\begin{list}{}{%
      \settowidth{\labelwidth}{#2{#1}}
      \setlength{\leftmargin}{\labelwidth}
         \addtolength{\leftmargin}{\labelsep}
      \setlength{\parsep}{0.5ex plus0.2ex minus0.2ex}
      \setlength{\itemsep}{0.3ex}
      \renewcommand{\makelabel}[1]{#2{##1}\hfill}}}
   {\end{list}}
%*********************************************************************************************

\newcommand{\bsp}[1]{\begin{Bsp}\hspace*{1cm}\newline#1\end{Bsp}}
\newcommand{\kartauf}[3]{
\newpage
\normalsize
\begin{tabular}{p{13cm}}
\textbf{#1 \hfill #2}\\\hline
\end{tabular}
\begin{enumerate}\small
#3
\end{enumerate}}

\newcommand{\bem}[1]{\begin{Bem}\normalfont \hspace*{1cm}\newline#1\end{Bem}}

\newcommand{\defi}[2]{\begin{Def}[#1]\hspace*{1cm}\newline#2\end{Def}}

\newcommand{\theo}[2]{\begin{Theo}[#1]\hspace*{1cm} #2\end{Theo}} %\newline

\newcommand{\satz}[2]{\begin{Satz}[#1]\hspace*{1cm}\newline#2\end{Satz}}

\newcommand{\folg}[2]{\begin{Folg}[#1]\hspace*{1cm}\newline#2\end{Folg}}

\newcommand{\auf}[1]{\begin{Auf} \normalfont#1\end{Auf}} % \hspace*{1cm}\newline

\newcommand{\lem}[1]{\begin{Lem}\hspace*{1cm}\newline#1\end{Lem}}

\newcommand{\cor}[1]{\begin{Cor}\hspace*{1cm}\newline#1\end{Cor}}

\newcommand{\tipp}[2]{\begin{Tipp}[#1]\hspace*{1cm}\newline#2\end{Tipp}}

\newcommand{\bew}[1]{\textsl{Beweis: }#1 \hfill $\Box$}  %ge{\"a}ndert (\!)

\newcommand{\loes}[1]{\textit{Lösung: }#1 \hfill $\Box$}  %ge{\"a}ndert (\!)

\newcommand{\lema}[2]{\begin{Lem}[#1]\hspace*{1cm}\newline#2\end{Lem}}

\newtheoremstyle{definition}% name
     {3pt}%      Space above
     {5pt}%      Space below
     {\itshape}%         Body font % evtl 
     {0ex}%         Indent amount (empty = no indent, \parindent = para indent)
     {\bfseries}% Thm head font
     {:}%        Punctuation after thm head
     {.5em}%     Space after thm head: " " = normal interword space;
           %       \newline = linebreak
     {}%         Thm head spec (can be left empty, meaning `normal')

\newtheoremstyle{Beweis}% name
     {30pt}%      Space above
     {3pt}%      Space below
     {}%         Body font
     {}%         Indent amount (empty = no indent, \parindent = para indent)
     {\itshape}% Thm head font
     {:}%        Punctuation after thm head
     {.5em}%     Space after thm head: " " = normal interword space;
           %       \newline = linebreak
     {}%         Thm head spec (can be left empty, meaning `normal')


\newtheoremstyle{break}% name
     {3pt}%      Space above
     {7pt}%      Space below
     {\itshape}%         Body font
     {}%         Indent amount (empty = no indent, \parindent = para indent)
     {\bfseries}% Thm head font
     {:}%        Punctuation after thm head
     {.5em}%     Space after thm head: " " = normal interword space;
           %       \newline = linebreak
     {}%         Thm head spec (can be left empty, meaning `normal')

\theoremstyle{definition}

\newtheorem{Def}{Definition}[section]


\theoremstyle{break}
\newtheorem{Satz}[Def]{Satz}
\newtheorem{Lem}[Def]{Lemma}
\newtheorem{Cor}[Def]{Korollar}
\newtheorem{Tipp}[Def]{Tipp}

\theoremstyle{definition}
\newtheorem{Theo}[Def]{Theorem}
\newtheorem{Auf}[Def]{Aufgabe}
\newtheorem{Bem}[Def]{Bemerkung}
\newtheorem{Bsp}[Def]{Beispiel}
\newtheorem{beispiel}[Def]{Beispiel}
\newtheorem{aufgabe}[Def]{Aufgabe}
\newtheorem{Folg}[Def]{Folgerung}
\newtheorem{obs}[Def]{Beobachtung}

\theoremstyle{Beweis}
\newtheorem{Bew}{Beweis}



\makeindex

\endinput

%\usepackage{mathtools} % für DeclarePairedDelimiter
%\usepackage{wasysym} % für das Winkel-Symbol
%\usepackage{tikz} % für die Skizzen
%\usetikzlibrary{positioning,calc,arrows.meta,shapes,decorations.pathmorphing,decorations.markings,hobby}
%\tikzset{every picture/.style={line width=0.6, line cap=round,line join=round}}
%\usepackage{csquotes} % für bessere Anführungszeichen
%\usepackage[shortlabels]{enumitem} % um Aufzählungen automatisch in der Form (a), (b), ... zu setzen
%\setlist[enumerate]{label={$(\alph*)$}, ref={$(\alph*)$},topsep=0pt}
%\setlist[itemize]{topsep=0pt,itemsep=0pt,parsep=\parskip}
%\usepackage{booktabs} % schönere Tabellen
%\usepackage{tabularx} % Ich benutze tabularx, um mehrere Bilder so nebeneinander anzuordnen, dass die Abstände alle gleich groß sind
%\usepackage{wrapfig} % für Bilder, die von Text umflossen werden sollen
%\usepackage{microtype} % typographische Mikrooptimierungen (verhindert overfull hboxes)
%\usepackage{xurl} % damit URLs automatisch umgebrochen werden (verhindert over/underfull hboxes).
% ************************************************************************************

% ************************* Kosmetik fürs Inhaltsverzeichnis *************************
% Ich finde es optisch ansprechend und inhaltlich sinnvoll, wenn die Beiträge im Inhaltsverzeichnis nach Theme sortiert auftauchen. Dazu wird das Erscheinungsbild des Inhaltsverzeichnisses ein wenig umgebaut. Wird das nicht gewünscht, dann können die folgenden Zeilen sowie alle \cftaddtitleline-Befehle im Dokument entfernt werden.
\usepackage{tocloft}

% passe Abstände und Erscheinungsbild für Sections an
\cftsetindents{section}{1.5em}{2.3em} % Einrückung
\setlength{\cftbeforesecskip}{0.25em} % Zeilenabstand
\renewcommand{\cftsecleader}{\cftdotfill{\cftdotsep}} % Punkte zwischen Section-Titel und Seitenzahl
\renewcommand{\cftsecfont}{} % Section-Titel nicht fett
\renewcommand{\cftsecpagefont}{} % Seitenzahl nicht fett

% passe Abstände und Erscheinungsbild für Parts an
\setlength{\cftbeforepartskip}{1.0em} % Zeilenabstand
\renewcommand{\cftpartfont}{\bfseries} % Part-Titel nur fett, aber nicht größer
\renewcommand{\cftpartpagefont}{\bfseries} % Seitenzahl nur fett, aber nicht größer
% ************************************************************************************


% **************************** Eine praktische Umgebungen ****************************
% Die amsthm-Umgebungen erlauben es nicht, Bilder in Form von Floats einzufügen. Deswegen wird hier die proof-Umgebungen neu definiert. Außerdem definieren wir neue Umgebungen für Aufgaben, Definitionen und Sätze mit Namen.

% Die proof-Umgebung wird neu definiert
\RenewDocumentEnvironment{proof}{ O{\proofname} }{
	\par\pushQED{\qed}
	\noindent\textbf{#1.}\ \ignorespaces
}{%
	\popQED\par
}

% Aufgaben-Umgebung. Das optionale Argument wird meistens benutzt, um schwere Aufgaben durch Asteriske zu kennzeichnen.
\newcounter{caufgabe}[section]
\NewDocumentEnvironment{aufgabe*}{ O{} }{
	\par\refstepcounter{caufgabe}
	\noindent\textbf{Aufgabe~\thecaufgabe#1.}\ \ignorespaces
}{
	\par
}

% Eine Umgebung für benannte Sätze. Der Name kommt in das optionale Argument. Zum Beispiel liefert "\begin{satzmitnamen}[Satz von Euler-Fermat] ..." im Text "Satz von Euler-Fermat. ..."
\NewDocumentEnvironment{satzmitnamen}{ O{Satz} }{
	\par\begingroup
	\noindent\textbf{#1.}\ \ignorespaces\itshape
}{
	\endgroup\par
}

% Eine Umgebung für Definitionen
\NewDocumentEnvironment{definition}{ O{Definition}}{
	\par
	\noindent\textbf{#1.}\ \ignorespaces
}{
	\par
}
% ************************************************************************************

% ****************************** Eine praktische Makros ******************************
% bessere Klammern in kursiven Umgebungen
\newcommand{\embrace}[1]{\textup{(}#1\textup{)}}

% Das \varangle-Symbol wird in \itshape-Umgebungen kursiv dargestellt; dieses Kommando behebt das Problem.
\newcommand{\winkel}{\textup{\varangle}}

% Ein Verkehrszeichen
\DeclareRobustCommand{\Warnung}{\smash{\tikz[baseline, anchor=center]\node[draw, regular polygon, regular polygon sides=3, rounded corners=2, thick, inner sep=-0.25pt] at (0,0) {\textbf{!}};}}

% Gepaarte Klammern
\DeclarePairedDelimiter{\parens}{\lparen}{\rparen}
\DeclarePairedDelimiter{\braces}{\lbrace}{\rbrace}
\DeclarePairedDelimiter{\brackets}{[}{]}
\DeclarePairedDelimiter{\abs}{\lvert}{\rvert}
% ************************************************************************************


% Wenn $\boldsymbol{...}$ in einem Titel verwendet wird, kommt es manchmal zu einer Warnung (aber alles sieht ok aus). Dieser Befehl unterdrückt die Warnung.
\SetSymbolFont{wasy}{bold}{U}{wasy}{m}{n}
\begin{document}
	\begin{titlepage}
		\centering\sffamily\Huge\bfseries
		\vspace*{0.75em}
		
		Sächsisches Landesseminar \\
		Mathematik 2025 \\
		\textmd{in Sayda} \\
		
		\vspace{1.75em}
		
		\textmd{\large XX.\,XX. -- XX.\,XX.\,2025} \\
		
		\vspace{6em}
		
		Begleitmaterial zum Seminarprogramm\\[.6\baselineskip]
		der Klassenstufe 11\\
		
		\vfill
		
		\raggedright\normalfont\normalsize
		Herausgegeben im Auftrag des Sächsischen Landeskomitees zur Förderung \\
		mathematisch-naturwissenschaftlich begabter und interessierter Schüler
	\end{titlepage}
	\setcounter{page}{2}
	
	\section*{Vorwort}
	
	Liebe Schülerinnen,\\
	liebe Schüler,
	
	ich freue mich, Euch im Sächsischen Landesseminar Mathematik begrüßen zu können.
	
	In den nächsten drei Tagen werdet Ihr zehn mathematische Seminare haben. Ich hoffe, dass Ihr während dieser Seminare nicht nur neue Lösungsmethoden oder mathematische Sachverhalte kennen lernt, sondern dass Ihr dabei auch Freude an der Mathematik und am Lösen von Aufgaben haben werdet.
	
	Am Donnerstag wird dann die Auswahlklausur geschrieben, für die ich Euch jetzt schon alles Gute und viel Erfolg wünsche. Während Ihr am vorbereiteten Freizeitprogramm teilnehmt, werden die Klausuren korrigiert. Am Freitag wird dann die Mannschaft, die Sachsen auf der Bundesrunde der Mathematik-Olympiade vertreten darf, feierlich bekannt gegeben.
	
	Ich hoffe, dass Ihr es in den nächsten Tagen neben der Mathematik auch genießen könnt, Euch mit Gleichgesinnten zu unterhalten, Euch auszutauschen und um Lösungsansätze gemeinsam zu ringen. Dazu soll insbesondere auch der MatBoj beitragen. Aber natürlich denke ich dabei auch an die vielen Spiele, die eine lange Tradition im Landesseminar haben.
	
	Ich wünsche Euch viel Erfolg und eine gute Woche.
	
	Joachim Lippert
	
	\section*{Über dieses Heft}
	Dieses Heft behandelt einige der wichtigsten Themen für die Mathematik-Olympiade in der Klassenstufe~11. Einige dieser Themen werden auch in den Seminaren besprochen, einige werdet ihr bestimmt schon kennen und andere werden euch neu sein. Es wird empfohlen, dass ihr das Heft zur Vorbereitung auf die Bundesrunde oder die nächste Olympiade durcharbeitet.
	
	In diesem Heft gibt es zwei Typen von Aufgaben: \emph{Beispielaufgaben} und \emph{Übungsaufgaben}. An Beispielaufgaben lassen sich die vorgestellten Methoden besonders gut vorführen. Manchmal lösen wir Beispielaufgaben direkt auf, aber meistens findet ihr am Ende des jeweiligen Kapitels Tipps und erst ganz am Ende des Heftes die Lösungen für die Beispielaufgaben. Die Beispielaufgaben solltet ihr zuerst bearbeiten, wenn ihr euch mit einer neuen Methode vertraut machen wollt. Bei den Übungsaufgaben hingegen seid ihr auf euch allein gestellt. Sie dienen zur weiteren Vertiefung der Inhalte.
	
	Schwere Aufgaben sind mit einem (*) bis drei (***) Sternen gekennzeichnet. Ein Stern bedeutet dabei, dass die Aufgabe schwerer als die durchschnittliche Bundesrunden-Aufgabe ist. Besonders bei solchen Aufgaben gilt: Wenn ihr nicht weiterkommt, dann holt euch einen Tipp und wenn ihr dann immer noch feststeckt, dann lest euch auch gern die Lösung durch -- dafür sind die Tipps und die Lösungen schließlich da.
	
	
	\vfill
	
	\scriptsize
	
	\emph{Texte:} Ferdinand Wagner. Mit tatkräftiger Unterstützung von Sebastian Bürger, Leo Gitin, Cara Hobohm, Tien Nguyen und Arne Wolf. Aufbauend auf früheren Texten von Ingolf Busch, Frank Göhring, Maximilian Keitel, Eric Müller, Jens Reinhold und Lisa Sauermann. Herzlichen Dank auch an alle weiteren, die in früheren Ausgaben dieses Begleitheftes Beiträge erstellt haben.
	
	\emph{Textsatz:} Joachim Lippert, Tien Nguyen, Ferdinand Wagner.
	
	\emph{Redaktion}: Joachim Lippert (\href{mailto:lippert@landesseminar-sachsen.de}{\texttt{lippert@landesseminar-sachsen.de}}).
	\normalsize
	
	\tableofcontents
	
	\newpage
	\phantomsection\cftaddtitleline{toc}{part}{Gleichungen und Ungleichungen}{\thepage}
	\section{Die Jensensche Ungleichung für nicht-konvexe Funktionen}\label{kapitel:LCF}
Wenn euch in der Oberstufe eine Ungleichung der Form $f(x_1)+f(x_2)+\dotsb+f(x_n)\geqslant c$ begegnet, dann ist $f$ fast nie konvex, sodass ihr die Jensensche Ungleichung nicht anwenden könnt -- das wäre ja auch zu einfach. In diesem Kapitel lernt ihr einige Tricks, mit denen ihr bei solchen Aufgaben trotzdem mit der Jensenschen Ungleichung argumentieren könnt.

Am Ende des Kapitels findet ihr Tipps und am Ende des Heftes findet ihr Musterlösungen zu den Beispielaufgaben.

\subsection*{Die Karamata-Schiebemethode}
Angenommen, wir haben es mit einer Ungleichung der Form $f(x_1)+f(x_2)+\dotsb+f(x_n)\geqslant c$ mit der Nebenbedingung $x_1+x_2+\dotsb+x_n=b$ zu tun, wobei $f$ nicht konvex ist. Stattdessen nehmen wir an, dass $f$ konvex für $x\geqslant a$ und konkav für $x\leqslant a$ ist. Analoge Argumente würden in dem Fall funktionieren, dass $f$ konvex für $x\leqslant a$ und konkav für $x\geqslant a$ ist. Wenn ihr dieses Unterkapitel fertig gelesen habt, wird euch auch klar sein, wie ihr die Argumente auf noch allgemeinere Situationen ausdehnen könnt.

\begin{wrapfigure}{r}{0.43\textwidth}
	\centering\vspace{-0.6cm}
	\begin{tikzpicture}[x=1cm,y=1cm]
		\draw[->] (-0.5,0) to node[pos=1,below=0.5ex] {$x$} (6,0);
		\draw[->] (0,-0.5) to  node[pos=1,right=0.5ex] {$y$} (0,2.5);
		\node[below left] at (0,0) {$0$};
		\coordinate (x1) at (1.136,1.828);
		\coordinate (x2) at (2.287,1.812);
		\coordinate (x3) at (3.473,0.934);
		\coordinate (x4) at (5.305,1.097);
		\coordinate (x5) at (5.697,1.79);
		\draw plot[domain=0.2:5.8,hobby] (\x,{0.8*((0.56*\x-1.7)*(0.56*\x-1.7)*(0.56*\x-1.7)-\x)+3.7});
		\draw (3.036,0) ++ (0,0.5ex) to ++(0ex,-1ex) node[below] {$a$};
		\draw [line width=0.3,dashed,dash phase=0.5] (3.036,1.125ex) to (3.036,2.4);
		\draw[fill=black] (x1) circle (2pt) node[left=0.25ex] {$\leftarrow$};
		\draw[fill=black] (x2) circle (2pt) node[right=0.25ex] {$\rightarrow$};
		\draw[fill=black] (x3) circle (2pt) node[right=0.25ex] {$\rightarrow$};
		\draw[fill=black] (x4) circle (2pt) node[left=0.25ex] {$\leftarrow$};
		\draw[fill=black] (x5) circle (2pt) node[left=0.25ex] {$\leftarrow$};
		\node[below,align=center] at (1.518,-0.5) {\glqq Abstoßung\grqq\\für $x<a$};
		\node[below,align=center] at (4.518,-0.5) {\glqq Anziehung\grqq\\für $x\geqslant a$};
	\end{tikzpicture}\vspace{-0.5cm}
\end{wrapfigure}
Wir beginnen mit einer beliebigen Wahl der Variablen $x_1,x_2,\dotsc,x_n$ mit $x_1+x_2+\dotsb+x_n=b$. Dann verschieben wir die Variablen schrittweise, sodass $f(x_1)+f(x_2)+\dotsb+f(x_n)$ in jedem Verschiebungsschritt kleiner wird. Zuerst können wir alle Variablen $x_i\geqslant a$ durch ihren Mittelwert ersetzen, denn für $x\geqslant a$ ist $f$ konvex und die Jensensche Ungleichung gilt. Angenommen, es gibt mindestens zwei Variablen $x_j,x_k<a$. Dann schieben wir $x_j$ und $x_k$ auseinander. Das bedeutet: Wenn $x_j\leqslant x_k$ ist, dann ersetzen wir $x_j$ durch $x_j-\delta$ und $x_k$ durch $x_k+\delta$ für passendes $\delta>0$. Aus der Ungleichung von Karamata folgt
\begin{equation*}
	f(x_j-\delta)+f(x_k+\delta)\geqslant f(x_j)+f(x_k)\,,
\end{equation*}
denn $(x_k+\delta,x_j-\delta)$ majorisiert offensichtlich $(x_k,x_j)$. Wir können $x_j$ und $x_k$ so lange auseinanderschieben, bis eine von beiden Variablen an einer Grenze \glqq anstößt\grqq. Denn sobald $x_k=a$ gilt, also sobald $x_k$ den Konkavitätsbereich verlassen hat, können wir Karamata nicht mehr anwenden. Andererseits kann es passieren, dass unsere Funktion $f$ nur für $x\geqslant a_0$ definiert ist, und sobald $x_j=a_0$ erreicht ist, können wir nicht mehr weiter schieben. Wenn $x_k=a$ erreicht ist, liegt $x_k$ im Konvexitätsbereich. Wir können also wieder Jensen anwenden, um $x_k$ sowie alle anderen Variablen im Konvexitätsbereich durch ihren Mittelwert zu ersetzen.

Wir wiederholen dieses Verfahren so lange, bis es nicht mehr geht. Dann sind wir in einem Spezialfall von der folgenden Form angekommen:
\begin{itemize}
	\item Einige Variablen sind gleich $a_0$, sagen wir, $x_1=x_2=\dotsb=x_i=a_0$ (möglicherweise ist $i=0$, zum Beispiel dann, wenn $a_0$ gar nicht existiert).
	\item Im Konkavitätsbereich liegt höchstens eine Variable $a_0<x_{i+1}<a$.
	\item Alle anderen Variablen liegen im Konvexitätsbereich und sind gleich. Es gilt folglich $x_{i+2}=x_{i+3}=\dotsb=x_n=x$ für ein $x\geqslant a$ bzw.\ $x_{i+1}=x_{i+2}=\dotsb=x_n=x$, falls keine von $a_0$ verschiedene Variable im Konkavitätsbereich liegt.
\end{itemize}
Falls im Konkavitätsbereich keine von $a_0$ verschiedene Variable liegt, folgt aus der Nebenbedingung $x_1+x_2+\dotsb+x_n=b$, dass $x=(b-ia_0)/(n-i)$ gilt, und wir können die Ungleichung $f(x_1)+f(x_2)+\dotsb+f(x_n)\geqslant c$ durch einfaches Einsetzen überprüfen. Falls ein $x_{i+1}$ mit $a_0<x_{i+1}<a$ existiert, können wir aus der Nebenbedingung $x_{i+1}=b-(n-i-1)x-ia_0$ folgern. Damit haben wir die Ungleichung also immerhin auf eine Variable $x$ reduziert.

Und Ungleichungen in einer Variablen lassen sich meistens durch \emph{brute force} lösen: Ihr multipliziert alles aus und erhaltet (meistens) ein Polynom in einer Variablen $x$. Dann versucht ihr, die Gleichheitsfälle der Ungleichung zu erraten. Die Gleichheitsfälle müssen offensichtlich Nullstellen des Polynoms sein. Tatsächlich müssen sie sogar Doppelnullstellen sein, ansonsten würde das Polynom an dieser Stelle sein Vorzeichen ändern und die Ungleichung wäre falsch. Ihr könnt also die Gleichheitsfälle quadratisch ausklammern. Für den übrigbleibenden Faktor gibt es üblicherweise ein einfaches Argument, warum er stets positiv sein muss.

\begin{aufgabe*}\label{aufgabe:KaramataSchieben}
	Gegeben seien positive reelle Zahlen $a,b,c,d,e>0$ mit $a+b+c+d+e=5$. Beweise, dass
	\begin{equation*}
		\frac1{a^2}+\frac1{b^2}+\frac1{c^2}+\frac1{d^2}+\frac1{e^2}+9\geqslant \frac{14}5\parens*{\frac1a+\frac1b+\frac1c+\frac1d+\frac1e}\,.
	\end{equation*}
\end{aufgabe*}
\begin{aufgabe*}[**]\label{aufgabe:log3log2}
	Ermittle die kleinste reelle Zahl $\kappa$, sodass für alle positiven reellen Zahlen $a,b,c>0$ die folgende Ungleichung gilt:
	\begin{equation*}
		\parens*{\frac{2a}{b+c}}^\kappa+\parens*{\frac{2b}{c+a}}^\kappa+\parens*{\frac{2c}{a+b}}^\kappa\geqslant 3\,.
	\end{equation*}
\end{aufgabe*}

\subsection*{Die Jensen-Tangenten-Methode}
Wenn $f$ eine differenzierbare konvexe Funktion ist, dann verläuft jede Tangente an den Funktionsgraphen von $f$ stets unterhalb des Graphen. 

\begin{wrapfigure}{r}{0.21\textwidth}
	\centering\vspace{-0.5cm}
	\begin{tikzpicture}[x=1cm,y=1cm]
		\draw[->] (-0.5,0) to node[pos=1,below=0.5ex] {$x$} (2.5,0);
		\draw[->] (0,-0.5) to  node[pos=1,right=0.5ex] {$y$} (0,2.3);
		\node[below left] at (0,0) {$0$};
		\draw plot[domain=3.7:5.8,hobby] (\x-3.5,{0.8*((0.56*\x-1.7)*(0.56*\x-1.7)*(0.56*\x-1.7)-\x)+3.5});
		\draw[dashed,line width=0.3] (2.3,1.052) to (0.2,-0.26);
		\draw [fill=white] (1.374,0.474) circle (2pt);
	\end{tikzpicture}
	Tangente an eine konvexe Funktion\vspace{-2cm}
\end{wrapfigure}
Betrachte nun $x_1,x_2,\dotsc,x_n$ mit Mittelwert $\overline{x}\coloneqq\frac{x_1+x_2+\dotsb+x_n}{n}$. Wenn die Tangente an den Graphen von $f$ in $\overline{x}$ die Gleichung $y=ax+b$ hat, dann ist also $f(x)\geqslant ax+b$ für alle $x$, mit Gleichheit für $x=\overline{x}$. Nun folgt
\begin{align*}
	\frac{f(x_1)+f(x_2)+\dotsb+f(x_n)}{n}&\geqslant \frac{(ax_1+b)+(ax_2+b)+\dotsb+(ax_n+b)}{n}\\
	&=a\parens*{\frac{x_1+x_2+\dotsb+x_n}{n}}+b\\
	&=f\parens*{\frac{x_1+x_2+\dotsb+x_n}{n}}
\end{align*}
und wir haben die Jensensche Ungleichung bewiesen. Auf die gleiche Weise lässt sich die gewichtete Jensen-Ungleichung beweisen. Insbesondere folgt:
\begin{itemize}\itshape
	\item[$(*)$] Um die Jensensche Ungleichung für eine differenzierbare Funktion zu verwenden, benötigen wir nur, dass die Tangente an den Funktionsgraphen im Mittelwert der Variablen stets unterhalb des Funktionsgraphen verläuft.
\end{itemize}
Diese Bedingung ist häufig nicht nur im Konvexitätsbereich von $f$ erfüllt, sondern noch ein gutes Stück darüber hinaus (im besten Fall überall). Außerhalb dieses Bereiches sind die Werte von $f$ dann hoffentlich so klein, dass die Ungleichung, die ihr zu beweisen versucht, aus trivialen Gründen gilt.
\begin{aufgabe*}\label{aufgabe:51}
	Gegeben seien nichtnegative reelle Zahlen $x,y,z\geqslant 0$ mit $x+y+z=9$. Beweise, dass
	\begin{equation*}
		\frac{1}{51+x^2}+\frac{1}{51+y^2}+\frac{1}{51+z^2}\leqslant \frac1{20}\,.
	\end{equation*}
\end{aufgabe*}
\begin{aufgabe*}\label{aufgabe:b+c-aUngleichung}
	Gegeben seien positive reelle Zahlen $a,b,c>0$. Beweise die Ungleichung
	\begin{equation*}
		\frac{(b+c-a)^2}{(b+c)^2+a^2}+\frac{(c+a-b)^2}{(c+a)^2+b^2}+\frac{(a+b-c)^2}{(a+b)^2+c^2}\geqslant \frac35\,.
	\end{equation*}
\end{aufgabe*}
\begin{aufgabe*}[*]\label{aufgabe:USAMO2017}
	Gegeben seien nichtnegative reelle Zahlen $a,b,c,d\geqslant 0$ mit $a+b+c+d=4$. Finde den minimalen Wert von
	\begin{equation*}
		\frac{a}{b^3+4}+\frac{b}{c^3+4}+\frac{c}{d^3+4}+\frac{d}{a^3+4}\,.
	\end{equation*}
\end{aufgabe*}

\subsection*{Linearisieren}
Im vorherigen Unterkapitel haben wir gesehen, dass es sehr praktisch sein kann, eine Funktion gegen ihre Tangente abzuschätzen. Allgemein ist es häufig eine gute Strategie, komplizierte Terme gegen lineare Terme abzuschätzen (achtet dabei darauf, dass die Gleichheitsfälle nicht verloren gehen). Wir werden diese Strategie an einem Beispiel demonstrieren und euch dann zwei weitere Beispiele als Beispielaufgaben stellen.
\begin{aufgabe*}\label{aufgabe:AIMO2014}
	Gegeben seien positive reelle Zahlen $a,b,c,d>0$ mit $abcd=1$. Beweise die Ungleichung
	\begin{equation*}
		\frac{a^2}{a^3+1}+\frac{b^2}{b^3+1}+\frac{c^2}{c^3+1}+\frac{d^2}{d^3+1}\leqslant 2\,.
	\end{equation*}
\end{aufgabe*}
\begin{proof}
	Wir versuchen, alle quadratischen und kubischen Terme durch lineare Terme zu ersetzen. Dazu substituieren wir zuerst $x=a^2$, $y=b^2$, $z=c^2$, und $w=d^2$. Dann gilt immer noch $xyzw=1$ und wir sind die quadratischen Terme im Zähler losgeworden. Dafür stehen im Nenner jetzt Terme der Form $x^{3/2}+1$. Um diese Terme zu linearisieren, benutzen wir gewichtetes AM-GM; dabei müssen wir aufpassen, dass bei unserer Abschätzung im Fall $x=1$ Gleichheit eintritt, denn in der ursprünglichen Ungleichung gilt ja auch Gleichheit für $a=b=c=d=1$. Nach etwas Rumprobieren kommen wir auf die Abschätzung
	\begin{equation*}
		x^{3/2}+1=\frac{x^{3/2}}2+\frac{x^{3/2}}2+\frac12+\frac12\geqslant 3\sqrt[3]{\frac{x^{3/2}}2\cdot \frac{x^{3/2}}2\cdot\frac12}+\frac12=\frac{3x+1}{2}\,.
	\end{equation*}
	Wir erhalten also
	\begin{equation*}
		\frac{x}{x^{3/2}+1}\leqslant \frac{2x}{3x+1}=\frac23-\frac23\cdot \frac{1}{3x+1}\,.
	\end{equation*}
	Folglich müssen wir nur die Ungleichung
	\begin{equation*}
		\frac1{3x+1}+\frac1{3y+1}+\frac1{3z+1}+\frac1{3w+1}\geqslant 1
	\end{equation*}
	beweisen. Durch Ausmultiplizieren sehen wir, dass diese Ungleichung zu
	\begin{equation*}
		27\sum xyz+18\sum xy+9\sum x+4\geqslant 81xyzw+27\sum xyz+9\sum xy+3\sum x+1
	\end{equation*}
	äquivalent ist. Durch $xyzw=1$ und Vereinfachen erhalten wir $9\sum xy+6\sum x\geqslant 78$. Aus AM-GM folgt $9\sum xy\geqslant 9\cdot 6\sqrt[6]{(xyzw)^3}=54$ sowie $6\sum x\geqslant 6\cdot 4\sqrt[4]{xyzw}=24$ und die gewünschte Ungleichung folgt sofort.
\end{proof}

\begin{aufgabe*}\label{aufgabe:MatBoj2015}
	Gegeben seien positive reelle Zahlen $a,b,c>0$ mit $abc=1$. Beweise die Ungleichung
	\begin{equation*}
		\frac{a}{a^2+2}+\frac{b}{b^2+2}+\frac{c}{c^2+2}\leqslant 1\,.
	\end{equation*}
\end{aufgabe*}

\begin{aufgabe*}\label{aufgabe:DEMO2013}
	Gegeben seien eine reelle Zahl $\alpha>1$. Betrachte die Zahlenfolge $(a_n)_{n\geqslant 1}$, die durch
	\begin{equation*}
		a_n=\sqrt[\alpha]{1+\sqrt[\alpha]{2+\sqrt[\alpha]{3+\sqrt[\alpha]{\dotsb+\sqrt[\alpha]{n+\sqrt[\alpha]{n+1}}}}}}
	\end{equation*}
	gegeben ist. Zeige, dass die Folge $(a_n)_{n\geqslant 1}$ \emph{beschränkt} ist, das heißt, dass es eine Konstante $C$ gibt, sodass $a_n<C$ für alle $n$ gilt.
\end{aufgabe*}

\vfill\hrule\vspace{-1em}

\subsection*{Tipps zu den Beispielaufgaben}
\textbf{Tipp zu Aufgabe~\ref{aufgabe:KaramataSchieben}.} Benutze die Karamata-Schiebemethode.%Die Ungleichung hat außer dem offensichtlichen Gleichheitsfall $a=b=c=d=e=1$ noch einen weiteren Gleichheitsfall. Kannst du ihn erraten?

\textbf{Tipps zu Aufgabe~\ref{aufgabe:log3log2}.} Offensichtlich ist $a=b=c$ ein Gleichheitsfall. Es ist naheliegend, dass es für das minimale $\kappa$ noch einen weiteren, \glqq asymptotischen\grqq\ Gleichheitsfall geben sollte. Kannst du diesen Gleichheitsfall erraten und damit $\kappa$ bestimmen?

Um die Ungleichung für dieses $\kappa$ nachzuweisen, benutze die Karamata-Schiebemethode, um sie auf eine Ungleichung in einer Variable zurückzuführen. Dann benutze Differentialrechnung, um diese Ungleichung zu beweisen (die übliche Methode wird nicht funktionieren, weil wir kein Polynom bekommen). Das wird leider etwas hässlich.

\textbf{Tipp zu Aufgabe~\ref{aufgabe:51}.} Benutze die Jensen-Tangentenmethode. Dort, wo die Abschätzung versagt, ist die Funktion $f(x)=\frac1{51+x^2}$ sehr klein.

\textbf{Tipp zu Aufgabe~\ref{aufgabe:b+c-aUngleichung}.} Die linke Seite ist homogen in $a$, $b$ und $c$, also darf $a+b+c=1$ angenommen werden.

\textbf{Tipps zu Aufgabe~\ref{aufgabe:USAMO2017}.} Errate, an welcher Stelle das Minimum angenommen wird (Spoiler: Es ist \emph{nicht} $a=b=c=d=1$).

Fasse einen Teil der Terme als Gewichte und einen Teil als Funktion auf.

\textbf{Tipp zu Aufgabe~\ref{aufgabe:MatBoj2015}.} Lass dich von der Lösung zu Aufgabe~\ref{aufgabe:AIMO2014} inspirieren.

\textbf{Tipp zu Aufgabe~\ref{aufgabe:DEMO2013}.} Schätze $\sqrt[\alpha]{x}$ gegen eine geeignete lineare Funktion ab und benutze Konvergenz von geometrischen Reihen.\newpage
	\section{Die \texorpdfstring{$\boldsymbol{uvw}$}{uvw}-Methode}\label{kapitel:uvw}
Die $uvw$-Methode ist eine Methode, um Ungleichungen in drei Variablen auf den Spezialfall zu reduzieren, dass eine Variable gleich $0$ ist oder dass zwei Variablen gleich sind. Zusammen mit einer Nebenbedingung (oder unter Ausnutzung von Homogenität) könnt ihr die Ungleichung dann weiter auf nur noch eine Variable reduzieren und spätestens an diesem Punkt habt ihr fast immer gewonnen. Die Methode ist überdies sehr flexibel und führt auch bei Ungleichungen mit untypischen Gleichheitsfällen zum Ziel, bei denen AM-GM oder Jensen zwangsläufig versagen würden.

Gegeben sei also eine Ungleichung in den drei\footnote{Wenn ihr den Beweis des $uvw$-Theorems verstanden habt, wird euch klar sein, dass die Methode auch für mehr als drei Variablen anwendbar ist. Nur seid ihr bei vier oder mehr Variablen noch nicht fertig, wenn ihr wisst, dass zwei Variablen gleich sein müssen. Um die Ungleichung weiter zu reduzieren, könntet ihr als nächstes zum Beispiel die Schiebemethode auf zwei der verbleibenden Variablen anwenden \ldots} Variablen $a,b,c\geqslant 0$, gerne auch mit einer Nebenbedingung. Wir substituieren
\begin{equation*}
	u=a+b+c\,,\quad v=ab+bc+ca\,,\quad w=abc
\end{equation*}
Häufig kommt es nun vor, dass die Ungleichung monoton in einer der Variablen $u$, $v$ und $w$ ist. In diesem Fall können wir die anderen beiden Variablen fixieren und müssen nur den Fall betrachten, dass die monotone Variable minimal oder maximal ist.\footnote{Dieses Argument funktioniert noch in weiteren Fällen. Zum Beispiel könnte es passieren, dass wir einen Ausdruck in $u$, $v$ und $w$ maximieren wollen, der in einer der drei Variablen konvex ist. Da konvexe Funktionen ihr Maximum immer am Rand des Definitionsbereiches annehmen, können wir wiederum die anderen beiden Variablen fixieren und die konvexe Variable minimieren oder maximieren.} Was bedeutet es nun, minimal oder maximal zu sein? Die einzige Einschränkung, die $u$, $v$ und $w$ erfüllen müssen, ist, dass das Polynom
\begin{equation*}
	p(X)=X^3-uX^2+vX-w=(X-a)(X-b)(X-c)
\end{equation*}
drei nichtnegative reelle Nullstellen hat, denn wir fordern ja, dass $a,b,c\geqslant 0$ nichtnegative reelle Zahlen sind. Diese Überlegung führt uns nun auf das folgende Theorem.
\begin{satzmitnamen}[Tejs' $\boldsymbol{uvw}$-Theorem]
	Seien $u$, $v$ und $w$ wie oben.
	\begin{enumerate}[label={$(\alph*)$},ref={$(\alph*)$}]
		\item Wir fixieren $u$ und $v$. Dann existiert ein maximales $w$ mit der Eigenschaft, dass das Polynom $p_w(X)=X^3-uX^2+vX-w$ drei nichtnegative reelle Nullstellen hat, und für dieses $w$ hat $p_w$ sogar eine Doppelnullstelle. Ebenso existiert ein minimales $w$ mit dieser Eigenschaft, und dieses $p_w$ hat eine Doppelnullstelle oder eine Nullstelle bei $X=0$.\label{behauptung:uv}
		\item Wir fixieren $u$ und $w$. Dann existiert ein maximales und ein minimales $v$ mit der Eigenschaft, dass das Polynom $p_v(X)=X^3-uX^2+vX-w$ drei nichtnegative reelle Nullstellen hat. Für diese Werte von $v$ hat $p_v$ sogar eine Doppelnullstelle.\label{behauptung:uw}
		\item Wir fixieren $v$ und $w$. Wenn $w>0$, dann existiert ein maximales und ein minimales $u$ mit der Eigenschaft, dass das Polynom $p_u(X)=X^3-uX^2+vX-w$ drei nichtnegative reelle Nullstellen hat. Für diese Werte von $u$ hat $p_u$ sogar eine Doppelnullstelle. Wenn $w=0$, dann hat $p_u$ eine Nullstelle bei $X=0$.\label{behauptung:vw}
	\end{enumerate}
\end{satzmitnamen}
\begin{proof}
	Wir beweisen nur \ref{behauptung:vw}; die anderen beiden Aussagen gehen analog. Im Fall $w=0$ ist klar, dass $p_u$ eine Nullstelle bei $X=0$ hat. Wir können uns also auf den Fall $w\neq0$ beschränken.
	
	
	Zu jedem $p_u$ betrachten wir die Funktion $f_u\colon \mathbb R_{>0}\rightarrow \mathbb R$, $f_u(x)=p_u(x)/x^2$. Wegen $w>0$ kann $p_u$ keine Nullstelle bei $x=0$ haben, also besitzen $p_u$ und $f_u$ die gleichen Nullstellen. Sei $u_0$ irgendein fester Wert von $u$, sodass $p_{u_0}$ drei nichtnegative Nullstellen hat; diese sind dann wegen $w>0$ automatisch positiv. Wir bezeichnen die Nullstellen mit $0<\alpha\leqslant \beta\leqslant \gamma$. Das Polynom $p_{u_0}$ ist dann nichtpositiv auf dem Intervall $(0,\alpha]$, nichtnegativ auf dem Intervall $[\alpha,\beta]$, wieder nichtpositiv auf dem Intervall $[\beta,\gamma]$ und schließlich wieder nichtnegativ auf dem Intervall $[\gamma,\infty)$. Selbiges gilt folglich auch für $f_{u_0}$. Weil $f_{u_0}$ außerdem stetig ist, nimmt es auf dem Intervall $[\alpha,\beta]$ einen maximalen Wert $\Delta^+$ und auf dem Intervall $[\beta,\gamma]$ einen minimalen Wert $\Delta^-$ an. 
	
	\begin{wrapfigure}{r}{0.51\textwidth}
		\centering\vspace{-0.5cm}
		\begin{tikzpicture}[x=1.25cm,y=1.25cm]
			\draw[->] (-0.4,0) to node[pos=1,below=0.5ex] {$x$} (4.8,0);
			\draw[->] (0,-2.1) to  node[pos=1,right=0.5ex] {$y$} (0,2.5);
			\node[below left] at (0,0) {$0$};
			\draw plot[domain=0.28:4.475,smooth,samples=200] (\x+0.125,{(\x-0.33)*(\x-1.14)*(\x-3.72)/(\x*\x)});
			\draw[line width=0.3,dashed] plot[domain=0.313:4.5,smooth,samples=200] (\x+0.125,{(\x-0.33)*(\x-1.14)*(\x-3.72)/(\x*\x)-1.401});
			\draw[line width=0.3,dashed] plot[domain=0.275:4.5,smooth,samples=200] (\x+0.125,{(\x-0.33)*(\x-1.14)*(\x-3.72)/(\x*\x)+0.625});
			\draw (0,1.405) ++ (0.5ex,0) to ++(-1ex,0ex) node[left] {$\Delta^+$};
			\draw [line width=0.3,dashed,dash phase=0.5] (1.125ex,1.405) to (4.65,1.405);
			\draw (0,-0.629) ++ (0.5ex,0) to ++(-1ex,0ex) node[left] {$\Delta^-$};
			\draw [line width=0.3,dashed,dash phase=0.5] (1.125ex,-0.629) to (4.65,-0.629);
			\draw (0.455,0) ++ (0,0.5ex) to ++(0,-1ex);
			\draw (1.265,0) ++ (0,0.5ex) to ++(0,-1ex);
			\draw (3.845,0) ++ (0,0.5ex) to ++(0,-1ex);
			\node at (0.25,-0.24) {$\alpha$};
			\node at (1.15,-0.23) {$\beta\vphantom{\gamma}$};
			\node at (3.845,-0.23) {$\gamma\vphantom{\beta}$};
			\node[right=0.25ex] at (4.5,0.54) {$f_{u_0}$};
			\node[right=0.25ex] at (4.5,1.165) {$f_{u_0+\Delta^+}$};
			\node[right=0.25ex] at (4.5,-0.861) {$f_{u_0+\Delta^-}$};
		\end{tikzpicture}
	\end{wrapfigure}	
	Für beliebiges $u$ gilt stets
	\begin{equation*}
		f_u(x)=f_{u_0}(x)-(u-u_0)\,.
	\end{equation*}
	Daraus folgt, dass $f_u$ für $u>u_0+\Delta^+$ oder $u<u_0+\Delta^-$ nur eine reelle Nullstelle haben kann. Für diese Werte von $u$ hat also auch $p_u$ nur eine reelle Nullstelle. Im Fall $u=u_0+\Delta^+$ berührt hingegen der Graph von $f_{u_0+\Delta^+}$ die $x$-Achse irgendwo auf dem Intervall $[\alpha,\beta]$ (denn $\Delta^+$ war als Extremwert der differenzierbaren Funktion $f_{u_0}$ gewählt, deren Verschiebung um $-\Delta^+$ genau $f_{u+\Delta^+}$ ist), und schneidet sie ein weiteres Mal im Intervall $[\gamma,\infty)$. Selbiges muss also auch für $p_u$ gelten! Folglich hat $p_u$ drei positive reelle Nullstellen, wobei die ersten beiden zu einer Doppelnullstelle zusammenfallen.
	
	Das zeigt, dass $u=u_0+\Delta^+$ der maximale Wert ist, für den $p_u$ drei nichtnegative reelle Nullstellen hat, und wie wir gerade gesehen haben, hat $p_u$ dann sogar eine Doppelnullstelle, wie behauptet. Analog ist $u=u_0+\Delta^-$ der minimale Wert, für den $p_u$ drei nichtnegative reelle Nullstellen hat, und $p_u$ hat dann sogar eine Doppelnullstelle. Das zeigt die Behauptung.
\end{proof}


Wir werden die Methode nun an einer Aufgabe demonstrieren und euch zwei weitere  Beispielaufgaben stellen. Am Ende dieses Kapitels findet ihr Tipps und am Ende des Heftes findet ihr Musterlösungen zu den Beispielaufgaben.
\begin{aufgabe*}
	Beweise, dass für alle nichtnegativen reellen Zahlen $x$, $y$, $z$ mit $x+y+z=1$ die folgende Ungleichung gilt:
	\begin{equation*}
		\frac{x}{1-yz}+\frac{y}{1-zx}+\frac{z}{1-xy}\leqslant\frac 98\,.
	\end{equation*}
\end{aufgabe*}
\begin{proof}[Lösung]
	Indem wir brutal mit allen Nennern durchmultiplizieren, erhalten wir die Ungleichung
	\begin{equation*}
		x+y+z-\sum x^2y+xyz\parens*{x^2+y^2+z^2}\leqslant \frac 98\parens*{1-\sum xy+xyz(x+y+z)-x^2y^2z^2}\,.
	\end{equation*}
	Wegen $x+y+z=1$ ist $\sum x^2y=(x+y+z)\sum xy-3xyz=\sum xy-3xyz=v-3w$ und außerdem $x^2+y^2+z^2=(x+y+z)^2-2\sum xy=1-2v$. Somit wird die obige Ungleichung zu
	\begin{equation*}
		0\leqslant \frac 98\parens*{1-v+w-w^2}-\parens*{1+3w-v+w-2wv}\,.
	\end{equation*}
	Und die rechte Seite ist linear in $v$! Wenn wir also $u=1$ und $w$ festhalten, nimmt die rechte Seite ihren minimalen Wert auf dem \enquote{Rand} an, sprich, wenn $v$ maximal oder minimal ist. Nach dem $uvw$-Theorem hat das Polynom $X^3-uX^2+vX-w=(X-x)(X-y)(X-z)$ in diesem Fall eine Doppelnullstelle. Das heißt, wir dürfen ohne Einschränkung $x=y$ annehmen. Wegen $x+y+z=1$ ist dann außerdem $z=1-2x$. Damit haben wir die Ungleichung auf eine Variable reduziert.
	
	Der Rest ist eine etwas hässliche, aber nicht mehr schwere Rechnung. Wir setzen $x=y$ und $z=1-2x$ ein und multiplizieren alles aus. Damit erhalten wir ein Polynom sechsten Grades in $x$, das wir gegen $0$ abschätzen müssen. Weil $x=y=z=\frac13$ offensichtlich ein Gleichheitsfall der ursprünglichen Ungleichung ist, muss das Polynom eine Nullstelle bei $x=\frac13$ haben. Tatsächlich muss es sogar eine Doppelnullstelle dort haben, sonst würde es sein Vorzeichen wechseln und die Ungleichung wäre falsch. Wir können also einen Faktor $(x-\frac13)^2$ (oder $(3x-1)^2$) ausklammern. Wenn wir das tun, erhalten wir die zu beweisende Ungleichung
	\begin{equation*}
		0\leqslant \frac18(3x-1)^2\parens*{-4x^4+12x^3-5x^2+4x+1}\,.
	\end{equation*}
	Für $0\leqslant x\leqslant 1$ ist $4x^4\leqslant 4x^3\leqslant 12x^3$ und $5x^2\leqslant 5x\leqslant 4x+1$, also ist der zweite Faktor auf der rechten Seite stets nichtnegativ. Der erste Faktor ist sowieso ein Quadrat. Also gilt die Ungleichung und wir sind fertig.
\end{proof}

\begin{aufgabe*}\label{aufgabe:DEMO2015Ungleichung}
	Beweise, dass für alle positiven reellen Zahlen $x,y,z>0$ die folgende Ungleichung gilt:
	\begin{equation*}
		\frac{x+y+z}{3}+\frac{3}{\frac1x+\frac1y+\frac1z}\geqslant 5\sqrt[3]{\frac{xyz}{16}}\,.
	\end{equation*}
\end{aufgabe*}
\begin{aufgabe*}[*]\label{aufgabe:UngleichungInvertieren}
	Gegeben seien nichtnegative reelle Zahlen $x,y,z\geqslant 0$ mit $x^2+y^2+z^2+2xyz=1$. Zeige, dass
	\begin{equation*}
		xy+yz+zx\leqslant \frac12+2xyz\,.
	\end{equation*}
\end{aufgabe*}

\vfill\hrule\vspace{-1em}

\subsection*{Tipps zu den Beispielaufgaben}
\textbf{Tipps zu Aufgabe~\ref{aufgabe:DEMO2015Ungleichung}.} Errate die Gleichheitsfälle (Spoiler: $x=y=z$ ist es nicht) und benutze die $uvw$-Methode.	

\textbf{Tipps zu Aufgabe~\ref{aufgabe:UngleichungInvertieren}.} Hier lässt sich die $uvw$-Methode nicht direkt anwenden, denn in der Nebenbedingung kommen sowohl $u$, $v$ als auch $w$ vor. Gibt es ein Argument, mit dem die Ungleichung und ihre Nebenbedingung vertauscht werden können? (Dieses Argument ist ein Trick, den ihr euch merken solltet!)

Es gibt bei dieser Aufgabe nicht nur den Gleichheitsfall $x=y=z=\frac12$. Errate die anderen Gleichheitsfälle.\newpage
	
	\phantomsection\cftaddtitleline{toc}{part}{Geometrie}{\thepage}
	\section{Projektive Geometrie}\label{kapitel:ProjektiveGeometrie}
In diesem Kapitel werden wir eine Einführung in die projektive Geometrie geben und an vielen Beispielen demonstrieren, wie projektive Geometrie zur Lösung von Olympiadeaufgaben eingesetzt werden können. Naturgemäß werden wir dafür viel Theorie einführen müssen, wovon euch sehr viel wahrscheinlich neu sein wird. Wenn ihr dieses Kapitel aber einmal verdaut und durchgearbeitet habt, werdet ihr ein ganzes Arsenal an neuen Werkzeugen zur Verfügung haben und ihr werdet Aufgaben mit Leichtigkeit lösen können, die euch vorher sehr schwer gefallen wären.

Die Beispielaufgaben in diesem Kapitel werden direkt aufgelöst, statt dass es wie sonst Tipps und die Lösungen erst am Ende des Heftes gibt. Der Grund dafür ist die Fülle an neuen Methoden, die in diesem Kapitel eingeführt werden. Bevor ihr diese Methoden realistisch selbst anwenden könnt, müsst ihr sie einmal in Aktion gesehen haben. Wenn ihr Lust habt, könnt ihr natürlich trotzdem zuerst selbst an den Aufgaben zu knobeln, bevor ihr weiterlest. Außerdem gibt es am Ende jede Menge Übungsaufgaben zum Vertiefen der Methoden.

Zur projektiven Geometrie gibt es auch ein exzellentes Skript von Jens Reinhold,\footnote{Online verfügbar unter \url{https://drive.google.com/uc?id=1aUdgB5zLRj8MPfwy3r1exr6idfGJR65Z}} an welchem sich dieses Kapitel in großen Teilen orientiert.

\subsection*{Die projektive Ebene}
In vielen geometrischen Konstellationen müssen Spezialfälle betrachtet werden, weil es passieren kann, dass zwei Geraden parallel sind, sodass ein gewünschter Schnittpunkt vielleicht gar nicht existiert. Zum Beispiel gibt es viele Aussagen (und Aufgaben) der Form \enquote{drei Geraden schneiden sich in einem Punkt oder sind paarweise parallel}. Die projektive Ebene ist ein Hilfsmittel, mit dem sich derartige Spezialfälle vermeiden lassen.

Die projektive Ebene entsteht aus der üblichen euklidischen Ebene, indem wir einige \enquote{unendlich weit entfernte Punkte} hinzufügen, sogenannte \emph{Fernpunkte}, die alle auf einer \enquote{unendlich weit entfernten Gerade} liegen, der \emph{Ferngerade}. Konkret stellen wir uns dafür vor, dass jede Gerade~$\ell$ durch einen Fernpunkt~$\infty_\ell$ verläuft. Für zwei Geraden $\ell_1$~und~$\ell_2$ soll genau dann $\infty_{\ell_1}=\infty_{\ell_2}$ gelten, wenn $\ell_1$~und~$\ell_2$ parallel sind. Alle Geraden, die zu einer gegebenen Geraden~$\ell$ parallel sind, schneiden $\ell$ also im Fernpunkt~$\infty_\ell$ (und alle anderen Geraden schneiden~$\ell$ natürlich in einem Nicht-Fernpunkt). Wenn $P$ ein beliebiger Punkt ist, dann definieren wir die \enquote{Gerade~$P\infty_\ell$} als die Parallele zu~$\ell$ durch~$P$. Wenn $\ell_1$ und $\ell_2$ zwei nicht-parallele Geraden sind, das heißt $\infty_{\ell_1}\neq \infty_{\ell_2}$ gilt, dann ist die \enquote{Gerade $\infty_{\ell_1}\infty_{\ell_2}$} genau die Ferngerade.

Es gibt einen Weg, sich die projektive Ebene etwas anschaulicher zu machen. Dazu stellen wir uns vor, dass die euklidische Ebene in Wirklichkeit eine Ebene~$\Sigma$ im Raum ist. Sei $Z$ ein Punkt, der nicht auf~$\Sigma$ liegt. Jedem Punkt~$X$ auf~$\Sigma$ ordnen wir die Gerade~$XZ$ zu und jeder Gerade~$\ell$ in~$\Sigma$ ordnen wir die Ebene~$\Pi_\ell$ zu, die von $\ell$~und~$Z$ aufgespannt wird. Dadurch erhalten wir Abbildungen	
\begin{equation*}
	\begin{aligned}[t]
		\{\text{Punkte auf~$\Sigma$}\}&\longrightarrow \{\text{Geraden durch~$Z$}\}\\
		X&\longmapsto XZ
	\end{aligned}\quad\text{und}\quad 
	\begin{aligned}[t]
		\{\text{Geraden auf~$\Sigma$}\} & \longrightarrow \{\text{Ebenen durch~$Z$}\}\\
		\ell & \longmapsto \Pi_\ell\,.
	\end{aligned}
\end{equation*}
Diese Abbildungen sind injektiv, aber nicht surjektiv. Denn die Geraden durch~$Z$, die zur Ebene~$\Sigma$ parallel sind, können nicht von der Form~$XZ$ für irgendeinen Punkt~$X$ auf~$\Sigma$ sein. Genauso wenig kann die Ebene durch~$Z$, die zu~$\Sigma$ parallel ist, von der Form~$\Pi_\ell$ für irgendeine Gerade~$\ell$ auf~$\Sigma$ sein. Die Fernpunkte und die Ferngerade, die wir in der projektiven Ebene hinzugefügt haben, entsprechen genau den Elementen, die von den obigen Abbildungen nicht getroffen werden.

Die Konvention, Punkte zur Ebene hinzuzufügen, erinnert euch bestimmt an Inversion am Kreis. Allerdings unterscheiden sich die beiden Konstruktionen deutlich: Bei der Inversion haben wir einen einzigen unendlich fernen Punkt~$\infty$ hinzugefügt und Geraden als Kreise durch~$\infty$ betrachtet. Insbesondere schneiden sich beliebige zwei Geraden in~$\infty$. Bei der projektiven Ebene haben wir stattdessen unendlich viele unendlich ferne Punkte hinzugefügt, nämlich einen für jede Klasse von parallelen Geraden. Im Studium werdet ihr erfahren, dass Inversion am Kreis in der \emph{komplex-projektiven Geraden~$\mathbb{CP}^1$} stattfindet (die uns als Ebene erscheint, weil wir uns die komplexen Zahlen~$\mathbb C$ ja selber als Ebene vorstellen), während wir hier mit der \emph{reell-projektiven Ebene~$\mathbb{RP}^2$} arbeiten.

\subsection*{Das Doppelverhältnis}
\begin{definition}
	Seien $A$,~$B$, $C$ und~$D$ vier verschiedene Punkte auf einer Geraden~$\ell$. Das \emph{Doppelverhältnis der Punktepaare $(A,B)$ und $(C,D)$} ist definiert als
	\begin{equation*}
		(A,B;C,D)\coloneqq \frac{AC}{CB}\bigg/\!\frac{AD}{DB}\,,
	\end{equation*}
	wobei alle Streckenlängen als gerichtet zu verstehen sind.
\end{definition}

Ihr könnt euch leicht überlegen, dass das Doppelverhältnis unabhängig von der Wahl der Richtung auf~$BC$ ist (so eine Wahl der Richtung findet ja immer implizit statt, wenn wir mit gerichteten Streckenlängen arbeiten). Das Doppelverhältnis ist außerdem unabhängig von der Reihenfolge der Punktepaare: $(A,B;C,D)=(C,D;A,B)$. Es ist allerdings \emph{nicht} unabhängig von der Reihenfolge der Punkte innerhalb der Punktepaare: Wenn $(A,B;C,D)=\lambda$, dann ist $(B,A;C,D)=1/\lambda$.

Ein wiederkehrendes Motiv in der projektiven Geometrie ist das \emph{Dualitätsprinzip}: Eine Aussage sollte auch dann noch wahr sein, wenn wir \enquote{die Rollen von Punkten und Geraden vertauschen}, das heißt, wenn jeder Punkt durch eine Gerade und jede Gerade durch einen Punkt ersetzt wird, wobei die Verbindungsgerade zweier Punkte durch den Schnittpunkt der entsprechenden Geraden ersetzt wird und umgekehrt. Im letzten Unterkapitel zu Dualität werdet ihr eine Methode kennenlernen, mit der sich dieses Prinzip konkret verwirklichen lässt. Für den Moment nehmen wir das Dualitätsprinzip als Motivation, das Doppelverhältnis nicht nur für Punkte auf einer Geraden, sondern auch für Geraden durch einen Punkt zu definieren. 

\begin{definition}
	Seien $a$,~$b$, $c$ und~$d$ vier verschiedene Geraden, die alle durch den Punkt~$P$ verlaufen. Dann definieren wir das \emph{Doppelverhältnis der Geradenpaare $(a,b)$ und $(c,d)$} als
	\begin{equation*}
		(a,b;c,d)\coloneqq \epsilon\cdot \frac{\sin\winkel(a,c)}{\sin\winkel(c,b)}\bigg/\frac{\sin\winkel (a,d)}{\sin\winkel (d,b)}\,,
	\end{equation*}
	wobei alle Schnittwinkel wie üblich zwischen $0^\circ$~und~$90^\circ$ liegen sollen und das Vorzeichen $\epsilon\in \braces{-1,+1}$ wie folgt definiert ist: Wenn $c$~und~$d$ im gleichen der von $a$~und~$b$ aufgespannten Winkel liegen, dann wählen wir $\epsilon\coloneqq+1$, ansonsten sei $\epsilon\coloneqq-1$.
\end{definition}

\begin{satzmitnamen}[Lemma]
	Seien $A$,~$B$, $C$ und~$D$ vier verschiedene Punkte auf einer Geraden~$\ell$ sind und sei $P$ ein beliebiger Punkt, der nicht auf~$\ell$ liegt. Dann ist %das Doppelverhältnis der Punktepaare $(A,B)$ und $(C,D)$ gleich dem Doppelverhältnis der Geradenpaare $(PA,PB)$ und $(PC,PD)$:
	\begin{equation*}
		\left(A,B;C,D\right)=\left(PA,PB;PC,PD\right)\,.
	\end{equation*}
\end{satzmitnamen}

\begin{proof}
	Es lässt sich leicht nachprüfen, dass $(A,B;C,D)$ und $(PA,PB;PC,PD)$ das gleiche Vorzeichen haben. Wir müssen also nur prüfen, dass auch ihre Beträge übereinstimmen. Die Winkel $\varphi$~und~$\psi$ seien wie in der Skizze definiert. Nach dem Sinussatz in den Dreiecken $ACP$ und $BCP$ gilt $\abs{AC}=\abs{PA}\cdot {\sin\winkel(PA,PC)}/{\sin\varphi}$ und $\abs{CB}=\abs{PB}\cdot {\sin\winkel(PC,PB)}/{\sin\varphi}$. Nach dem Sinussatz in den Dreiecken $ADP$ und $BDP$ gilt $\abs{AD}=\abs{PA}\cdot {\sin\winkel(PA,PD)}/{\sin\psi}$ und $\abs{DB}=\abs{PB}\cdot {\sin\winkel(PD,PB)}/{\sin\psi}$. Durch Einsetzen folgt
	\begin{equation*}
		\frac{\abs{AC}}{\abs{CB}}\bigg/\frac{\abs{AD}}{\abs{DB}}=\frac{\abs{PA}\cdot\sin\winkel(a,c)}{\abs{PB}\cdot \sin\winkel(c,b)}\bigg/\frac{\abs{PA}\cdot\sin\winkel(a,d)}{\abs{PB}\cdot \sin\winkel(d,b)}=\frac{\sin\winkel(a,c)}{\sin\winkel(c,b)}\bigg/\frac{\sin\winkel(a,d)}{\sin\winkel(d,b)}\,,
	\end{equation*}
	also $\abs{(A,B;C,D)}=\abs{(PA,PB;PC,PD)}$, wie gewünscht.
\end{proof}
	\begin{figure}[ht]
	\centering
	\begin{tikzpicture}[x=0.6cm,y=0.6cm]
		\coordinate (A) at (-1.006,0.404);
		\coordinate (B) at (0.306,-0.462);
		\coordinate (C) at (2.406,-1.849);
		\coordinate (D) at (4.972,-3.543);
		\coordinate (P) at (7.023,2.358);
		\draw [line width=0.3,shorten <=-3em,shorten >=-2.25em] (A) to (D);
		\draw [shorten <=-3ex,shorten >=-1.5em] (P) to (A);
		\draw [shorten <=-3ex,shorten >=-1.5em] (P) to (B);
		\draw [shorten <=-3ex,shorten >=-1.5em] (P) to (C);
		\draw [shorten <=-3ex,shorten >=-1.5em] (P) to (D);
		\draw [line width=0.3, shift={(C)}] (42.34:3.25ex) arc (42.34:146.567:3.25ex);
		\draw [line width=0.3, shift={(D)}] (70.836:3.75ex) arc (70.836:146.567:3.75ex);
		\draw [fill=black] (A) circle (2pt) node[shift={(250:2ex)}] {$A$};
		\draw [fill=black] (B) circle (2pt) node[shift={(260:2ex)}] {$B$};
		\draw [fill=black] (C) circle (2pt) node[shift={(270:2ex)}] {$C$};
		\draw [fill=black] (D) circle (2pt) node[shift={(200:2ex)}] {$D$};
		\draw [fill=white] (P) circle (2pt) node[shift={(135:2ex)}] {$P$};
		\node [shift={(92:1.75ex)}] at (C) {$\varphi$};
		\node [shift={(105:2.25ex)}] at (D) {$\psi$};
		\node at (-1.85,1.65) {$\ell$};
	\end{tikzpicture}
\end{figure}



Das Doppelverhältnis von zwei Punktepaaren auf einer Geraden~$\ell$ lässt sich auch dann definieren, wenn einer der vier Punkte der Fernpunkt~$\infty_\ell$ ist. Wenn zum Beispiel $D=\infty_\ell$ gilt, dann definieren wir $(A,B;C,\infty_\ell)\coloneqq -AC/CB$. Die Logik dahinter ist, dass das Verhältnis $AD/DB$ gegen~$-1$ konvergiert, wenn sich $D$ immer weiter von der Strecke~$\overline{AB}$ entfernt. Wir können das Doppelverhältnis sogar dann definieren, wenn die Gerade~$\ell$ die Ferngerade ist. Wir können nämlich Geraden $a$,~$b$, $c$ und~$d$ wählen, sodass $A=\infty_a$, $B=\infty_b$, $C=\infty_c$ und $D=\infty_d$. Durch Parallelverschiebung können wir erreichen, dass sich $a$,~$b$, $c$ und~$d$ in einem Punkt~$P$ schneiden. Dann definieren wir $(\infty_a,\infty_b;\infty_c,\infty_d)\coloneqq (a,b;c,d)$. Aus dem Lemma folgt, dass diese Definition unabhängig von der Wahl von~$P$ ist. Mit ähnlichen Tricks lässt sich das Doppelverhältnis schließlich auch für Geradenpaare definieren, bei denen eine der Geraden die Ferngerade ist. Durch eine etwas umständliche Fallunterscheidung lässt sich zeigen, dass das vorherige Lemma auch für diese erweiterte Definition des Doppelverhältnisses gültig ist.
% Tien: Da du diese Verallgemeinerung des Doppelverhältnisses auch später im Beweis benutzt, willst du vielleicht sagen, dass das Lemma auch für diesen erweiterten Begriff gilt.

Nachdem wir Doppelverhältnisse für Punktepaare auf einer Geraden und Geradenpaare durch einen Punkt definiert haben, wollen wir zum Schluss noch das Doppelverhältnis für Punktepaare auf einem Kreis definieren.

\begin{definition}
	Seien $A$,~$B$, $C$ und~$D$ vier verschiedene Punkte auf einem Kreis~$\omega$. Das \emph{Doppelverhältnis der Punktepaare $(A,B)$ und $(C,D)$} ist definiert als
	\begin{equation*}
		\left(A,B;C,D\right)\coloneqq \epsilon\cdot \left.\frac{\abs{AC}}{\abs{CB}}\middle/\frac{\abs{AD}}{\abs{DB}}\right.\,,
	\end{equation*}
	wobei wir $\epsilon\coloneqq +1$ wählen, wenn $C$ und $D$ auf dem gleichen der beiden Kreisbögen $\wideparen{AB}$ liegen, sonst wählen wir $\epsilon\coloneqq -1$.
\end{definition}

\begin{satzmitnamen}[Lemma]
	Seien $A$,~$B$, $C$ und~$D$ vier verschiedene Punkte auf einem Kreis~$\omega$ und sei $P$ ein weiterer Punkt auf~$\omega$. Dann ist% das Doppelverhältnis der Punktepaare $(A,B)$ und $(C,D)$ gleich dem Doppelverhältnis der Geradenpaare $(PA,PB)$ und $(PC,PD)$:
	\begin{equation*}
		\left(A,B;C,D\right)=\left(PA,PB;PC,PD\right)
	\end{equation*}
	\embrace{in dem Fall, dass $P$ mit einem der Punkte $A$,~$B$, $C$ oder~$D$ zusammenfällt, interpretieren wir die \enquote{Gerade~$PP$} als die Tangente an~$\omega$ in~$P$}.
\end{satzmitnamen}

\begin{proof}
	Mit einer einfachen Fallunterscheidung lässt sich zeigen, dass die Doppelverhältnisse $(A,B;C,D)$ und $(PA,PB;PC,PD)$ stets das gleiche Vorzeichen haben. Wir müssen also nur prüfen, dass auch ihre Beträge übereinstimmen. Durch Einsetzen der Formeln folgt
	\begin{equation*}
		\frac{\abs{(PA,PB;PC,PD)}}{\abs{(A,B;C,D)}}=\frac{\sin\winkel(PA,PC)}{\abs{AC}}\cdot \frac{\abs{CB}}{\sin\winkel (PC,PB)}\cdot \frac{\abs{AD}}{\sin\winkel(PA,PD)}\cdot \frac{\sin\winkel(PD,PB)}{\abs{DB}}\,.
	\end{equation*}
	Nach dem erweiterten Sinussatz im Dreieck $ACP$ gilt aber $\abs{AC}/\!\sin\winkel(PA,PC)=2r$, wobei $r$ der Radius von~$\omega$ ist.
	% Tien: Vielleicht noch ein Kommentar, dass im Spezialfall, falls P mit einer der vier Punkte zusammenfällt, du den Sehnen-Tangentenwinkelsatz benutzt.
	Analoge Argumente lassen sich auf die anderen Faktoren anwenden. Das Produkt auf der rechten Seite der obigen Gleichung ist also $(2r)^{-1}\cdot (2r)\cdot (2r)\cdot (2r)^{-1}=1$.
	Damit haben wir, wie gewünscht, $\abs{(A,B;C,D)}=\abs{(PA,PB;PC,PD)}$ gezeigt. Das Argument funktioniert auch in dem Fall, dass~$P$ mit einem der Punkte $A$, $B$, $C$ oder $D$ zusammenfällt. In diesem Fall müssen wir den Sehnen-Tangentenwinkelsatz benutzen.
\end{proof}

\subsection*{Projektive Abbildungen}
\begin{definition}
	%Seien $\mathcal X$ und $\mathcal Y$ zwei Mengen von Punkten, die jeweils ein Kreis oder eine Gerade sein können.
	Seien $\mathcal X$~und~$\mathcal Y$ jeweils ein Kreis oder eine Gerade, die wir als Menge von Punkten auffassen.
	% Tien: Ich habe das mal umformuliert.
	Eine Abbildung $\pi\colon \mathcal X\to \mathcal Y$ heißt \emph{projektiv}, wenn sie das Doppelverhältnis erhält, das heißt, wenn für beliebige vier verschiedene Punkte $A,B,C,D\in\mathcal X$ folgendes gilt:
	\begin{equation*}
		(A,B;C,D)=\parens[\big]{\pi(A),\pi(B);\pi(C),\pi(D)}\,.
	\end{equation*}
\end{definition}

\begin{satzmitnamen}[Lemma]
	In den folgenden Fällen ist die Projektion durch einen Punkt eine projektive Abbildung:
	\begin{enumerate}
		\item \label{itm:projAbb:GG}
		Seien $\ell_1$~und~$\ell_2$ zwei Geraden, sei $P$ ein Punkt, der nicht auf~$\ell_1$ liegt, und sei $\pi_1\colon \ell_1\to \ell_2$ die Projektion durch~$P$ von~$\ell_1$ auf~$\ell_2$. Das bedeutet: Für jeden Punkt~$X$ auf~$\ell_1$ ist $\pi_1(X)$ der Schnittpunkt von~$PX$ mit~$\ell_2$; falls $PX$ parallel zu~$\ell_2$ ist definieren wir $\pi_1(X)\coloneqq \infty_{\ell_2}$. Dann ist $\pi_1$ eine projektive Abbildung.
		\item \label{itm:projAbb:GK}
		Sei $\ell$ eine Gerade, sei $\omega$ ein Kreis und sei $P$ ein Punkt auf~$\omega$, der nicht auf~$\ell$ liegt.
		% Tien: Ich bin mir zu 100% sicher, dass P nicht auf \ell liegen darf; habe das korrigiert.
		Sei $\pi_2\colon \ell\to \omega$ die Projektion durch~$P$ von~$\ell$ auf~$\omega$. Das bedeutet: Für jeden Punkt~$X$ auf~$\ell$ ist $\pi_2(X)$ der von~$P$ verschiedene Schnittpunkt von~$PX$ mit~$\omega$; falls $PX$ den Kreis~$\omega$ in~$P$ tangiert, definieren wir $\pi_2(X)\coloneqq P$.
		% und falls $PX$ parallel zu~$\ell$ ist, definieren wir $\pi_2(X)\coloneqq \infty_\ell$.
		% Tien: Der auskommentierte Teilsatz macht keinen Sinn. Es ist doch egal, ob PX parallel zu \ell ist, da es doch immer noch einen Schnittpunkt mit \omega gibt (auch liegt \infty_\ell nicht auf \omega). Was du wahrscheinlich damit meinst ist der Spezialfall für \pi_2'.
		Dann ist $\pi_2$ eine projektive Abbildung. Wenn $\pi_2'\colon\omega\to \ell$ die umgekehrte Projektion durch~$P$ von~$\omega$ auf~$\ell$ ist, dann ist auch $\pi_2'$ projektiv.
		\item \label{itm:projAbb:KK}
		Seien $\omega_1$~und~$\omega_2$ zwei Kreise, die sich in einem Punkt~$P$ schneiden. Sei $\pi_3\colon \omega_1\to \omega_2$ die Projektion durch~$P$ von~$\omega_1$ auf~$\omega_2$. Dann ist $\pi_3$ eine projektive Abbildung.
		% Tien: Willst du vielleicht auch hier Projektion durch einen Punkt erklären, oder gehst du davon aus, das alle Elftklässler das Heft von Klasse 10 kennen?
		\item \label{itm:projAbb:K}
		Sei $\omega$ ein Kreis, sei $P$ ein Punkt, der nicht auf~$\omega$ liegt, und sei $\pi_4\colon \omega\to \omega$ die Projektion durch~$P$ von~$\omega$ auf sich selbst. Das bedeutet: Für jeden Punkt~$X$ auf~$\omega$ ist $\pi_4(X)$ der von~$X$ verschiedene Schnittpunkt von~$PX$ mit~$\omega$; falls $PX$ den Kreis~$\omega$ in~$X$ tangiert, definieren wir $\pi_4(X)\coloneqq X$. Dann ist $\pi_4$ eine projektive Abbildung.
	\end{enumerate}
\end{satzmitnamen}

\begin{proof}
	In jedem der Fälle betrachten wir vier verschiedene Punkte $A$,~$B$, $C$ und~$D$ sowie ihre Bildpunkte $A'$,~$B'$, $C'$ und~$D'$ unter der entsprechenden Abbildung. In den Fällen~\ref{itm:projAbb:GG},~\ref{itm:projAbb:GK} und~\ref{itm:projAbb:KK} können wir die beiden Lemmata aus dem vorherigen Abschnitt anwenden und erhalten sofort
	\begin{equation*}
		(A,B;C,D)=(PA,PB;PC,PD)=(A',B';C',D')\,. 
	\end{equation*}
	Die Abbildungen $\pi_1$,~$\pi_2$, $\pi_2'$ und~$\pi_3$ erhalten also in der Tat das Doppelverhältnis. Im Fall~\ref{itm:projAbb:KK} können wir sogar noch mehr sagen: Die Abbildung~$\pi_3$ ist identisch zur Drehstreckung um den zweiten Schnittpunkt von $\omega_1$~und~$\omega_2$, die $\omega_1$~auf~$\omega_2$ abbildet (falls ihr diesen Fakt noch nicht kanntet, schaut einmal ins Kapitel \emph{\embrace{Dreh-}Streckungen} im Heft für Klasse~10). Und es ist klar, dass Drehstreckungen das Doppelverhältnis erhalten.
	
	\begin{figure}[ht]
		\centering
		\begin{tabularx}{\textwidth}{X c X c X}
			& \begin{tikzpicture}[x=0.65cm,y=0.65cm]
				\clip (-0.08,-1.35) rectangle (7.28,4.42);
				\coordinate (P) at (5.03,3.713);
				\coordinate (A) at (2.979,1.43);
				\coordinate (B) at (3.697,1.357);
				\coordinate (C) at (4.613,1.264);
				\coordinate (D) at (5.842,1.139);
				\coordinate (A1) at (1.058,-0.709);
				\coordinate (B1) at (2.691,-0.423);
				\coordinate (C1) at (4.377,-0.127);
				\coordinate (D1) at (6.144,0.183);
				\draw [line width=0.3,shorten <=-3ex,shorten >=-1.5em] (P) to (A1);
				\draw [line width=0.3,shorten <=-3ex,shorten >=-1.5em] (P) to (B1);
				\draw [line width=0.3,shorten <=-3ex,shorten >=-1.5em] (P) to (C1);
				\draw [line width=0.3,shorten <=-3ex,shorten >=-1.5em] (P) to (D1);
				\draw [shorten <=-4em,shorten >=-2.5em] (A) to (D);
				\draw [shorten <=-2em,shorten >=-2em] (A1) to (D1);
				\draw [fill=white] (P) circle (2pt) node[shift={(170:2ex)}] {$P$};
				\draw [fill=black] (A) circle (2pt) node[shift={(260:2ex)}] {$A$};
				\draw [fill=black] (B) circle (2pt) node[shift={(285:2ex)}] {$B$};
				\draw [fill=black] (C) circle (2pt) node[shift={(302:2.25ex)}] {$C$};
				\draw [fill=black] (D) circle (2pt) node[shift={(320:2.5ex)}] {$D$};
				\draw [fill=black] (A1) circle (2pt) node[shift={(310:2ex)}] {$A'$};
				\draw [fill=black] (B1) circle (2pt) node[shift={(310:2ex)}] {$B'$};
				\draw [fill=black] (C1) circle (2pt) node[shift={(325:2.25ex)}] {$C'$};
				\draw [fill=black] (D1) circle (2pt) node[shift={(335:2.75ex)}] {$D'$};
				\node at (0.9,1.1) {$\ell_1$};
				\node at (0.3,-0.35) {$\ell_2$};
			\end{tikzpicture} & & \begin{tikzpicture}[x=0.52cm,y=0.52cm]
				\clip (-0.09,-1.76) rectangle (9.13,5.5);
				\draw (2.853,3.042) circle (1.735);
				\coordinate (P) at (1.945,4.521);
				\coordinate (A) at (1.644,1.797);
				\coordinate (B) at (3.54,1.449);
				\coordinate (C) at (4.275,2.049);
				\coordinate (D) at (4.587,2.999);
				\coordinate (A1) at (1.378,-0.615);
				\coordinate (B1) at (4.156,0.264);
				\coordinate (C1) at (5.544,0.703);
				\coordinate (D1) at (7.499,1.321);
				\draw [line width=0.3,shorten <=-3ex,shorten >=-1.5em] (P) to (A1);
				\draw [line width=0.3,shorten <=-3ex,shorten >=-1.5em] (P) to (B1);
				\draw [line width=0.3,shorten <=-3ex,shorten >=-1.5em] (P) to (C1);
				\draw [line width=0.3,shorten <=-3ex,shorten >=-1.5em] (P) to (D1);
				\draw [shorten <=-2em,shorten >=-2.25em] (A1) to (D1);
				\draw [fill=white] (P) circle (2pt) node[shift={(193:2.25ex)}] {$P$};
				\draw [fill=black] (A) circle (2pt) node[shift={(210:2ex)}] {$A$};
				\draw [fill=black] (B) circle (2pt) node[shift={(250:2ex)}] {$B$};
				\draw [fill=black] (C) circle (2pt) node[shift={(10:2ex)}] {$C$};
				\draw [fill=black] (D) circle (2pt) node[shift={(30:2ex)}] {$D$};
				\draw [fill=black] (A1) circle (2pt) node[shift={(230:2.5ex)}] {$A'$};
				\draw [fill=black] (B1) circle (2pt) node[shift={(240:2.25ex)}] {$B'$};
				\draw [fill=black] (C1) circle (2pt) node[shift={(250:2.25ex)}] {$C'$};
				\draw [fill=black] (D1) circle (2pt) node[shift={(255:2ex)}] {$D'$};
				\node at (8.6,2.3) {$\ell$};
				\node at (4,5) {$\omega$};
			\end{tikzpicture} & \\
			& Fall~\ref{itm:projAbb:GG} & & Fall~\ref{itm:projAbb:GK} &
		\end{tabularx}
	\end{figure}
	\begin{figure}[ht]
		\centering
		\begin{tabularx}{\textwidth}{X c X c X}
			& \begin{tikzpicture}[x=0.6cm,y=0.6cm]
				\clip (-1.2,-2.02) rectangle (5.12,4.62);
				\draw (2.671,2.952) circle (1.622);
				\draw (1.832,0.774) circle (2.23);
				\coordinate (P) at (3.812,1.8);
				\coordinate (A) at (1.218,3.674);
				\coordinate (B) at (1.799,1.584);
				\coordinate (C) at (3.231,1.43);
				\coordinate (D) at (4.292,2.911);
				\coordinate (A1) at (2.184,2.976);
				\coordinate (B1) at (-0.32,1.358);
				\coordinate (C1) at (0.064,-0.584);
				\coordinate (D1) at (2.444,-1.371);
				\draw [line width=0.3,shorten <=-2ex,shorten >=-2ex] (P) to (A);
				\draw [line width=0.3,shorten <=-2ex,shorten >=-2ex] (P) to (B1);
				\draw [line width=0.3,shorten <=-2ex,shorten >=-2ex] (P) to (C1);
				\draw [line width=0.3,shorten <=-2ex,shorten >=-2ex] (D) to (D1);
				\draw [fill=white] (P) circle (2pt) node[shift={(0:3ex)}] {$P$};
				\draw [fill=white] (1.051,2.863) circle (2pt);
				\draw [fill=black] (A) circle (2pt) node[shift={(100:2.25ex)}] {$A$};
				\draw [fill=black] (B) circle (2pt) node[shift={(250:2ex)}] {$B$};
				\draw [fill=black] (C) circle (2pt) node[shift={(251:2.125ex)}] {$C$};
				\draw [fill=black] (D) circle (2pt) node[shift={(-5:2ex)}] {$D$};
				\draw [fill=black] (A1) circle (2pt) node[shift={(70:2ex)}] {$A'$};
				\draw [fill=black] (B1) circle (2pt) node[shift={(220:2.5ex)}] {$B'$};
				\draw [fill=black] (C1) circle (2pt) node[shift={(255:2.25ex)}] {$C'$};
				\draw [fill=black] (D1) circle (2pt) node[shift={(320:2.125ex)}] {$D'$};
				\node at (3.4,3.85) {$\omega_1$};
				\node at (1.4,-0.90) {$\omega_2$};
			\end{tikzpicture} & & \begin{tikzpicture}[x=0.55cm,y=0.55cm]
				\clip (-3.52,-1.62) rectangle (7.93,6.29);
				\draw (4.275,2.339) circle (3.134);
				\coordinate (P) at (-2.574,0.772);
				\coordinate (A) at (1.441,3.676);
				\coordinate (B) at (1.717,0.528);
				\coordinate (C) at (1.148,2.127);
				\coordinate (D) at (4.45,-0.791);
				\coordinate (A1) at (3.891,5.45);
				\coordinate (B1) at (6.612,0.25);
				\coordinate (C1) at (6.806,4.188);
				\coordinate (D1) at (2.79,-0.421);
				\draw [line width=0.3,shorten <=-3ex,shorten >=-1.5em] (P) to (A1);
				\draw [line width=0.3,shorten <=-3ex,shorten >=-1.5em] (P) to (B1);
				\draw [line width=0.3,shorten <=-3ex,shorten >=-1.5em] (P) to (C1);
				\draw [line width=0.3,shorten <=-3ex,shorten >=-1.5em] (P) to (D);
				\draw [line width=0.3] (A) to (C);
				\draw [line width=0.3] (A1) to (C1);
				\draw [line width=0.3,shift={(A)}] (215.888:0.32cm) arc (215.888:259.306:0.32cm);
				\draw [line width=0.3,shift={(A)}] (215.888:0.37cm) arc (215.888:259.306:0.37cm);
				\draw [line width=0.3,shift={(C)}] (79.306:0.27cm) arc (79.306:200.012:0.27cm);
				\draw [line width=0.3,shift={(A1)}] (215.888:0.27cm) arc (215.888:336.594:0.27cm);
				\draw [line width=0.3,shift={(C1)}] (156.594:0.32cm) arc (156.594:200.012:0.32cm);
				\draw [line width=0.3,shift={(C1)}] (156.594:0.37cm) arc (156.594:200.012:0.37cm);
				\draw [fill=white] (P) circle (2pt) node[shift={(270:2ex)}] {$P$};
				\draw [fill=black] (A) circle (2pt) node[shift={(150:2ex)}] {$A$};
				\draw [fill=black] (B) circle (2pt) node[shift={(60:2ex)}] {$B$};
				\draw [fill=black] (C) circle (2pt) node[shift={(235:2.25ex)}] {$C$};
				\draw [fill=black] (D) circle (2pt) node[shift={(250:2ex)}] {$D$};
				\draw [fill=black] (A1) circle (2pt) node[shift={(120:2ex)}] {$A'$};
				\draw [fill=black] (B1) circle (2pt) node[shift={(300:2ex)}] {$B'$};
				\draw [fill=black] (C1) circle (2pt) node[shift={(-14:2.6ex)}] {$C'$};
				\draw [fill=black] (D1) circle (2pt) node[shift={(225:2.5ex)}] {$D'$};
				\node at (6.85,2.25) {$\omega$};
			\end{tikzpicture} & \\
			& Fall~\ref{itm:projAbb:KK} & & Fall~\ref{itm:projAbb:K} &
		\end{tabularx}
	\end{figure}
	
	Im Fall~\ref{itm:projAbb:K} müssen wir ein wenig mehr arbeiten. Mit einer einfachen Fallunterscheidung sehen wir zunächst, dass $(A,B;C,D)$ und $(A',B';C',D')$ das gleiche Vorzeichen haben. Wir müssen also nur ihre Beträge vergleichen. Dazu beobachten wir, dass die Dreiecke $PCA$ und $PA'C'$ ähnlich sind; je nachdem, ob $P$ innerhalb oder außerhalb von~$\omega$ liegt, benutzen wir dazu das Argument aus dem Beweis des Sehnen- oder des Sekantensatzes. Es folgt $\abs{A'C'}/\abs{AC}=\abs{PA'}/\abs{PC}$. Analog erhalten wir $\abs{C'B'}/\abs{CB}=\abs{PC'}/\abs{PB}$, $\abs{A'D'}/\abs{AD}=\abs{PA'}/\abs{PD}$ und $\abs{D'B'}/\abs{DB}=\abs{PD'}/\abs{PB}$. Setzen wir diese Gleichungen ein, erhalten wir
	\begin{equation*}
		\frac{\abs{(A',B';C',D')}}{\abs{(A,B;C,D)}}=\frac{\abs{PA'}}{\abs{PC}}\cdot \frac{\abs{PB}}{\abs{PC'}}\cdot \frac{\abs{PD}}{\abs{PA'}}\cdot\frac{\abs{PD'}}{\abs{PB}}=\frac{\abs{PD}\cdot\abs{PD'}}{\abs{PC}\cdot \abs{PC'}}=1\,,
	\end{equation*}
	wie gewünscht. Im letzten Schritt haben wir, je nach Lage von $P$, den Sehnensatz oder den Sekantensatz benutzt.
\end{proof}

Um die bisher ziemlich trockene Theorie auf vortreffliche Weise zu illustrieren, werden wir jetzt den Satz von Pascal auf sehr elegante Weise beweisen.

\begin{satzmitnamen}[Satz von Pascal]
	Seien $A$,~$B$, $C$, $D$, $E$ und~$F$ sechs Punkte auf einem Kreis~$\omega$. Sei $P$ der Schnittpunkt von $AB$ und~$DE$, sei $Q$ der Schnittpunkt von $BC$ und~$EF$ und schließlich sei $R$ der Schnittpunkt von $CD$ und~$FA$. Dann liegen $P$,~$Q$ und~$R$ auf einer Geraden.
\end{satzmitnamen}

\begin{figure}[ht]
	\centering
	\begin{tikzpicture}[x=0.45cm,y=0.45cm]
		\clip (-3.25,-0.05) rectangle (20.25,12.7);
		\draw (5.3,9.241) circle (2.566);
		\coordinate (A) at (2.985,8.134);
		\coordinate (B) at (3.884,7.101);
		\coordinate (C) at (5.835,6.731);
		\coordinate (D) at (7.689,8.305);
		\coordinate (E) at (7.654,10.263);
		\coordinate (F) at (4.961,11.785);
		\coordinate (P) at (7.792,2.611);
		\coordinate (Q) at (17.972,4.433);
		\coordinate (R) at (-0.841,1.065);
		\coordinate (S) at (12.498,3.453);
		\draw [line width=0.3,shorten <=-3em,shorten >=-2ex] (A) to (P);
		\draw [line width=0.3,shorten <=-2em,shorten >=-2ex] (E) to (P);
		\draw [line width=0.3,shorten <=-4em,shorten >=-2ex] (B) to (Q);
		\draw [line width=0.3,shorten <=-1em,shorten >=-2ex] (F) to (Q);
		\draw [line width=0.3,shorten <=-1em,shorten >=-2ex] (F) to (R);
		\draw [line width=0.3,shorten <=-4em,shorten >=-2ex] (D) to (R);
		\draw [line width=0.3,dashed,shorten <=-2.5em,shorten >=-2.5em] (R) to (Q);
		\draw [line width=0.3,shorten <=-3em,shorten >=-2ex] (A) to (S);
		\draw [fill=black] (A) circle (2pt) node[shift={(187:2.75ex)}] {$A$};
		\draw [fill=black] (B) circle (2pt) node[shift={(230:2ex)}] {$B$};
		\draw [fill=black] (C) circle (2pt) node[shift={(270:2ex)}] {$C$};
		\draw [fill=black] (D) circle (2pt) node[shift={(335:2ex)}] {$D$};
		\draw [fill=black] (E) circle (2pt) node[shift={(40:2ex)}] {$E$};
		\draw [fill=black] (F) circle (2pt) node[shift={(280:2ex)}] {$F$};
		\draw [fill=white] (P) circle (2pt) node[shift={(232:2.75ex)}] {$P$};
		\draw [fill=white] (Q) circle (2pt) node[shift={(250:2.25ex)}] {$Q$};
		\draw [fill=white] (R) circle (2pt) node[shift={(300:2ex)}] {$R$};
		\draw [fill=black] (S) circle (2pt) node[shift={(250:2.25ex)}] {$S$};
		\node at (2.6,11) {$\omega$};
	\end{tikzpicture}
\end{figure}

\begin{proof}
	Seien $R'$~und~$R''$ die Schnittpunkte von $CD$ und~$FA$ mit~$PQ$. Wir werden $R'=R''$ und damit die Behauptung zeigen. Zu diesem Zweck betrachten wir die folgenden vier Abbildungen:
	\begin{itemize}
		\item Sei $\pi_C\colon PQ\to \omega$ die Projektion durch~$C$.
		\item Sei $\pi_P\colon \omega\to \omega$ die Projektion durch~$P$.
		\item Sei $\pi_Q\colon \omega\to \omega$ die Projektion durch~$Q$.
		\item Sei $\pi_A\colon \omega\to PQ$ die Projektion durch~$A$.
	\end{itemize}
	Wir haben gesehen, dass alle diese Abbildungen projektiv sind. Sei nun $S$ der Schnittpunkt von $AC$ und~$PQ$. Indem wir die obigen Abbildungen nacheinander anwenden, erhalten wir
	\begin{equation*}
		(P,Q;S,R')\overset{\raisebox{0.5ex}{$\scriptstyle\pi_C$}}{=}(PC\cap\omega,B;A,D)\overset{\raisebox{0.5ex}{$\scriptstyle\pi_P$}}{=}(C,A;B,E)\overset{\raisebox{0.5ex}{$\scriptstyle\pi_Q$}}{=}(B,QA\cap\omega;C,F)\overset{\raisebox{0.5ex}{$\scriptstyle\pi_A$}}{=}(P,Q;S,R'')
	\end{equation*} 
	(hierbei bezeichnet $PC\cap \omega$ den von~$C$ verschiedenen Schnittpunkt von~$PC$ mit~$\omega$ und analog bezeichnet $QA\cap \omega$ den von~$A$ verschiedenen Schnittpunkt von~$QA$ mit~$\omega$). Für jede reelle Zahl $\lambda\neq 0,1$ gibt es aber genau einen Punkt~$Y_\lambda$ auf~$PQ$, für den $(P,Q;S,Y_\lambda)=\lambda$ gilt (wir erlauben natürlich auch $Y_\lambda=\infty_{PQ}$). Also muss $R'=R''$ gelten und wir sind fertig.
\end{proof}

Mit dem gleichen Argument, mit dem wir im obigen Beweis $R'=R''$ schlussfolgern, lässt sich beweisen, dass zwei projektive Abbildungen, die in drei Punkten übereinstimmen, schon gleich sein müssen. Dieser Fakt ist sehr wichtig für die allgemeine Theorie von projektiven Abbildungen. Wir hätten den obigen Beweis auch so formulieren können, dass wir die Hintereinanderausführung $\pi_A\circ\pi_Q\circ\pi_P\circ\pi_C\colon PQ\to PQ$ betrachten und bemerken, dass diese die Punkte $P$,~$Q$ und~$S$ auf sich selbst abbildet, sodass sie mit der Identitätsabbildung $\mathrm{id}_{PQ}\colon PQ\to PQ$ übereinstimmen muss. Für Olympiadezwecke ist es aber meistens einfacher, direkt auf Doppelverhältnisjagd zu gehen. 

An dem Beweis des Satzes von Pascal ist unter anderem bemerkenswert, dass er vollkommen ohne Lagediskussion auskommt. Die Punkte $A$,~$B$, $C$, $D$, $E$ und~$F$ dürfen also in beliebiger Reihenfolge auf dem Kreis~$\omega$ liegen. Dank der projektiven Ebene müssen wir noch nicht einmal die Spezialfälle betrachten, dass einige der Geraden parallel sind, da in der projektiven Ebene auch in diesem Fall die entsprechenden Schnittpunkte definiert sind.

Außerdem lässt sich der Beweis leicht auf andere Situationen übertragen. Betrachte zum Beispiel den Fall, dass die sechs Punkte $A$,~$B$, $C$, $D$, $E$ und~$F$ nicht auf einem Kreis~$\omega$, sondern (abwechselnd) auf zwei Geraden $\ell_1$~und~$\ell_2$ liegen. Die Projektionen durch $C$,~$P$, $Q$ und~$A$ sind immer noch projektive Abbildungen $\pi_C\colon PQ\to \ell_2$, $\pi_P\colon \ell_2\to \ell_1$, $\pi_Q\colon \ell_1\to\ell_2$ und $\pi_A\colon \ell_2\to PQ$. Wir können also mit haargenau dem gleichen Argument auch den folgenden Satz zeigen:

\begin{satzmitnamen}[Satz von Pappos]
	Seien $A$,~$C$ und~$E$ drei Punkte auf einer Geraden~$\ell_1$ und $B$,~$D$ und~$F$ drei Punkte auf einer Geraden~$\ell_2$. Sei $P$ der Schnittpunkt von $AB$ und~$DE$, sei $Q$ der Schnittpunkt von $BC$ und~$EF$ und schließlich sei $R$ der Schnittpunkt von $CD$ und~$FA$. Dann liegen $P$,~$Q$ und~$R$ auf einer Geraden.
\end{satzmitnamen}

Die Sätze von Pascal und Pappos sind häufig nützlich in Olympiadeaufgaben, nicht zuletzt deshalb, weil sie nicht zum Standardrepertoire gehören und euch deshalb einen Vorteil verschaffen können.

Wir werden jetzt den Einsatz von Pascal und Pappos an zwei Beispielaufgaben demonstrieren.

\begin{aufgabe*}\label{aufgabe:InkreisPascal}
	Sei $ABC$ ein Dreieck mit Umkreis~$\Omega$. Ein Kreis~$\omega$ berühre die Strecke~$\overline{AB}$ in~$P$, die Strecke~$\overline{AC}$ in~$Q$ sowie den Umkreis~$\Omega$ von innen. Zeige, dass der Inkreismittelpunkt von $ABC$ auf der Geraden~$PQ$ liegt.
\end{aufgabe*}
\begin{figure}[ht]
	\centering
	\begin{tikzpicture}[x=0.45cm,y=0.45cm]
		\draw [line width=0.3] (8.382,6.366) circle (5.549);
		\draw [line width=0.3] (9.397,4.201) circle (3.158);
		\coordinate (A) at (10.536,11.48);
		\coordinate (B) at (4.411,2.491);
		\coordinate (C) at (12.972,3.248);
		\coordinate (I) at (9.606,5.538);
		\coordinate (M) at (3.797,9.491);
		\coordinate (N) at (13.703,7.941);
		\coordinate (P) at (6.788,5.979);
		\coordinate (Q) at (12.425,5.097);
		\coordinate (T) at (10.737,1.342);
		\draw (A) to (B) to (C) to cycle;
		\draw [line width=0.3] (T) to (M) to (C);
		\draw [line width=0.3] (T) to (N) to (B);
		\draw [dashed,line width=0.3] (P) to (Q);
		\draw [fill=black] (A) circle (2pt) node[shift={(80:2ex)}] {$A$};
		\draw [fill=black] (B) circle (2pt) node[shift={(220:2ex)}] {$B$};
		\draw [fill=black] (C) circle (2pt) node[shift={(330:2ex)}] {$C$};
		\draw [fill=white] (I) circle (2pt) node[shift={(85:2ex)}] {$I$};
		\draw [fill=black] (M) circle (2pt) node[shift={(145:2ex)}] {$M$};
		\draw [fill=black] (N) circle (2pt) node[shift={(30:2ex)}] {$N$};
		\draw [fill=white] (P) circle (2pt) node[shift={(190:2ex)}] {$P$};
		\draw [fill=white] (Q) circle (2pt) node[shift={(-10:2ex)}] {$Q$};
		\draw [fill=white] (T) circle (2pt) node[shift={(290:2ex)}] {$T$};
		\node at (8.1,2) {$\omega$};
		\node at (6.5,10.8) {$\Omega$};
	\end{tikzpicture}
\end{figure}

\begin{proof}[Lösung]
	Sei $T$ der Berührpunkt der Kreise $\omega$~und~$\Omega$. Sei $M$ der von~$T$ verschiedene Schnittpunkt von~$TP$ mit~$\Omega$ und sei $N$ der von~$T$ verschiedene Schnittpunkt von~$TQ$ mit~$\Omega$. Nach dem Kreisberührungslemma ist $M$ der Mittelpunkt des Bogens~$\wideparen{AB}$ und $N$ der Mittelpunkt des Bogens~$\wideparen{CA}$. Dieses Lemma findet ihr zum Beispiel im Kapitel \emph{\embrace{Dreh-}Streckungen} im Heft für Klasse~10 (und ihr solltet es auf jeden Fall kennen); es lässt sich leicht beweisen, indem ihr die Streckung mit Zentrum~$T$ betrachtet, die $\omega$~auf~$\Omega$ abbildet. Insbesondere sind $BN$ und~$CM$ die Winkelhalbierenden von $\winkel CBA$ und $\winkel ACB$ und schneiden sich im Inkreismittelpunkt von $ABC$. Nach dem Satz von Pascal im Sehnensechseck $ABNTMC$ liegen $P$,~$Q$ und der Schnittpunkt von $BN$ und~$CM$ auf einer Geraden. Damit sind wir schon fertig.
\end{proof}


\begin{aufgabe*}[*]\label{aufgabe:VAIMO2020_2}
	Sei $ABCD$ ein Parallelogramm mit $\abs{AC}=\abs{AD}$. Sei $P$ ein Punkt auf der Verlängerung von~$\overline{AB}$ über~$B$ hinaus. Der Umkreis $\odot ACD$ und die Strecke~$\overline{PD}$ schneiden sich außer in~$D$ noch in einem Punkt~$Q$. Der Umkreis $\odot APQ$ und die Strecke~$\overline{PC}$ schneiden sich außer in~$P$ noch noch in einem Punkt~$R$. Beweise, dass sich die drei Geraden $CD$, $AQ$ und~$BR$ in einem Punkt schneiden.
\end{aufgabe*}

\begin{figure}[ht]
	\centering
	\begin{tikzpicture}[x=0.5cm,y=0.5cm]
		\draw (8.241,5.617) circle (4.279);
		\draw [shift={(15.121,0.688)}] (-5:6.174) arc (-5:185:6.174);
		\draw (12.884,9.2) circle (2.965);
		\coordinate (A) at (8.988,1.404);
		\coordinate (B) at (13.631,2.227);
		\coordinate (C) at (9.939,9.544);
		\coordinate (D) at (5.296,8.721);
		\coordinate (M) at (9.464,5.474);
		\coordinate (P) at (20.633,3.469);
		\coordinate (Q) at (12.47,6.264);
		\coordinate (R) at (14.687,6.847);
		\coordinate (S) at (15.531,10.536);
		\draw [shorten <=-6em,shorten >=-3em] (A) to (P);
		\draw [shorten <=-3em,shorten >=-6em] (D) to (S);
		\draw [line width=0.3] (A) to (S) to (B) to (D) to (P) to (C) to (Q);
		\draw [line width=0.3,shorten <=-1.65em,shorten >=-1.65em] (C) to (A); 
		\draw [line width=0.3,dashed,shorten <=-3em,shorten >=-3em] (R) to (M);
		\draw [line width=0.3, shift={(R)}] (150.398:0.32cm) arc (150.398:194.721:0.32cm);
		\draw [line width=0.3, shift={(R)}] (150.398:0.37cm) arc (150.398:194.721:0.37cm);
		\draw [line width=0.3, shift={(A)}] (10.054:0.32cm) arc (10.054:54.377:0.32cm);
		\draw [line width=0.3, shift={(A)}] (10.054:0.37cm) arc (10.054:54.377:0.37cm);
		\draw [line width=0.3, shift={(C)}] (263.335:0.32cm) arc (263.335:307.658:0.32cm);
		\draw [line width=0.3, shift={(C)}] (263.335:0.37cm) arc (263.335:307.658:0.37cm);
		\draw [line width=0.3, shift={(D)}] (-18.904:0.42cm) arc (-18.904:10.054:0.42cm);
		\draw [line width=0.3, shift={(A)}] (54.377:0.47cm) arc (54.377:83.335:0.47cm);
		\draw [line width=0.3, shift={(P)}] (161.096:0.42cm) arc (161.096:190.054:0.42cm);
		\draw [fill=black] (A) circle (2pt) node[shift={(225:2.5ex)}] {$A$};
		\draw [fill=black] (B) circle (2pt) node[shift={(260:2ex)}] {$B$};
		\draw [fill=black] (C) circle (2pt) node[shift={(121:2.5ex)}] {$C$};
		\draw [fill=black] (D) circle (2pt) node[shift={(125:2ex)}] {$D$};
		\draw [fill=black] (M) circle (2pt) node[shift={(228:2.75ex)}] {$M$};
		\draw [fill=black] (P) circle (2pt) node[shift={(240:2.25ex)}] {$P$};
		\draw [fill=black] (Q) circle (2pt) node[shift={(305:2.75ex)}] {$Q$};
		\draw [fill=black] (R) circle (2pt) node[shift={(292:2.5ex)}] {$R$};
		\draw [fill=white] (S) circle (2pt) node[shift={(70:2ex)}] {$S$};
	\end{tikzpicture}
\end{figure}

\begin{proof}[Lösung]
	Sei $S$ der Schnittpunkt von $AQ$ und~$CD$ und sei $R'$ der Schnittpunkt von $BS$ mit~$PC$. Wir müssen also $R=R'$ zeigen. Nach dem Satz von Pappos im Sechseck $ASBDPC$ liegen $Q$,~$R'$ und der Schnittpunkt von $BD$ und~$AC$ auf einer Geraden. Bei letzterem handelt es sich offenbar um den Diagonalenschnittpunkt des Parallelogramms $ABCD$, also um den Mittelpunkt von~$\overline{AC}$. Um $R=R'$ zu zeigen, müssen wir also nur zeigen, dass auch $QR$ durch den Mittelpunkt der Strecke~$\overline{AC}$ verläuft.
	
	Dazu bemerken wir, dass $QR$ die Potenzgerade der Kreise $\odot APQ$ und $\odot QRC$ ist. Wenn wir zeigen können, dass $AC$ eine Tangente an beide Kreise ist, sind wir fertig, denn die Potenzgerade verläuft bekanntlich durch den Mittelpunkt der Strecke zwischen den beiden Berührpunkten. Das ist nun eine einfache Winkeljagd: Nach dem Peripheriewinkelsatz im Sehnenviereck $AQCD$ und dem Wechselwinkelsatz gilt $\winkel QAC=\winkel QDC=\winkel QPA$. Nach der Umkehrung des Sehnen-Tangentenwinkelsatzes ist $AC$ also eine Tangente an $\odot APQ$. Im Sehnenviereck $APRQ$ gilt $\winkel CRQ=180^\circ-\winkel QRP=\winkel PAQ$. Nach Voraussetzung ist das Dreieck $ACD$ gleichschenklig mit Spitze~$A$. Also ist die Parallele zu~$CD$ durch~$A$ eine Tangente an $\odot ACD$. Nach dem Sehnen-Tangentenwinkelsatz folgt $\winkel PAQ=\winkel ACQ$. Also gilt $\winkel CRQ=\winkel ACQ$ und nach der Umkehrung des Sehnen-Tangentenwinkelsatzes folgt, dass $AC$ auch eine Tangente an $\odot QRC$ ist. Wie oben erklärt wurde, sind wir damit fertig.
\end{proof}


Es mag bisher der Eindruck entstanden sein, dass alle projektiven Abbildungen durch Projektionen gegeben sind. Das ist tatsächlich \emph{nicht} der Fall, wie ihr in der folgenden Übungsaufgabe zeigen sollt.

\begin{aufgabe*}
	Seien $A$,~$B$, $C$ und~$D$ vier Punkte auf einer Gerade oder auf einem Kreis. Wir fassen $A$,~$B$, $C$ und~$D$ als Punkte in der komplexen Ebene auf und bezeichnen mit $a$,~$b$, $c$ und~$d$ die zugehörigen komplexen Zahlen.
	\begin{enumerate}
		\item Zeige, dass das Doppelverhältnis von $A$,~$B$, $C$ und~$D$ wie folgt gegeben ist:
		\begin{equation*}
			(A,B;C,D)=\frac{a-b}{c-b}\bigg/\frac{a-d}{d-b}\,.
		\end{equation*}
		\item Sei $\iota$ die Inversion am Einheitskreis in der komplexen Ebene. Sei $z\neq 0$ eine komplexe Zahl. Zeige $\iota(z)=1/\overline{z}$.
		\item Sei $\omega$ ein Kreis oder eine Gerade und sei $\omega'$ das Bild von~$\omega$ unter~$\iota$. Zeige, dass $\iota$ eine projektive Abbildung $\iota|_{\omega}\colon \omega\to\omega'$ induziert.
	\end{enumerate}
\end{aufgabe*}

\subsection*{Harmonische Punktepaare und der Satz vom vollständigen Vierseit}
\begin{definition}
	Seien $A$,~$B$, $C$ und~$D$ vier verschiedene Punkte auf einer Gerade oder einem Kreis. Wir sagen, dass $(A,B)$ und $(C,D)$ \emph{harmonische Punktepaare} sind, wenn $(A,B;C,D)=-1$ gilt.
\end{definition}

Für zwei beliebige Punktepaare hängt das Doppelverhältnis im Allgemeinen von der Reihenfolge der Punkte innerhalb der Punktepaare ab. Für harmonische Punktepaare ist das allerdings nicht der Fall: Wenn $(A,B)$ und $(C,D)$ harmonisch sind, dann sind auch $(B,A)$ und $(C,D)$ harmonisch; es gilt $(A,B;C,D)=-1=(B,A;C,D)$.

Im Unterkapitel zu Polaren werden wir sehen, wie sich harmonische Punktepaare auf einem Kreis charakterisieren lassen. Fürs Erste betrachten wir den Fall, dass die Punkte $A$,~$B$, $C$ und~$D$ auf einer Geraden liegen. Dann gibt es für jede reelle Zahl $\lambda>0$ genau zwei Punkte auf der Geraden~$AB$, die die Strecke~$\overline{AB}$ im (ungerichteten) Verhältnis~$\lambda$ teilen: Nämlich einen Punkt~$X_\lambda$ im Inneren von~$\overline{AB}$, für den (in gerichteten Streckenlängen) $AX_{\lambda}/X_{\lambda}B=\lambda$ gilt, und einen Punkt~$X_{-\lambda}$ außerhalb von~$\overline{AB}$, für den (in gerichteten Streckenlängen) mit $AX_{-\lambda}/X_{-\lambda}B=-\lambda$ gilt. Im Fall $\lambda=1$ gilt $X_{-\lambda}=\infty_{AB}$. Somit sind die Punktepaare $(A,B)$ und $(C,D)$ genau dann harmonisch, wenn es ein $\lambda >0$ gibt, sodass $C=X_\lambda$ und $D=X_{-\lambda}$ oder $C=X_{-\lambda}$ und $D=X_\lambda$ gilt.

Harmonische Punktepaare werden dadurch interessant, dass sie häufig auf natürliche Weise in euren Skizzen auftauchen. Sei zum Beispiel $ABC$ ein Dreieck und seien $W$,~$W'$ die Schnittpunkte der Innen- bzw.\ Außenwinkelhalbierenden von $\winkel ACB$ mit~$AB$. Dann teilen $W$ und $W'$ die Strecke $\overline{AB}$ bekanntlich im Verhältnis $\abs{CA}/\abs{BC}$. Aus den obigen Überlegungen folgt also, dass $(A,B)$ und $(W,W')$ harmonische Punktepaare sind. Für ein weiteres Beispiel betrachte den Mittelpunkt~$M$ einer Strecke~$\overline{AB}$; dann sind $(A,B)$ und $(M,\infty_{AB})$ harmonische Punktepaare. Ihr werdet in den Beispielaufgaben sehen, wie sich diese triviale Beobachtung auf nicht-triviale Weise ausnutzen lässt.

Die allgemeinste Quelle von harmonischen Punktepaaren ist aber durch den folgenden Satz gegeben:
\begin{satzmitnamen}[Satz vom vollständigen Vierseit]
	Gegeben sei ein vollständiges Vierseit, das heißt vier Geraden $a$,~$b$, $c$ und~$d$, die sich in sechs paarweise verschiedenen Punkten schneiden. Sei $P$ der Schnittpunkt von $a$~und~$b$ sowie $P'$~der Schnittpunkt von $c$~und~$d$. Sei $Q$ der Schnittpunkt von $a$~und~$c$ sowie $Q'$~der Schnittpunkt von $b$~und~$d$. Sei $R$ der Schnittpunkt von $a$~und~$d$ sowie $R'$~der Schnittpunkt von $b$~und~$c$. Schließlich sei $K$ der Schnittpunkt von $QQ'$ und~$RR'$, $L$~der Schnittpunkt von $RR'$ und~$PP'$ sowie $M$~der Schnittpunkt von $PP'$ und~$QQ'$. Dann sind $(P,P')$ und $(L,M)$ harmonische Punktepaare. Ebenso sind $(Q,Q')$ und $(M,K)$ sowie $(R,R')$ und $(K,L)$ harmonische Punktepaare.
\end{satzmitnamen}

\begin{figure}[ht]
	\centering
	\begin{tikzpicture}[x=0.4cm,y=0.4cm]
		\clip (-5.1,-4.45) rectangle (14.62,11.65);
		\coordinate (P) at (5.195,-0.484);
		\coordinate (P1) at (-2.588,-2.008);
		\coordinate (Q) at (3.667,2.851);
		\coordinate (Q1) at (-0.802,3.903);
		\coordinate (R) at (0.782,9.146);
		\coordinate (R1) at (2.197,1.709);
		\coordinate (K) at (1.901,3.267);
		\coordinate (L) at (2.707,-0.971);
		\coordinate (M) at (12.093,0.867);
		\draw [shorten <=-2.25em,shorten >=-2.25em] (P) to (R);
		\draw [shorten <=-2.25em,shorten >=-2.25em] (P1) to (R);
		\draw [shorten <=-2.25em,shorten >=-2.25em] (P) to (Q1);
		\draw [shorten <=-2.25em,shorten >=-4em] (P1) to (Q);
		\draw [line width=0.3,dashed,shorten <=-2.25em,shorten >=-2.0em] (M) to (P1);
		\draw [line width=0.3,dashed,shorten <=-2.25em,shorten >=-2.15em] (M) to (Q1);
		\draw [line width=0.3,dashed,shorten <=-2.25em,shorten >=-2.25em] (R) to (L);
		\draw [fill=black] (P) circle (2pt) node[shift={(240:2.25ex)}] {$P$};
		\draw [fill=black] (P1) circle (2pt) node[shift={(320:2ex)}] {$P'$};
		\draw [fill=black] (Q) circle (2pt) node[shift={(75:2.5ex)}] {$Q$};
		\draw [fill=black] (Q1) circle (2pt) node[shift={(220:2.5ex)}] {$Q'$};
		\draw [fill=black] (R) circle (2pt) node[shift={(190:2ex)}] {$R$};
		\draw [fill=black] (R1) circle (2pt) node[shift={(180:2.75ex)}] {$R'$};
		\draw [fill=white] (K) circle (2pt) node[shift={(135:3ex)}] {$K$};
		\draw [fill=white] (L) circle (2pt) node[shift={(230:2.3ex)}] {$L$};
		\draw [fill=white] (M) circle (2pt) node[shift={(260:2ex)}] {$M$};
		\node at (-0.5,10.5) {$a$};
		\node at (6.8,-0.85) {$b$};
		\node at (6,5.35) {$c$};
		\node at (1.8,10.63) {$d$};
	\end{tikzpicture}
\end{figure}

\begin{proof}
	Der Satz lässt sich leicht durch eine Kombination der Sätze von Ceva und Menelaos beweisen. Mit projektiven Abbildungen ergibt sich aber eine noch kürzere Lösung. Indem wir die Gerade~$PP'$ durch~$R$ auf die Gerade~$QQ'$ projizieren, erhalten wir $(P,P';L,M)=(Q,Q';K,M)$. Indem wir die Gerade~$QQ'$ durch~$R'$ auf die Gerade~$PP'$ projizieren, erhalten wir $(Q,Q';K,M)=(P',P;L,M)$. Also ist $(P,P';L,M)=(P',P;L,M)$. Andererseits gilt ganz allgemein $(P',P;L,M)=1/(P,P';L,M)$. Also muss $(P,P';L,M)^2=1$ gelten. Da Doppelverhältnisse nie den Wert~$1$ annehmen, kommt nur $(P,P';L,M)=-1$ in Frage und es folgt, dass $(P,P')$ und $(L,M)$ in der Tat harmonisch sind. Die anderen beiden Behauptungen folgen völlig analog (oder durch geeignete Projektion).
\end{proof}

Wenn $(A,B)$ und $(C,D)$ harmonische Punktepaare sind, dann erfüllt der Mittelpunkt von $\overline{AB}$ einige sehr nützliche Relationen.

\begin{satzmitnamen}[Lemma]
	Seien $(A,B)$ und $(C,D)$ harmonische Punktepaare auf einer Geraden. Sei $M$ der Mittelpunkt der Strecke $\overline{CD}$. Dann gilt \embrace{in gerichteten Streckenlängen}
	\begin{equation*}
		AB\cdot AM=AC\cdot AD\quad\text{und}\quad AM\cdot BM=CM^2=DM^2\,.
	\end{equation*}
\end{satzmitnamen}

\begin{proof}
	Der Beweis ist eine formale Rechnung. Die Bedingung $(A,B;C,D)=-1$ impliziert $AC\cdot DB=-AD\cdot CB$. Indem wir $DB=AB-AD$ und $-CB=AC-AB$ einsetzen, erhalten wir
	\begin{equation*}
		AC\,(AB-AD)=AD\,(AC-AB)\quad\iff\quad (AC+AD)\,AB=2AC\cdot AD\,.
	\end{equation*}
	Weil $M$ der Mittelpunkt von~$\overline{CD}$ ist, gilt $AC+AD=2AM$ und die erste der beiden behaupteten Gleichheiten folgt. Andererseits gilt $AC=AM+MC$ und $AD=AM+MD=AM-MC$. Also
	\begin{equation*}
		AC\cdot AD=(AM+MC)\,(AM-MC)=AM^2-MC^2=AM^2-CM^2\,.
	\end{equation*}
	Ebenso ist $AB=AM-BM$ und somit $AB\cdot AM=(AM-BM)\,AM=AM^2-AM\cdot BM$. Indem wir die erste der beiden behaupteten Gleichheiten einsetzen, folgt die zweite.
\end{proof}

Wir werden nun harmonische Punktepaare, projektive Abbildungen, den Satz vom vollständigen Vierseit sowie das obige Lemma benutzen, um zwei Beispielaufgaben zu lösen.

\begin{aufgabe*}[*]\label{aufgabe:BWM}
	Sei $ABC$ ein nicht-gleichschenkliges Dreieck mit Inkreis $\omega$. Die Winkelhalbierende von $\winkel BAC$ schneide $BC$ in~$W$. Der Lotfußpunkt von $A$ auf $BC$ sei $L$. Schließlich sei $M$ der Mittelpunkt von~$\overline{BC}$. Die von~$BC$ verschiedene Tangente an~$\omega$ durch~$M$ berühre den Kreis~$\omega$ in~$T$. Beweise, dass $\winkel MTW=\winkel TLM$.
\end{aufgabe*}

	\begin{figure}[ht]
	\centering
	\begin{tikzpicture}[x=0.5cm,y=0.5cm]
		\clip (-3.93,-7.8) rectangle (11.21,8.93);
		\draw [line width=0.3] (5.824,2.18) coordinate (I) circle (2.18);
		\draw [line width=0.3,shift={(1.859,-4.966)}] (-24:4.966) arc (-24:61:4.966);
		\draw [line width=0.3,shift={(1.859,-4.966)}] (72:4.966) arc (72:107:4.966);
		\draw [line width=0.3,shift={(1.859,-4.966)}] (115.5:4.966) arc (115.5:204:4.966);
		\coordinate (A) at (8.927,7.774);
		\coordinate (B) at (0,0);
		\coordinate (C) at (7.683,0);
		\coordinate (D) at (5.824,0);
		\coordinate (E) at (1.859,0);
		\coordinate (L) at (8.927,0);
		\coordinate (M) at (3.842,0);
		\coordinate (N) at (5.824,4.361);
		\coordinate (T) at (3.653,1.973);
		\coordinate (W) at (4.614,0);
		\draw [shorten >=-6em] (A) to (B);
		\draw [shorten >=-9.175em] (A) to (C);
		\draw [shorten <=-2em,shorten >=-4em] (B) to (C);
		\draw [line width=0.3,dashed] (A) to (E);
		\draw [line width=0.3,dashed] (A) to (W);
		\draw [line width=0.3,dashed] (A) to (L);
		\draw [line width=0.3] (W) to (T) to (L);
		\draw [line width=0.3,shorten >=-4em] (M) to (T);
		\draw [shift={(L)},line width=0.3] (159.484:0.47cm) arc (159.484:180:0.47cm);
		\draw [shift={(L)},line width=0.3] (159.484:0.42cm) arc (159.484:180:0.42cm);
		\draw [shift={(T)},line width=0.3] (275.445:0.47cm) arc (275.445:295.961:0.47cm);
		\draw [shift={(T)},line width=0.3] (275.445:0.42cm) arc (275.445:295.961:0.42cm);
		\draw [shift={(L)},line width=0.3] (0:0.32cm) arc (0:90:0.32cm);
		\fill [shift={(L)}] (45:0.18cm) circle (1pt);
		\draw [shift={(D)},line width=0.3] (0:0.32cm) arc (0:90:0.32cm);
		\fill [shift={(D)}] (45:0.18cm) circle (1pt);
		\draw [line width=0.3] (D) to (N);
		\draw [fill=black] (A) circle (2pt) node[shift={(70:2ex)}] {$A$};
		\draw [fill=black] (B) circle (2pt) node[shift={(280:1.8ex)}] {$B$};
		\draw [fill=black] (C) circle (2pt) node[shift={(310:2.25ex)}] {$C$};
		\draw [fill=white] (D) circle (2pt) node[shift={(310:2.25ex)}] {$D$};
		\draw [fill=white] (E) circle (2pt) node[shift={(270:1.8ex)}] {$E$};
		\draw [fill=black] (I) circle (2pt) node[shift={(350:1.5ex)}] {$I$};
		\draw [fill=black] (L) circle (2pt) node[shift={(310:2.25ex)}] {$L$};
		\draw [fill=black] (M) circle (2pt) node[shift={(270:1.8ex)}] {$M$};
		\draw [fill=black] (N) circle (2pt) node[shift={(308:2.125ex)}] {$N$};
		\draw [fill=white] (T) circle (2pt) node[shift={(160:1.4ex)}] {$T$};
		\draw [fill=black] (W) circle (2pt) node[shift={(315:2.5ex)}] {$W$};
		\node at (7.3,3) {$\omega$};
	\end{tikzpicture}
\end{figure}

\begin{proof}[Lösung]
	Sei $D$ der Berührpunkt des Inkreises~$\omega$ mit~$\overline{BC}$ und sei $E$ der Berührpunkt des Ankreises gegenüber~$A$ mit~$\overline{BC}$. Schließlich sei $I$ der Inkreismittelpunkt und $N$ der Punkt auf~$\omega$, der $D$~gegenüber liegt. Bekanntlich sind $A$,~$N$ und~$E$ kollinear (wenn ihr diesen Fakt noch nicht kanntet, betrachtet die zentrische Streckung um~$A$, die den Inkreis auf den Ankreis abbildet). Außerdem ist bekannt, dass die Tangentenabschnitte am In-~und Ankreis gleich lang sind, sodass $M$ auch der Mittelpunkt von~$\overline{DE}$ ist.
	
	Nun ist $I$ der Mittelpunkt der Strecke~$\overline{DN}$, also sind $(D,N)$ und $(I,\infty_{DN})$ harmonische Punktepaare. Betrachte die Projektion $\pi_A\colon DN\to BC$ durch~$A$. Bei dieser Projektion werden offenbar $D$,~$I$ und~$N$ auf $D$,~$W$ und~$E$ abgebildet. Außerdem ist $A\infty_{DN}$ die Parallele zu~$DN$ durch~$A$. Weil $DN$ senkrecht auf~$BC$ steht, muss $A\infty_{DN}$ das Lot von~$A$ auf~$BC$ sein und es folgt $\pi_A(\infty_{DN})=L$. Also sind $(D,E)$ und $(W,L)$ harmonische Punktepaare. Weil $M$ der Mittelpunkt von $\overline{DE}$ ist, zeigt das Lemma $WM\cdot LM=DM^2$. Weil die Tangentenabschnitte $\overline{DM}$ und~$\overline{TM}$ gleich lang sind, gilt außerdem $DM^2=TM^2$. Indem wir zu ungerichteten Streckenlängen übergehen, folgt $\abs{MW}\cdot\abs{ML}=\abs{MT}^2$ und nach Umstellen $\abs{MW}/\abs{MT}=\abs{MT}/\abs{ML}$. Die Dreiecke $MWT$ und $MLT$ haben den Winkel $\winkel WMT=\winkel LMT$ gemeinsam und stimmen im Verhältnis der anliegenden Seiten überein. Also sind sie ähnlich und es folgt wie gewünscht $\winkel MTW=\winkel TLM$.
\end{proof}

Die Skizze legt außerdem nahe, dass $T$ auf der Geraden durch $A$,~$N$ und~$E$ liegt. Das ist in der Tat der Fall und ihr sollt es in Übungsaufgabe~\ref{aufgabe:BWM2} zeigen.

\begin{aufgabe*}[*]\label{aufgabe:VAIMO2008_5}
	Sei $ABCD$ ein konvexes Sehnenviereck. Die Diagonalen $AC$ und~$BD$ schneiden sich in~$E$ und die Geraden $AD$ und~$BC$ schneiden sich in~$F$. Seien $M$~und~$N$ die Mittelpunkte der Seiten $\overline{AB}$ und~$\overline{CD}$. Zeige, dass die Gerade~$EF$ den Umkreis $\odot ENM$ berührt.
\end{aufgabe*}

Aufgabe~\ref{aufgabe:VAIMO2008_5} wird wesentlich machbarer, wenn ihr den folgenden Satz kennt.

\begin{satzmitnamen}[Satz von der Gauß-Newton Geraden]
	Gegeben sei ein vollständiges Vierseit, das von den Geraden $a$,~$b$, $c$ und~$d$ sowie ihren Schnittpunkten $P$,~$Q$,~$R$ und $P'$,~$Q'$,~$R'$ gebildet wird \embrace{wobei die Bezeichnungen wie im Satz vom vollständigen Vierseit gewählt werden}. Seien $L$,~$M$ und~$N$ die Mittelpunkte der Diagonalen $\overline{PP'}$, $\overline{QQ'}$ und~$\overline{RR'}$. Dann sind $L$,~$M$ und~$N$ kollinear.
\end{satzmitnamen}

\begin{figure}[ht]
	\centering
	\begin{tikzpicture}[x=0.5cm,y=0.5cm]
		%\clip (-4.8,-4.15) rectangle (14.32,11.35);
		\coordinate (A) at (2.45,2.301);
		\coordinate (B) at (9.321,5.65);
		\coordinate (C) at (3.573,9.04);
		\coordinate (C1) at (8.198,-1.09);
		\coordinate (P) at (-3.381,-0.541);
		\coordinate (P1) at (6.125,1.909);
		\coordinate (Q) at (2.72,3.918);
		\coordinate (Q1) at (5.856,0.292);
		\coordinate (R) at (6.979,7.031);
		\coordinate (R1) at (8.467,0.527);
		\coordinate (L) at (1.372,0.684);
		\coordinate (M) at (4.288,2.105);
		\coordinate (N) at (7.723,3.779);
		\draw [line width=0.3,shorten <=-2ex,shorten >=-2em] (P) to (R);
		\draw [line width=0.3,shorten <=-2ex,shorten >=-2em] (P) to (R1);
		\draw [line width=0.3,shorten <=-2ex,shorten >=-2em] (P) to (B);
		\draw (A) to (C1) to (B) to (C) to cycle; 
		\draw [line width=0.3] (Q1) to (R);
		\draw [line width=0.3] (Q) to (R1);
		\draw [line width=0.3,dashed] (P) to (P1);
		\draw [line width=0.3,dashed,dash phase=3.5] (Q) to (Q1);
		\draw [line width=0.3,dashed] (R) to (R1);
		\draw [line width=0.3,dashed,shorten <=-4.4em,shorten >=-4.4em] (N) to (L);
		\draw [fill=black] (A) circle (2pt) node[shift={(240:2ex)}] {$A$};
		\draw [fill=black] (B) circle (2pt) node[shift={(-30:1.75ex)}] {$B$};
		\draw [fill=black] (C) circle (2pt) node[shift={(120:2ex)}] {$C$};
		\draw [fill=black] (C1) circle (2pt) node[shift={(300:2ex)}] {$C'$};
		\draw [fill=black] (P) circle (2pt) node[shift={(270:2ex)}] {$P$};
		\draw [fill=black] (P1) circle (2pt) node[shift={(30:2.25ex)}] {$P'$};
		\draw [fill=black] (Q) circle (2pt) node[shift={(145:2ex)}] {$Q$};
		\draw [fill=black] (Q1) circle (2pt) node[shift={(250:2.25ex)}] {$Q'$};
		\draw [fill=black] (R) circle (2pt) node[shift={(90:2ex)}] {$R$};
		\draw [fill=black] (R1) circle (2pt) node[shift={(320:2ex)}] {$R'$};
		\draw [fill=white] (L) circle (2pt) node[shift={(140:1.65ex)}] {$L$};
		\draw [fill=white] (M) circle (2pt) node[shift={(168:2.125ex)}] {$M$};
		\draw [fill=white] (N) circle (2pt) node[shift={(160:1.8ex)}] {$N$};
	\end{tikzpicture}
\end{figure}
\begin{proof}
	Betrachte die Streckung mit Zentrum~$P'$ und Faktor~$2$. Diese Streckung bildet den Mittelpunkt~$L$ von~$\overline{PP'}$ auf~$P$ ab. Der Mittelpunkt~$M$ von~$\overline{QQ'}$ wird auf denjenigen Punkt~$A$ abgebildet, für den $AQ'P'Q$ ein Parallelogramm ist und der Mittelpunkt~$N$ von~$\overline{RR'}$ wird auf denjenigen Punkt~$B$ abgebildet, für den $P'R'BR$ ein Parallelogramm ist.
	% Tien: Vielleicht willst du noch als Begründung hinschreiben, dass a priori bekannt ist das P auf QR und Q'R' liegt.
	Es genügt also, zu zeigen, dass $P$,~$A$ und~$B$ kollinear sind. Dazu sei $P_1$ der Schnittpunkt von~$QR$ mit~$AB$ und $P_2$~der Schnittpunkt von~$Q'R'$ mit~$AB$. Wir müssen dann $P_1=P_2$ zeigen.
	
	Zu diesem Zweck betrachte den Schnittpunkt~$C$ von $AQ$ und~$BR$ sowie den Schnittpunkt~$C'$ von $AQ'$ und~$BR'$. Nach dem Satz von Menelaos in den Dreiecken $ABC$ und $AC'B$ gilt
	\begin{equation*}
		\frac{AP_1}{P_1B}=-\frac{QA}{CQ}\cdot \frac{RC}{BR}\quad \text{und}\quad \frac{AP_2}{P_2B}=-\frac{Q'A}{C'Q'}\cdot \frac{R'C'}{BR'}\,.
	\end{equation*}
	Nach Konstruktion gilt aber $AC\parallel Q'R\parallel C'B$ und $AC'\parallel QR'\parallel CB$. Insbesondere ist $AC'R'Q$ ein Parallelogramm und es gilt $QA=R'C'$. Analog gilt $CQ=BR'$, $RC=Q'A$ und $BR=C'Q'$. Indem wir dies in die obigen Gleichungen einsetzen, folgt $AP_1/P_1B=AP_2/P_2B$. Für jede reelle Zahl $\lambda\neq 0$ gibt es aber genau einen Punkt~$X_\lambda$ auf~$AB$, für den (in gerichteten Streckenlängen) $AX_\lambda/X_\lambda B=\lambda$ gilt. Also muss $P_1=P_2$ gelten und wir sind fertig.
\end{proof}
\begin{figure}[ht]
	\centering
	\begin{tikzpicture}[x=0.5cm,y=0.5cm]
		\clip (-1.65,-3.4) rectangle (11.65,11.15);
		\draw [line width=0.3,shift={(5.516,5.616)}] (-95.5:4.38) arc (-95.5:234.5:4.38);
		\draw [line width=0.3,shift={(5.516,5.616)}] (244.5:4.38) arc (244.5:255.5:4.38);
		\draw [line width=0.3,shift={(3.497,5.898)}] (-87:3.876) arc (-87:262:3.876);
		\draw [line width=0.3,dashed] (5.411,4.498) circle (2.765);
		\coordinate (A) at (2.108,2.864);
		\coordinate (B) at (6.076,1.272);
		\coordinate (C) at (9.018,2.985);
		\coordinate (D) at (5.637,9.995);
		\coordinate (E) at (5.993,2.932);
		\coordinate (F) at (-0.606,-2.62);
		\coordinate (G) at (10.749,-0.603);
		\coordinate (L) at (2.694,0.156);
		\coordinate (M) at (4.092,2.068);
		\coordinate (N) at (7.328,6.49);
		\coordinate (P) at (4.684,1.831);
		\coordinate (Q) at (8.163,4.758);
		\draw (A) to (B) to (C) to (D) to cycle;
		\draw [line width=0.3] (A) to (C);
		\draw [line width=0.3] (B) to (D);
		\draw [line width=0.3,shorten >=-2ex] (A) to (F);
		\draw [line width=0.3,shorten >=-2ex] (B) to (F);
		\draw [line width=0.3,shorten >=-2ex] (B) to (G);
		\draw [line width=0.3,shorten >=-2ex] (C) to (G);
		\draw [line width=0.3,shorten >=-2ex] (Q) to (F);
		\draw [dashed,line width=0.3,shorten >=-4.4em,shorten <=-4.3em] (N) to (L);
		\draw [fill=black] (A) circle (2pt) node[shift={(180:2ex)}] {$A$};
		\draw [fill=black] (B) circle (2pt) node[shift={(270:2ex)}] {$B$};
		\draw [fill=black] (C) circle (2pt) node[shift={(350:2ex)}] {$C$};
		\draw [fill=black] (D) circle (2pt) node[shift={(90:2ex)}] {$D$};
		\draw [fill=white] (E) circle (2pt) node[shift={(310:2ex)}] {$E$};
		\draw [fill=black] (F) circle (2pt) node[shift={(150:2ex)}] {$F$};
		\draw [fill=black] (G) circle (2pt) node[shift={(50:2ex)}] {$G$};
		\draw [fill=white] (L) circle (2pt) node[shift={(135:2ex)}] {$L$};
		\draw [fill=white] (M) circle (2pt) node[shift={(205:2.5ex)}] {$M$};
		\draw [fill=white] (N) circle (2pt) node[shift={(8:2ex)}] {$N$};
		\draw [fill=white] (P) circle (2pt) node[shift={(270:2ex)}] {$P$};
		\draw [fill=black] (Q) circle (2pt) node[shift={(25:2ex)}] {$Q$};
	\end{tikzpicture}
\end{figure}
\begin{proof}[Lösung zu Aufgabe~\ref{aufgabe:VAIMO2008_5}]
	Betrachte außerdem den Schnittpunkt~$G$ von $AB$ und~$CD$ sowie die Schnittpunkte $P$~und~$Q$ von~$EF$ mit $AB$ und~$CD$. Nach dem Satz vom vollständigen Vierseit (angewendet auf das Vierseit, das von $CF$, $DF$, $AC$ und $BD$ gebildet wird)
	% Tien: Vielleicht willst du noch schreiben, dass du das Vierseit FC, FD, AC, BD betrachtest.
	sind $(A,B)$ und $(P,G)$ sowie $(C,D)$ und $(Q,G)$ harmonische Punktepaare. Nach dem Lemma gilt $GP\cdot GM=GA\cdot GB$ und $GQ\cdot GN=GC\cdot GD$. Nach dem Sekantensatz gilt aber auch $GA\cdot GB=GC\cdot GD$. Also ist $GP\cdot GM=GQ\cdot GN$ und nach der Umkehrung des Sekantensatzes folgt, dass $PQNM$ ein Sehnenviereck ist.
	
	Sei nun $L$ der Mittelpunkt von $\overline{EF}$. Nach dem Satz von der Gauß-Newton-Geraden liegt $L$ auf der Geraden $MN$. Nach dem Satz vom vollständigen Vierseit (im gleichen Vierseit wie vorhin) sind $(E,F)$ und $(P,Q)$ harmonische Punktepaare. Nach dem Lemma über harmonische Punktepaare gilt somit $PL\cdot QL=EL^2$. Nach dem Sekantensatz ist $ML\cdot NL=PL\cdot QL$. Also folgt $EL^2=ML\cdot NL$ und die Umkehrung des Sekanten-Tangentensatzes liefert uns, dass $EF$ in der Tat eine Tangente an $\odot ENM$ ist. Damit sind wir fertig.
\end{proof}
Eine etwas einfachere Aufgabe, in der ähnliche Methoden zum Einsatz kommen, ist die folgende Übungsaufgabe.
\begin{aufgabe*}
	Sei $ABC$ ein Dreieck. Ein Kreis $\omega$ verläuft durch $B$ und $C$ und schneidet $AB$ in $Y$ sowie $AC$ in $Z$. Sei $P$ der Schnittpunkt von $BZ$ und $CY$ und sei $X$ der Schnittpunkt von $AP$ mit $BC$. Beweise, dass der Mittelpunkt der Strecke $\overline{BC}$ auf dem Umkreis $\odot XYZ$ liegt.
\end{aufgabe*}


\subsection*{Polaren}
%In diesem Kapitel lernt ihr eine Konstruktion kennen, mit der sich die das Dualitätsprinzip auf konkrete Weise verwirklichen lässt.
\begin{definition}
	Sei $\omega$ ein Kreis mit Mittelpunkt~$O$ und Radius~$r$. Sei $P$ ein von~$O$ verschiedener Punkt und sei $P'$ das Bild von~$P$ unter Inversion an~$\omega$, also derjenige Punkt auf dem Strahl~$\overrightarrow{OP}$, für den $\abs{OP}\cdot \abs{OP'}=r^2$ gilt. Sei $p$ die Senkrechte auf~$OP$ in~$P'$. Dann wird $p$ die \emph{Polare von~$P$} genannt. Umgekehrt nennen wir $P$ den \emph{Pol von~$p$}.
	
	Polaren lassen sich auch für Fernpunkte definieren. Wenn $\infty_\ell$ der Fernpunkt auf der Geraden~$\ell$ ist, dann definieren wir die \emph{Polare von~$\infty_\ell$} als die Senkrechte~$p$ auf~$\ell$ durch~$O$. Wenn umgekehrt $p$ eine Gerade durch~$O$ ist, definieren wir den \emph{Pol von~$p$} als~$\infty_\ell$, wobei $\ell$ die Senkrechte auf~$p$ in~$O$ ist. Die Polare von~$O$ ist schließlich die Ferngerade und umgekehrt ist $O$ der Pol der Ferngeraden. 
\end{definition}

Der wichtigste Fakt über Pole und Polaren ist der folgende:
\begin{satzmitnamen}[Satz von La~Hire, Pol-Polare-Dualität]
	Ein Punkt~$P$ liegt auf der Polaren eines weiteren Punktes~$Q$ genau dann, wenn $Q$ auf der Polaren von~$P$ liegt.
\end{satzmitnamen}

\begin{figure}[ht]
	\centering
	\begin{tikzpicture}[x=2.35cm,y=2.35cm]
		\clip (-0.35,-0.382) rectangle (1.464,1.192);
		\draw[line width=0.3] (-22:1) arc (-22:110:1);
		\coordinate (O) at (0,0);
		\draw[line width=0.3,dashed] (0.978,0.521) circle (0.477);
		\coordinate (P1) at (1.22,0.11);
		\coordinate (P) at (0.81,0.07);
		\coordinate (Q) at (1.14,0.97);
		\coordinate (Q1) at (0.509,0.433);
		\draw [line width=0.3] (P1) to (O) to (Q);
		\draw[shorten <=-3ex,shorten >=-6.9ex] (Q) to (P1);
		\draw[shorten <=-13.75ex,shorten >=-8.15ex] (Q1) to (P);
		\draw[line width=0.3, shift={(P1)}] (95.1:0.32cm) arc (95.1:185.1:0.32cm);
		\fill[shift={(P1)}] (140.1:0.18cm) circle (1pt);
		\draw[line width=0.3, shift={(Q1)}] (310.29:0.32cm) arc (310.29:400.29:0.32cm);
		\fill[shift={(Q1)}] (355.29:0.18cm) circle (1pt);
		\draw[fill=white] (O) circle (2pt) node[shift={(220:2ex)}] {$O$};
		\draw[fill=black] (P) circle (2pt) node[shift={(240:2ex)}] {$P$};
		\draw[fill=black] (P1) circle (2pt) node[shift={(-20:2ex)}] {$P'$};
		\draw[fill=black] (Q) circle (2pt) node[shift={(40:2ex)}] {$Q$};
		\draw[fill=black] (Q1) circle (2pt) node[shift={(185:2.5ex)}] {$Q'$};
		\node at (-0.2,0.85) {$\omega$};
		\node at (0.075,0.75) {$q$};
		\node at (1.35,-0.25) {$p$};
	\end{tikzpicture}
\end{figure}

\begin{proof}
	Seien $P'$~und~$Q'$ die Bilder von $P$~und~$Q$ unter Inversion an~$\omega$. Angenommen, $Q$~liegt auf der Polaren von~$P$, also der Senkrechten auf~$OP$ in~$P'$. Nach Konstruktion gilt $\abs{OP}\cdot \abs{OP'}=r^2=\abs{OQ}\cdot \abs{OQ'}$. Nach der Umkehrung des Sekantensatzes muss $PP'QQ'$ ein Sehnenviereck sein. Mit $\winkel QP'P=90^\circ$ folgt also auch $\winkel PQ'Q=90^\circ$. Also liegt $P$ auf der Senkrechten auf~$OQ$ in~$Q'$, sprich der Polaren von~$Q$. Die andere Implikation lässt sich völlig analog beweisen.
\end{proof}

Mithilfe des Satzes von La~Hire lassen sich Pole und Polaren bestimmen (auch wenn das erstmal ziemlich verwirrend ist).
% Tien: Die Aussage in Klammern ist nicht vollständig. Hier fehlt wohl etwas.
% Tien: Ich habe den folgenden Abschnitt etwas (meiner Meinung nach klarer) umformuliert.
Wenn zum Beispiel $P$ der Schnittpunkt zweier Geraden $a$~und~$b$ mit den Polen $A$~und~$B$ ist, dann impliziert der Satz von La~Hire, dass $AB$ die Polare von~$P$ sein muss. Für einen gegebenen Punkt~$P$, der außerhalb von~$\omega$ liegt, können wir $A$~und~$B$ als die Berührpunkte der Tangenten an~$\omega$ durch~$P$ wählen. Weil $A$~und~$B$ auf~$\omega$ liegen, werden sie durch Inversion an~$\omega$ auf sich selbst abgebildet. Folglich sind die Polaren von $A$~und~$B$ die Tangenten an~$\omega$ in $A$~und~$B$. Ihr Schnittpunkt~$P$ muss also der Pol von~$AB$ sein und umgekehrt muss $AB$ die Polare von~$P$ sein.

\begin{figure}[ht]
	\centering
	\begin{tikzpicture}[x=1.65cm,y=1.65cm]
		\clip (-2.356,-1.209) rectangle (1.012,1.287);
		%\draw[line width=0.3] (-22:1) arc (-22:110:1);
		\draw [line width=0.3] (0,0) circle (1);
		\coordinate (A) at (-0.387,0.922);
		\coordinate (B) at (-0.533,-0.846);
		\coordinate (P) at (-2.159,0.178);
		\draw [line width=0.3,shorten <=-2ex,shorten >=-2em] (P) to (A);
		\draw [line width=0.3,shorten <=-2ex,shorten >=-2em] (P) to (B);
		\draw[shorten <=-1.5em,shorten >=-1.5em] (A) to (B);
		\draw[fill=black] (P) circle (2pt) node[shift={(260:2ex)}] {$P$};
		\draw[fill=black] (A) circle (2pt) node[shift={(140:2ex)}] {$A$};
		\draw[fill=black] (B) circle (2pt) node[shift={(210:2ex)}] {$B$};
		\node at (0.6,-0.6) {$\omega$};
		\node at (-0.625,0.05) {$p$};
	\end{tikzpicture}
\end{figure}

Der Satz von La~Hire lässt sich verwenden, um Aufgaben umzuformulieren (wie immer in der Hoffnung, dass die neue Aufgabe einfacher ist). Wenn in einer Aufgabe zum Beispiel zu zeigen ist, dass sich drei Geraden in einem Punkt schneidet, könnt ihr stattdessen die Pole der Geraden bezüglich eines geeigneten Kreises betrachten und beweisen, dass die drei Pole auf einer Geraden liegen. Ein perfektes Beispiel hierfür ist die folgende Aufgabe, die ohne diesen Trick wesentlich schwieriger wäre.

\begin{aufgabe*}[*]
	Sei $ABC$ ein Dreieck mit Inkreis~$\omega$ und Inkreismittelpunkt~$I$. Sei $t$ eine Tangente an~$\omega$, die nicht durch $A$,~$B$ oder~$C$ verläuft. Wähle Punkte $A'$,~$B'$ und~$C'$ auf~$t$, sodass $\winkel AIA'=90^\circ$, $\winkel BIB'=90^\circ$ und $\winkel CIC'=90^\circ$. Beweise, dass sich die Geraden $AA'$, $BB'$ und~$CC'$ in einem Punkt schneiden.
\end{aufgabe*}

\begin{figure}[ht]
	\centering
	\begin{tikzpicture}[x=1.25cm,y=1.25cm]
		\clip (-2.97,-1.705) rectangle (3.403,3.391);
		%\draw[line width=0.3] (-22:1) arc (-22:110:1);
		\draw [line width=0.3] (-0.206,1.865) coordinate (I) circle (1.031);
		\coordinate (A) at (-0.748,3.03);
		\coordinate (A1) at (-1.315,1.348);
		\coordinate (B) at (-2.672,-0.536);
		\coordinate (B1) at (2.904,-1.329);
		\coordinate (C) at (2.892,2.275);
		\coordinate (C1) at (-0.029,0.532);
		\coordinate (D) at (0.259,0.945);
		\coordinate (E) at (0.004,2.874);
		\coordinate (F) at (-1.113,2.354);
		\coordinate (T) at (-0.758,0.995);
		\draw (A) to (B) to (C) to cycle;
		\draw [line width=0.3,shorten <=-3em,shorten >=-2em,dashed] (A1) to (B1);
		\draw[line width=0.3] (A) to (I) to (A1);
		\draw[line width=0.3] (B) to (I) to (B1);
		\draw[line width=0.3] (C) to (I) to (C1);
		\draw [dashed,line width=0.3] (D) to (E) to (F) to cycle;
		\draw [shift={(I)},line width=0.3] (114.972:0.32cm) arc (114.972:204.972:0.32cm);
		\fill [shift={(I)}] (159.972:0.18cm) circle (1pt);
		\draw [shift={(I)},line width=0.3] (277.548:0.37cm) arc (277.548:367.548:0.37cm);
		\fill [shift={(I)}] (322.548:0.208cm) circle (1pt);
		\draw [shift={(I)},line width=0.3] (224.233:0.32cm) arc (224.233:314.233:0.32cm);
		\fill [shift={(I)}] (269.233:0.18cm) circle (1pt);
		\draw[fill=black] (A) circle (2pt) node[shift={(120:2ex)}] {$A$};
		\draw[fill=black] (B) circle (2pt) node[shift={(230:2ex)}] {$B$};
		\draw[fill=black] (C) circle (2pt) node[shift={(10:2ex)}] {$C$};
		\draw[fill=black] (D) circle (2pt) node[shift={(300:2ex)}] {$D$};
		\draw[fill=black] (E) circle (2pt) node[shift={(70:2ex)}] {$E$};
		\draw[fill=black] (F) circle (2pt) node[shift={(150:2ex)}] {$F$};
		\draw[fill=black] (I) circle (2pt) node[shift={(70:2ex)}] {$I$};
		\draw[fill=white] (T) circle (2pt) node[shift={(250:2ex)}] {$T$};
		\draw[fill=black] (A1) circle (2pt) node[shift={(250:2ex)}] {$A'$};
		\draw[fill=black] (B1) circle (2pt) node[shift={(250:2ex)}] {$B'$};
		\draw[fill=black] (C1) circle (2pt) node[shift={(250:2ex)}] {$C'$};
		\node at (0.45,2.3) {$\omega$};
		\node at (0.9,-0.35) {$t$};
	\end{tikzpicture}
\end{figure}

\begin{proof}[Lösung]
	Wir wollen stattdessen zeigen, dass die Pole von $AA'$, $BB'$ und~$CC'$ bezüglich~$\omega$ auf einer Geraden liegen. Was ist also der Pol~$X$ von~$AA'$ bezüglich~$\omega$? Nach dem Satz von La~Hire muss $X$ der Schnittpunkt der Polaren von~$A$ und der Polaren von~$A'$ sein. Wie wir weiter oben gesehen haben, können wir die Polare von~$A$ direkt als die Gerade~$EF$ identifizieren, wobei $E$~und~$F$ die Berührpunkte des Inkreises~$\omega$ mit den Seiten $\overline{CA}$ und~$\overline{AB}$ sind. Wir müssen also nur die Polare von~$A'$ identifizieren. Sei $T$ der Berührpunkt von~$t$ an~$\omega$. Weil $A'$ auf~$t$ liegt, also auf der Polaren von~$T$, muss $T$ auf der Polaren von~$A'$ liegen. Andererseits steht die Polare von~$A'$ senkrecht auf~$A'I$. Die Polare von~$A'$ ist also die Senkrechte von~$T$ auf~$A'I$. Nach Voraussetzung ist $A'I\perp AI$. Andererseits ist aber auch $EF\perp AI$. Die Senkrechte von~$T$ auf~$A'I$ ist also auch die Senkrechte von~$T$ auf~$EF$. Insgesamt folgt, dass der Pol von~$AA'$, also der Schnittpunkt der Polaren von $A$~und~$A'$, durch den Lotfußpunkt von~$T$ auf~$EF$ gegeben ist. Analog sind die Pole von $BB'$ und~$CC'$ die Lotfußpunkte von~$T$ auf $FD$ und~$DE$, wobei $D$ der Berührpunkt von~$\omega$ mit~$\overline{BC}$ ist. Wir müssen somit zeigen, dass diese drei Lotfußpunkte kollinear sind. Das ist aber genau der Satz von der Simson-Geraden, angewendet auf das Dreieck $DEF$ und den Punkt~$T$ auf seinem Umkreis.
\end{proof}

Harmonische Punktepaare auf Kreisen lassen sich sehr elegant durch Polaren charakterisieren.

\begin{satzmitnamen}[Lemma]
	Seien $A$,~$B$, $C$ und~$D$ vier verschiedene Punkte auf einem Kreis~$\omega$. Die Punktepaare $(A,B)$ und $(C,D)$ sind genau dann harmonisch, wenn der Pol~\(P\) von~$AB$ auf~$CD$ liegt. In diesem Fall sind auch die Punktepaare $(P,Q)$ und $(C,D)$ harmonisch, wobei $Q$ den Schnittpunkt von $AB$ und~$CD$ bezeichnet.
\end{satzmitnamen}

\begin{figure}[ht]
	\centering
	\begin{tikzpicture}[x=1.65cm,y=1.65cm]
		\clip (-2.372,-1.11) rectangle (1.012,1.15);
		%\draw[line width=0.3] (-22:1) arc (-22:110:1);
		\draw [line width=0.3] (0,0) circle (1);
		\coordinate (A) at (-0.387,0.922);
		\coordinate (B) at (-0.533,-0.846);
		\coordinate (C) at (-0.931,0.365);
		\coordinate (D) at (0.78,0.626);
		\coordinate (P) at (-2.159,0.178);
		\coordinate (Q) at (-0.427,0.442);
		\draw [shorten <=-2ex,shorten >=-2em] (P) to (A);
		\draw [shorten <=-2ex,shorten >=-2em] (P) to (B);
		\draw[shorten <=-2ex,shorten >=-2ex] (A) to (B);
		\draw [shorten <=-2ex,shorten >=-2ex] (P) to (D);
		\draw[fill=black] (P) circle (2pt) node[shift={(260:2ex)}] {$P$};
		\draw[fill=black] (Q) circle (2pt) node[shift={(310:2ex)}] {$Q$};
		\draw[fill=black] (A) circle (2pt) node[shift={(140:2ex)}] {$A$};
		\draw[fill=black] (B) circle (2pt) node[shift={(210:2ex)}] {$B$};
		\draw[fill=black] (C) circle (2pt) node[shift={(310:2ex)}] {$C$};
		\draw[fill=black] (D) circle (2pt) node[shift={(250:2ex)}] {$D$};
		\node at (0.6,-0.6) {$\omega$};
	\end{tikzpicture}
\end{figure}

\begin{proof}
	Der Pol von~$AB$ ist der Schnittpunkt der Tangenten an~$\omega$ in $A$~und~$B$. Sei $D'$ der von~$C$ verschiedene Schnittpunkt von~$PC$ mit~$\omega$. Wie wir in einem vorherigen Lemma gezeigt haben, ist die Projektion $\pi_P\colon \omega\to \omega$ durch~$P$ eine projektive Abbildung. Offenbar bildet $\pi_P$ die Punkte $A$~und~$B$ auf sich selbst ab und vertauscht $C$~und~$D'$. Also gilt $(A,B;C,D')=(A,B;D',C)$. Ganz allgemein gilt aber auch $(A,B;C,D')=1/(A,B;D',C)$. Also muss $(A,B;C,D')^2=1$ sein. Weil das Doppelverhältnis von vier verschiedenen Punkten nie den Wert~$1$ annimmt, kommt nur $(A,B;C,D)=-1$ in Frage und die Punktepaare $(A,B)$ und $(C,D')$ sind harmonisch. Wenn $P$ auf~$CD$ liegt, sodass $D=D'$ gilt, haben wir damit gezeigt, dass $(A,B)$ und $(C,D)$ harmonisch sind. Wenn umgekehrt $(A,B)$ und $(C,D)$ harmonisch sind, muss $D=D'$ gelten, denn für jede reelle Zahl $\lambda\neq 0,1$ gibt es genau einen Punkt~$Y_\lambda$ auf~$\omega$ mit $(A,B;C,Y_\lambda)=\lambda$. Also liegt $P$ in diesem Fall auf~$CD$. Damit ist gezeigt, dass $(A,B)$ und $(C,D)$ genau dann harmonische Punktepaare sind, wenn der Pol~$P$ von~$AB$ auf~$CD$ liegt.
	
	Wenn dies erfüllt ist, können wir die Projektion $\pi_A\colon \omega\to CD$ durch~$A$ betrachten. Diese Projektion bildet offenbar $C$~und~$D$ auf sich selbst ab und $B$~auf~$Q$. Ferner wird $A$ auf den Schnittpunkt der Tangente an~$\omega$ in~$A$ mit der Geraden~$CD$ abgebildet (denn die \enquote{Gerade~$AA$} interpretieren wir wie üblich als die Tangente an~$\omega$ in~$A$). Also gilt $\pi_A(A)=P$. Somit ist $(P,Q;C,D)=(A,B;C,D)=-1$, wie gewünscht.
\end{proof}

In der folgenden Übungsaufgabe sollt ihr Polaren benutzen, um mehr über die Situation von Aufgabe~\ref{aufgabe:BWM} herauszufinden.

\begin{aufgabe*}\label{aufgabe:BWM2}
	In der Situation von Aufgabe~\ref{aufgabe:BWM} sei $E$ der Berührpunkt des Ankreises gegenüber~$A$ mit~$\overline{BC}$. Zeige, dass $A$,~$E$ und~$T$ kollinear sind.
\end{aufgabe*}

Der nun folgende Satz kann sehr nützlich in Olympiadeaufgaben sein. Besonders die Aussage, dass $O$ der Höhenschnittpunkt von $PQR$ ist, ist alles andere als offensichtlich und wird gerne in schweren Aufgaben verwendet. Ihr solltet sie kennen!

\begin{satzmitnamen}[Satz von Brocard]
	Sei $ABCD$ ein Sehnenviereck mit Umkreis~$\omega$. Sei $P$ der Schnittpunkt von $AB$ und~$CD$, sei $Q$ der Schnittpunkt von $AC$ und~$BD$ und sei $R$ der Schnittpunkt von $AD$ und~$BC$. Dann ist $QR$ die Polare von~$P$, $RP$~die Polare von~$Q$ und $PQ$~die Polare von~$R$.
	
	Insbesondere folgt: Wenn $O$ der Umkreismittelpunkt ist, dann gilt $OP\perp QR$, $OQ\perp RP$ und $OR\perp PQ$ und somit ist $O$ der Höhenschnittpunkt des Dreiecks $PQR$.
\end{satzmitnamen}

\begin{figure}[ht]
	\centering
	\begin{tikzpicture}[x=1.125cm,y=1.125cm]
		%\clip (-2.49,-1.484) rectangle (5.224,5.035);
		\draw [line width=0.3] (2.11,2.896) coordinate (O) circle (1.899);
		\coordinate (A) at (0.569,1.787);
		\coordinate (B) at (2,1);
		\coordinate (C) at (3.765,1.965);
		\coordinate (D) at (2.907,4.619);
		\coordinate (P) at (4.526,-0.389);
		\coordinate (Q) at (2.22,1.879);
		\coordinate (R) at (-1.792,-1.072);
		\coordinate (X) at (1.097,1.053);
		\coordinate (Y) at (4.773,3.756);
		\coordinate (P1) at (2.634,2.183);
		\coordinate (Q1) at (2.489,-0.609);
		\coordinate (R1) at (1.656,2.434);
		\draw (B) to (D) to (R) to (C) to (P) to (A) to (C) to (D);
		\draw [dash phase=-1.2,line width=0.3,dashed] (O) to (P);
		\draw [line width=0.3,dashed] (Q1) to (O);
		\draw [dash phase=0.4,line width=0.3,dashed] (O) to (R);
		\draw [line width=0.3,shorten <=-2ex,shorten >=-2em] (X) to (A);
		\draw [line width=0.3,shorten <=-2ex,shorten >=-2em] (X) to (B);
		\draw [line width=0.3,shorten <=-2ex,shorten >=-2em] (Y) to (C);
		\draw [line width=0.3,shorten <=-2ex,shorten >=-2em] (Y) to (D);
		\draw [line width=0.3,shorten <=-2em,shorten >=-2em] (P) to (R);
		\draw [line width=0.3,shorten <=-2ex,shorten >=-2ex] (Y) to (R);
		\draw [line width=0.3,shorten <=-2ex,shorten >=-9em] (P) to (Q);
		\draw [line width=0.3,shift={(P1)}] (36.333:0.32cm) arc (36.333:126.333:0.32cm);
		\fill [shift={(P1)}] (81.333:0.18cm) circle (1pt);
		\draw [line width=0.3,shift={(Q1)}] (6.171:0.32cm) arc (6.171:96.171:0.32cm);
		\fill [shift={(Q1)}] (51.171:0.18cm) circle (1pt);
		\draw [line width=0.3,shift={(R1)}] (45.475:0.32cm) arc (45.475:135.475:0.32cm);
		\fill [shift={(R1)}] (90.475:0.18cm) circle (1pt);
		\draw[fill=black] (A) circle (2pt) node[shift={(170:2ex)}] {$A$};
		\draw[fill=black] (B) circle (2pt) node[shift={(270:2ex)}] {$B$};
		\draw[fill=black] (C) circle (2pt) node[shift={(-5:2ex)}] {$C$};
		\draw[fill=black] (D) circle (2pt) node[shift={(75:2ex)}] {$D$};
		\draw[fill=black] (O) circle (2pt) node[shift={(80:2ex)}] {$O$};
		\draw[fill=white] (P) circle (2pt) node[shift={(245:2ex)}] {$P$};
		\draw[fill=white] (Q) circle (2pt) node[shift={(335:3.25ex)}] {$Q$};
		\draw[fill=white] (R) circle (2pt) node[shift={(290:2ex)}] {$R$};
		\draw[fill=black] (X) circle (2pt) node[shift={(255:2.25ex)}] {$X$};
		\draw[fill=black] (Y) circle (2pt) node[shift={(290:2.25ex)}] {$Y$};
		\node at (1.15,4.25) {$\omega$};
	\end{tikzpicture}
\end{figure}

\begin{proof}
	Wir werden nur zeigen, dass $QR$ die Polare von~$P$ ist; die anderen beiden Aussagen sind völlig analog. Aus der Definition der Polaren folgt dann auch sofort $OP\perp QR$.
	
	Sei $X$ der Schnittpunkt der Tangenten an~$\omega$ in $A$~und~$B$ und sei $Y$ der Schnittpunkt der Tangenten an~$\omega$ in $C$~und~$D$. Dann ist $X$ der Pol von~$AB$ und $Y$~der Pol von~$CD$. Weil $P$ der Schnittpunkt von $AB$ und~$CD$ ist, muss nach dem Satz von La~Hire $XY$ die Polare von~$P$ sein. Wir müssen also nur zeigen, dass $Q$~und~$R$ auf~$XY$ liegen. Aus dem Satz von Pascal im entarteten Sehnensechseck $AACBBD$ folgt, dass $X$,~$Q$ und~$R$ kollinear sind (wobei wir die \enquote{Geraden $AA$ und~$BB$} wie üblich als die Tangenten an~$\omega$ in $A$~und~$B$ interpretieren). Analog folgt aus dem Satz von Pascal im entarteten Sehnensechseck $CCADDB$, dass $Y$,~$Q$ und~$R$ kollinear sind.
\end{proof}

Der entscheidende Trick im Beweis war, den Satz von Pascal auf ein entartetes Sehnensechseck anzuwenden. Das ist ganz allgemein ein sehr nützlicher Trick und ihr solltet ihn im Hinterkopf behalten.

An dem folgenden Satz demonstrieren wir ein letztes Mal die Methoden dieses Abschnittes.

\begin{satzmitnamen}[Schmetterlingssatz]
	Sei $ABCD$ ein konvexes Sehnenviereck mit Umkreis~$\omega$. Sei $Q$ der Schnittpunkt der Diagonalen $AC$ und~$BD$. Eine Gerade~$\ell$ durch~$Q$ schneidet den Umkreis~$\omega$ in $X$~und~$Y$ und die Geraden $AB$ und~$CD$ in $K$~und~$L$. Der Punkt~$Q$ ist der Mittelpunkt von~$\overline{XY}$ genau dann, wenn $Q$ der Mittelpunkt von~$\overline{KL}$ ist.
\end{satzmitnamen}

\begin{figure}[ht]
	\centering
	\begin{tikzpicture}[x=0.8cm,y=0.8cm]
		%\clip (-4.945,-3.4) rectangle (5.704,5.518);
		\draw [line width=0.3] (1.844,2.986) coordinate (O) circle (1.993);
		\coordinate (A) at (0.035,2.153);
		\coordinate (B) at (2,1);
		\coordinate (C) at (3.581,2.009);
		\coordinate (D) at (2.281,4.931);
		\coordinate (P) at (4.747,-0.611);
		\coordinate (Q) at (2.076,2.07);
		\coordinate (R) at (-3.988,-2.822);
		\coordinate (S) at (2.698,1.446);
		\coordinate (K) at (0.749,1.734);
		\coordinate (L) at (3.404,2.406);
		\coordinate (X) at (0.376,1.64);
		\coordinate (Y) at (3.777,2.5);
		\coordinate (Q1) at (2.875,-1.085);
		\draw (A) to (B) to (D) to (R) to (C) to (P) to (B) to (C) to (D);
		\draw [shorten >=3.05em] (A) to (Q);
		\draw [shorten <=1.95em] (A) to (C);
		\draw [line width=0.3,dashed] (Q1) to (O);
		\draw [line width=0.3,shorten <=-7em,shorten >=-3em] (X) to (Y);
		\draw [line width=0.3,shorten <=-2em,shorten >=-2em] (P) to (R);
		\draw [line width=0.3,shorten <=-2ex,shorten >=-7em] (P) to (Q);
		\draw [line width=0.3,shift={(Q1)}] (14.201:0.32cm) arc (14.201:104.201:0.32cm);
		\fill [shift={(Q1)}] (59.201:0.18cm) circle (1pt);
		\draw[fill=black] (A) circle (2pt) node[shift={(165:2ex)}] {$A$};
		\draw[fill=black] (B) circle (2pt) node[shift={(270:2ex)}] {$B$};
		\draw[fill=black] (C) circle (2pt) node[shift={(0:2ex)}] {$C$};
		\draw[fill=black] (D) circle (2pt) node[shift={(75:2ex)}] {$D$};
		\draw[fill=black] (O) circle (2pt) node[shift={(100:2ex)}] {$O$};
		\draw[fill=black] (P) circle (2pt) node[shift={(245:2ex)}] {$P$};
		\draw[fill=black] (Q) circle (2pt) node[shift={(50:2.35ex)}] {$Q$};
		\draw[fill=black] (R) circle (2pt) node[shift={(250:2ex)}] {$R$};
		\draw[fill=black] (S) circle (2pt) node[shift={(80:1.5ex)}] {$S$};
		\draw[fill=white] (K) circle (2pt) node[shift={(80:2ex)}] {$K$};
		\draw[fill=white] (L) circle (2pt) node[shift={(152:2ex)}] {$L$};
		\draw[fill=white] (X) circle (2pt) node[shift={(255:2ex)}] {$X$};
		\draw[fill=white] (Y) circle (2pt) node[shift={(45:2.25ex)}] {$Y$};
		\node at (1.05,4.4) {$\omega$};
		\node at (-2.6,1.25) {$\ell$};
	\end{tikzpicture}
\end{figure}

\begin{proof}
	Wir betrachten den Schnittpunkt~$P$ von $AB$ und~$CD$ sowie den Schnittpunkt~$R$ von $AD$ und~$BC$. Außerdem sei $O$ der Umkreismittelpunkt von $ABCD$. Nach dem Satz von Brocard steht $OQ$ senkrecht auf~$RP$. Andererseits ist $Q$ der Mittelpunkt von~$\overline{XY}$ genau dann, wenn $OQ$ auch auf~$XY$ senkrecht steht. Diese Bedingung ist folglich äquivalent zu $RP\parallel XY$.
	
	Sei $S$ der Schnittpunkt von $PQ$ und~$BC$. Nach dem Satz vom vollständigen Vierseit sind $(B,C)$ und $(R,S)$ harmonische Punktepaare. Die Projektion $\pi_P\colon BC\to XY$ durch~$P$ ist eine projektive Abbildung. Also sind auch sind $(K,L)$ und $(Q,T)$ harmonische Punktepaare, wobei $T$ der Schnittpunkt von $RP$ und~$XY$ ist. Somit ist $Q$ genau dann der Mittelpunkt von~$\overline{KL}$, wenn $T=\infty_{XY}$ der unendlich ferne Punkt auf der Geraden~$XY$ ist~-- mit anderen Worten, wenn $RP\parallel XY$. Das zeigt die Behauptung.
\end{proof}

\subsection*{Dualität}
Ein beherrschendes Prinzip in der projektiven Geometrie ist das \emph{Dualitätsprinzip}.

\begin{satzmitnamen}[Dualitätsprinzip]
	Eine geometrische Aussage sollte auch dann noch wahr sein, wenn die Rollen von Punkten und Geraden vertauscht werden, das heißt, wenn die Begriffe \enquote{Punkt} und \enquote{Gerade} sowie \enquote{Schnittpunkt} und \enquote{Verbindungsgerade} vertauscht werden.
\end{satzmitnamen}

Ihr solltet das Dualitätsprinzip nicht allzu wörtlich nehmen. Zum Beispiel ist ja überhaupt nicht klar, was mit Kreisen oder Winkeln passieren soll, wenn Punkte und Geraden vertauscht werden. Solange aber in einer geometrischen Aussage nur Punkte und Geraden vorkommen, ist das Dualitätsprinzip wortwörtlich wahr~-- dank dem Satz von La~Hire! Wenn wir nämlich eine Aussage~$\mathcal A$ bereits bewiesen haben und versuchen, ihre \enquote{duale} Aussage~$\overline{\mathcal A}$ zu beweisen, können wir einen beliebigen Kreis~$\omega$ wählen und alle Punkte durch ihre Polaren sowie alle Geraden durch ihre Pole bezüglich~$\omega$ ersetzen. Dadurch erhalten wir eine äquivalente Aussage. Nach dem Satz von La~Hire ist diese äquivalente Aussage aber genau~$\mathcal A$, was wir schon bewiesen haben.

Ein interessantes Beispiel hierfür ist der Satz von Desargues, der gelegentlich in Olympiadeaufgaben zur Anwendung kommt: Er ist zu seiner eigenen Umkehrung dual! Insbesondere haben wir, sobald wir den Satz bewiesen haben, auch seine Umkehrung bewiesen.

\begin{satzmitnamen}[Satz von Desargues]
	Seien $ABC$ und $A'B'C'$ zwei Dreiecke, sodass sich die Geraden $AA'$, $BB'$ und~$CC'$ in einem Punkt~$Z$ schneiden. Sei $P$ der Schnittpunkt von $BC$ und~$B'C'$, $Q$~der Schnittpunkt von $CA$ und~$C'A'$ sowie $R$~der Schnittpunkt von $AB$ und~$A'B'$. Dann sind $P$,~$Q$ und~$R$ kollinear.
\end{satzmitnamen}

\begin{figure}[ht]
	\centering
	\begin{tikzpicture}[x=0.65cm,y=0.65cm]
		\clip (-7.753,-3.27) rectangle (5.15,6.1);
		\coordinate (A) at (-2.449,0.391);
		\coordinate (B) at (-3.435,-0.507);
		\coordinate (C) at (-3.035,-1.405);
		\coordinate (A1) at (0.876,1.396);
		\coordinate (B1) at (0.031,-0.124);
		\coordinate (C1) at (2.636,-2.299);
		\coordinate (U) at (-5.752,4.703);
		\coordinate (V) at (-0.902,5.129);
		\coordinate (W) at (3.148,5.485);
		\coordinate (Z) at (-6.553,-0.85);
		\draw (A) to (B) to (C) to cycle;
		\draw (A1) to (B1) to (C1) to cycle;
		\draw [line width=0.3,shorten <=-2ex,shorten >=-6.375em] (Z) to (A1);
		\draw [line width=0.3,shorten <=-2ex,shorten >=-7.5em] (Z) to (B1);
		\draw [line width=0.3,shorten <=-2ex,shorten >=-3em] (Z) to (C1);
		\draw [line width=0.3,shorten <=-2ex] (U) to (B);
		\draw [line width=0.3,shorten <=-2ex] (U) to (B1);
		\draw [line width=0.3,shorten <=-2ex] (V) to (A);
		\draw [line width=0.3,shorten <=-2ex] (V) to (A1);
		\draw [line width=0.3,shorten <=-2ex] (W) to (A);
		\draw [line width=0.3,shorten <=-2ex] (W) to (A1);
		\draw[dashed,line width=0.3,shorten <=-3em,shorten >=-3em] (U) to (W);
		\draw[fill=black] (A) circle (2pt) node[shift={(135:2ex)}] {$A$};
		\draw[fill=black] (B) circle (2pt) node[shift={(235:2ex)}] {$B$};
		\draw[fill=black] (C) circle (2pt) node[shift={(270:2ex)}] {$C$};
		\draw[fill=black] (A1) circle (2pt) node[shift={(160:2.4ex)}] {$A'$};
		\draw[fill=black] (B1) circle (2pt) node[shift={(250:2ex)}] {$B'$};
		\draw[fill=black] (C1) circle (2pt) node[shift={(270:2ex)}] {$C'$};
		\draw[fill=white] (U) circle (2pt) node[shift={(240:2ex)}] {$P$};
		\draw[fill=white] (V) circle (2pt) node[shift={(325:2.5ex)}] {$Q$};
		\draw[fill=white] (W) circle (2pt) node[shift={(320:2.25ex)}] {$R$};
		\draw[fill=black] (Z) circle (2pt) node[shift={(270:2ex)}] {$Z$};
	\end{tikzpicture}
\end{figure}

\begin{proof}
	Der Beweis ist sehr ähnlich zum Satz von Pascal. Sei $R'$ der Schnittpunkt von~$AB$ mit~$PQ$ und sei $R''$ der Schnittpunkt von~$A'B'$ mit~$PQ$. Es genügt, $R'=R''$ zu zeigen. Dazu betrachten wir die Projektionen $\pi_A\colon PQ\to BC$, $\pi_Z\colon BC\to B'C'$ und $\pi_{A'}\colon B'C'\to PQ$ durch $A$,~$Z$ und~$A'$. Sei $S$ der Schnittpunkt von $AA'$ und~$PQ$. Weil die Abbildungen $\pi_A$,~$\pi_Z$ und~$\pi_{A'}$ das Doppelverhältnis erhalten, können wir wie folgt umformen:
	\begin{equation*}
		(P,Q;S,R')\overset{\raisebox{0.5ex}{$\scriptstyle\pi_A$}}{=} (P,C;AA'\cap BC,B)\overset{\raisebox{0.5ex}{$\scriptstyle\pi_Z$}}{=} (P,C';AA'\cap B'C',B')\overset{\raisebox{0.5ex}{$\scriptstyle\pi_{A'}$}}{=} (P,Q;S,R'')\,.
	\end{equation*}
	(hierbei bezeichnen $AA'\cap BC$ und $AA'\cap B'C'$ die Schnittpunkte von~$AA'$ mit $BC$ und~$B'C'$). Für jede reelle Zahl $\lambda\neq 0,1$ gibt es aber genau einen Punkt~$Y_\lambda$ auf~$PQ$, für den $(P,Q;S,Y_\lambda)=\lambda$ gilt. Also muss $R'=R''$ gelten und wir sind fertig.
\end{proof}

Gelegentlich lassen sich auch Aussagen dualisieren, die Kreise beinhalten. Ein gutes Beispiel ist der Satz von Pascal: Er ist eine Aussage über sechs Punkte auf einem Kreis, also sollte der duale Satz eine Aussage über sechs \enquote{Geraden auf einem Kreis}, sprich sechs Tangenten sein. Im Satz von Pascal sind die Schnittpunkte der gegenüberliegenden Verbindungsgeraden kollinear. Im dualen Satz sollten dann die Verbindungsgeraden der gegenüberliegenden Schnittpunkte kopunktal sein. Und tatsächlich erhalten wir so einen bekannten Satz:
\begin{satzmitnamen}[Satz von Brianchon]
	Sei $ABCDEF$ ein Tangentensechseck \embrace{das auch überschlagen sein darf}. Dann schneiden sich die Hauptdiagonalen $AD$, $BE$ und~$CF$ in einem Punkt.
\end{satzmitnamen}

Den Satz von Brianchon kennt ihr vielleicht schon aus dem Kapitel \emph{Potenzgeraden} im Heft für Klasse~10 (allerdings haben wir ihn dort nur für nicht-überschlagene Tangentensechsecke bewiesen). Ihr sollt ihn nun mit den Methoden der projektiven Geometrie noch einmal beweisen.

\begin{aufgabe*}
	Beweise den Satz von Brianchon! (\emph{Tipp: Benutze Polaren, um die Aussage auf den Satz von Pascal zurückzuführen.})
\end{aufgabe*}

Wie der Satz von Pascal ist auch der Satz von Brianchon insbesondere dann nützlich, wenn er in entarteten Fällen angewendet wird.

\begin{aufgabe*}
	Sei $ABCD$ ein Tangentenviereck und seien $W$,~$X$, $Y$ und~$Z$ die Berührpunkte des Inkreises mit den Seiten $\overline{AB}$, $\overline{BC}$, $\overline{CD}$ und~$\overline{DA}$.
	\begin{enumerate}
		\item Beweise, dass sich die Geraden $WY$ und~$XZ$ sowie die Diagonalen $AC$ und~$BD$ in einem Punkt~$S$ schneiden.
		\item Beweise, dass sich die Geraden $AX$ und~$CW$ auf~$BD$ schneiden.
		\item Beweise, dass sich die Geraden $AC$, $WX$ und~$YZ$ in einem Punkt~$T$ schneiden. Zeige außerdem, dass $(A,C)$ und $(S,T)$ harmonische Punktepaare sind.
	\end{enumerate}
\end{aufgabe*}

Nachdem wir nun die duale Aussage zum Satz von Pascal ermittelt haben, drängt sich die Frage auf, was denn die duale Aussage zum Satz von Pappos ist. Eine solche duale Aussage muss auf jeden Fall existieren, weil im Satz von Pappos ja nur Punkte und Geraden vorkommen. Statt euch die Antwort direkt zu verraten, werden wir den dualen Satz von Pappos benutzen, um eine Aufgabe von der IMO 2019 zu lösen.

\begin{aufgabe*}[*]
	Im Dreieck $ABC$ liegt ein Punkt~$B_1$ auf der Seite~$\overline{CA}$ und ein Punkt~$C_1$ auf der Seite~$\overline{AB}$. Auf den Strecken $\overline{BB_1}$ und~$\overline{CC_1}$ werden Punkte $P$~und~$Q$ gewählt, sodass $PQ$ parallel zu~$BC$ ist. Ferner sei $P_1$ ein Punkt auf der Verlängerung von~$PC_1$ über~$C_1$ hinaus, für den $\winkel PP_1A=\winkel CBA$ gilt, und sei $Q_1$ ein Punkt auf der Verlängerung von~$QB_1$ über~$B_1$ hinaus, für den $\winkel AQ_1Q=\winkel ACB$ gilt. Beweise, dass $PQQ_1P_1$ ein Sehnenviereck ist.
\end{aufgabe*}

\begin{figure}[ht]
	\centering
	\begin{tikzpicture}[x=0.8cm,y=0.8cm]
		\clip (-4.315,-0.601) rectangle (7.919,7.585);
		\draw [line width=0.3] (-0.614,2.35) circle (2.728);
		\draw [line width=0.3] (3.032,1.573) circle (1.847);
		\draw [dashed,line width=0.3] (1.616,3.128) circle (2.569);
		\draw [dashed,line width=0.3, shift={(3.244,6.599)}] (165:3.703) arc (165:375:3.703);
		\coordinate (A) at (2.007,3.11);
		\coordinate (B) at (-2,0);
		\coordinate (B1) at (3.076,1.442);
		\coordinate (C) at (4,0);
		\coordinate (C1) at (0.441,1.894);
		\coordinate (X) at (0.773,0);
		\coordinate (Y) at (2.063,0);
		\coordinate (P) at (0.641,0.75);
		\coordinate (Q) at (2.59,0.75);
		\coordinate (P1) at (-0.11,5.031);
		\coordinate (Q1) at (4.183,3.018);
		\coordinate (U) at (-0.443,6.932);
		\coordinate (V) at (6.932,6.932);
		\coordinate (K) at (-1.033,0.75);
		\coordinate (L) at (3.519,0.75);
		\draw (A) to (B) to (C) to cycle;
		\draw [line width=0.3] (B) to (B1);
		\draw [line width=0.3] (C) to (C1);
		\draw [line width=0.3, shorten <=-2ex,shorten >=-2ex] (X) to (U);
		\draw [line width=0.3, shorten <=-2ex,shorten >=-2ex] (Y) to (V);
		\draw [line width=0.3] (P1) to (A) to (Q1);
		\draw [line width=0.3, shorten <=-6.125em,shorten >=-7em] (K) to  node[pos=-0.575] {$\scriptscriptstyle/$} (L);
		\draw [line width=0.3, shorten <=-8em,shorten >=-2em] (U) to  node[pos=-0.435] {$\scriptscriptstyle/$} (V);
		\draw [line width=0.3,shorten >=-2ex] (A) to (U);
		\draw [line width=0.3,shorten >=-2ex] (A) to (V);
		\draw [line width=0.3, shorten <=-4em,shorten >=-5.5em] (B) to node[pos=-0.275] {$\scriptscriptstyle/$} (C);
		\draw [line width=0.3,shift={(X)}] (99.951:0.32cm) arc (99.951:180:0.32cm);
		\draw [line width=0.3,shift={(U)}] (-80.049:0.32cm) arc (-80.049:0:0.32cm);
		\draw [line width=0.3,shift={(A)}] (137.765:0.32cm) arc (137.765:217.814:0.32cm);
		\draw [line width=0.3,shift={(Y)}] (0:0.32cm) arc (0:54.917:0.32cm);
		\draw [line width=0.3,shift={(Y)}] (0:0.37cm) arc (0:54.917:0.37cm);
		\draw [line width=0.3,shift={(V)}] (180:0.32cm) arc (180:234.917:0.32cm);
		\draw [line width=0.3,shift={(V)}] (180:0.37cm) arc (180:234.917:0.37cm);
		\draw [line width=0.3,shift={(A)}] (302.659:0.32cm) arc (302.659:357.578:0.32cm);
		\draw [line width=0.3,shift={(A)}] (302.659:0.37cm) arc (302.659:357.578:0.37cm);
		\draw [line width=0.3,shift={(B)}] (0:0.32cm) arc (0:37.815:0.32cm);
		\draw [line width=0.3,shift={(B)}] (0:0.37cm) arc (0:37.815:0.37cm);
		\draw [line width=0.3,shift={(B)}] (0:0.42cm) arc (0:37.815:0.42cm);
		\draw [line width=0.3,shift={(P1)}] (279.951:0.32cm) arc (279.951:317.766:0.32cm);
		\draw [line width=0.3,shift={(P1)}] (279.951:0.37cm) arc (279.951:317.766:0.37cm);
		\draw [line width=0.3,shift={(P1)}] (279.951:0.42cm) arc (279.951:317.766:0.42cm);
		\draw [line width=0.3,shift={(C)}] (122.659:0.27cm) arc (122.659:180:0.27cm);
		\draw [line width=0.3,shift={(C)}] (122.659:0.32cm) arc (122.659:180:0.32cm);
		\draw [line width=0.3,shift={(C)}] (122.659:0.37cm) arc (122.659:180:0.37cm);
		\draw [line width=0.3,shift={(C)}] (122.659:0.42cm) arc (122.659:180:0.42cm);
		\draw [line width=0.3,shift={(Q1)}] (177.578:0.27cm) arc (177.578:234.919:0.27cm);
		\draw [line width=0.3,shift={(Q1)}] (177.578:0.32cm) arc (177.578:234.919:0.32cm);
		\draw [line width=0.3,shift={(Q1)}] (177.578:0.37cm) arc (177.578:234.919:0.37cm);
		\draw [line width=0.3,shift={(Q1)}] (177.578:0.42cm) arc (177.578:234.919:0.42cm);
		\draw[fill=black] (A) circle (2pt) node[shift={(76:2.5ex)}] {$A$};
		\draw[fill=black] (B) circle (2pt) node[shift={(250:2ex)}] {$B$};
		\draw[fill=black] (C) circle (2pt) node[shift={(290:2ex)}] {$C$};
		\draw[fill=black] (B1) circle (2pt) node[shift={(155:2.5ex)}] {$B_1$};
		\draw[fill=black] (C1) circle (2pt) node[shift={(160:2.25ex)}] {$C_1$};
		\draw[fill=white] (P) circle (2pt) node[shift={(235:2ex)}] {$P$};
		\draw[fill=white] (Q) circle (2pt) node[shift={(282:2ex)}] {$Q$};
		\draw[fill=white] (P1) circle (2pt) node[shift={(143:2.75ex)}] {$P_1$};
		\draw[fill=white] (Q1) circle (2pt) node[shift={(-14:3ex)}] {$Q_1$};
		\draw[fill=black] (K) circle (2pt) node[shift={(120:2ex)}] {$K$};
		\draw[fill=black] (L) circle (2pt) node[shift={(60:2ex)}] {$L$};
		\draw[fill=black] (U) circle (2pt) node[shift={(225:2ex)}] {$U$};
		\draw[fill=black] (V) circle (2pt) node[shift={(-45:2ex)}] {$V$};
		\draw[fill=black] (X) circle (2pt) node[shift={(244:2.25ex)}] {$X$};
		\draw[fill=black] (Y) circle (2pt) node[shift={(285:2.125ex)}] {$Y$};
	\end{tikzpicture}
\end{figure}

\begin{proof}[Lösung]
	Sei $X$ der Schnittpunkt von $PC_1$ mit~$BC$ und $Y$~der Schnittpunkt von $QB_1$ mit~$BC$. Dann gilt $\winkel XP_1A=\winkel PP_1A=\winkel CBA=\winkel XBA$, also ist $BXAP_1$ ein Sehnenviereck. Analog ist $YCQ_1A$ ein Sehnenviereck. Sei ferner $U$ der Schnittpunkt von $PC_1$ mit~$AC$ und $V$~der Schnittpunkt von $QB_1$ mit~$AB$. Wir lösen die Aufgabe zuerst durch eine einfache Winkeljagd unter der Annahme, dass $UV$ parallel zu $PQ$ und~$BC$ ist. Unter dieser Annahme folgt aus dem Wechselwinkelsatz $\winkel P_1XB=\winkel P_1UV$. Andererseits ist nach Peripheriewinkelsatz $\winkel P_1XB=\winkel P_1AB$. Es folgt $\winkel VAP_1=180^\circ -\winkel P_1AB=180^\circ -\winkel P_1UV$, also ist $P_1AVU$ ein Sehnenviereck. Mit analoger Begründung ist auch $AQ_1VU$ ein Sehnenviereck, sodass die fünf Punkte $P_1$,~$A$, $Q_1$, $V$ und~$U$ auf einem Kreis liegen. Nach dem Stufen- und Nebenwinkelsatz an den Parallelen $PQ$ und~$UV$ gilt nun $\winkel QPP_1=180^\circ-\winkel P_1UV$. Im Sehnenviereck $P_1Q_1VU$ gilt aber $180^\circ-\winkel P_1UV=\winkel VQ_1P_1=180^\circ-\winkel P_1Q_1Q$. Es folgt $\winkel QPP_1=180^\circ-\winkel P_1Q_1Q$ und somit muss $PQQ_1P_1$ in der Tat ein Sehnenviereck sein.
	
	Es bleibt zu zeigen, dass $UV$ wirklich parallel zu $PQ$ und~$BC$ ist. Wenn wir in der projektiven Ebene arbeiten, müssen wir also zeigen, dass sich $UV$, $PQ$ und~$BC$ in einem Punkt schneiden (dieser ist dann zwangsläufig der Schnittpunkt von $BC$ und~$PQ$, also~$\infty_{PQ}$). Und das folgt direkt aus dem dualen Satz von Pappos! Der Satz von Pappos ist nämlich eine Aussage über sechs Punkte, von denen jeweils drei auf einer Geraden liegen. Und hier haben wir es umgekehrt mit sechs Geraden $AB$, $CC_1$, $PP_1$ und $AC$, $BB_1$, $QQ_1$ zu tun, von denen sich jeweils drei in einem Punkt schneiden.
\end{proof}

Statt mit dem dualen Satz von Pappos lässt sich auch direkt über Doppelverhältnisse argumentieren. Dazu führen wir die Schnittpunkte $K$~und~$L$ von~$PQ$ mit $AB$ und~$AC$ ein. Die Projektion $\pi_{B_1}\colon PQ\to AB$ durch~$B_1$ erhält das Doppelverhältnis, also gilt $(P,Q;K,L)=(B,V;K,A)$. Die Projektion $\pi_{C_1}\colon PQ\to CA$ durch~$C_1$ erhält ebenfalls das Doppelverhältnis, also gilt $(P,Q;K,L)=(U,C;A,L)$. Es folgt
\begin{equation*}
	\frac{BK}{KV}\bigg/\frac{BA}{AV}=\frac{UA}{AC}\bigg/\frac{UL}{LC}\,.
\end{equation*}
Weil $KL$ parallel zu~$BC$ ist, folgt aus dem Strahlensatz (der auch für gerichtete Streckenlängen gültig ist) $BK/BA=LC/AC$.
% Tien: Vielleicht willst du zu ungerichteten Streckenlängen übergehen. Oder erwähnst kurz, dass der Strahlensatz so auch für gerichtete Streckenlängen gilt.
Die obige Gleichung impliziert also $AV/KV=UA/UL$. Nach der Umkehrung des Strahlensatzes muss dann $UV$ parallel zu~$KL$ sein.

Hier sehen wir einen letzten Trick in Aktion: Wenn ihr ein Verhältnis von Strecken berechnen müsst --~zum Beispiel, für ein Strahlensatzargument wie oben, oder für Argumente mit den Sätzen von Ceva und Menelaos~-- dann könnt ihr versuchen, dieses Verhältnis zu einem Doppelverhältnis zu ergänzen, das ihr wiederum mithilfe geeigneter Projektionen auf ein Doppelverhältnis zurückführen könnt, welches einfacher zu berechnen ist.

\subsection*{Weitere Übungsaufgaben}
\begin{aufgabe*}
	Einem Dreieck $ABC$ sei ein Halbkreis~$\omega$ so einbeschrieben, dass der Durchmesser~$\overline{EF}$ auf der Seite~$\overline{AB}$ liegt. Dabei befindet sich $E$ zwischen $F$~und~$A$. Der Halbkreis~$\omega$ berühre die Seiten $\overline{BC}$ und~$\overline{CA}$ in $G$~und~$H$. Schließlich sei $S$ der Schnittpunkt von $GE$ und~$FH$ sowie $T$~der Schnittpunkt von $EH$ und~$FG$. Zeige, dass $C$ der Mittelpunkt der Strecke~$\overline{ST}$ ist.
\end{aufgabe*}

\begin{aufgabe*}
	Sei $ABC$ ein Dreieck mit Inkreis~$\omega$. Sei $D$ der Berührpunkt von~$\omega$ mit der Seite~$\overline{BC}$, sei $X$ der von~$D$ verschiedene Schnittpunkt von~$AD$ mit~$\omega$ und seien $Y$,~$Z$ die von~$X$ verschiedenen Schnittpunkte von $BX$,~$CX$ mit~$\omega$. Beweise, dass sich $BZ$ und~$CY$ auf~$AD$ schneiden.
\end{aufgabe*}

\begin{aufgabe*}[*]
	Sei $ABC$ ein Dreieck mit Inkreis~$\omega$ und Inkreismittelpunkt~$I$. Seien $D$,~$E$ und~$F$ die Berührpunkte von~$\omega$ mit den Seite $\overline{BC}$, $\overline{CA}$ und~$\overline{AB}$. Schließlich sei $M$ der Mittelpunkt von~$\overline{BC}$. Beweise, dass sich die Geraden $AM$, $DI$ und~$EF$ in einem Punkt schneiden.
\end{aufgabe*}

\begin{aufgabe*}[*]
	Die Punkte $A$,~$B$, $C$ und~$D$ liegen in dieser Reihenfolge auf einem Kreis~$\omega$ mit Mittelpunkt~$O$. Sei $K$ der Schnittpunkt von $AC$ und~$BD$. Ein weiterer Kreis~$\omega_1$ mit Mittelpunkt auf der Strecke~$\overline{OK}$ schneide die Strecken $\overline{AB}$ und~$\overline{CD}$ in den Punkten $A_1$~und~$B_1$ bzw.\ $C_1$~und~$D_1$. Angenommen, $K$~liegt auf der Geraden~$A_1C_1$ und es gilt $\abs*{A_1K}\neq \abs*{KC_1}$. 
	% Tien: Gibt es einen bestimmten Grund, warum A_1K und KC_1 gerichtete Streckenlängen sind? Ansonsten finde ich, dass sich Betragsstriche besser lesen lassen.
	Beweise, dass $K$ dann auch auf~$B_1D_1$ liegt.
\end{aufgabe*}

\begin{aufgabe*}[*]
	Sei $ABCD$ ein konvexes Sehnenviereck. Sei $K$ der Schnittpunkt der Diagonalen $AC$ und~$BD$. Die Mittelpunkte von $\overline{AC}$ und~$\overline{BD}$ bezeichnen wir mit $M$~und~$N$. Die Umkreise $\odot ABM$ und $\odot CDM$ schneiden sich außer in~$M$ noch in~$L$. Zeige, dass $KLMN$ ein Sehnenviereck ist.
\end{aufgabe*}

\begin{aufgabe*}[*]
	Sei $ABCD$ ein konvexes Viereck, sodass die Diagonale~$BD$ zugleich auch die Winkelhalbierende von $\winkel CBA$ ist. Der Umkreis $\odot ABC$ schneide die Strecken $\overline{CD}$ und~$\overline{DA}$ in den inneren Punkten $P$~und~$Q$. Die Parallele zu~$AC$ durch~$D$ scheide die Geraden $BA$ und~$BC$ in den Punkten $R$~und~$S$. Zeige, dass die vier Punkte $P$,~$Q$, $R$ und~$S$ auf einem Kreis liegen.
\end{aufgabe*}

\begin{aufgabe*}[*]
	In einem gleichschenkligen Dreieck $ABC$ mit $\abs{AB}=\abs{AC}$ sei $M$ der Mittelpunkt der Seite~$\overline{BC}$. Weiter sei $P$ ein Punkt auf der Parallelen zu~$BC$ durch~$A$, für den $\abs{PB}<\abs{PC}$ gilt. Ferner sei $X$ ein Punkt auf der Verlängerung von~$\overline{PB}$ über~$B$ hinaus und $Y$~ein Punkt auf der Verlängerung von~$\overline{PC}$ über~$C$ hinaus, sodass $\winkel MXP=\winkel PYM$ gilt. Beweise, dass $APXY$ ein Sehnenviereck ist.
\end{aufgabe*}
\newpage
	
	\phantomsection\cftaddtitleline{toc}{part}{Kombinatorik}{\thepage}
	\section{Kombinatorikaufgaben mit Graphentheorie lösen}\label{kapitel:GraphenInCombo}
Viele Kombinatorik-Aufgaben lassen sich lösen, indem das Problem auf geeignete Weise als Graph interpretiert wird. Häufig ist das recht offensichtlich, zum Beispiel immer dann, wenn in einer Aufgabe von Freundschaften, Feindschaften, Bekanntschaften oder von Straßennetzen in Ländern mit phantasievollen Namen die Rede ist. Aber es gibt auch Aufgaben, bei denen alles andere als offensichtlich ist, dass sie sich mit Graphentheorie lösen lassen.

Hier sind einige Lösungsstrategien, die bei solchen Aufgaben häufig hilfreich sind:
\begin{itemize}
	\item Nach dem Handschlagslemma ist in jedem Graphen die Anzahl der Knoten von ungeradem Grad gerade. Insbesondere folgt: Wenn ihr einen Graphen konstruiert habt und wisst, dass ein bestimmter Knoten ungeraden Grad hat, dann muss es einen weiteren Knoten mit ungeradem Grad geben. Diese simple Beobachtung lässt sich erstaunlich oft auf nichttriviale Weise anwenden!
	\item Bei vielen Aufgaben könnt ihr euren Graphen in Wege (oder Pfade oder Kreise) zerlegen und entlang dieser Wege etwas abwechselnd tun, zum Beispiel Knoten oder Kanten abwechselnd einfärben.
	\item Allgemein könnt ihr versuchen, euren Graphen schrittweise zu vereinfachen. Zum Beispiel könnt ihr häufig Knoten von Grad $1$ entfernen oder Kreise \emph{kontrahieren}: Das bedeutet, ihr nehmt euch einen Kreis und ersetzt alle seine Knoten und Kanten durch einen einzigen Knoten.
\end{itemize}
Ansonsten solltet ihr natürlich die üblichen Lösungsstrategien in der Kombinatorik im Kopf behalten (Extremalprinzip, Invarianzprinzip, Schubfachprinzip, \ldots).

Wir werden euch nun vier Beispielaufgaben stellen, die sich mit Graphentheorie lösen lassen (obwohl manche dieser Aufgaben überhaupt nicht danach aussehen). Die letzten beiden Aufgaben sind richtig schwer. Im Anschluss an die Aufgaben findet ihr Tipps dazu und am Ende des Heftes könnt ihr die Lösungen nachlesen. Wenn ihr nicht weiterkommt, benutzt gerne die Tipps oder lest (besonders bei den schweren Aufgaben) die Lösungen.
\begin{aufgabe*}\label{aufgabe:Feldwege}
	Zwischen den Orten einer Insel verlaufen einige Feldwege, die zum Spazierengehen genutzt werden. Jeder Feldweg beginnt an einem Ort und endet an einem anderen Ort, wobei von jedem Ort genau drei Wege ausgehen. Um das touristische Angebot der Insel zu erweitern, sollen einige dieser Feldwege zu Radwegen ausgebaut werden. Damit aber auch das Spaziergehen nicht zu sehr beeinträchtigt wird, soll anschließend von jedem Ort mindestens ein Radweg und mindestens ein weiterhin unausgebauter Feldweg ausgehen. Zeige, dass das stets möglich ist.
\end{aufgabe*}
\begin{aufgabe*}\label{aufgabe:50Laender}\leavevmode
	\begin{enumerate}
		\item 100 Leute aus 50 Ländern, zwei aus jedem Land, stehen im Kreis. Zeige, dass die Leute so in zwei Gruppen aufgeteilt werden können, dass weder zwei Leute aus einem Land noch drei im Kreis aufeinanderfolgende Leute zu einer Gruppe gehören.\label{teilaufgabe:50}
		\item 100 Leute aus 25 Ländern, vier aus jedem Land, stehen im Kreis. Zeige, dass die Leute so in vier Gruppen aufgeteilt werden können, dass weder zwei Leute aus einem Land noch zwei im Kreis aufeinanderfolgende Leute zu einer Gruppe gehören.\label{teilaufgabe:25}
	\end{enumerate}
\end{aufgabe*}
\begin{aufgabe*}[**]\label{aufgabe:Kartenspiel}
	Zwei Mathematikerinnen werden gezwungen, ein Kartenspiel zu spielen, und dürfen erst aufhören, wenn eine von beiden keine Karten mehr hat. Das Kartendeck besteht aus $n$ verschiedenen Karten. Von je zwei Karten schlägt eine die andere (aber wenn $A$ von $B$ geschlagen wird und $B$ von $C$ geschlagen wird, muss $A$ nicht unbedingt von $C$ geschlagen werden). Zu Beginn wird das Deck zufällig in zwei Stapel aufgeteilt, von denen jede Mathematikerin einen bekommt. Sie dürfen sich ihre Stapel anschauen und sich absprechen, aber die Reihenfolge der Karten nicht verändern. In jedem Zug decken beide die oberste Karte ihres Stapels auf. Diejenige Mathematikerin, deren Karte die andere schlägt, bekommt beide Karten und legt sie unter ihren Stapel (sie bestimmt die Reihenfolge). Zeige, dass die beiden Mathematikerinnen stets das Spiel beenden können.
\end{aufgabe*}

\begin{aufgabe*}[***]\label{aufgabe:Rechtecksparkettierung}
	Gegeben seien positive ganze Zahlen $m$, $n$, $a$ und $b$. Angenommen, ein $m\times n$ Rechteck lässt sich lückenlos und überscheidungsfrei mit horizontalen $a\times1$-Rechtecken und vertikalen $1\times b$-Rechtecken parkettieren. Zeige, dass $m$ durch $a$ oder $n$ durch $b$ teilbar sein muss.
\end{aufgabe*}
\subsection*{Tipps zu den Beispielaufgaben}
\textbf{Tipp zu Aufgabe~\ref{aufgabe:Feldwege}.} Füge geeignete Kanten hinzu und benutze den Satz von Euler-Hierholzer.

\textbf{Tipp zu Aufgabe~\ref{aufgabe:50Laender}.} Für~\ref{teilaufgabe:50} betrachte den Graphen, in dem je zwei Leute aus einem Land mit einer Kante verbunden sind sowie außerdem die erste und die zweite Person im Kreis, die dritte und die vierte Person und so weiter.

Für~\ref{teilaufgabe:25} benutze zuerst~\ref{teilaufgabe:50} und führe dann ein ähnliches Argument noch einmal durch.

\textbf{Tipp zu Aufgabe~\ref{aufgabe:Kartenspiel}.} Betrachte den gerichteten Graphen aller Spielsituationen. Was kannst du über die Eingangs- und Ausgangsgrade der Knoten aussagen?

\textbf{Tipp zu Aufgabe~\ref{aufgabe:Rechtecksparkettierung}.} Betrachte den folgenden Graphen $G$: Die Knoten sind der Mittelpunkt eines jeden Rechtecks in der Parkettierung sowie diejenigen Eckpunkte von Rechtecken, deren $x$-Koordinate durch $a$ und deren $y$-Koordinate durch $b$ teilbar ist. Jeder Mittelpunkt eines Rechtecks wird mit allen Eckpunkten verbunden, die in $G$ liegen. Was kannst du über die Knotengrade in $G$ aussagen?
\newpage
	
	\phantomsection\cftaddtitleline{toc}{part}{Zahlentheorie}{\thepage}
	\section{Pellsche Gleichungen}\label{kapitel:Pell}
\emph{Pellsche Gleichungen}, fälschlicherweise benannt nach John Pell (1611--1685) sind Diophantische Gleichungen der Form $x^2-dy^2=k$. Hierin ist $k$ eine ganze Zahl und $d$ eine positive ganze Zahl, die keine Quadratzahl ist.\footnote{Wäre $d=a^2$ eine Quadratzahl, ließe sich die Gleichung zu $(x-ay)(x+ay)=k$ faktorisieren und aus der Primfaktorzerlegung von $k$ ließen sich alle Lösungen ablesen. Dieser Fall ist also trivial.} Gesucht sind alle ganzzahligen Lösungspaare $(x,y)$.

Zahlreiche Olympiade-Aufgaben lassen sich, nach geschickter Vereinfachung, auf Pellsche Gleichungen zurückführen. Deswegen ist es wichtig, dass ihr wisst, wie sich solche Gleichungen lösen lassen. Unser Ziel in diesem Kapitel ist, den folgenden, alles andere als offensichtlichen Satz zu beweisen:
\begin{satzmitnamen}[Lösbarkeit der Pellschen Gleichung]
	Sei $k$ eine ganze Zahl und sei $d$ eine positive ganze Zahl, die keine Quadratzahl ist.
	\begin{enumerate}[label={$(\alph*)$},ref={$(\alph*)$}]
		\item Die Pellsche Gleichung $x^2-dy^2=1$ hat stets unendlich viele ganzzahlige Lösungen $(x,y)$.
		\item Wenn die Pellsche Gleichung $x^2-dy^2=k$ eine ganzzahlige Lösung $(x,y)$ hat, dann hat sie unendlich viele ganzzahlige Lösungen.
	\end{enumerate}
\end{satzmitnamen}
Der Beweis bedarf einiger Vorbereitungen.

\subsection*{Die Norm in $\boldsymbol{\mathbb Z[\sqrt{d}]}$}
Wenn $d$ eine Quadratzahl wäre, könnten wir den Ausdruck $x^2-dy^2$ gemäß der dritten binomischen Formel in zwei ganzzahlige Faktoren zerlegen. Die entscheidende Idee zum Lösen Pellscher Gleichungen besteht darin, die gleiche Faktorisierung auch dann zu betrachten, wenn $d$ keine Quadratzahl ist: Die Gleichungen $x^2-dy^2=1$ und $x^2-dy^2=k$ werden dann zu
\begin{equation*}
	\parens*{x+y\sqrt{d}}\parens*{x-y\sqrt{d}}=1\quad\text{und}\quad \parens*{x+y\sqrt{d}}\parens*{x-y\sqrt{d}}=k\,.
\end{equation*}
Freilich sind die Faktoren keine ganzen Zahlen mehr. Wir müssen also anstelle von $\mathbb Z$ die Menge $\braces{a+b\sqrt{d}\ |\ a,b\in\mathbb Z}$ betrachten. Diese Menge wollen wir mit $\mathbb Z[\sqrt{d}]$ bezeichnen. 

Wir überlegen uns zuerst, dass die Darstellung $z=a+b\sqrt{d}$ mit $a,b\in \mathbb Z$ eindeutig ist. Wären nämlich $a_1+b_1\sqrt{d}=a_2+b_2\sqrt{d}$ zwei verschiedene Darstellungen, dann muss $b_1\neq b_2$ sein, denn aus $b_1=b_2$ folgt auch $a_1=a_2$. Durch Umformen folgt nun $\sqrt{d}=(a_1-a_2)/(b_2-b_1)$. Somit wäre $\sqrt{d}$ wäre eine rationale Zahl, was aber nicht sein kann, denn $d$ ist keine Quadratzahl.

Zahlen aus $\mathbb Z[\sqrt{d}]$ können wir addieren, subtrahieren und multiplizieren und das Ergebnis wird stets wieder in $\mathbb Z[\sqrt{d}]$ liegen. Außerdem lassen sich Zahlen \emph{konjugieren}: Zu einer Zahl $z=x+y\sqrt{d}\in\mathbb Z[\sqrt{d}]$ definieren wir die \emph{zu $z$ konjugierte Zahl} als $\overline{z}\coloneqq x-y\sqrt{d}$. Es lässt sich unmittelbar nachprüfen, dass für die Konjugation die Rechenregeln
\begin{equation*}
	\overline{\overline{z}}=z\,,\quad \overline{z_1\pm z_2}=\overline {z_1}\pm \overline{z_2}\quad\text{und}\quad \overline{z_1\cdot z_2}=\overline{z_1}\cdot\overline{z_2}
\end{equation*}
für alle $z,z_1,z_2\in\mathbb Z[\sqrt{d}]$ gelten. 

Wenn euch $\mathbb Z[\sqrt{d}]$ jetzt an die komplexen Zahlen erinnert, dann liegt ihr vollkommen richtig. Die komplexen Zahlen $\mathbb C$ entstehen aus den reellen Zahlen, indem wir eine Lösung der in $\mathbb R$ nicht lösbaren quadratischen Gleichung $X^2+1=0$ hinzufügen.\footnote{Statt \emph{hinzufügen} wird in der Mathematik meistens \emph{adjungieren} gesagt.} Ebenso entsteht $\mathbb Z[\sqrt{d}]$ aus $\mathbb Z$, indem wir eine Lösung der in $\mathbb Z$ nicht lösbaren quadratischen Gleichung $X^2-d=0$ hinzufügen. Für komplexe Zahlen gibt es den \emph{Betrag}, der duch $\abs{z}^2=z\cdot \overline{z}$ definiert ist. Analog definieren wir für $z=x+y\sqrt{d}\in\mathbb Z[\sqrt{d}]$ die \emph{Norm von $z$} durch
\begin{equation*}
	N(z)=z\cdot \overline z=\left(x+y\sqrt{d}\right)\left(x-y\sqrt{d}\right)=x^2-dy^2\,.
\end{equation*}
Genau wie der komplexe Betrag verträgt sich auch die Norm mit Multiplikation, das heißt es gilt $N(z_1z_2)=N(z_1)N(z_2)$ für alle $z_1,z_2\in\mathbb Z[\sqrt{d}]$.

Die Pellschen Gleichungen $x^2-dy^2=1$ und $x^2-dy^2=k$ können wir jetzt wie folgt umschreiben:
\begin{equation*}
	N\parens*{x+y\sqrt{d}}=1\quad\text{und}\quad N\parens*{x+y\sqrt{d}}=k\,.
\end{equation*}
Statt nach Paaren $(x,y)$ von ganzen Zahlen mit $x^2-dy^2=1$ können wir also äquivalenterweise nach Zahlen $z\in\mathbb Z[\sqrt{d}]$ mit $N(z)=1$ suchen. Offenbar gilt $N(1)=N(-1)=1$, korrespondierend zu den beiden trivialen Lösungen $(x,y)=(1,0)$ und $(x,y)=(-1,0)$ von $x^2-dy^2=1$. Wenn $N(z)=1$, dann gilt auch $N(\overline{z})=1$, $N(-z)=1$ und $N(-\overline{z})=1$. Das korrespondiert zu der Beobachtung, dass mit $(x,y)$ auch $(\pm x,\pm y)$ Lösungen der Gleichung $x^2-dy^2=1$ sind.

Diese Beobachtung erlaubt uns, den Zahlenraum, in dem wir nach Lösungen von $N(z)=1$ suchen, ein wenig einzuschränken. Indem wir $z$ gegebenenfalls durch $-z$ ersetzen, dürfen wir nämlich $z>0$ annehmen. Indem wir $z$ gegebenenfalls durch $\overline{z}=\frac 1z$ ersetzen (denn $z\cdot \overline{z}=N(z)=1$), dürfen wir ferner $z\geqslant 1$ annehmen. Wir dürfen uns somit auf Lösungen von $N(z)=1$ mit $z\geqslant 1$ beschränken. Das korrespondiert zu der Einschränkung $x\geqslant 1$, $y\geqslant 0$.

Wir können jetzt eine Verschärfung des gewünschten Satzes formulieren.
\begin{satzmitnamen}[Lösbarkeit der Pellschen Gleichung \textmd{(verbesserte Version)}]
	Sei $k$ eine ganze Zahl und $d$ eine natürliche Zahl, die keine Quadratzahl ist.
	\begin{enumerate}
		\item Die Gleichung $N(z)=1$ in $\mathbb Z[\sqrt{d}]$ hat unendlich viele Lösungen. Unter allen Lösungen mit $z>1$ gibt es eine minimale Lösung $z_0$, die sogenannte Fundamentallösung. Alle weiteren Lösungen $z$ sind durch $z=\pm z_0^n$ mit $n\in\mathbb Z$ gegeben.\label{behauptung:PellscheGleichung=1}
		\item Wenn die Gleichung $N(z)=k$ in $\mathbb Z[\sqrt{d}]$ eine Lösung $z=u$ hat, dann lassen sich unendlich viele weitere Lösungen durch $z=\pm u\cdot z_0^n$ finden, wobei~$z_0$ die Fundamentallösung aus~\ref{behauptung:PellscheGleichung=1} ist. Im Fall $k=-1$ sind alle Lösungen von dieser Form \embrace{für allgemeines $k$ muss das jedoch nicht der Fall sein}.\label{behauptung:PellscheGleichung=k}
	\end{enumerate}
\end{satzmitnamen}
\begin{proof}
	Der komplizierteste Teil des Beweises besteht darin, zu zeigen, dass eine nichttriviale Lösung $z\neq \pm1$ der Gleichung $N(z)=1$ existiert. Diesen Beweis werden wir im nächsten Unterabschnitt führen. Wir werden einstweilen die übrigen Aussagen beweisen.
	
	Aus dem obigen Argument folgt: Wenn eine Lösung mit $z\neq \pm 1$ existiert, dann existiert auch eine Lösung mit $z>1$. Schreibe $z=x+y\sqrt{d}$ mit $x> 1$, $y> 0$. Wenn $z_0=x_0+y_0\sqrt{d}$ eine weitere Lösung mit $z\geqslant z_0>1$ ist, dann müssen die Ungleichungen $x\geqslant x_0> 1$ und $y\geqslant y_0> 0$ gelten (eine von beiden Ungleichungen muss auf jeden Fall gelten, sonst wäre nicht $z\geqslant z_0$; die andere folgt dann automatisch aus $x^2-dy^2=1=x_0^2-dy_0^2$). Also gibt es nur endlich viele Lösungen $z_0$ mit $z\geqslant z_0>1$. Insbesondere muss tatsächlich eine minimale Lösung $z_0$ existieren.
	
	Jede weitere Lösung mit $z> 0$ muss dann von der Form $z=z_0^n$ für eine ganze Zahl $n$ sein. Ansonsten gäbe es nämlich ein $n$ mit $z_0^{n}<z<z_0^{n+1}$. Dann ist $1<z\cdot z_0^{-n}<z_0$. Wegen $z_0\cdot \overline{z_0}=N(z_0)=1$ ist $z\cdot z_0^{-n}=z\cdot \overline{z_0}^{n}$ ein Element von $\mathbb Z[\sqrt{d}]$. Schließlich gilt
	\begin{equation*}
		N\parens*{z\cdot z_0^{-n}}=N(z)\cdot N(z_0)^{-n}=1\cdot 1^{-n}=1
	\end{equation*}
	Dann widerspricht $z\cdot z_0^{-n}$ aber der Minimalität von $z_0$. Unsere Annahme war somit falsch und $z$ muss tatsächlich von der Form $z=z_0^n$ sein. Analog muss jede Lösung mit $z<0$ von der Form $z=-z_0^n$ sein. Ansonsten gäbe es ein $n$ mit $-z_0^{n+1}<z<-z_0^n$. Analog zu oben würde dann $-z\cdot z_0^n$ der Minimalität von $z_0$ widersprechen. Damit ist~\ref{behauptung:PellscheGleichung=1} gezeigt (bis auf die Existenz einer nichttrivialen Lösung).
	
	Für~\ref{behauptung:PellscheGleichung=k} benutzen wir, dass die Norm multiplikativ ist, um $N(\pm u\cdot z_0^n)=N(\pm 1)\cdot N(u)\cdot N(z_0)^n=1\cdot k\cdot 1^n=k$, zu rechnen. Im Fall~$k=-1$ gilt $u\cdot \overline{u}=N(u)=-1$. Also können wir in $\mathbb Z[\sqrt{d}]$ durch $u$ dividieren, denn Division durch $u$ ist das Gleiche wie Multiplikation mit $\frac 1u=-\overline{u}$. Wenn~$z$ eine Lösung von $N(z)=-1$ ist, die nicht von der Form $\pm u\cdot z_0^n$ ist, dann wäre $z/u$ nicht von der Form $\pm z_0^n$. Es gilt aber $-1=N(z)=N(z/u\cdot u)=N(z/u)\cdot N(u)=-N(z/u)$, also $N(z/u)=1$. Damit würde $z/u$ der in \ref{behauptung:PellscheGleichung=1} bewiesenen Klassifikation aller Elemente von Norm~$1$ widersprechen.
\end{proof}

Wenn ihr Lösungen der Gleichung $N(z)=1$ mit $z\in\mathbb Z[\sqrt{d}]$ zurück in Lösungen der Pellschen Gleichung $x^2-dy^2=1$ übersetzen wollt, könnt ihr die Bedingung $z=\pm z_0^n$ folgendermaßen umformulieren: Wenn die Pellsche Gleichung $x^2-dy^2$ die minimale nichttriviale Lösung $(x,y)=(x_0,y_0)$ mit $x_0>1$ und $y_0>0$ besitzt, dann ist jede weitere Lösung $(x,y)$ mit $x>1$ und $y>0$ durch die folgende Rekursion gegeben:
\begin{equation*}
	x_{n+1}=x_0x_n+dy_0y_n\quad\text{und}\quad y_{n+1}=x_0y_n+x_ny_0
\end{equation*}
(die Gleichungen sind äquivalent zu $x_{n+1}+y_{n+1}\sqrt{d}=(x_0+y_0\sqrt{d})(x_n+y_n\sqrt{d})$). Statt dieser Rekursion solltet ihr euch aber lieber die Gleichung $z=\pm z_0^n$ merken -- das ist wesentlich einfacher und erklärt, woher die Rekursion wirklich kommt.

\subsection*{Existenz einer nichttrivialen Lösung}
Wir machen uns jetzt daran, den schwierigsten Schritt im Beweis der Lösbarkeit der Pellschen Gleichung durchzuführen und eine nichttriviale Lösung zu konstruieren. Die zentrale Idee ist hierbei folgende: Wenn $(x,y)$ die Pellsche Gleichung $x^2-dy^2=1$ löst und $x>1$, $y>0$ gilt, dann ist $x^2\approx dy^2$, also $x/y\approx \sqrt{d}$. Mit anderen Worten: $x/y$ ist eine \emph{rationale Approximation} der irrationalen Zahl $\sqrt{d}$. Um die Pellsche Gleichung zu lösen, gehen wir umgekehrt vor. Wir konstruieren zuerst sehr genaue rationale Approximationen der irrationalen Zahl $\sqrt{d}$ und beweisen dann, dass wir dadurch (manchmal) Lösungen der Pellschen Gleichung bekommen.

Um genaue Approximationen von irrationalen Zahlen zu konstruieren, benutzen wir das folgende Lemma, das auf Gustav Lejeune Dirichlet (1805--1859) zurückgeht.

\begin{satzmitnamen}[Dirichletscher Approximationssatz]
	Sei $\alpha$ eine irrationale reelle Zahl. Dann gibt es unendlich viele Paare $(p,q)$ von ganzen Zahlen mit $q>0$ und
	\begin{equation*}
		\abs*{\alpha-\frac{p}{q}}<\frac{1}{q^2}\;.
	\end{equation*}
\end{satzmitnamen}
Intuitiv besagt der Dirichletsche Approximationssatz besagt, dass wir $\alpha$ \enquote{effizient} (relativ zum Nenner) durch rationale Zahlen annähern können.

\begin{wrapfigure}{r}{0.2\textwidth}
	\centering\vspace{-0.7cm}
	\begin{tikzpicture}[x=0.55cm,y=0.55cm, rotate=22.5]
		%\draw(0,0) circle (2);
		\draw[loosely dotted] (22:2) arc (22:158:2);
		\draw (158:2) arc (158:382:2);
		\draw (0:2) ++ (0:0.5ex) to ++(0:-1ex);
		\draw (315:2) ++ (315:0.5ex) to ++(315:-1ex);
		\draw (270:2) ++ (270:0.5ex) to ++(270:-1ex);
		\draw (225:2) ++ (225:0.5ex) to ++(225:-1ex);
		\draw (180:2) ++ (180:0.5ex) to ++(180:-1ex);
		\node[right] at (337.5:2) {$\frac{1}{N}$};
		\node[below right] at (292.5:2) {$\frac{1}{N}$};
		\node[below] at (247.5:2) {$\frac{1}{N}$};
		\node[below left] at (202.5:2) {$\frac{1}{N}$};
	\end{tikzpicture}
	\vspace{-0.5cm}
\end{wrapfigure}
\begin{proof}
	Wir stellen uns einen Kreis vom mit Umfang $1$ vor, um den wir mit Schritten der Länge $\alpha$ herumlaufen. Wähle eine positive ganze Zahl $N$. Wir unterteilen den Kreis in $N$ Abschnitte der Länge $\frac1N$ (wobei die \glqq Grenzen\grqq\ zwischen zwei Abschnitten immer zu demjenigen Abschnitt dazuzählen sollen, der -- gegen den Uhrzeigersinn betrachtet -- vor dem anderen Abschnitt liegt). Wenn wir den Anfangspunkt als nullten Schritt mitzählen, müssen wir gemäß dem Schubfachprinzip nach $N$ Schritten zweimal im gleichen Abschnitt gelandet sein. Es gibt also natürliche Zahlen $0\leqslant m<n\leqslant N$, sodass die Endpunkte nach $m\alpha$ und $n\alpha$ Umläufen um den Kreis weniger als $\frac 1N$ auseinanderliegen, das heißt wir finden eine ganze Zahl $p$ mit 
	\begin{equation*}
		\abs*{n\alpha-m\alpha-p}<\frac1N\,.
	\end{equation*}
	Mit $q=n-m$ gilt dann $q\leqslant N$ und damit
	\begin{align*}
		\abs*{\alpha-\frac{p}{q}}<\frac{1}{Nq}\leqslant \frac{1}{q^2}\,.
	\end{align*}
	Ganz fertig sind wir noch nicht, denn wir müssen noch zeigen, dass unendlich viele solche $q$ gefunden werden können. Angenommen, wir haben schon $q_1,\ldots,q_n$ und zugehörige Zähler $p_1,\ldots,p_n$ gefunden, sodass $\abs{\alpha-p_i/q_i}<1/q_i^2$. Weil $\alpha$ irrational ist, kann keine der Differenzen $\abs{q_i\alpha-p_i}$ gleich Null sein. Wir können also $N$ so groß wählen, dass
	\begin{equation*}
		\frac{1}{N}<\min\braces[\big]{\abs*{q_1\alpha-p_1},\abs*{q_2\alpha-p_2},\dotsc,\abs*{q_n\alpha-p_n}}\,.
	\end{equation*}
	Wenn wir $q$ und $p$ dann wie oben konstruieren, gilt $|q\alpha-p|<\frac 1N$, also kann $q$ nicht schon unter den Zahlen $q_1,\ldots,q_n$ sein.
\end{proof}

\begin{proof}[Beweis der Existenz einer nichttrivialen Lösung]
	Wir wenden den Dirichletschen Approximationssatz auf die irrationale Zahl $\sqrt{d}$ an. Dann gibt es unendlich viele Paare ganzer Zahlen $(x,y)$ mit $y>0$ und
	\begin{equation*}
		\abs*{\frac xy-\sqrt{d}}<\frac{1}{y^2}\,.
	\end{equation*}
	Mithilfe dieser Abschätzung und der Dreiecksungleichung können wir wie folgt abschätzen:
	\begin{equation*}
		\abs*{x^2-dy^2}=\abs*{\frac xy+\sqrt{d}}\cdot y^2\cdot \abs*{\frac{x}{y}-\sqrt{d}}<\left|\frac{x}{y}+\sqrt{d}\right|\leqslant \abs*{\frac xy-\sqrt{d}}+2\sqrt{d}<1+2\sqrt{d}\,.
	\end{equation*}
	Für $z=x+y\sqrt{d}\in\mathbb Z[\sqrt{d}]$ gilt also $\abs{N(z)}<1+2\sqrt{d}$. Weil $d$ keine Quadratzahl ist, kann $N(z)=x^2-dy^2$ nicht Null sein. Also gibt es unendlich viele $z\in\mathbb Z[\sqrt{d}]$ mit $0<\abs{N(z)}<1+2\sqrt{d}$.
	
	Nach dem unendlichen Schubfachprinzip gibt es eine ganze Zahl $0<k<1+2\sqrt{d}$, sodass $N(z)=k$ unendlich viele Lösungen in $\mathbb Z[\sqrt{d}]$ hat. Unser Ziel ist nun, zwei solche Lösungen $z_1=x_1+y_1\sqrt{d}$ und $z_2=x_2+y_2\sqrt{d}$ mit $z_1\neq \pm z_2$ zu konstruieren, sodass der Quotient $z\coloneqq z_1/z_2$ wieder in $\mathbb Z[\sqrt{d}]$ liegt. Dann wäre $k=N(z_1)=N(z_1/z_2\cdot z_2)=N(z_1/z_2)N(z_2)=N(z_1/z_2)\cdot k$, sodass $N(z_1/z_2)=1$ sein müsste. Also wäre $z_1/z_2$ eine nichttriviale Lösung und wir wären fertig.
	
	Wir untersuchen nun, wann $z_1/z_2\in \mathbb Z[\sqrt{d}]$ erfüllt ist. Wegen $z_2\cdot\overline{z_2}=N(z_2)=k$ gilt
	\begin{equation*}
		\frac{z_1}{z_2}=\frac{z_1\cdot \overline{z_2}}{z_2\cdot \overline{z_2}}=\frac{\parens*{x_1+y_1\sqrt{d}}\parens*{x_2-y_2\sqrt{d}}}{k}=\frac{x_1x_2-dy_1y_2+(x_2y_1-x_1y_2)\sqrt{d}}k\,.
	\end{equation*}
	Wir brauchen also, dass $x_1x_2-dy_1y_2$ und $x_2y_1-x_1y_2$ durch $k$ teilbar sind. Weil die Gleichung aber $N(z)=k$ unendlich viele Lösungen hat, finden wir nach dem unendlichen Schubfachprinzip auch zwei Lösungen $z_1=x_1+y_1\sqrt{d}$ und $z_2=x_2+y_2\sqrt{d}$ darunter, für die $x_1\equiv x_2\mod k$ und $y_1\equiv y_2\mod k$ sowie $z_1\neq \pm z_2$ gilt. Für $z_1$ und $z_2$ gilt dann 
	\begin{equation*}
		x_1x_2-dy_1y_2\equiv x_1^2-dy_1^2\equiv N(z_1)\equiv 0\mod k
	\end{equation*}
	und
	\begin{equation*}
		x_2y_1-x_1y_2\equiv x_1y_1-x_1y_1\equiv 0\mod k\,.
	\end{equation*}
	Also haben wir $z_1$ und $z_2$ mit den gewünschten Eigenschaft konstruiert. Das beendet den Beweis, dass stets eine nichttriviale Lösung existiert.
\end{proof}

\subsection*{Abschließende Bemerkungen}
Wie lässt sich die Fundamentallösung einer Pellschen Gleichung $x^2-dy^2=1$ bestimmen? Für Olympiadezwecke: Durch Ausprobieren! Wenn in einer Olympiadeaufgabe eine explizite Lösung benötigt wird, dann könnt ihr davon ausgehen, dass ihr euch dafür nicht totrechnen bzw.\ totprobieren müsst. Selbiges gilt für die Lösung einer Pellschen Gleichung $x^2-dy^2=k$ (vorausgesetzt, so eine Lösung existiert; wenn nicht, dann lässt sich das meistens durch Modulo-Betrachtungen zeigen).

Wenn $(x,y)$ eine Lösung von $x^2-dy^2=1$ ist, dann lässt sich zeigen, dass $x/y$ in der Kettenbruchdarstellung von $\sqrt{d}$ auftauchen muss.\footnote{Für jede reelle Zahl $\alpha$ existiert eine eindeutige Darstellung der Form
	\begin{equation*}
		\alpha=a_0+\frac{1}{\displaystyle a_1+\frac{1^{\vphantom{t}}}{\displaystyle a_2+\frac{1^{\vphantom{t}}}{a_3+\dotsb}}}\,,
	\end{equation*}
	wobei $a_0$ eine ganze Zahl und $a_1,a_2,\dotsc$ eine endliche oder unendliche Folge von positiven ganze Zahlen ist. Diese Darstellung wird \emph{Kettenbruchdarstellung von~$\alpha$} genannt. Wir sagen \emph{$x/y$ taucht in der Kettenbruchdarstellung von~$\alpha$ auf}, wenn ein $n\geqslant 0$ existiert, sodass $x/y$ der Bruch ist, den wir erhalten, wenn wir die Kettenbruchdarstellung nach~$a_n$ abbrechen.} Wenn ihr jemals in die Verlegenheit kommt, einen Computer nach einer Fundamentallösung suchen zu lassen, dann lässt sich die Suche mit dieser Beobachtung wesentlich beschleunigen. Für Olympiade-Zwecke ist das aber Overkill und ihr seid durch Ausprobieren einfach schneller.

Für allgemeinere Gleichungen der Form $ax^2-by^2=k$ könnt ihr folgendermaßen unendlich viele Lösungen generieren (vorausgesetzt, es gibt mindestens eine Lösung): Ratet zuerst eine Lösung $(x,y)=(x_1,y_1)$ dieser Gleichung. Sei $u\coloneqq x_1\sqrt{a}+y_1\sqrt{b}$ und sei $z_0\coloneqq x_0+y_0\sqrt{ab}\in\mathbb Z[\sqrt{ab}]$ eine Lösung von $N(z_0)=1$. Dann liefert $\pm u\cdot z_0^n$ für $n\in\mathbb Z$ unendlich viele weitere Lösungen der Gleichung $ax^2-by^2=k$. Genauer: Wir können $\pm u\cdot z_0^n$ in der Form $x\sqrt{a}+y\sqrt{b}$ für ganze Zahlen~$x$ und~$y$ schreiben (das gilt allgemein für Produkte der Form $u\cdot z$ mit $z\in\mathbb Z[\sqrt{ab}]$). Dann ist $(x,y)$ eine Lösung von $ax^2-by^2=k$. Denn $\sqrt{a} u$ ist ein Element von $\mathbb Z[\sqrt{ab}]$ und
\begin{equation*}
	a\parens*{ax^2-by^2}=(ax)^2-aby^2=N\parens*{\sqrt{a} u\cdot z_0^n}=N\parens*{\sqrt{a}u}\cdot N(z_0)^n=ak\cdot 1^n=ak\,.
\end{equation*}

Für noch allgemeinere Gleichungen vom Grad 2 gilt: \emph{Substituieren!} Mit einer geschickten Substitution könnt ihr alle gemischten Terme töten und gelangt zu einer Gleichung von bekanntem Typ.

\subsection*{Beispielaufgaben}
Die folgenden ehemaligen Bundesrundenaufgaben lassen sich mithilfe von Pellschen Gleichungen lösen. Am Ende des Kapitels findet ihr Tipps zu den Aufgaben und am Ende des Heftes findet ihr Lösungen.
\begin{aufgabe*}\label{aufgabe:561246}
	Beweise, dass es unendlich viele positive ganze Zahlen~$m$ gibt, für die eine Folge von~$m$ aufeinanderfolgenden Quadratzahlen existiert, deren Summe $m^3$ ist. Finde außerdem ein Beispiel mit $m>1$.
\end{aufgabe*}
\begin{aufgabe*}[***]\label{aufgabe:521246}
	Sei $(a_n)_{n\geqslant 1}$ eine Zahlenfolge, die durch die rekursive Vorschrift $a_1=1$, $a_2=2$ und $a_{n+2}=2a_{n+1}+a_n$ für $n\geqslant 0$ definiert wird. Bestimme alle positiven reellen Zahlen $\beta$ mit den beiden folgenden Eigenschaften:
	\begin{enumerate}[label={$(\Alph*)$},ref={$(\Alph*)$}]
		\item Es gibt unendlich viele Paare $(p,q)$ von positiven ganzen Zahlen mit\label{bedingung:RationaleApproximation}
		\begin{equation*}
			\abs*{\frac pq-\sqrt{2}}<\frac{\beta}{q^2}\,.
		\end{equation*}
		\item Nur für endlich viele dieser Paare $(p,q)$ taucht~$q$ nicht in der Folge $(a_n)_{n\geqslant 1}$ auf.\label{bedingung:NurEndlichVieleNichtInDerFolge}
	\end{enumerate}
\end{aufgabe*}

\subsection*{Weitere Übungsaufgaben}
\begin{aufgabe*}
	Beweise: Wenn $n$ eine positive ganze Zahl ist, sodass $n^2$ als Differenz von zwei aufeinanderfolgenden Kubikzahlen geschrieben werden kann, dann ist $2n-1$ eine Quadratzahl.
\end{aufgabe*}
\begin{aufgabe*}
	Eine positive ganze Zahl heiße \emph{heinersch}, wenn sie sich als Summe einer positiven Quadratzahl und einer positiven Kubikzahl darstellen lässt. Beweise, dass es unendlich viele positive ganze Zahlen~$n$ gibt, sodass $n$, $n+1$ und $n+2$ allesamt heinersch sind. 
\end{aufgabe*}

\subsection*{Tipps zu den Beispielaufgaben}

\textbf{Tipps zu Aufgabe~\ref{aufgabe:561246}.} Benutze die Summenformel
\begin{equation*}
	\sum_{k=1}^nk^2=\frac{n(n+1)(2n+1)}{6}\,.
\end{equation*}
Welche Bedingung an~$m$ ergibt sich? Substituiere geeignet, um diese Bedingung in eine Gleichung der Form $ax^2-by^2=k$ umzuformen.

\textbf{Tipps zu Aufgabe~\ref{aufgabe:521246}.} Bevor wir Tipps zur Lösung geben, sollten wir erstmal einen Tipp zum Verstehen der Aufgabe geben. Offenbar geht es in dieser Aufgabe darum, wie gut die irrationale Zahl $\sqrt{2}$ durch rationale Zahlen approximiert werden kann. Die Aufgabenstellung legt nahe, dass der Fehler einer rationalen Approximation $p/q$ mindestens von der Größenordnung $1/q^2$ ist. Es wird also vermutlich ein \emph{minimales} $\beta$ geben, sodass der Fehler nicht $<\beta/q^2$ sein kann (bis auf endlich viele Ausnahmen). Dieses minimale $\beta$ zu bestimmen ist ein Teil der Aufgabe.

Dann gibt es aber auch noch Bedingung~\ref{bedingung:NurEndlichVieleNichtInDerFolge}. Diese Bedingung legt nahe, dass es unter allen rationalen Approximationen $p/q$ einige besonders gute gibt, deren Nenner allesamt in der Folge $(a_n)_{n\geqslant1}$ auftreten. Die Aufgabe fragt dann nach dem \emph{maximalen} $\beta$, sodass ein Fehler $<\beta/q^2$ nur von den besonders guten Approximationen erreicht werden kann (bis auf endlich viele Ausnahmen).

Nun kommen die Tipps zur Lösung:
\begin{itemize}
	\item Es ist intuitiv einleuchtend, dass die \enquote{bestmöglichen} Approximationen $p/q$ durch die Lösungen der Pellschen Gleichungen $p^2-2q^2 =\pm1$ gegeben sein sollten.
	\item Ebenso einleuchtend ist, dass die \enquote{zweitbestmöglichen} Approximationen durch die Lösungen der Pellsche Gleichungen $p^2-2q^2=\pm2$ gegeben sein sollten.
\end{itemize}
Um die Aufgabe zu lösen, finde zuerst heraus, was die Folge $(a_n)_{n\geqslant 1}$ mit diesen Pellschen Gleichungen zu tun hat. Dann untersuche die \enquote{best-} und \enquote{zweitbestmöglichen} Approximationen von $\sqrt{2}$, um eine untere und eine obere Schranke für $\beta$ herzuleiten.
\newpage
	\section{Nullstellen von Polynomen in der Zahlentheorie}\label{kapitel:Galois}
Es gibt einen Typus von Olympiade-Aufgaben, in denen ihr zahlentheoretische Eigenschaften von bestimmten irrationalen Zahlen untersuchen sollt. Solche Aufgaben wirken auf den ersten Blick völlig unlösbar, aber in Wirklichkeit gibt es einen einfachen Trick, mit dem sie sich fast immer lösen lassen.

Statt euch den Trick direkt zu verraten, demonstrieren wir ihn zuerst an einem Beispiel.
\begin{aufgabe*}\label{aufgabe:1+sqrt2Hoch1000}
	Bestimme die erste Stelle vor dem Komma und die erste Stelle nach dem Komma von $(1+\sqrt{2})^{1000}$.
\end{aufgabe*}
\begin{proof}[Lösung]
	 Betrachte die Zahlenfolge $(a_n)_{n\geqslant 0}$, die durch $a_n = (1+\sqrt{2})^n + (1-\sqrt{2})^n$ gegeben ist. Dann sind alle $a_n$ ganze Zahlen. Eine Methode, das zu zeigen, haben wir in Kapitel~\ref{kapitel:Pell}: \emph{Pellsche Gleichungen} kennengelernt: Wenn $z=x+y\sqrt{2}\in\mathbb Z[\sqrt{2}]$, dann ist $z+\overline{z}=2x$ eine ganze Zahl. Diese Beobachtung ist hier anwendbar, denn $\overline{(1+\sqrt{2})^n}=(\overline{1+\sqrt{2}})^n=(1-\sqrt{2})^n$
	
	Im Hinblick auf spätere Anwendungen geben wir noch einen weiteren Beweis für die Ganzzahligkeit von $a_n$. Es ist klar, dass $a_0 = a_1 = 2$ ganze Zahlen sind. Ferner sind $1\pm\sqrt{2}$ die Nullstellen des Polynoms $(X-1)^2-2=X^2-2X-1$. Nach der Theorie der linearen Rekursionen (siehe das Kapitel \emph{Lineare Rekursionen} im Heft für Klasse~10) erfüllt die Folge $(a_n)_{n\geqslant 0}$ demnach die Rekursionsgleichung $a_{n+2} = 2a_{n+1} + a_n$. Per Induktion ist dann klar, dass alle $a_n$ ganze Zahlen sein müssen.
	
	Insbesondere ist $a_{1000}$ eine ganze Zahl. Nun ist $-\frac12<1-\sqrt{2}<0$, folglich ist $(1-\sqrt{2})^{1000}$ eine mikroskopisch kleine positive reelle Zahl (kleiner als $2^{-1000}$). In jedem Fall beginnt ihre Dezimaldarstellung mit $0{,}000\dotso$, sodass die Dezimaldarstellung von $(1+\sqrt{2})^{1000}$ mit $\ldots{,}999\dotso$ aufhören muss. Folglich ist die erste Stelle nach dem Komma auf jeden Fall eine~$9$. 
	
	Um die erste Stelle vor dem Komma zu bestimmen, betrachten wir die Rekursionsgleichung $a_{n+2} = 2a_{n+1} + a_n$ modulo~$10$ und schauen, ab wann die Folge periodisch wird. Wir erhalten:
	\begin{center}
		\begin{tabular}{r | c c c c c c c c c c c c c c }\toprule
			$n$ & $0$ & $1$ & $2$ & $3$ & $4$ & $5$ & $6$ & $7$ & $8$ & $9$ & $10$ & $11$ & $12$ & $13$ \\
			$a_n \mod 10$ & $2$ & $2$ & $6$ & $4$ & $4$ & $2$ & $8$ & $8$ & $4$ & $6$ & $6$ & $8$ & $2$ & $2$ \\\bottomrule
		\end{tabular}
	\end{center}
	Aus der Tabelle folgt, dass $(a_n)_{n\geqslant 0}$ periodisch modulo~$10$ mit Periodenlänge~$12$ ist. Wegen $1000\equiv 4\mod 12$ können wir aus der Tabelle ablesen, dass $a_{1000}$ auf eine~$4$ endet. Weil nun $(1+\sqrt{2})^{1000}$ ein winziges Stückchen kleiner ist, muss die letzte Stelle vor dem Komma eine~$3$ sein. Damit ist die Aufgabe gelöst.
\end{proof}

Hinter der Lösung dieser Aufgabe steckt das folgende allgemeine Prinzip, das ihr euch unbedingt merken solltet:
\begin{enumerate}\itshape
	\item[$(*)$] Wenn in einer Aufgabe eine Nullstelle $\alpha$ eines irreduziblen\footnote{\emph{Irreduzibel} bedeutet, dass $P$ sich nicht als Produkt $P=QR$ mit zwei nicht-konstanten Polynomen $Q$ und $R$ mit rationalen Koeffizienten schreiben lässt.} Polynoms $P$ vorkommt, dann betrachte auch die anderen Nullstellen dieses Polynoms.
\end{enumerate}
Wenn $P$ ein normiertes quadratisches Polynom mit ganzzahligen Koeffizienten ist (so wie $P=X^2-2X-1$ in der obigen Lösung), dann ist die andere Nullstelle von $P$ genau die zu~$\alpha$ konjugierte Zahl. Wenn~$P$ hingegen höheren Grad hat, dann gibt es nicht länger nur \emph{eine} zu $\alpha$ konjugierte Zahl; stattdessen sind \emph{alle} anderen Nullstellen von $P$ zu $\alpha$ konjugiert.

Die Konjugierten von~$\alpha$ verhalten sich auch im allgemeinen Fall sehr ähnlich wie im quadratischen Fall. Wenn zum Beispiel $P$ ein normiertes Polynom vom Grad~$d$ mit ganzzahligen Koeffizienten ist und~$P$ die komplexen Nullstellen $\alpha_1,\alpha_2,\dotsc,\alpha_d\in\mathbb C$ besitzt (wobei jede Nullstelle entsprechend ihrer Vielfachheit oft gezählt wird), dann ist $\alpha_1^n+\alpha_2^n+\dotsb+\alpha_d^n$ für jedes $n\geqslant 0$ eine ganze Zahl, obwohl jedes einzelne $\alpha_i$ im Allgemeinen irrational ist. Es gilt sogar noch allgemeiner: Jeder \emph{symmetrische}\footnote{\emph{Symmetrisch} bedeutet, dass der Ausdruck in sich selbst übergeht, wenn beliebige $\alpha_i$ und $\alpha_j$ vertauscht werden.} polynomielle Ausdruck in $\alpha_1,\alpha_2,\dotsc,\alpha_d$ ist ganzzahlig. Um das einzusehen, betrachten wir die \emph{elementarsymmetrischen Polynome} in $d$ Variablen $X_1,X_2,\dotsc,X_d$:
\begin{equation*}
	\sigma_m\coloneqq \sum_{1\leqslant i_1 < i_2 < \dotsb < i_m\leqslant d} X_{i_1}X_{i_2}\dotsm X_{i_m}\quad\text{für }m=1,2,\dotsc,d\,.
\end{equation*}
Dann sind $\sigma_m(\alpha_1,\alpha_2,\dotsb,\alpha_d)$ ganze Zahlen, denn das Polynom
\begin{equation*}
	P(X)=(X-\alpha_1)(X-\alpha_2)\dotsm(X-\alpha_d)=X^m+\sum_{m=1}(-1)^m\sigma_m(\alpha_1,\alpha_2,\dotsc,\alpha_d)X^{m-d}
\end{equation*}
hat nach Annahme ganzzahlige Koeffizienten. Um zu zeigen, dass ein allgemeiner symmetrischer Ausdruck in $\alpha_1,\alpha_2,\dotsc,\alpha_d$ ebenfalls ganzzahlig ist, benutzen wir dann den folgenden Satz:
\begin{satzmitnamen}[Hauptsatz über symmetrische Funktionen]
	Jedes symmetrische Polynom in den Variablen $X_1,X_2,\dotsc,X_d$ mit ganzzahligen Koeffizienten lässt sich als Polynom in den elementarsymmetrischen Polynomen $\sigma_1,\sigma_2,\dotsc,\sigma_m$ mit ganzzahligen Koeffizienten schreiben.%Analoge Aussagen gelten auch für rationale/reelle/komplexe Koeffizienten.
\end{satzmitnamen}
\begin{proof}
	Wir benutzen Induktion nach $d$. Der Fall $d=1$ ist trivial. Sei nun $d\geqslant 2$ und wir nehmen an, dass die Behauptung für symmetrische Polynome in $d-1$ Variablen bereits bewiesen ist. Für alle $m=1,2,\dotsc,d-1$ betrachten wir das Polynom $\overline{\sigma}_m\coloneqq \sigma_m(X_1,X_2,\dotsc,X_{d-1},0)$. Dann ist $\overline{\sigma}_m$ genau das $m$-te elementarsymmetrische Polynom in den Variablen $X_1,X_2,\dotsc,X_{d-1}$.
	
	Sei nun $P$ ein beliebiges symmetrisches Polynom in den Variablen $X_1,X_2,\dotsc,X_d$ mit ganzzahligen Koeffizienten. Dann ist $\overline{P}\coloneqq P(X_1,X_2,\dotsc,X_{d-1},0)$ ein symmetrisches Polynom in den Variablen $X_1,X_2,\dotsc,X_{d-1}$ mit ganzzahligen Koeffizienten. Nach Voraussetzung gibt es ein Polynom $Q$ mit ganzzahligen Koeffizienten, sodass $\overline{P}=Q(\overline{\sigma}_1,\overline{\sigma}_2,\dotsc,\overline{\sigma}_{d-1})$. Betrachte nun das Polynom
	\begin{equation*}
		R\coloneqq P-Q(\sigma_1,\sigma_2,\dotsc,\sigma_{d-1})\,.
	\end{equation*}
	Weil $P$ und $Q(\sigma_1,\sigma_2,\dotsc,\sigma_{d-1})$ symmetrisch sind, muss auch~$R$ symmetrisch sein. Nach Konstruktion gilt ferner $R(X_1,X_2,\dotsc,X_{d-1},0)=0$. Also muss $R$ durch $X_d$ teilbar sein. Weil $R$ symmetrisch ist, muss~$R$ auch durch $X_1,X_2,\dotsc,X_{d-1}$ teilbar sein. Folglich ist $R$ durch das Produkt $X_1X_2\dotsm X_d=\sigma_d$ teilbar. Wir können also $R=\sigma_d\cdot P_1$ schreiben, wobei $P_1$ wiederum ein symmetrisches Polynom mit ganzzahligen Koeffizienten ist.
	
	Nun können wir das gleiche Argument mit $P_1$ statt $P$ wiederholen und iterieren. Weil in jedem Iterationsschritt der Totalgrad der betrachteten Polynome mindestens um~$d$ kleiner wird, haben wir nach endlich viele Schritten eine Darstellung von $P$ als Polynom in $\sigma_1,\sigma_2,\dotsc,\sigma_d$ mit ganzzahligen Koeffizienten gefunden.
\end{proof}

Der Hauptsatz über symmetrische Funktionen gilt völlig analog auch für Polynome mit rationalen, reellen oder komplexen Koeffizienten.\footnote{Allgemein sind Koeffizienten in jedem beliebigen \emph{Ring} erlaubt.}

Das Studium der Symmetrien von Nullstellen von Polynomen führt euch in die \emph{Galois-Theorie}. Die Strukturen, die entstehen, wenn wir zu $\mathbb Z$ die Nullstellen eines normierten irreduziblen Polynoms mit ganzzahligen Koeffizienten hinzufügen, werden in der \emph{Algebraischen Zahlentheorie} untersucht. Beides sind wunderschöne Gebiete der Mathematik, die ihr spätestens im Studium kennenlernen werdet.

\subsection*{Beispielaufgaben}

Die folgenden Aufgaben sind nach einem ähnlichen Muster wie Aufgabe~\ref{aufgabe:1+sqrt2Hoch1000} gestrickt. Am Ende des Kapitels findet ihr Tipps zu den Aufgaben und am Ende des Heftes könnt ihr die Lösungen nachlesen.

\begin{aufgabe*}\label{aufgabe:621246}
	Die Funktion $f$ mit der Gleichung $f(x)=x^3-3x^2+1$ hat drei reelle Nullstellen $\alpha<\beta<\gamma$.
	\begin{enumerate}[label={$(\alph*)$},ref={$(\alph*)$}]
		\item Zeige, dass $\lceil\gamma^n\rceil$ für jede positive ganze Zahl $n\geqslant 1$ durch~$3$ teilbar ist.\label{teilaufgabe:621246}
		\item Zeige, dass $\lfloor \gamma^{2020}\rfloor$ und $\lfloor \gamma^{2220}\rfloor$ durch~$17$ teilbar sind.\label{teilaufgabe:IMOSL1988}
	\end{enumerate}
	(\emph{Hierbei bezeichnet $\lfloor x\rfloor$ die größte ganze Zahl $\leqslant x$ und $\lceil x\rceil$ die kleinste ganze Zahl $\geqslant x$.}) 
\end{aufgabe*}
\begin{aufgabe*}\label{aufgabe:VAIMO2011_2}
	Sei $n$ eine positive ganze Zahl und sei
	\begin{equation*}
		b\coloneqq \left\lfloor \parens*{\sqrt[3]{28}-3}^{-n}\right\rfloor\,.
	\end{equation*}
	Zeige, dass~$b$ nicht durch~$6$ teilbar ist.
\end{aufgabe*}

\vfill\hrule\vspace{-1em}

\subsection*{Tipps zu den Beispielaufgaben}

\textbf{Tipp zu Aufgabe~\ref{aufgabe:621246}.} Betrachte $\alpha^n+\beta^n+\gamma^n$.

\textbf{Tipp zu Aufgabe~\ref{aufgabe:VAIMO2011_2}.} Betrachte $(\sqrt[3]{28}-3)^{-n}+(\sqrt[3]{28}\zeta-3)^{-n}+(\sqrt[3]{28}\zeta^2-3)^{-n}$, wobei $\zeta\coloneqq \mathrm{e}^{2\pi\mathrm{i}/3}$ eine dritte Einheitswurzel ist.
\newpage
	
	\phantomsection\cftaddtitleline{toc}{part}{Lösungen zu den Beispielaufgaben}{\thepage}
	% Damit in der PDF-Navigationsleiste auch der Abschnitt "Lösungen" auftaucht, muss ein zusätzliches Bookmark gesetzt werden. Irrelevant für die Druckversion.
	\pdfbookmark{Lösungen zu den Beispielaufgaben}{Loesungen}
	\section*{Lösungen zu den Beispielaufgaben}
	Die Lösungen sind nicht immer so formuliert, wie ihr das in der Olympiade tun solltet. Zum Teil sind sie sehr knapp -- zum Beispiel überspringen wir triviale Umformungsschritte oder lassen die Probe weg. In der Olympiade solltet ihr etwas ausführlicher sein und immer die Probe machen. Umgekehrt erklären wir gelegentlich (vor allem bei besonders schweren Aufgaben), wie wir auf die Lösung gekommen sind. In der Olympiade müsst ihr solche Überlegungen natürlich nicht aufschreiben, sondern könnt eure ausgefuchste Lösung einfach vom Himmel fallen lassen.
	
	Soweit bekannt ist außerdem angegeben, aus welchem Wettbewerb die betreffende Aufgabe stammt, damit ihr (zum Beispiel in einschlägigen Foren) nach Alternativlösungen suchen könnt.
	\subsection*{Lösungen zu Kapitel~\ref{kapitel:LCF}: \emph{Die Jensensche Ungleichung für nicht-konvexe Funktionen}}

\begin{proof}[Lösung zu Aufgabe~\ref{aufgabe:KaramataSchieben}]
	Betrachte die Funktion $f\colon \mathbb R_{>0}\rightarrow \mathbb R$, $f(x)=\frac1{x^2}-\frac{14}{5x}$. Die zweite Ableitung $f''(x)=\frac{6}{x^4}-\frac{28}{5x^3}$ hat eine Nullstelle bei $x=\frac{15}{14}$ und ist vorher negativ und nachher positiv. Folglich ist die Funktion $f$ auf dem Intervall $\bigl(0,\frac{15}{14}\bigr]$ konvex und auf dem Intervall $\bigl[\frac{15}{14},\infty\bigl)$ konkav. Mit der Karamata-Schiebemethode können wir die Ungleichung auf die beiden Spezialfälle reduzieren, dass $a=b=c=d=e=1$ gilt oder dass $a=b=c=d=x$, $e=5-4x$ gilt. Der erste Fall ist trivial. Der zweite Fall führt auf die Ungleichung
	\begin{equation*}
		\frac{4}{x^2}-\frac{56}{5x}+\frac{1}{(5-4x)^2}-\frac{14}{5(5-4x)}+9\geqslant 0\,.
	\end{equation*}
	Durch Ausmultiplizieren und quadratisches Ausklammern des Gleichheitsfalles $x=1$ erhalten wir $20(x-1)^2(36x^2-60x+25)\geqslant 0$. Den zweiten Faktor erkennen wr als $(6x-5)^2$. Insbesondere ist er stets nichtnegativ. Außerdem sehen wir, dass $a=b=c=d=\frac{5}{6}$ und $e=\frac53$ sowie alle Permutationen davon weitere Gleichheitsfälle sind.
\end{proof}

\begin{proof}[Lösung zu Aufgabe~\ref{aufgabe:log3log2}]
	Wir raten, dass es für das minimale $\kappa$ noch einen weiteren, \enquote{asymptotischen} Gleichheitsfall geben muss, der durch $a=0$, $b=c$ gegeben ist (die eigentliche Aufgabe verbietet $a=0$ natürlich). Diese Überlegung für auf $\kappa=\log_2\parens[\big]{\frac32}$. Für alle $\kappa'<\kappa$ ist die Ungleichung im Fall $a=0$, $b=c$ verletzt, also auch dann, wenn $a>0$ hinreichend klein gewählt wurde. Wenn die Ungleichung für $\kappa$ erfüllt ist, dann ist sie nach der allgemeinen Potenzmittel-Ungleichung auch für alle $\kappa'>\kappa$ erfüllt. Um zu zeigen, dass $\kappa=\log_2\parens[\big]{\frac32}$ tatsächlich minimal ist, müssen wir also nur zeigen, dass die Ungleichung in diesem Fall wirklich gilt.
	
	Weil die Ungleichung homogen in $a$, $b$ und $c$ ist, dürfen wir $a+b+c=1$ annehmen. Betrachte nun die Funktion $f\colon (0,1)\rightarrow \mathbb R$, $f(x)=\parens[\big]{\frac{2x}{1-x}}^\kappa$. Nach etwas anstrengender Rechnung folgt
	\begin{equation*}
		f''(x)=2\kappa\frac{(2x)^{\kappa-2}(1-x)^{\kappa}}{(1-x)^{2(\kappa+1)}}\parens[\big]{2(\kappa-1)(1-x)+(\kappa+1)x}\,.
	\end{equation*}
	Der Term in der Klammer hat eine Nullstelle bei $x=\frac{1-\kappa}{2}$ und ist vorher negativ und nachher positiv. Alle anderen Faktoren sind überall positiv. Folglich ist $f$ auf dem Intervall $\bigl(0,\frac{1-\kappa}{2}\bigr]$ konkav und auf dem Intervall $\bigl[\frac{1-\kappa}{2},1\bigr)$ konvex. Mit der Karamata-Schiebemethode können wir die Ungleichung auf vier Spezialfälle reduzieren. Zwei dieser Fälle lassen sich durch simples Einsetzen überprüfen. Die anderen beiden Fälle sind wie folgt:
	
	\emph{Fall~1: $a=0$, $b$ im Konkavitätsbereich, $c$ im Konvexitätsbereich.} Nach AM-GM und unserer Wahl von $\kappa=\log_2\parens[\big]{\frac32}$ gilt
	\begin{equation*}
		\parens*{\frac{2b}{c}}^\kappa+\parens*{\frac{2c}{b}}^\kappa\geqslant 2\cdot 2^\kappa=3\,.
	\end{equation*}
	
	\emph{Fall~2: $a$ im Konkavitätsbereich, $b=c$ im Konvexitätsbereich.} In diesem Fall müssen wir die Ungleichung
	\begin{equation*}
		\parens*{\frac{2a}{2b}}^\kappa+2\parens*{\frac{2b}{a+b}}^\kappa\geqslant 3
	\end{equation*}
	zeigen. Setze $t=\frac ab$, sodass $\frac{2b}{a+b}=\frac{2}{t+1}$. Die Ungleichung, die wir zeigen müssen, wird also zu
	\begin{equation*}
		g(t)\coloneqq t^\kappa+\frac{3}{(t+1)^\kappa}\geqslant 3\,,
	\end{equation*}
	wobei $0\leqslant t<1$ (dadurch, dass $a$ im Konkavitätsbereich und $b$ im Konvexitätsbereich liegt, muss nämlich $a<b$ gelten). Damit haben wir die Ungleichung auf eine Variable reduziert. Da $\kappa$ irrational ist, haben wir es hier jedoch leider nicht mit einer polynomiellen Ungleichung zu tun. Stattdessen fassen wir die Ungleichung als Extremwertaufgabe auf. Für $t=0$ und $t=1$ ist die Ungleichung offenbar erfüllt, also müssen wir nur nach lokalen Minima Ausschau halten. Alle lokalen Extrema von $g$ sind Lösungen der Gleichung
	\begin{equation*}
		0=g'(t)=\kappa t^{\kappa-1}-\frac{3\kappa}{(t+1)^{\kappa+1}}\,,
	\end{equation*}
	also muss $t^{\kappa-1}(t+1)^{\kappa+1}=3$ gelten. Eine Lösung hiervon ist offensichtlich $t=1$. Um zu schauen, ob noch weitere Lösungen von $t^{\kappa-1}(t+1)^{\kappa+1}=3$ existieren können, leiten wir auch noch die Funktion $h(t)\coloneqq t^{\kappa-1} (t+1)^{\kappa+1}$ ab:
	\begin{equation*}
		h'(t)=t^{\kappa-2}(t+1)^\kappa\parens[\big]{(\kappa+1)t+(\kappa-1)(t+1)}\,.
	\end{equation*}
	Es folgt, dass $h'$ genau bei $t=\frac{\kappa-1}{2\kappa}$ eine Nullstelle hat. Folglich kann es außer $t=1$ höchstens eine weitere Lösung von $h(t)=3$ geben, denn zwischen je zwei Lösungen müsste $h'$ eine Nullstelle haben. Das bedeutet, dass $g'$ im Intervall $(0,1)$ höchstens eine Nullstelle $t_0$ haben kann. Für $t\rightarrow 0$ gilt $\kappa t^{\kappa-1}\rightarrow \infty$, also auch $g'(t)\rightarrow \infty$, während für $t=\frac12$ mit einer einfachen Rechnung $g'(t)<0$ gilt. Also muss $g'(t)$ bei seiner einzigen Nullstelle $t=t_0$ das Vorzeichen von $+$ zu $-$ wechseln. Es folgt, dass $g$ bei $t_0$ ein lokales Maximum hat. Unsere Suche nach lokalen Minima von $g$ im Intervall $(0,1)$ ist somit abgeschlossen: Es gibt keine. Damit ist die Aufgabe gelöst.
\end{proof}

\begin{proof}[Lösung zu Aufgabe~\ref{aufgabe:51}]
	Wir zeigen zuerst die Ungleichung $\frac{13-t}{600}\geqslant\frac1{51+t^2}$ für alle $0\leqslant t\leqslant7$ (die rechte Seite ist genau die Tangente an den Graphen von $f(t)=\frac{1}{51+t^2}$ in $t=3$). Diese Abschätzung folgt aus
	\begin{equation*}
		\frac{13-t}{600}-\frac1{51+t^2}=\frac{663-51t+13t^2-t^3-600}{600(51+t^2)}=\frac{(t-3)^2(7-t)}{600(51+t^2)}\geqslant0
	\end{equation*}für $0\leqslant t\leqslant7$. Wenn $x,y,z\leqslant7$ gilt, können wir also
	\begin{equation*}
		\frac1{51+x^2}+\frac1{51+y^2}+\frac1{51+z^2}\leqslant\frac{13-x}{600}+\frac{13-y}{600}+\frac{13-z}{600}=\frac{39-(x+y+z)}{600}=\frac1{20}
	\end{equation*}
	abschätzen und sind fertig. Falls eine der Variablen größer als 7 ist, können wir sogar deutlich unschärfer abschätzen, und zwar $\sum\frac1{51+x^2}<\frac1{51+7^2}+\frac1{50}+\frac1{50}= \frac1{100}+\frac1{25}=\frac1{20}$. Auch hier sind wir fertig.\qed
	
	\textbf{Lösung zu Aufgabe~\ref{aufgabe:b+c-aUngleichung}.} Weil die Ungleichung homogen ist, dürfen wir $a+b+c=1$ annehmen. Dann gilt
	\begin{equation*}
		\frac{(b+c-a)^2}{(b+c)^2+a^2}=\frac{(1-2a)^2}{(1-a)^2+a^2}=2-\frac{1}{1-2a+2a^2}\,.
	\end{equation*}
	Die behauptete Ungleichung ist also äquivalent zu
	\begin{equation*}
		\frac{1}{1-2a+2a^2}+\frac{1}{1-2b+2b^2}+\frac{1}{1-2c+2c^2}\leqslant 6-\frac35=\frac{27}{5}\,.
	\end{equation*}
	Nun behaupten wir, dass $\frac{1}{1-2x+2x^2}\leqslant \frac{54}{27}x+\frac{27}{25}$ für alle $0\leqslant x\leqslant 1$ gilt (die rechte Seite ist genau die Tangente an den Graphen von $f(x)=\frac{1}{1-2x+2x^2}$ in $x=\frac13$). Diese Abschätzung folgt aus
	\begin{equation*}
		\frac{54x+27}{25}-\frac{1}{1-2x+2x^2}=\frac{108x^3-54x^2+2}{25\parens*{1-2x+2x^2}}=\frac{2(3x-1)^2(6x+1)}{25(1-2x+2x^2)}\geqslant 0\,.
	\end{equation*}
	Es folgt
	\begin{equation*}
		\frac{1}{1-2a+2a^2}+\frac{1}{1-2b+2b^2}+\frac{1}{1-2c+2c^2}\leqslant \frac{54(x+y+z)+3\cdot 27}{25}=\frac{105}{25}=\frac{27}{5}\,.\tag*{\qed}
	\end{equation*}
	
	\textbf{Lösung zu Aufgabe~\ref{aufgabe:USAMO2017}.} Sei $T$ der betrachtete Ausdruck. Durch geschicktes Raten vermuten wir, dass das Minimum von $T$ nicht bei $a=b=c=d=1$ angenommen wird, was auf den Wert $T=\frac45$ führt, sondern bei $a=2$, $b=2$, $c=0$ und $d=0$ (sowie zyklischen Vertauschungen davon), was auf den Wert $T=\frac23$ führt. Das inspiriert uns dazu, die Tangente an $f(x)=\frac{1}{x^3+4}$ in $x=2$ zu betrachten. Wir behaupten dann, dass stets $ \frac{1}{x^3+4}\geqslant \frac14-\frac{x}{12}$ gilt, was aus
	\begin{equation*}
		\frac1{x^3+4}-\frac14-\frac{x}{12}=\frac {12-(3-b)(b^3+4)}{12(b^3+4)}=\frac{b(b+1)(b-2)^2}{12\parens*{b^3+4}}\geqslant 0
	\end{equation*}
	folgt. Mit dieser Abschätzung erhalten wir
	\begin{equation*}
		T\geqslant \frac{a+b+c+d}4-\frac{ab+bc+cd+da}{12}=1-\frac{(a+c)(b+d)}{12}
	\end{equation*}
	Nach AM-GM ist $(a+c)(b+d)\leqslant \frac14(a+b+c+d)^2=4$ und nach Einsetzen sind wir fertig.\qed
	
	\textbf{Lösung zu Aufgabe~\ref{aufgabe:MatBoj2015}.} Um die Nenner zu linearisieren, schätzen wir nach AM-GM wie folgt ab: $a^2+2=a^2+1+1\geqslant 2a+1$. Folglich genügt es, die Ungleichung
	\begin{equation*}
		\frac{a}{2a+1}+\frac{b}{2b+1}+\frac{c}{2c+1}\leqslant 1
	\end{equation*}
	zu beweisen. Nach Ausmultiplizieren, Vereinfachen und Einsetzen von $abc=1$ wird diese Ungleichung zu $3\leqslant a+b+c$, was direkt aus AM-GM folgt.\qed
	
	\textbf{Lösung zu Aufgabe~\ref{aufgabe:DEMO2013}.} Der Einfachheit halber schreiben wir $\beta\coloneqq \frac 1\alpha$, sodass $\beta<1$. Wir wollen die Wurzeln durch lineare Terme ersetzen. Dazu schätzen wir die Funktion $f(x)=\sqrt[\alpha]{x}=x^\beta$ durch ihre Tangente in $x=1$ ab und erhalten $x^\beta\leqslant 1+\beta(x-1)$. Diese Ungleichung ist für alle $x\geqslant 0$ gültig, denn $f$ ist konkav: Wegen $\beta<1$ gilt $f''(x)=\beta(\beta-1)x^{\beta-2}<0$ für alle $x>0$.
	
	Mit dieser Abschätzung folgt
	\begin{align*}
		\sqrt[\alpha]{1+\sqrt[\alpha]{2+\sqrt[\alpha]{\dotsb+\sqrt[\alpha]{n+\sqrt[\alpha]{n+1}}}}}&\leqslant 1+\beta\parens[\bigg]{1+\beta\parens[\Big]{2+\beta\parens[\big]{\dotsb+\beta(n+\beta n)}}}\\
		&=1+\beta+2\beta^2+\dotsb+n\beta^n+n\beta^{n+1}\\
		&<\frac1{(1-\beta)^2}\,.
	\end{align*}
	In der letzten Abschätzung haben wir die für $\abs{x}<1$ gültige Identität
	\begin{equation*}
		\sum_{i=0}^\infty ix^{i-1}=\frac1{(1-x)^2}
	\end{equation*}
	verwendet, welche aus der bekannten geometrischen Summenformel $\sum_{i=0}^\infty x^i=\frac1{1-x}$ durch Ableiten folgt.
\end{proof}
	
\subsection*{Lösungen zu Kapitel~\ref{kapitel:uvw}: \emph{Die \texorpdfstring{$uvw$}{uvw}-Methode}}

\begin{proof}[Lösung zu Aufgabe~\ref{aufgabe:DEMO2015Ungleichung} \textmd{(\href{https://www.mathematik-olympiaden.de/moev/index.php?option=com_download&thema=a&format=raw&datei=A54124b.pdf}{MO 541246})}]
	Durch Ausprobieren finden wir heraus, dass Gleichheit für $x=y=4z$ sowie Permutationen davon angenommen wird. Weil die Ungleichung homogen in $x$, $y$ und $z$ ist, dürfen wir ohne Einschränkung $w=16$ annehmen (diese Wahl ist von den Gleichheitsfällen inspiriert). Mit dieser Wahl lässt sich die Ungleichung als
	\begin{equation*}
		\frac{u}{3}+\frac{3w}{v}\geqslant 5
	\end{equation*}
	schreiben. Fixiere $v$ und $w=16$. Die linke Seite nimmt ihr Minimum an, wenn $u$ minimal ist. Nach dem $uvw$-Theorem müssen dann zwei der Variablen gleich sein. Wir dürfen also $x=y$ und somit $z=\frac{16}{x^2}$ annehmen und erhalten die Ungleichung
	\begin{equation*}
		\frac{2x+\frac{16}{x^2}}{3}+\frac{3}{\frac2x+\frac{x^2}{16}}\geqslant 5\,.
	\end{equation*}
	Indem wir Ausmultiplizieren und den Gleichheitsfall $x=y=4$ ausklammern, erhalten wir nach kurzer Rechnung die äquivalente Ungleichung
	\begin{equation*}
		(x-4)^2\parens*{2x^4+x^3-24x^2+16x+32}\geqslant 0\,.
	\end{equation*}
	Der zweite Faktor lässt sich als $x(x-4)^2+2(x^2-4)^2$ schreiben. Mindestens eines der Quadrate ist stets positiv, also ist der zweite Faktor für $x>0$ stets positiv. Somit gilt die Ungleichung.
\end{proof}

\begin{proof}[Lösung zu Aufgabe~\ref{aufgabe:UngleichungInvertieren}]
	Wir bemerken zuerst, dass in der Nebenbedingung $u^2-2v+2w=1$ alle Variablen enthält und die $uvw$-Methode damit nicht anwendbar ist. Deshalb führen wir zuerst einen Trick durch, den ihr euch merken solltet: Wir \emph{invertieren} die Ungleichung! Das geht folgendermaßen: Angenommen, wir haben $xy+yz+zx>\frac12+2xyz$. Dann müssen wir zeigen, dass die Nebenbedingung verletzt ist, dass also $x^2+y^2+z^2+2xyz\neq 1$ gilt. Tatsächlich werden wir zeigen, dass sogar $x^2+y^2+z^2+2xyz>1$ gilt.\footnote{Allgemein können wir in solchen Situationen erwarten, dass die Nebenbedingung \enquote{nur in eine Richtung} verletzt ist, dass also entweder immer \enquote{$>$} oder immer \enquote{$<$} erfüllt ist. Durch Ausprobieren finden wir dann heraus, welche von beiden Möglichkeiten gilt.} Wenn wir $x$, $y$ und $z$ durch $\lambda x$, $\lambda y$ und $\lambda z$ für ein $\lambda <1$ ersetzen, wird $x^2+y^2+z^2+2xyz$ strikt kleiner (der Fall $x=y=z=0$ kann nicht eintreten, weil wir $xy+yz+zx>\frac12+2xyz$ annehmen). Wenn indem wir ein geeignetes $\lambda<1$ wählen, können wir erreichen, dass die Ungleichung $xy+yz+zx>\frac12+2xyz$ zu einer Gleichheit wird. Wenn wir in diesem Fall $x^2+y^2+z^2+2xyz\geqslant 1$ zeigen können, sind wir fertig, denn bei der Ersetzung wurde der Term $x^2+y^2+z^2+2xyz$ strikt kleiner.
	
	Insgesamt erhalten wir also die folgende, modifizierte Aufgabenstellung, in der Nebenbedingung und Behauptung vertauscht wurden:\addtocounter{caufgabe}{-1}
	\begin{aufgabe*}[$'$]
		Gegeben seien nichtnegative reelle Zahlen $x,y,z\geqslant 0$ mit $xy+yz+zx=\frac12+2xyz$. Zeige, dass
		\begin{equation*}
			x^2+y^2+z^2+2xyz\geqslant 1\,.
		\end{equation*}
	\end{aufgabe*}
	Die Nebenbedingung hat nun die Form $v=\frac12+2w$ und die zu zeigende Ungleichung die Form $u^2-2v+2w\geqslant 1$. Wir können also $v$ und $w$ fixieren und müssen $u$ minimieren. Im Fall $w=0$ muss eine der Variablen verschwinden, sagen wir, $z=0$. Die Nebenbedingung wird dann zu $xy=\frac12$ und die Ungleichung zu $x^2+y^2\geqslant 1$, was sofort aus AM-GM folgt.
	
	Für $w>0$ können wir das $uvw$-Theorem anwenden und finden heraus, dass das Minimum nur dann angenommen wird, wenn zwei der Variablen gleich sind. Wir dürfen also $x=y$ annehmen. Die Nebenbedingung wird dann zu $x^2+2xz=\frac12+2x^2z$ und die Behauptung zu $2x^2+z^2+2x^2z\geqslant 1$. Aus der Nebenbedingung folgt
	\begin{equation*}
		z=\frac{1-2x^2}{4x(1-x)}
	\end{equation*}
	für $x\neq 0,1$; für $x=0$ ist die Nebenbedingung nie erfüllt, während für $x=1$ die Nebenbedingung immer erfüllt ist, aber auch die gewünschte Ungleichung aus trivialen Gründen gilt.
	
	Wenn wir die Gleichung für $z$ in die Behauptung einsetzen, haben wir die Ungleichung auf eine Variable reduziert, allerdings wird sie ziemlich hässlich (es treten Terme 6.\ Grades auf). Wir sollten uns also gut überlegen, was wir ausklammern können. Für $x=y=z=\frac12$ tritt offensichtlich Gleichheit ein, also erwarten wir, dass wir $(2x-1)^2$ ausklammern können. Für $x=y=\frac{\sqrt{2}}{2}$ und $z=0$ tritt ebenfalls Gleichheit ein. Bei diesem Gleichheitsfall können wir nicht erwarten, ihn quadratisch ausklammern zu können, denn er liegt ja wegen $z=0$ am \enquote{Rand} des Definitionsbereiches. Andererseits hat jedes Polynom mit rationalen Koeffizienten, das $\frac{\sqrt{2}}{2}$ als Nullstelle hat, auch $-\frac{\sqrt{2}}{2}$ als Nullstelle. Wir können also erwarten, dass wir zwar nicht $\parens[\big]{x-\frac{\sqrt{2}}{2}}^2$ ausklammern können, dafür aber $2x^2-1$. Tatsächlich ergibt sich nach etwas umständlicher Rechnung die Ungleichung
	\begin{equation*}
		\frac{(2x-1)^2\parens*{2x^2-1}\parens*{6x^2-4x+1}}{16x^2(1-x)^2}\geqslant 0\,.
	\end{equation*}
	Der Term $6x^2-4x+1=6\parens[\big]{x-\frac13}^2+\frac13$ ist für alle reellen Zahlen $x$ positiv. Der Term $2x^2-1$ ist nichtnegativ, weil sich für $x>\frac{\sqrt{2}}{2}$ ein negativer Wert für $z$ ergäbe. Somit gilt diese Ungleichung und wir sind fertig.
\end{proof}
	
\subsection*{Lösungen zu den Kapitel~\ref{kapitel:GraphenInCombo}: \emph{Kombinatorikaufgaben mit Graphentheorie lösen}}

\begin{proof}[Lösung zu Aufgabe~\ref{aufgabe:Feldwege} \textmd{(\href{https://www.mathematik-olympiaden.de/moev/index.php?option=com_download&thema=a&datei=A59123a.pdf&format=raw}{MO 591233})}]
	Wir betrachten den Graphen $G$, dessen Knoten die Orte und dessen Kanten die Feldwege sind. Wir dürfen ohne Einschränkung annehmen, dass $G$ zusammenhängend ist (ansonsten führen wir das folgende Argument in jeder Zusammenhangskomponente durch). Dann hat jeder Knoten von $G$ Grad $3$. Weil die Anzahl der Knoten ungeraden Grades stets gerade ist, muss $G$ gerade viele Knoten haben. Wir teilen die Knoten von $G$ beliebig in Paare auf und fügen für jedes Paar eine Kante hinzu (wenn dadurch parallele Kanten entstehen, ist das kein Problem). Nun hat jeder Knoten Grad $4$, also gibt es nach dem Satz von Euler-Hierholzer einen geschlossenen Weg, der jede Kante genau einmal durchläuft. Entlang dieses Weges färben wir die Kanten abwechselnd rot und grün. Dann hat jeder Knoten zwei rote und zwei grüne ausgehende Kanten. Das gilt insbesondere auch für den ersten (und letzten) Knoten des Weges, denn nach dem Handschlagslemma ist $4\abs{V}=\sum_{v\in V}d(v)=2\abs{E}$, also ist $\abs{E}$ gerade. Nun entfernen wir die hinzugefügten Kanten wieder. Dann hat jeder Knoten immer noch mindestens eine rote und mindestens eine grüne ausgehende Kante. Wenn wir alle roten Kanten zu Radwegen ausbauen, haben wir also die Aufgabe gelöst.
\end{proof}

\begin{proof}[Lösung zu Aufgabe~\ref{aufgabe:50Laender}]
	Für~\ref{teilaufgabe:50} betrachten wir den Graphen $G$, dessen Knoten $v_1,v_2,\dotsc,v_{100}$ die 100 Leute sind. Je zwei Leute aus einem Land verbinden wir mit einer Kante; diese Kanten nennen wir \emph{Länderkanten}. Außerdem verbinden wir $v_1$ mit $v_2$, $v_3$ mit $v_4$, \ldots, $v_{99}$ mit $v_{100}$; diese Kanten nennen wir \emph{Nachbarskanten}. Wenn dabei parallele Kanten entstanden sind, ist das nicht schlimm. Jeder Knoten von $G$ hat Grad $2$, also ist $G$ eine disjunkte Vereinigung von Kreisen. In jedem Kreis wechseln sich außerdem Länder- und Nachbarskanten ab, also hat jeder Kreis gerade Länge. Nun durchlaufen wir jeden Kreis und färben die Knoten abwechselnd rot und grün. Indem wir alle roten Knoten die eine Gruppe bilden lassen und alle grünen Knoten die andere, haben wir die Aufgabe gelöst.
	
	Für~\ref{teilaufgabe:25} teilen wir jedes der 25 Länder temporär in zwei Teile auf, sodass aus jedem Teil genau zwei Leute kommen. Wir teilen die 100 Leute zunächst wie in~\ref{teilaufgabe:50} in zwei Gruppen $G_1$ und $G_2$ auf. In $G_1$ verbinden wir je zwei Leute aus einem Land mit einer Kante. Außerdem verbinden wir alle Paare von Leuten, die vorher im Kreis nebeneinander standen. Diese Kanten nennen wir wie vorher \emph{Länder-} und \emph{Nachbarskanten}. Nach Konstruktion geht von jedem Knoten von $G_1$ eine Länderkante aus. Da es in $G_1$ keine drei Leute gibt, die vorher im Kreis nebeneinander standen, geht von jedem Knoten maximal eine Nachbarskante aus. Die Knoten, von denen noch keine Nachbarskante ausgeht, teilen wir beliebig in Paare auf und verbinden jedes Paar. Etwaige parallele Kanten sind uns dabei wieder egal. Analog zu~\ref{teilaufgabe:50} zerfällt $G_1$ in Kreise gerader Länge, in denen sich Länder- und Nachbarskanten abwechseln. Indem wir die Kreise wieder alternierend färben, haben wir $G_1$ in zwei Gruppen mit den gewünschten Eigenschaften aufgeteilt. Analog verfahren wir mit $G_2$.
\end{proof}

\begin{proof}[Lösung zu Aufgabe~\ref{aufgabe:Kartenspiel}]
	Wir betrachten den gerichteten Graphen $G$, dessen Knoten die möglichen Spielsituationen sind (eine Spielsituation wird stets eindeutig durch die beiden Kartenstapel beschrieben). Zwischen zwei Spielsituationen $s_1$ und $s_2$ ziehen wir genau dann eine gerichtete Kante $s_1s_2$, wenn $s_1$ im nächsten Zug zu $s_2$ führen kann. Fast jeder Knoten hat dann Ausgangsgrad $2$, denn in jedem Zug gibt es zwei Möglichkeiten, in welcher Reihenfolge die Gewinnerin ihre beiden gewonnenen Karten unter ihren Stapel legt. Die einzigen Knoten, die nicht Ausgangsgrad $2$ haben, sind die Spielsituationen, in denen eine Mathematikerin keine Karten mehr hat. In diesem Fall ist das Spiel endlich vorbei und der Ausgangsgrad ist $0$.
	
	Andererseits hat jeder Knoten höchstens Eingangsgrad $2$. Denn für jede Spielsituation gibt es höchstens zwei Möglichkeiten, welche Mathematikerin den letzten Zug gewonnen hat (falls eine Mathematikerin maximal zwei Karten in ihrem Stapel hat, gibt es sogar nur eine Möglichkeit, denn sonst wäre das Spiel schon vorbei gewesen). Für jede Möglichkeit ist die vorherige Verteilung durch die beiden untersten Karten im Stapel der Gewinnerin eindeutig bestimmt: Die Gewinnerin hatte die stärkere der beiden Karten oben auf ihrem Stapel liegen und die Verliererin die schwächere.
	
	Betrachte nun eine beliebige Spielsituation $s$. Sei $G_s=(V_s,E_s)$ der gerichtete Graph aller Spielsituationen, die von $s$ aus erreichbar sind. Wir müssen zeigen, dass $s$ eine Endsituation enthält, also einen Knoten mit Ausgangsgrad $0$. Angenommen, das wäre nicht der Fall. Dann muss jeder Knoten in $G_s$ Ausgangsgrad $2$ haben, sodass einerseits $\sum_{v\in V_s}d^+(v)=2\abs{V_s}$ gilt. Andererseits hat jeder Knoten in $G_s$ Eingangsgrad höchstens $2$, sodass $\sum_{v\in V_s}d^-(v)\leqslant 2\abs{V_s}$ gilt. Wenn wir zeigen können, dass in dieser Ungleichung in Wirklichkeit \enquote{$<$} statt \enquote{$\leqslant$} gilt, haben wir einen Widerspruch zu dem allgemeinen Fakt, dass in jedem gerichteten Graphen die Summe aller Eingangsgrade gleich der Summe aller Ausgangsgrade ist.
	
	Es genügt also, einen einzigen Knoten in $G_s$ zu finden, dessen Eingangsgrad kleiner als $2$ ist. Nur wo bekommen wir einen solchen Knoten her? An dieser Stelle erinnern wir uns an das Extremalprinzip! Wir wählen eine der beiden Mathematikerinnen aus und betrachten eine Spielsituation $s_\mathrm{max}$ in $G_s$, in der sie maximal viele Karten in ihrem Stapel hatte. Dann kann sie den vorherigen Zug nicht verloren haben, sonst hätte sie vor diesem Zug noch mehr Karten in ihrem Stapel gehabt. Folglich muss $s_\mathrm{max}$ Eingangsgrad $1$ haben und wir sind fertig.\qed
	
	\textbf{Lösung zu Aufgabe~\ref{aufgabe:Rechtecksparkettierung}.} Wir legen das $m\times n$-Rechteck $ABCD$ so in ein Koordinatensystem, dass $A=(0,0)$, $B=(m,0)$, $C=(m,n)$ und $D=(0,n)$ gilt. %\footnote{Die Schule hat euch möglicherweise beigebracht, dass\enquote{$A=(0,0)$} ein Formfehler ist und ihr die Notation \enquote{$A\parens*{0\ \middle|\ 0}$} verwenden sollt. Ich kann euch versichern, dass niemand außerhalb der Schule diese Notation verwendet, genausowenig wie irgendjemand \enquote{ganzrationale Funktion} statt \enquote{Polynom} sagt oder eine Rennstrecke näherungsweise durch den Graphen der Funktion $f(x)=\frac x2+\frac6{x^2+3}$ beschreibt.}
	Dann haben alle Rechtecke in der Parkettierung Eckpunkte mit ganzzahligen Koordinaten und ihre Seiten liegen parallel zu den Koordinatenachsen. 
	
	Wir konstruieren einen Graphen $G$ wie folgt: Die Knoten von $G$ sind die Mittelpunkte aller kleinen Rechtecke in der Parkettierung und außerdem alle Eckpunkte, deren $x$-Koordinate durch $a$ und deren $y$-Koordinate durch $b$ teilbar ist. Zum Beispiel ist $A$ ein Knoten von $G$. Jeden Knoten, der der Mittelpunkt eines kleinen Rechtecks ist, verbinden wir mit allen Eckpunkten dieses Rechtecks, die in $G$ liegen. Es kann durchaus passieren, dass manch ein Mittelpunkt mit gar keinem Eckpunkt verbunden wird, das ist aber nicht schlimm. Im folgenden Bild seht ihr eine Parkettierung für $a=3$ und $b=2$ sowie den zugehörigen Graphen $G$. Die Mittelpunkte sind schwarz gefüllt, die anderen Knoten weiß. Wie ihr seht, hat $G$ im Allgemeinen nicht sonderlich viele Kanten.
	
	\begin{figure}[ht]
		\centering
		\begin{tabularx}{\textwidth}{X c X c X}
			& \begin{tikzpicture}[x=0.65cm,y=0.65cm]
				\draw (0,0) to (6,0) to (6,5) to (0,5) to cycle;
				\draw (0,1) to (3,1) to (3,0);
				\draw (3,1) to (6,1);
				\draw (0,3) to (1,3);
				\draw (1,1) to (1,5);
				\draw (4,3) to (6,3);
				\draw (4,1) to (4,5);
				\draw (5,1) to (5,5);
				\draw (1,2) to (4,2);
				\draw (1,4) to (4,4);
				\draw (2,2) to (2,4);
				\draw (3,2) to (3,4);
			\end{tikzpicture} & & \begin{tikzpicture}[x=0.65cm,y=0.65cm]
				\begin{scope}[line width=0.3,dashed]
					\draw (0,0) to (6,0) to (6,5) to (0,5) to cycle;
					\draw (0,1) to (3,1) to (3,0);
					\draw (3,1) to (6,1);
					\draw (0,3) to (1,3);
					\draw (1,1) to (1,5);
					\draw (4,3) to (6,3);
					\draw (4,1) to (4,5);
					\draw (5,1) to (5,5);
					\draw (1,2) to (4,2);
					\draw (1,4) to (4,4);
					\draw (2,2) to (2,4);
					\draw (3,2) to (3,4);
				\end{scope}
				\coordinate (A) at (0,0);
				\coordinate (B) at (6,0);
				\coordinate (E) at (0.5,2);
				\coordinate (F) at (0.5,4);
				\coordinate (G) at (1.5,0.5);
				\coordinate (H) at (4.5,0.5);
				\coordinate (I) at (2.5,1.5);
				\coordinate (J) at (4.5,2);
				\coordinate (K) at (4.5,4);
				\coordinate (L) at (5.5,2);
				\coordinate (M) at (5.5,4);
				\coordinate (N) at (1.5,3);
				\coordinate (O) at (2.5,3);
				\coordinate (P) at (3.5,3);
				\coordinate (Q) at (2.5,4.5);
				\coordinate (R) at (3,0);
				\coordinate (S) at (3,2);
				\coordinate (T) at (3,4);
				\draw (A) to (G) to (R) to (H) to (B);
				\draw (O) to (S) to (P) to (T) to cycle;
				\draw[fill=white] (A) circle (2pt);
				\draw[fill=white] (B) circle (2pt);
				\draw[fill=black] (E) circle (2pt);
				\draw[fill=black] (F) circle (2pt);
				\draw[fill=black] (G) circle (2pt);
				\draw[fill=black] (H) circle (2pt);
				\draw[fill=black] (I) circle (2pt);
				\draw[fill=black] (J) circle (2pt);
				\draw[fill=black] (K) circle (2pt);
				\draw[fill=black] (L) circle (2pt);
				\draw[fill=black] (M) circle (2pt);
				\draw[fill=black] (N) circle (2pt);
				\draw[fill=black] (O) circle (2pt);
				\draw[fill=black] (P) circle (2pt);
				\draw[fill=black] (Q) circle (2pt);
				\draw[fill=white] (R) circle (2pt);
				\draw[fill=white] (S) circle (2pt);
				\draw[fill=white] (T) circle (2pt);
			\end{tikzpicture} &
		\end{tabularx}
	\end{figure}	
	Nun betrachten wir die Grade der Knoten in $G$. Wenn $v$ der Mittelpunkt eines kleinen Rechtecks $R$ ist, dann ist $d(v)$ gerade. Denn wenn $R$ ein $a\times 1$-Rechteck ist, dann muss für jeden Eckpunkt, der in $G$ liegt, auch der Eckpunkt am anderen Ende der horizontalen Kante mit Länge $a$ in $G$ liegen. Wenn $R$ ein $1\times b$-Rechteck ist, muss analog der Eckpunkt am anderen Ende der vertikalen Kante mit Länge $b$ ebenfalls in $G$ liegen. Wenn $v\neq A,B,C,D$ der Eckpunkt eines Rechtecks ist, dann muss $v$ ebenfalls geraden Grad haben, denn $v$ ist stets Eckpunkt von zwei oder vier kleinen Rechtecken.
	
	Wir wissen aber, dass $G$ einen Knoten mit ungeradem Grad enthält, denn $A$ liegt in $G$ und hat Grad~1. Weil die Anzahl der Knoten ungeraden Grades stets gerade ist, muss es in $G$ mindestens einen weiteren Knoten $w$ mit ungeradem Grad geben. Nach den obigen Überlegungen muss $w$ einer der Eckpunkte $B$, $C$ oder $D$ sein. Dann muss die $x$-Koordinate eines der Eckpunkte $B$, $C$ oder $D$ durch $a$ und seine $y$-Koordinate durch $b$ teilbar sein. Daraus folgt aber genau, dass $m$ durch $a$ oder $n$ durch $b$ teilbar ist.
\end{proof}

	\subsection*{Lösungen zu Kapitel~\ref{kapitel:Pell}: \emph{Pellsche Gleichungen}}

\begin{proof}[Lösung zu Aufgabe~\ref{aufgabe:561246} \textmd{(\href{https://www.mathematik-olympiaden.de/moev/index.php?option=com_download&thema=a&datei=A56124b.pdf&format=raw}{MO 561246})}]
	Wir benutzen die bekannte Summenformel
	\begin{equation*}
		\sum_{k=1}^nk^2=\frac{n(n+1)(2n+1)}{6}\,.
	\end{equation*}
	Damit die Bedingung für~$m$ erfüllt ist, muss eine nichtnegative ganze Zahl~$n\geqslant 0$ existieren, sodass Folgendes gilt:
	\begin{align*}
		m^3=\sum_{k=1}^{n+m}k^2-\sum_{k=1}^nk^2&=\frac{(n+m)(n+m+1)(2n+2m+1)}{6}-\frac{n(n+1)(2n+1)}{6}\\
		&=\frac{m}{6}\parens*{2m^2+6n^2+6nm+3m+6n+1}\,.
	\end{align*}
	Indem wir $m^3$ auf die andere Seite bringen, durch $m$ teilen (was wegen $m>0$ eine Äquivalenzumformung ist) und mit~$6$ multiplizieren, erhalten wir
	\begin{equation*}
		0=-4m^2+6n^2+6nm+3m+6n+1\,.
	\end{equation*}
	Wir wollen diese Gleichung durch quadratische Ergänzung auf eine Pellsche Gleichung zurückführen. Nach etwas Rumprobieren erhalten wir
	\begin{equation*}
		0=6\parens*{n+\frac{m}{2}+\frac12}^2-\frac{11}2m^2-\frac12\quad\Longleftrightarrow\quad 3\parens*{2n+m+1}^2-11m^2=1\,.
	\end{equation*}
	Die Substitution $\ell\coloneqq 2n+m+1$ führt nun auf $3\ell^2-11m^2=1$, was eine Gleichung der Form ist, die wir am Ende des Kapitels untersucht haben. An dieser Stelle ist ein guter Moment, uns klar zu machen, dass wir durch die Substitution nichts verschenkt haben: Für jedes Lösungspaar $(\ell,m)$ von $3\ell^2-11m^2$ müssen~$\ell$ und~$m$ von unterschiedlicher Parität sein. Ferner muss offensichtlich $\ell>m$ gelten. Es gibt also stets ein $n\geqslant 0$ mit $\ell=2n+m+1$.
	
	Die Gleichung $3\ell^2-11m^2=1$ hat die Lösung $(\ell,m)=(2,1)$. Wie wir gesehen haben, existieren dann unendlich viele weitere Lösungen $(\ell_i,m_i)$, $i=1,2,\dotsc$, die durch
	\begin{equation*}
		\ell_i\sqrt{3}+m_i\sqrt{11}=\parens*{2\sqrt{3}+\sqrt{11}}\parens*{x_0+y_0\sqrt{33}}^i
	\end{equation*}
	gegeben sind, wobei $(x_0,y_0)$ die Fundamentallösung der Pellschen Gleichung $x^2-33y^2=1$ ist. Damit haben wir bereits gezeigt, dass unendlich viele~$m$ mit der gewünschten Eigenschaft existieren. Was noch zu tun ist, ist eine Lösung mit $m>1$ zu konstruieren. Nach etwas Rumprobieren erhalten wir, dass die Fundamentallösung durch $(x_0,y_0)=(23,4)$ gegeben ist. Nun ist $(2\sqrt{3}+\sqrt{11})(23+4\sqrt{33})=90\sqrt{3}+47\sqrt{11}$, also ist $(\ell_1,m_1)=(90,47)$ eine weitere Lösung von $3\ell^2-11m^2=1$. Aus $90=\ell_1=2n+m_1+1=2n+48$ folgt $n=21$. Somit ist $m=47$ eine Zahl mit der gewünschten Eigenschaft und
	\begin{equation*}
		\underbrace{22^2+23^2+\dotsb+68^2}_{\text{$47$ Summanden}}=47^3\,.\qedhere
	\end{equation*}
\end{proof}

\begin{proof}[Lösung zu Aufgabe~\ref{aufgabe:521246} \textmd{(\href{https://www.mathematik-olympiaden.de/moev/index.php?option=com_download&thema=a&datei=A52124b.pdf&format=raw}{MO 521246})}]
	Wir lösen zuerst die Rekursionsgleichung für $(a_n)_{n\geqslant 1}$ mit der Standardmethode für lineare Rekursionen (siehe das Kapitel \emph{Lineare Rekursionen} im Heft für Klasse~$10$) und erhalten
	\begin{equation*}
		a_n=\frac{1}{2\sqrt{2}}\parens*{\parens*{1+\sqrt{2}}^n-\parens*{1-\sqrt{2}}^n}\,.
	\end{equation*}
	Für $z=x+y\sqrt{2}\in \mathbb Z[\sqrt{2}]$ gilt stets $y=\frac{1}{2\sqrt{2}}(z-\overline{z})$. Wegen $\overline{(1+\sqrt{2})^n}=(\overline{1+\sqrt{2}})^n=(1-\sqrt{2})^n$ ist die Gleichung für $a_n$ genau von dieser Form. Es folgt: Wenn wir $(1+\sqrt{2})^n$ in der Form $x_n+y_n\sqrt{2}$ mit ganzen Zahlen $x_n$ und $y_n$ schreiben, dann ist $a_n=y_n$.
	
	Als nächstes bemerken wir, dass $z\coloneqq 1+\sqrt{2}$ eine Lösung der Gleichung $N(z)=-1$ ist. Ferner ist $z=(1+\sqrt{2})^2=3+2\sqrt{2}$ die Fundamentallösung der Gleichung $N(z)=1$. Nach dem Satz über die Lösbarkeit der Pellschen Gleichung sind alle Lösungen der Gleichung $N(z)=1$ durch $z=\pm (3+2\sqrt{2})^n=\pm (1+\sqrt{2})^{2n}$ gegeben, wobei $n$ durch alle ganzen Zahlen läuft. Indem wir uns auf $z=(1+\sqrt{2})^{2n}$ für $n\geqslant 1$ einschränken, durchlaufen wir genau diejenigen Lösungen mit $z>1$. Das entspricht genau denjenigen Lösungen der Pellschen Gleichung $x^2-2y^2=1$, die $x>1$ und $y>0$ erfüllen. Analog sind alle Lösungen von $N(z)=-1$ durch $z=\pm (1+\sqrt{2})(3+2\sqrt{2})^n=(1+2\sqrt{2})^{2n+1}$ gegeben, wobei $n$ alle ganzen Zahlen durchläuft. Wenn wir nur $z=(1+2\sqrt{2})^{2n+1}$ für $n\geqslant 0$ betrachten, schränken wir uns wieder auf die Lösungen mit $z>1$ ein. Das entspricht den Lösungen der Pellschen Gleichnung $x^2-2y^2=-1$, die $x,y>0$ erfüllen.
	
	Insgesamt sehen wir: Die Folge $(a_n)_{n\geqslant 1}$ besteht genau aus denjenigen positiven ganzen Zahlen $y>0$, die Teil eines Lösungspaares $(x,y)$ der Pellschen Gleichungen $x^2-2y^2=\pm 1$ sind. Für gerades~$n$ bekommen wir Lösungen von $x^2-2y^2=1$ und für ungerades~$n$ bekommen wir Lösungen von $x^2-2y^2=-1$.
	
	Jetzt wenden wir uns der eigentlichen Aufgabe zu und untersuchen nacheinander die beiden Bedingungen (und zwar fast wortwörtlich auf die gleiche Weise).
	
	\emph{Bedingung~\ref{bedingung:RationaleApproximation}.} Nach dem Satz über die Lösbarkeit der Pellschen Gleichung gibt es unendlich viele Paare $(p,q)$ mit $p>1$ und $q>0$, welche die Gleichung $p^2-2q^2=1$ erfüllen. In diesem Fall gilt $p/q>\sqrt{2}$ und folglich
	\begin{equation*}
		\abs*{\frac pq-\sqrt{2}}=\frac{1}{\abs*{\frac{p}{q}+\sqrt{2}}q^2}<\frac{1}{2\sqrt{2}q^2}\,.
	\end{equation*}
	Folglich ist~\ref{bedingung:RationaleApproximation} für alle $\beta\geqslant \frac{1}{2\sqrt{2}}$ erfüllt. Wir zeigen nun, dass~\ref{bedingung:RationaleApproximation} für $\beta<\frac{1}{2\sqrt{2}}$ nicht erfüllt sein kann. Angenommen, das Paar $(p,q)$ erfüllt $\abs{p/q-\sqrt{2}}<\beta/q^2$. Dann gilt $p/q<\sqrt{2}+\beta/q^2$. Andererseits gilt stets $\abs{p^2-2q^2}\geqslant 1$ und somit
	\begin{equation*}
		\frac{\beta}{q^2}>\abs*{\frac pq-\sqrt{2}}\geqslant \frac{1}{\abs*{\frac{p}{q}+\sqrt{2}}q^2}>\frac{1}{2\sqrt{2}q^2+\beta}\,.
	\end{equation*}
	Diese Ungleichungskette führt auf $\beta^2>(1-2\sqrt{2}\beta)q^2$, was für $\beta<\frac1{2\sqrt{2}}$ nur für endlich viele~$q$ erfüllt sein kann.
	
	\emph{Bedingung~\ref{bedingung:NurEndlichVieleNichtInDerFolge}}. Nach dem Satz Nach dem Satz über die Lösbarkeit der Pellschen Gleichung gibt es unendlich viele Paare $(p,q)$ mit $p>1$ und $q>0$, welche die Gleichung $p^2-2q^2=2$ erfüllen (denn $(p,q)=(2,1)$ ist eine Lösung, also gibt es unendlich viele weitere). In diesem Fall gilt $p/q>\sqrt{2}$ und folglich
	\begin{equation*}
		\abs*{\frac pq-\sqrt{2}}=\frac{2}{\abs*{\frac{p}{q}+\sqrt{2}}q^2}<\frac{1}{\sqrt{2}q^2}\,.
	\end{equation*}
	Nach unserer obigen Beobachtung kann~$q$ aber nicht in der Zahlenfolge $(a_n)_{n\geqslant 1}$ auftauchen. Somit ist~\ref{bedingung:NurEndlichVieleNichtInDerFolge} für alle $\beta\geqslant \frac{1}{\sqrt{2}}$ verletzt. Umgekehrt werden wir zeigen, dass~\ref{bedingung:NurEndlichVieleNichtInDerFolge} für $\beta<\frac{1}{\sqrt{2}}$ wahr ist. Wenn $(p,q)$ ein beliebiges Zahlenpaar ist, sodass $q$ nicht in der Folge $(a_n)_{n\geqslant 1}$ auftritt, dann muss $\abs{p^2-2q^2}\geqslant 2$ gelten. Wenn gleichzeitig $\abs{p/q-\sqrt{2}}<\beta/q^2$ erfüllt ist, dann folgt $p/q<\sqrt{2}+\beta/q^2$ und somit
	\begin{equation*}
		\frac{\beta}{q^2}>\abs*{\frac pq-\sqrt{2}}\geqslant \frac{2}{\abs*{\frac{p}{q}+\sqrt{2}}q^2}>\frac{2}{2\sqrt{2}q^2+\beta}\,.
	\end{equation*}
	Diese Ungleichungskette führt auf $\beta^2>(2-2\sqrt{2}\beta)q^2$, was für $\beta<\frac1{\sqrt{2}}$ nur für endlich viele $q$ erfüllt sein kann. Also gilt~\ref{bedingung:NurEndlichVieleNichtInDerFolge} für alle $\beta<\frac{1}{\sqrt{2}}$.
	
	Insgesamt sehen wir, dass~\ref{bedingung:RationaleApproximation} und~\ref{bedingung:NurEndlichVieleNichtInDerFolge} genau dann beide erfüllt sind, wenn $\frac{1}{2\sqrt{2}}\leqslant \beta<\frac{1}{\sqrt{2}}$.
\end{proof}
	\subsection*{Lösungen zu Kapitel~\ref{kapitel:Galois}: \emph{Nullstellen von Polynomen in der Zahlentheorie}}

\begin{proof}[Lösung zu Aufgabe~\ref{aufgabe:621246}]
	Durch Ausprobieren finden wir heraus, dass
	\begin{equation*}
		-0{,}6<\alpha<-0{,}5\,,\quad 0{,}6<\beta<0{,}7\quad\text{und}\quad 2<\gamma <3
	\end{equation*}
	gilt. Etwas genauer können wir hierfür wie folgt argumentieren: Durch Einsetzen erhalten wir $f(-0{,}6)<0<f(-0{,}5)$, also muss $f$ nach dem Zwischenwertsatz zwischen $-0{,}6$ und $-0{,}5$ eine Nullstelle haben. Analog sehen wir, dass $f$ zwischen $0{,}6$ und $0{,}7$ sowie zwischen $2$ und $3$ jeweils eine Nullstelle haben muss. Da $f$ ein Polynom dritten Grades ist, kann es insgesamt höchstens drei Nullstellen haben, sodass wir damit alle Nullstellen gefunden haben müssen.
	
	Es folgt $0<\alpha^n+\beta^n$ für alle $n\geqslant 1$. Ferner ist $\alpha^n+\beta^n<0{,}36+0{,}49<1$ für alle $n\geqslant 2$. Für $n=1$ gilt ebenfalls $\alpha+\beta <0{,7} -0{,}5<1$. Schließlich haben wir im Theorieteil des Kapitels gezeigt, dass $u_n\coloneqq \alpha^n+\beta^n+\gamma^n$ für alle $n\geqslant 0$ eine ganze Zahl ist. Es folgt
	\begin{equation*}
		\lceil \gamma^n\rceil=u_n\quad\text{und}\quad\lfloor \gamma^n\rfloor =u_n-1\quad \text{für alle }n\geqslant 1\,.
	\end{equation*}
	Damit können wir nun beide Teilaufgaben lösen.
	
	Für~\ref{teilaufgabe:621246} bemerken wir, dass die Folge $(u_n)_{n\geqslant 0}$ die Rekursionsgleichung $u_{n+3}=3u_{n+2}-u_n$ erfüllt. Anhand der Koeffizienten von $f$ können wir $\alpha+\beta+\gamma=3$ und $\alpha\beta+\beta\gamma+\gamma\alpha=0$ ablesen. Es folgt, dass $u_0=\alpha^0+\beta^0+\gamma^0=3$, $u_1=\alpha+\beta+\gamma=3$ und $u_2=\alpha^2+\beta^2+\gamma^2=(\alpha+\beta+\gamma)^2-2(\alpha\beta+\beta\gamma+\gamma\alpha)=9$ allesamt durch~$3$ teilbar sind. Aus der Rekursionsgleichung folgt dann sofort, dass $u_n$ für jedes $n\geqslant 0$ durch~$3$ teilbar sein muss. Damit ist~\ref{teilaufgabe:621246} gezeigt.
	
	Für~\ref{teilaufgabe:IMOSL1988} können wir die obige Rekursion modulo~$17$ betrachten. Alternativ können wir benutzen, dass $x^3-3x^2+1\equiv (x-4)(x-5)(x+6)\mod 17$ gilt. Diese Faktorisierung legt die Vermutung $u_n\equiv 4^n+5^n+(-6)^n\mod 17$ nahe. Und tatsächlich: Wenn $u_n'\coloneqq 4^n+5^n+(-6)^n$, dann lässt sich sofort nachprüfen, dass $u_n\equiv u_n'\mod 17$ für $n=0,1,2$ gilt. Ferner erfüllt $u_n'$ ebenfalls die Rekursion $u_{n+3}'\equiv 3u_{n+2}-u_n\mod 17$. Somit gilt in der Tat $u_n\equiv u_n'\mod 17$ für alle $n\geqslant 0$. Nach dem kleinen Satz von Fermat ist $4^{16}\equiv 5^n\equiv (-6)^n\mod 17$. Also ist $(u_n)_{n\geqslant 0}$ periodisch modulo~$17$ mit Periodenlänge~$16$. Wegen $2020\equiv 4\mod 17$ und $2220\equiv -4\mod 16$ müssen wir nur zeigen, dass $u_4\equiv u_{-4}\equiv 1\mod 17$ gilt (wobei $u_n$ für $n<0$ rekursiv durch $u_n=3u_{n+2}-u_{n+3}$ definiert ist). Aus der Rekursionsgleichung erhalten wir $u_3=24$ und $u_4=69$ sowie $u_{-1}=0$, $u_{-2}=6$, $u_{-3}=-3$ und $u_{-4}=18$. Und tatsächlich ist $69\equiv 18\equiv 1\mod 17$.
\end{proof}

\begin{proof}[Lösung zu Aufgabe~\ref{aufgabe:VAIMO2011_2}]
	Wir bemerken zunächst, dass $\sqrt[3]{28}-3$ eine Nullstelle des kubischen Polynoms $(X+3)^3-28=X^3+9X^2+27X-1$ ist. Die anderen beiden Nullstellen sind durch $\sqrt[3]{28}\zeta-3$ und $\sqrt[3]{28}\zeta^2-3$ gegeben, wobei $\zeta\coloneqq \mathrm{e}^{2\pi\mathrm{i}/3}=\frac{1+\sqrt{3}\mathrm{i}}{2}$ eine dritte Einheitswurzel ist.
	
	Betrachte nun die Folge $(u_m)_{m\in\mathbb Z}$, die durch
	\begin{equation*}
		u_m\coloneqq \parens*{\sqrt[3]{28}-3}^m+\parens*{\sqrt[3]{28}\zeta-3}^m+\parens*{\sqrt[3]{28}\zeta^2-3}^m
	\end{equation*}
	gegeben ist. Wir haben im Theorieteil des Kapitels gezeigt, dass $u_m$ für alle $m\geqslant 0$ eine ganze Zahl ist. Ferner erfüllt $(u_m)_{m\in\mathbb Z}$ die Rekursionsgleichung $u_{m+3}=9u_{m+2}+27u_{m+1}-u_m$. Indem wir die Gleichung in der Form $u_{m}=27u_{m+1}+9u_{m+2}-u_{m+3}$ schreiben, sehen wir, dass $u_m$ auch für alle $m<0$ eine ganze Zahl ist. Analog zur Lösung von Aufgabe~\ref{aufgabe:VAIMO2011_2} können wir die Werte von $u_0$, $u_1$ und $u_2$ aus den Koeffizienten von $X^3+9X^2+27X-1$ ablesen. Wir finden $u_0=3$, $u_1=-9$ sowie $u_2=(-9)^2-2\cdot 27=27$. Nun fällt uns auf, dass $u_0\equiv u_1\equiv u_3\equiv 3\mod 6$ gilt. Aus der Rekursion ist dann klar, dass $u_m\equiv 3\mod 6$ auch für alle $m<0$ gilt.
	
	Schließlich bemerken wir $\abs{\sqrt[3]{28}\zeta-3}>3$. Das lässt sich am einfachsten geometrisch sehen: Die komplexen Zahlen $0$, $3$ und $\sqrt[3]{28}\zeta$ bilden ein Dreieck in der komplexen Ebene, dessen Winkel bei $0$ genau $120^\circ$ beträgt. Die Seite, die diesem Winkel gegenüberliegt, hat die Länge $\abs{\sqrt[3]{28}\zeta-3}$. Aber dem größten Winkel eines Dreiecks liegt stets die größte Seite gegenüber, also gilt in der Tat $\abs{\sqrt[3]{28}\zeta-3}>3$ (und sogar $\abs{\sqrt[3]{28}\zeta-3}>\sqrt[3]{28}$). Analog gilt $\abs{\sqrt[3]{28}\zeta^2-3}>3$. Nach der Dreiecksungleichung und der Definition von $b$ folgt nun
	\begin{equation*}
		\abs*{b-u_{-n}}\leqslant \abs[\Big]{b-\parens*{\sqrt[3]{28}-3}^{-n}}+\abs*{\sqrt[3]{28}\zeta-3}^{-n}+\abs*{\sqrt[3]{28}\zeta-3}^{-n}<1+3^{-n}+3^{-n}<2\,.
	\end{equation*}
	Weil $b$ und $u_{-n}$ ganze Zahlen sind, muss also $\abs*{b-u_{-n}}\leqslant 1$ sein. Wegen $u_{-n}\equiv 3\mod 6$ kann $b$ modulo~$6$ nur die Reste~$2$, $3$ oder~$4$ annehmen. Insbesondere kann $b$ nicht durch~$6$ teilbar sein, wie behauptet.
\end{proof}
	
	\newpage
	\phantomsection\cftaddtitleline{toc}{part}{MatBoj-Regeln}{\thepage}
	% Damit in der PDF-Navigationsleiste auch der Abschnitt "MatBoj-Regeln" auftaucht, muss ein zusätzliches Bookmark gesetzt werden. Irrelevant für die Druckversion.
	\pdfbookmark{MatBoj-Regeln}{MatBoj}
	\section*{MatBoj-Regeln} 

MatBoj -- abgeleitet aus dem Russischen -- steht für \enquote{mathematischer Kampf}.

Zwei Teams lösen Aufgaben und präsentieren anschließend ihre Lösungen.

\subsection*{Phase 1: Das Lösen der Aufgaben}
Jede Mannschaft gibt sich einen Namen und wählt einen Mannschaftskapitän und einen Stellvertreter. Diese vertreten die Mannschaft als Sprecher. Nur sie können für die Mannschaft verbindliche Entscheidungen verkünden.

Beide Teams erhalten den gleichen Satz von Aufgaben. Ihnen steht eine vorher bekanntgegebene Zeit zur Verfügung, um die Aufgaben getrennt voneinander zu lösen. 

Sollte einem Teammitglied eine Aufgabe bereits bekannt sein, so ist es aus Fairnessgründen dazu aufgefordert, dies der Jury bekanntzumachen (eventuell wird die betreffende Aufgabe durch eine neue ersetzt).


\subsection*{Phase 2: Das Vorstellen der Lösungen}
\begin{itemize}
	\item Den beiden Kapitänen wird gleichzeitig eine leichte Einstiegsaufgabe gestellt, die sie ohne Hilfsmittel lösen müssen. Keines der anderen Teammitglieder darf ihnen dabei helfen. Wer die richtige Antwort gibt, gewinnt für sein Team das Recht zu entscheiden, welches Team als erstes herausfordert. Gibt einer der Kapitäne eine falsche Antwort, erhält das Team des anderen Kapitäns dieses Recht.  
	\item \textbf{Herausfordern:} Das entsprechende Team fordert vom gegnerischen Team eine Aufgabe. Das herausgeforderte Team kann die Herausforderung annehmen oder ablehnen:
	\begin{itemize}
		\item Die \textit{Herausforderung wird angenommen}: Das herausgeforderte Team entsendet ein Teammitglied als \textit{Referenten}, der eine Lösung der Aufgabe vorstellt, das herausfordernde Team entsendet einen \textit{Kritiker}, der Lücken in der Lösung zu finden versucht. Nach der Vorstellung der Lösung darf der Kritiker erst Verständnisfragen stellen und dann die vorgetragene Lösung kritisieren und die von ihm aufgedeckten Lücken füllen. Hilfe aus dem Team ist unzulässig.
		\item Die \textit{Herausforderung wird abgelehnt}: Das herausfordernde Team entsendet ein Teammitglied, das eine Lösung der Aufgabe vorstellt, das herausgeforderte  Team entsendet einen Kritiker, der Lücken in der Lösung zu finden versucht. Nach der Vorstellung der Lösung darf der Kritiker erst Verständnisfragen stellen und dann die vorgetragene Lösung kritisieren. Er darf jedoch keine von ihm aufgedeckten Lücken füllen. Hilfe aus dem Team ist unzulässig.
	\end{itemize}
	\item \textbf{Bewertung:} Jede Aufgabe ist 12 Punkte wert. Der Referent erhält eine der Punktzahlen $0, 2, 4, 6, 8, 10, 12$, je nachdem, wie richtig und vollständig die von ihm vorgetragene Lösung ist. Der Kritiker erhält für das Aufdecken der Lücken in der vorgetragenen Lösung und für das Füllen dieser Lücken jeweils die Hälfte der noch nicht vergebenen Punkte. Wie weit Referent und Kritiker ihren Aufgaben im Einzelnen gerecht wurden, liegt im Ermessen der Jury.
	\item \textbf{Invalid challenge:} Wird die Herausforderung abgelehnt und kann das herausfordernde Team keine Lösung präsentieren, liegt ein \emph{invalid challenge} vor. Die Einschätzung, ob es sich um eine Lösung handelt, liegt im Ermessen der Jury.
	
	In diesem Fall erhält das herausgeforderte Team $6$ Punkte.
	\item Die \textit{nächste Herausforderung}: Es wird abwechselnd herausgefordert. Liegt ein invalid challenge vor, muss das herausfordernde Team erneut eine Aufgabe fordern.
	\item Die Endphase des Wettbewerbs: Zu einem beliebigen Zeitpunkt kann jedes der beiden Teams beschließen, keine Lösungen mehr zu präsentieren. Das betreffende Team muss aber weiter Kritiker entsenden, da das andere Team solange weiter Lösungen vorstellen kann, wie es will. Die Kritiker dürfen in dieser Endphase des Wettbewerbs nur noch Lücken in den vorgetragenen Lösungen aufzeigen, aber nicht mehr füllen.
	\item Am Ende des Wettbewerbs muss jedes Teammitglied mindestens einmal als Referent bzw. als Kritiker entsandt worden sein.
	\item \textbf{Time-Out:} Jedes Team hat dreimal im ganzen Wettbewerb die Möglichkeit, ein Time-Out (1 Minute) zu fordern. In dieser Zeit dürfen sich die Repräsentanten beider Teams  mit ihren Teammitgliedern absprechen und auch ausgewechselt werden.
	\item Am Ende des MatBojs gewinnt das Team mit der größeren Punktsumme. 
\end{itemize}
%\end{document}\newpage
	
	\section*{Notizen:}
\end{document}