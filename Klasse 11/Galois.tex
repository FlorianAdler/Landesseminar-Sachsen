\section{Nullstellen von Polynomen in der Zahlentheorie}\label{kapitel:Galois}
Es gibt einen Typus von Olympiade-Aufgaben, in denen ihr zahlentheoretische Eigenschaften von bestimmten irrationalen Zahlen untersuchen sollt. Solche Aufgaben wirken auf den ersten Blick völlig unlösbar, aber in Wirklichkeit gibt es einen einfachen Trick, mit dem sie sich fast immer lösen lassen.

Statt euch den Trick direkt zu verraten, demonstrieren wir ihn zuerst an einem Beispiel.
\begin{aufgabe*}\label{aufgabe:1+sqrt2Hoch1000}
	Bestimme die erste Stelle vor dem Komma und die erste Stelle nach dem Komma von $(1+\sqrt{2})^{1000}$.
\end{aufgabe*}
\begin{proof}[Lösung]
	 Betrachte die Zahlenfolge $(a_n)_{n\geqslant 0}$, die durch $a_n = (1+\sqrt{2})^n + (1-\sqrt{2})^n$ gegeben ist. Dann sind alle $a_n$ ganze Zahlen. Eine Methode, das zu zeigen, haben wir in Kapitel~\ref{kapitel:Pell}: \emph{Pellsche Gleichungen} kennengelernt: Wenn $z=x+y\sqrt{2}\in\mathbb Z[\sqrt{2}]$, dann ist $z+\overline{z}=2x$ eine ganze Zahl. Diese Beobachtung ist hier anwendbar, denn $\overline{(1+\sqrt{2})^n}=(\overline{1+\sqrt{2}})^n=(1-\sqrt{2})^n$
	
	Im Hinblick auf spätere Anwendungen geben wir noch einen weiteren Beweis für die Ganzzahligkeit von $a_n$. Es ist klar, dass $a_0 = a_1 = 2$ ganze Zahlen sind. Ferner sind $1\pm\sqrt{2}$ die Nullstellen des Polynoms $(X-1)^2-2=X^2-2X-1$. Nach der Theorie der linearen Rekursionen (siehe das Kapitel \emph{Lineare Rekursionen} im Heft für Klasse~10) erfüllt die Folge $(a_n)_{n\geqslant 0}$ demnach die Rekursionsgleichung $a_{n+2} = 2a_{n+1} + a_n$. Per Induktion ist dann klar, dass alle $a_n$ ganze Zahlen sein müssen.
	
	Insbesondere ist $a_{1000}$ eine ganze Zahl. Nun ist $-\frac12<1-\sqrt{2}<0$, folglich ist $(1-\sqrt{2})^{1000}$ eine mikroskopisch kleine positive reelle Zahl (kleiner als $2^{-1000}$). In jedem Fall beginnt ihre Dezimaldarstellung mit $0{,}000\dotso$, sodass die Dezimaldarstellung von $(1+\sqrt{2})^{1000}$ mit $\ldots{,}999\dotso$ aufhören muss. Folglich ist die erste Stelle nach dem Komma auf jeden Fall eine~$9$. 
	
	Um die erste Stelle vor dem Komma zu bestimmen, betrachten wir die Rekursionsgleichung $a_{n+2} = 2a_{n+1} + a_n$ modulo~$10$ und schauen, ab wann die Folge periodisch wird. Wir erhalten:
	\begin{center}
		\begin{tabular}{r | c c c c c c c c c c c c c c }\toprule
			$n$ & $0$ & $1$ & $2$ & $3$ & $4$ & $5$ & $6$ & $7$ & $8$ & $9$ & $10$ & $11$ & $12$ & $13$ \\
			$a_n \mod 10$ & $2$ & $2$ & $6$ & $4$ & $4$ & $2$ & $8$ & $8$ & $4$ & $6$ & $6$ & $8$ & $2$ & $2$ \\\bottomrule
		\end{tabular}
	\end{center}
	Aus der Tabelle folgt, dass $(a_n)_{n\geqslant 0}$ periodisch modulo~$10$ mit Periodenlänge~$12$ ist. Wegen $1000\equiv 4\mod 12$ können wir aus der Tabelle ablesen, dass $a_{1000}$ auf eine~$4$ endet. Weil nun $(1+\sqrt{2})^{1000}$ ein winziges Stückchen kleiner ist, muss die letzte Stelle vor dem Komma eine~$3$ sein. Damit ist die Aufgabe gelöst.
\end{proof}

Hinter der Lösung dieser Aufgabe steckt das folgende allgemeine Prinzip, das ihr euch unbedingt merken solltet:
\begin{enumerate}\itshape
	\item[$(*)$] Wenn in einer Aufgabe eine Nullstelle $\alpha$ eines irreduziblen\footnote{\emph{Irreduzibel} bedeutet, dass $P$ sich nicht als Produkt $P=QR$ mit zwei nicht-konstanten Polynomen $Q$ und $R$ mit rationalen Koeffizienten schreiben lässt.} Polynoms $P$ vorkommt, dann betrachte auch die anderen Nullstellen dieses Polynoms.
\end{enumerate}
Wenn $P$ ein normiertes quadratisches Polynom mit ganzzahligen Koeffizienten ist (so wie $P=X^2-2X-1$ in der obigen Lösung), dann ist die andere Nullstelle von $P$ genau die zu~$\alpha$ konjugierte Zahl. Wenn~$P$ hingegen höheren Grad hat, dann gibt es nicht länger nur \emph{eine} zu $\alpha$ konjugierte Zahl; stattdessen sind \emph{alle} anderen Nullstellen von $P$ zu $\alpha$ konjugiert.

Die Konjugierten von~$\alpha$ verhalten sich auch im allgemeinen Fall sehr ähnlich wie im quadratischen Fall. Wenn zum Beispiel $P$ ein normiertes Polynom vom Grad~$d$ mit ganzzahligen Koeffizienten ist und~$P$ die komplexen Nullstellen $\alpha_1,\alpha_2,\dotsc,\alpha_d\in\mathbb C$ besitzt (wobei jede Nullstelle entsprechend ihrer Vielfachheit oft gezählt wird), dann ist $\alpha_1^n+\alpha_2^n+\dotsb+\alpha_d^n$ für jedes $n\geqslant 0$ eine ganze Zahl, obwohl jedes einzelne $\alpha_i$ im Allgemeinen irrational ist. Es gilt sogar noch allgemeiner: Jeder \emph{symmetrische}\footnote{\emph{Symmetrisch} bedeutet, dass der Ausdruck in sich selbst übergeht, wenn beliebige $\alpha_i$ und $\alpha_j$ vertauscht werden.} polynomielle Ausdruck in $\alpha_1,\alpha_2,\dotsc,\alpha_d$ ist ganzzahlig. Um das einzusehen, betrachten wir die \emph{elementarsymmetrischen Polynome} in $d$ Variablen $X_1,X_2,\dotsc,X_d$:
\begin{equation*}
	\sigma_m\coloneqq \sum_{1\leqslant i_1 < i_2 < \dotsb < i_m\leqslant d} X_{i_1}X_{i_2}\dotsm X_{i_m}\quad\text{für }m=1,2,\dotsc,d\,.
\end{equation*}
Dann sind $\sigma_m(\alpha_1,\alpha_2,\dotsb,\alpha_d)$ ganze Zahlen, denn das Polynom
\begin{equation*}
	P(X)=(X-\alpha_1)(X-\alpha_2)\dotsm(X-\alpha_d)=X^m+\sum_{m=1}(-1)^m\sigma_m(\alpha_1,\alpha_2,\dotsc,\alpha_d)X^{m-d}
\end{equation*}
hat nach Annahme ganzzahlige Koeffizienten. Um zu zeigen, dass ein allgemeiner symmetrischer Ausdruck in $\alpha_1,\alpha_2,\dotsc,\alpha_d$ ebenfalls ganzzahlig ist, benutzen wir dann den folgenden Satz:
\begin{satzmitnamen}[Hauptsatz über symmetrische Funktionen]
	Jedes symmetrische Polynom in den Variablen $X_1,X_2,\dotsc,X_d$ mit ganzzahligen Koeffizienten lässt sich als Polynom in den elementarsymmetrischen Polynomen $\sigma_1,\sigma_2,\dotsc,\sigma_m$ mit ganzzahligen Koeffizienten schreiben.%Analoge Aussagen gelten auch für rationale/reelle/komplexe Koeffizienten.
\end{satzmitnamen}
\begin{proof}
	Wir benutzen Induktion nach $d$. Der Fall $d=1$ ist trivial. Sei nun $d\geqslant 2$ und wir nehmen an, dass die Behauptung für symmetrische Polynome in $d-1$ Variablen bereits bewiesen ist. Für alle $m=1,2,\dotsc,d-1$ betrachten wir das Polynom $\overline{\sigma}_m\coloneqq \sigma_m(X_1,X_2,\dotsc,X_{d-1},0)$. Dann ist $\overline{\sigma}_m$ genau das $m$-te elementarsymmetrische Polynom in den Variablen $X_1,X_2,\dotsc,X_{d-1}$.
	
	Sei nun $P$ ein beliebiges symmetrisches Polynom in den Variablen $X_1,X_2,\dotsc,X_d$ mit ganzzahligen Koeffizienten. Dann ist $\overline{P}\coloneqq P(X_1,X_2,\dotsc,X_{d-1},0)$ ein symmetrisches Polynom in den Variablen $X_1,X_2,\dotsc,X_{d-1}$ mit ganzzahligen Koeffizienten. Nach Voraussetzung gibt es ein Polynom $Q$ mit ganzzahligen Koeffizienten, sodass $\overline{P}=Q(\overline{\sigma}_1,\overline{\sigma}_2,\dotsc,\overline{\sigma}_{d-1})$. Betrachte nun das Polynom
	\begin{equation*}
		R\coloneqq P-Q(\sigma_1,\sigma_2,\dotsc,\sigma_{d-1})\,.
	\end{equation*}
	Weil $P$ und $Q(\sigma_1,\sigma_2,\dotsc,\sigma_{d-1})$ symmetrisch sind, muss auch~$R$ symmetrisch sein. Nach Konstruktion gilt ferner $R(X_1,X_2,\dotsc,X_{d-1},0)=0$. Also muss $R$ durch $X_d$ teilbar sein. Weil $R$ symmetrisch ist, muss~$R$ auch durch $X_1,X_2,\dotsc,X_{d-1}$ teilbar sein. Folglich ist $R$ durch das Produkt $X_1X_2\dotsm X_d=\sigma_d$ teilbar. Wir können also $R=\sigma_d\cdot P_1$ schreiben, wobei $P_1$ wiederum ein symmetrisches Polynom mit ganzzahligen Koeffizienten ist.
	
	Nun können wir das gleiche Argument mit $P_1$ statt $P$ wiederholen und iterieren. Weil in jedem Iterationsschritt der Totalgrad der betrachteten Polynome mindestens um~$d$ kleiner wird, haben wir nach endlich viele Schritten eine Darstellung von $P$ als Polynom in $\sigma_1,\sigma_2,\dotsc,\sigma_d$ mit ganzzahligen Koeffizienten gefunden.
\end{proof}

Der Hauptsatz über symmetrische Funktionen gilt völlig analog auch für Polynome mit rationalen, reellen oder komplexen Koeffizienten.\footnote{Allgemein sind Koeffizienten in jedem beliebigen \emph{Ring} erlaubt.}

Das Studium der Symmetrien von Nullstellen von Polynomen führt euch in die \emph{Galois-Theorie}. Die Strukturen, die entstehen, wenn wir zu $\mathbb Z$ die Nullstellen eines normierten irreduziblen Polynoms mit ganzzahligen Koeffizienten hinzufügen, werden in der \emph{Algebraischen Zahlentheorie} untersucht. Beides sind wunderschöne Gebiete der Mathematik, die ihr spätestens im Studium kennenlernen werdet.

\subsection*{Beispielaufgaben}

Die folgenden Aufgaben sind nach einem ähnlichen Muster wie Aufgabe~\ref{aufgabe:1+sqrt2Hoch1000} gestrickt. Am Ende des Kapitels findet ihr Tipps zu den Aufgaben und am Ende des Heftes könnt ihr die Lösungen nachlesen.

\begin{aufgabe*}\label{aufgabe:621246}
	Die Funktion $f$ mit der Gleichung $f(x)=x^3-3x^2+1$ hat drei reelle Nullstellen $\alpha<\beta<\gamma$.
	\begin{enumerate}[label={$(\alph*)$},ref={$(\alph*)$}]
		\item Zeige, dass $\lceil\gamma^n\rceil$ für jede positive ganze Zahl $n\geqslant 1$ durch~$3$ teilbar ist.\label{teilaufgabe:621246}
		\item Zeige, dass $\lfloor \gamma^{2020}\rfloor$ und $\lfloor \gamma^{2220}\rfloor$ durch~$17$ teilbar sind.\label{teilaufgabe:IMOSL1988}
	\end{enumerate}
	(\emph{Hierbei bezeichnet $\lfloor x\rfloor$ die größte ganze Zahl $\leqslant x$ und $\lceil x\rceil$ die kleinste ganze Zahl $\geqslant x$.}) 
\end{aufgabe*}
\begin{aufgabe*}\label{aufgabe:VAIMO2011_2}
	Sei $n$ eine positive ganze Zahl und sei
	\begin{equation*}
		b\coloneqq \left\lfloor \parens*{\sqrt[3]{28}-3}^{-n}\right\rfloor\,.
	\end{equation*}
	Zeige, dass~$b$ nicht durch~$6$ teilbar ist.
\end{aufgabe*}
\subsection*{Tipps zu den Beispielaufgaben}

\textbf{Tipp zu Aufgabe~\ref{aufgabe:621246}.} Betrachte $\alpha^n+\beta^n+\gamma^n$.

\textbf{Tipp zu Aufgabe~\ref{aufgabe:VAIMO2011_2}.} Betrachte $(\sqrt[3]{28}-3)^{-n}+(\sqrt[3]{28}\zeta-3)^{-n}+(\sqrt[3]{28}\zeta^2-3)^{-n}$, wobei $\zeta\coloneqq \mathrm{e}^{2\pi\mathrm{i}/3}$ eine dritte Einheitswurzel ist.
